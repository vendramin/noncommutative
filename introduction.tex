\section*{Introduction}

The notes correspond to the master  
course \emph{Non-commutative Algebra} of the 
Vrije Universiteit Brussel, 
Faculty of Sciences, 
Department of Mathematics and Data Sciences. The course
is divided into twelve two-hour lectures. 

Most of the material is based on standard
results of finite group theory.
Basic texts are... \cite{MR2426855}.

The notes include many exercises, some with full detailed solutions. Mandatory exercises have a \colorbox{green!5!white}{green background}, while optional ones
(bonus exercises) have a \colorbox{yellow!15!white}{yellow background}.

The notes also include some additional comments. While these are entirely optional, I hope they offer further insight. They are highlighted with a \colorbox{red!5!white}{pink background}.

The notes include Magma code, which we use to verify examples and offer alternative solutions to certain exercises. Magma \cite{zbMATH01077111} is a powerful software tool designed for working with algebraic structures. There is a free \href{https://magma.maths.usyd.edu.au/calc/}{online} version of Magma available.
% Part of the material is based on results on groups
% covered in the VUB course \emph{Associative Algebras}. Lecture  
% notes for this course are freely available  
% \href{https://github.com/vendramin/associative}{here}. 
% Basic texts on group algebras are Lam's book \cite{MR1125071}
% and Passman's book \cite{MR798076}.
% % are for example \cite{MR1645586}\dots

% As usual, we also mention a set of great expository papers by 
% Keith Conrad available at 
% \url{https://kconrad.math.uconn.edu/blurbs/}. 
% The notes are extremely well-written and are useful at  
% at every stage of a mathematical career. 
 
Thanks go to Arjen Elbert Dujardin, Ilaria Colazzo, Robynn Corveleyn, 
Luca Descheemaeker, Wannes Malfait, Lucas Simons, 
and
José Navarro Villegas. 


This version 
was compiled on \today~at~\currenttime.

% \bigskip
% \begin{flushright}
% Leandro Vendramin\\Brussels, Belgium\par
% \end{flushright}
