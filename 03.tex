\section{29/02/2024}

For subgroups $H$ and $K$ of $G$, let 
\[
[H,K]=\langle [h,k]:h\in H,\,k\in K\rangle.
\]

\index{Normalizer}
Let $G$ be a group and $K$ be a subgroup of $G$. We say that $K$ \textbf{normalizes} 
$H$ if $K\subseteq N_G(H)$.
\index{Centralizer}
We say that $K$ \textbf{centralizes} 
$H$ if $K\subseteq C_G(H)$, that is if and only if $[H,K]=\{1\}$.

\begin{exercise}
Let $K$ and $H$ be subgroups of $G$ such that $K\subseteq H$ and $K$ is normal in $G$.
Prove that $[H,G]\subseteq K$ if and only if $H/K\subseteq Z(G/K)$. 
\end{exercise}

\begin{lemma}
\label{lem:gamma_zeta}
Let $G$ be a group. There exists an integer $c$ such that 
$\zeta_c(G)=G$ if and only if 
$\gamma_{c+1}(G)=\{1\}$. In this case, 
\[
\gamma_{i+1}(G)\subseteq\zeta_{c-i}(G)
\]
for all $i\in\{0,1,\dots,c\}$. 
\end{lemma}

\begin{proof}
Assume first that $\zeta_c(G)=G$. To prove that 
$\gamma_{i+1}(G)\subseteq\zeta_{c-i}(G)$ holds for all $i$, we proceed by induction. 
The case $i=0$ is trivial. So assume that the result holds for some $i\geq0$. If
$g\in\gamma_{i+2}(G)=[G,\gamma_{i+1}(G)]$, then     
\[
g=\prod_{k=1}^N [x_k,g_k],
\]
for some $g_1,\dots,g_N\in\gamma_{i+1}(G)$ and $x_1,\dots,x_N\in G$. By the inductive 
hypothesis, 
	\[
	g_k\in\gamma_i(G)\subseteq\zeta_{c-i}(G)
	\]
for all $k$. Hence $[x_k,g_k]\in\zeta_{c-i-1}(G)$ for all $k$. Therefore   
$g\in\zeta_{c-(i+1)}(G)$. 
	
We now assume that $\gamma_{c+1}(G)=\{1\}$. We aim to prove that 
$\gamma_{i+1}(G)\subseteq\zeta_{c-i}(G)$ holds for all $i$. We proceed by backward induction on $i$. 
The case $i=c$ is trivial. So assume the result holds for some $i+1\leq c$. 
Let $g\in\gamma_{i}(G)$. By the inductive hypothesis, 
	\[
	[x,g]\in [G,\gamma_i(G)]=\gamma_{i+1}(G)\subseteq\zeta_{c-i}(G).
	\]
Thus $g\in\zeta_{c-i+1}(G)$ by definition. 
\end{proof}

\begin{example}
Let $G=\Sym_3$. Then $\zeta_j(G)=\{1\}$ for all $j\geq 0$: 
\begin{lstlisting}
gap> UpperCentralSeries(SymmetricGroup(3));
[ Group(()) ]
\end{lstlisting}
\end{example}

\begin{definition}
\index{Central series}
Let $G$ be a group. A \textbf{central series} for $G$ 
is a sequence 
	\[
		G=G_0\supseteq G_1\supseteq\cdots\supseteq G_n=\{1\}
	\]
of normal subgroups of $G$ such that 
for each $i\in\{1,\dots,n\}$, 
$\pi_i(G_{i-1})$ is a subgroup of $Z(G/G_i)$, where  $\pi_i\colon G\to
G/G_i$ is the canonical map.
\end{definition}

\begin{lemma}
\label{lem:central_series}
Let $G=G_0\supseteq G_1\supseteq\cdots\supseteq G_n=\{1\}$ be 
a central series of a group $G$. Then $\gamma_{i+1}(G)\subseteq G_i$ for all $i$.
\end{lemma}

\begin{proof}
We proceed by induction on $i$. The case $i=0$ is trivial. So assume the result holds for some
$i\geq0$. Let  
	$\pi_i\colon G\to
	G/G_i$ be the canonical map. 
	Then  
	\[
	\gamma_{i+1}(G)=[G,\gamma_i(G)]\subseteq [G,G_{i-1}].
	\]
	Since $\pi_i(G_{i-1})\subseteq Z(G/G_{i})$, 
	\[
        \pi_i([G,G_{i-1}])=[\pi_i(G),\pi_i(G_{i-1})]=\{1\}.
        \]
        Hence $\gamma_{i+1}(G)=[G,G_{i-1}]\subseteq G_i$. 
\end{proof}

% extender el lema para ver qué pasa con zeta_i

\begin{theorem}
A group is nilpotent if and only if it admits a central series. 
\end{theorem}

\begin{proof}
Let $G$ be a group. If $G$ is nilpotent, then the $\gamma_j(G)$ form a central series of 
$G$. Conversely, if $G=G_0\supseteq
G_1\supseteq\cdots\supseteq G_n=\{1\}$ is a central series of $G$, 
then, by the previous lemma,  
	\[
	\gamma_{n+1}(G)\subseteq G_n=\{1\}.
	\]
Hence $G$ is nilpotent. 
\end{proof}

\begin{exercise}
\label{xca:nilpotente_central}
Let $G$ be a group. Prove that if $K$ is a subgroup of $Z(G)$ such that 
$G/K$ is nilpotent, then $G$ is nilpotent. 
\end{exercise}

\subsection{Hirsch's theorem}

\begin{theorem}[Hirsch]
\label{thm:Hirsch}
\index{Hirsch's theorem}
Let $G$ be a nilpotent group. If $H$ is a non-trivial normal subgroup of $G$, 
then $H\cap Z(G)\ne\{1\}$. In particular, $Z(G)\ne\{1\}$. 
\end{theorem}

\begin{proof}
Since $\zeta_0(G)=\{1\}$ and there exists an integer $c$ such that $\zeta_c(G)=G$, 
there exists 
\[
m=\min\{k:H\cap\zeta_k(G)\ne\{1\}\}.
\]
Since $H$ is normal in $G$, 
\[
[G,H\cap\zeta_m(G)]\subseteq H\cap[G,\zeta_m(G)]\subseteq H\cap\zeta_{m-1}(G)=\{1\}.
\]
Therefore $\{1\}\ne H\cap\zeta_m(G)\subseteq H\cap Z(G)$. If $H=G$, then $Z(G)\ne\{1\}$. 
\end{proof}

\begin{exercise}
\label{xca:nilpotente_minimalnormal}
Let $G$ be a nilpotent group and $M$ be a minimal normal subgroup of $G$. Prove 
that $M\subseteq Z(G)$.
\end{exercise}

% \begin{svgraybox}
% 	Como $M\cap Z(G)$ es normal en $G$, la minimalidad de $M$ implica que hay
% 	dos posibilidades: $M\cap Z(G)$ es trivial o bien $M=M\cap Z(G)\subseteq Z(G)$.
% 	Por el teorema~\ref{theorem:Z(nilpotent)}, $M\cap Z(G)\ne 1$.
% \end{svgraybox}

\begin{definition}
\index{Subgroup|Maximal normal}
    Let $G$ be a group. A subgroup $M$ is said to be \textbf{maximal normal} in $G$
    if $M\ne G$ and $M$ is the only proper normal subgroup of $G$ containing $M$. 
\end{definition}

\begin{corollary}
Let $G$ be a non-abelian nilpotent group and $A$ be a maximal normal and abelian 
subgroup un subgroup of $G$. Then $A=C_G(A)$.
\end{corollary}

\begin{proof}
Since $A$ is abelian, $A\subseteq C_G(A)$. Assume that $A\ne C_G(A)$.
The centralizer $C_G(A)$ is normal in $G$, as, since $A$ is normal in $G$, 
\[
gC_G(A)g^{-1}=C_G(gAg^{-1})=C_G(A).
\]
for all $g\in G$. Let $\pi\colon G\to G/A$ be the canonical map. 
Then $\pi(C_G(A))$ is a non-trivial normal subgroup of $\pi(G)$. Since 
$G$ is nilpotent, $\pi(G)$ is nilpotent. By Hirsch's theorem, 
\[
\pi(C_G(A))\cap Z(\pi(G))\ne\{1\}.
\]
Let $x\in C_G(A)\setminus A$ be such that $\pi(x)$ is central in $\pi(G)$.  Then 
$\langle A,x\rangle$ is abelian, as $x\in C_G(A)$. Moreover,  $\langle
A,x\rangle$ is normal in $G$, as $A$ is normal in $G$ and 
$gxg^{-1}x^{-1}\in A$ for all  $g\in G$ (because $\pi(x)$ is central). Hence
	$gxg^{-1}\in \langle A,x\rangle$ and therefore $A\subsetneq \langle
	A,x\rangle\subsetneq G$, a contradiction.
\end{proof}

\begin{theorem}
Let $G$ be a nilpotent group. The following statements hold: 
\begin{enumerate}
\item Every minimal normal subgroup of $G$ has prime order and is central. 
\item Every maximal subgroup of $G$ is normal of prime index and contains $[G,G]$. 
\end{enumerate}
\end{theorem}

\begin{proof}\
\begin{enumerate}
    \item Let $N$ be a minimal normal subgroup of $G$. Since 
        $N\cap Z(G)\ne\{1\}$ by Hirsch's theorem, $N\cap Z(G)$ is a normal subgroup of 
        $G$ contained in $N$. Then $N=N\cap Z(G)\subseteq
	Z(G)$ by the minimality of $N$. In particular, $N$ is abelian. Since every subgroup of 
         $N$ is normal in $G$, $N$ is simple. Hence $N\simeq
	C_p$ for some prime number $p$.
 \item  If $M$ is a maximal subgroup, then $M$
is normal in $G$ by the normalizer condition (Lemma \ref{lem:normalizadora}). By the maximality
of $M$, the quotient $G/M$ contains no proper non-trivial subgroups. Thus 
$G/M\simeq C_p$ for some prime $p$. Since 
	$G/M$ is abeliano, $[G,G]\subseteq M$. \qedhere 
\end{enumerate}
\end{proof}

The previous theorem does not prove the existence of maximal subgroups. For example, 
$\Q$ is a nilpotent group (as it is abelian) 
that contains no maximal subgroups. 

\begin{proposition}
\label{pro:g^n}
Let $G$ be a nilpotent group and $H$ be a subgroup with $(G:H)=n$. If 
$g\in G$, then $g^n\in H$.
\end{proposition}

\begin{proof}
We proceed by induction on $n$. The case $n=1$ is trivial. The case  
$n=2$ follows from the normality of  $H$. So assume the result 
holds for all groups of index $<n$. Let $H$ be a subgroup of $G$ such that $(G:H)=n$. 
Let $H_0=H$ and $H_{i+1}=N_G(H_i)$ for all $i\geq0$. By definition, $H_{i}$ is normal in 
$H_{i+1}$. Since $G$ is nilpotent, $H_i\ne G$
implies that $H_i\subsetneq H_{i+1}$ by the normalizer condition. 
Since $(G:H)$ is finite, there exists $k$ such that  $H_k=G$. Since
$(H_j:H_{j-1})<n$ for all $j$, the inductive hypothesis implies that 
	$x^{(H_j:H_{j-1})}\in H_{j-1}$ for all $x\in H_j$ and all $j$. Hence 
	\[
		g^{(G:H)}=g^{(H_k:H_{k-1})(H_{k-1}:H_{k-2})\cdots (H_1:H_0)}\in H.\qedhere 
	\]
% 	El resultado es obvio en el caso en que $H$ sea un subgrupo normal.  Sea
% 	$H_0=H$ y $H_{i+1}=N_G(H_i)$ para $i\geq0$. Por definición, $H_{i}$ es
% 	normal en $H_{i+1}$ y además, como $G$ es nilpotente, si $H_i\ne G$
% 	entonces $H_i\subsetneq H_{i+1}$ por la condición normalizadora. 
% 	Como $(G:H)$ es finito, existe $k$ tal que $H_k=G$. Veamos que 
% 	\[
% 		g^{(G:H)}=g^{(H_k:H_{k-1})(H_{k-1}:H_{k-2})\cdots (H_1:H_0)}\in H.
% 	\]
% 	Observemos que $g^{(H_k:H_{k-1})}\in H_{k-1}$ pues $H_{k-1}$ es normal en $H_k=G$, y que, como 
% 	$g^{(H_k:H_{k-1})}\in H_k$, entonces 
% 	\[
% 	g^{(H_k:H_{k-2})}=g^{(H_k:H_{k-1})(H_{k-1}:H_{k-2})}=\left(g^{(H_k:H_{k-1})}\right)^{(H_{k-1}:H_{k-2})}\in H_{k-2}
% 	\]
% 	pues $H_{k-2}$ es normal en $H_{k-1}$. Al repetir este argumento, $g^{(G:H)}\in H$. 
\end{proof}

\begin{exercise}
\label{xca:g^n}
    Does the previous proposition hold for non-nilpotent groups? 
\end{exercise}

% \begin{sol}{xca:g^n}
% No. Take for example 
% $G=\Sym_3$ and $H=\{\id,(12)\}$. Then $H$ is index three in $G$ and 
% for  
% $g=(13)$ one has $g^{3}=(13)\not\in H$.
% \end{sol}

The following lemma is useful for performing induction 
in the nilpotency index of nilpotent groups. 

\begin{lemma}
\label{lem:a[GG]}
Let $G$ be a nilpotent group of class $c\geq2$. If $x\in G$, then 
the subgroup 
$\langle x,[G,G]\rangle$ is nilpotente of class $<c$.
\end{lemma}

\begin{proof}
Let $H=\langle x,[G,G]\rangle$.  If $x\in [G,G]$, the there is nothing to prove. 
So assume that $x\not\in [G,G]$. Note that 
	\[
		H=\{x^nc:n\in\Z,c\in [G,G]\},
	\]
as $[G,G]$ is normal in $G$. We need to show that
$[H,H]\subseteq\gamma_3(G)$. Let $h=x^nc,k=x^md\in H$
be such that $c,d\in [G,G]$. 
Since 
	\[
	[h,x^m]=[x^n,[c,x^m]][c,x^m]\in\gamma_4(G)\gamma_3(G)\subseteq\gamma_3(G),
	\]
then  
	\begin{align*}
		[h,k]&=[h,x^m][x^m,[h,d]][h,d]\\
			&=[x^n,[c,x^m]][c,x^m][x^m,[h,d]][h,d]\in\gamma_3(G).\qedhere
	\end{align*}
%	Como
%	$[G,G]=\gamma_2(G)\subseteq\zeta_{c-1}(G)$, al usar la definición de $H$ se
%	obtiene que $[G,G]\subseteq \zeta_{c-1}(G)\cap H\subseteq\zeta_{c-1}(H)$.
%	Luego $H/\zeta_{c-1}(H)$ es cíclico generado por $a\zeta_{c-1}(H)$. Si
%	$\zeta_{c-1}(H)=H$, no hay nada para demostrar. En caso contrario,
%	$Z(H)\subseteq\zeta_{c-1}(H)$ implicaría que $H$ es abeliano y luego $H$ es
%	nilpotente de clase uno.
\end{proof}

\begin{example}
Let $G=\D_{8}=\langle r,s:r^{8}=s^2=1,srs=r^{-1}\rangle$ the dihedral group of order 16.
Then $G$ is nilpotent of class three and 
$[G,G]=\{1,r^2,r^4,r^6\}\simeq C_4$. The subgroup $\langle
s,[G,G]\rangle\simeq\D_4$ is nilpotent of class two. 
\begin{lstlisting}
gap> G := DihedralGroup(IsPermGroup,16);;
gap> gens := GeneratorsOfGroup(G);;
gap> r := gens[1];;
gap> s := gens[2];;
gap> D := DerivedSubgroup(G);;
gap> S := Subgroup(G, Concatenation(Elements(D), [s]));;
gap> StructureDescription(S);
"D8"
gap> NilpotencyClassOfGroup(G);
3
gap> NilpotencyClassOfGroup(S);
2
	\end{lstlisting}
\end{example}

Let us discuss a concrete application of Lemma \ref{lem:a[GG]}.

\begin{theorem}
\label{thm:T(nilpotent)}
If $G$ is a nilpotent group, then 
\[
T(G)=\{g\in G:g^n=1\text{ for some $n\geq1$}\}
\]
is a subgroup of $G$. 
\end{theorem}

\begin{proof}
We proceed by induction on the nilpotency class of $G$. 
Let $a,b\in T(G)$ and 
\[
	A=\langle a,[G,G]\rangle,\quad
	B=\langle b,[G,G]\rangle.
\]
Since  $A$ and $B$ are nilpotent of class $<c$ by the previous lemma, 
the inductive hypothesis 
implies that $T(A)$ is a subgroup of $A$ and $T(B)$ is a subgroup of $B$.
Since $T(A)$ is characteristic in $A$ and $A$ is normal in $G$, $T(A)$ is normal in $G$. 
Similarly, $T(B)$ is normal in $G$. 

We claim that every element of 
$T(A)T(B)$ has finite order. If 
$x\in T(A)T(B)$, say $x=a_1b_1$ with
$a_1$ of order $m$, then $x$ has finite order, as 
\begin{align*}
x^m=(a_1b_1)^m&=%(a_1b_1a_1^{-1})(a_1^2b_1a_1^{-2})\cdots (a_1^{m-1} b_1 a_1^{-m+1}b_1)\\
(a_1b_1a_1^{-1})(a_1^2b_1a_1^{-2})\cdots (a^{m-1} b_1 a^{-m+1})b_1\in T(B).
\end{align*}

To see clearly what is what we did, let us work out a concrete 
example, say $m=3$. In this case, we obtain the following formula:
\begin{align*}
(a_1b_1)^3&=(a_1b_1)(a_1b_1)(a_1b_1)\\
&=(a_1b_1a_1^{-1})(a_1^2b_1a_1^{-2})a_1^3b_1
=(a_1b_1a_1^{-1})(a_1^2b_1a_1^{-2})b_1,
\end{align*}
as $a_1^3=1$.

With this trick, we prove that $ab$ and $a^{-1}$ have finite order. Hence $T(G)$ 
is a subgroup of $G$. 
\end{proof}

Another application:

\begin{theorem}
\label{thm:a=b}
Let $G$ be a torsion-free nilpotent group and $a,b\in G$. If there exists 
$n\ne 0$ such that $a^n=b^n$, then $a=b$.
\end{theorem}

\begin{proof}
We proceed by induction on the nilpotency order $c$ of $G$. 
The result clearly holds for abelian groups. Assume that $G$ is nilpotent of class $c\geq2$. 
Since $\langle a,[G,G]\rangle$ is a nilpotent subgroup of $G$ of class $<c$
and $bab^{-1}=[b,a]a\in \langle
a,[G,G]\rangle$, the inductive hypotehsis implies that 
$ba=ab$, as 
\[
	a^n=(bab^{-1})^n=b^n.
\]
Thus $(ab^{-1})^n=a^nb^{-n}=1$. Since $G$ has no torsion, we conclude that 
$a=b$.
\end{proof}

\begin{corollary}
Let $G$ be a torsion-free nilpotent group. If $x,y\in G$ are such that 
$x^ny^m=y^mx^n$ for some $n,m\ne 0$, then $xy=yx$.
\end{corollary}

\begin{proof}
Let $a=x$ and $b=y^nxy^{-n}$. Since $a^m=b^m$, the previous theorem implies that 
$a=b$. Thus $xy^n=y^nx$. Apply the previous theorem again, this time with 
$a=y$ and $b=xyx^{-1}$. Then we conclude that 
$xy=yx$. 
\end{proof}

Before proving another theorem, we recall a basic 
lemma about finitely generated groups. 

\begin{lemma}
\label{lem:fg}
Let $G$ be a finitely generated group and 
$H$ a finite-index subgroup. 
Then $H$ is finitely generated. 
\end{lemma}

\begin{proof}
Assume that $G$ is generated by $\{g_1,\dots,g_m\}$. Without loss of generality, we may assume that 
for each $i$ there exists $k$ such that $g_i^{-1}=g_k$. 
	
Let $\{1=t_1,\dots,t_n\}$ be a transversal of $H$ in $G$, that is 
a complete set of representatives of $G/H$. For $i\in\{1,\dots,n\}$ 
and $j\in\{1,\dots,m\}$, write 
\[
t_ig_j=h(i,j)t_{k(i,j)}.
\]
We claim that $H$ is generated by the $h(i,j)$. Let $x\in H$. Then  
	\begin{align*}
	x &=g_{i_1}\cdots g_{i_s}\\
	&= (t_1g_{i_1})g_{i_2}\cdots g_{i_s}\\
	&= h(1,i_1)t_{k_1}g_{i_2}\cdots g_{i_s}\\
	&= h(1,i_1)h(k_1,i_2)t_{k_2}g_{i_3}\cdots g_{i_s}\\
	&= h(1,i_1)h(k_1,i_2)\cdots h(k_{s-1},i_s)t_{k_s},
	\end{align*}
where $k_1,\dots,k_{s-1}\in\{1,\dots,n\}$. Since $t_{k_s}\in H$ (because $x\in H$), 
$t_{k_s}=1\in H$. Hence  $x$ is generated by the $h(i,j)$.
\end{proof}

Now the theorem:

\begin{theorem}
\label{thm:T(G)finito}
Let $G$ be a finitely generated torsion group that is nilpotent. 
Then $G$ is finite 
\end{theorem}

\begin{proof}
We proceed by induction on the nilpotency class $c$ of $G$. The case
$c=1$ is true, as $G$ is abelian. So assume the result holds for 
groups of class $c\geq1$. Since $[G,G]$ and $G/[G,G]$ are 
finitely generated (Lemma \ref{lem:fg}) torsion 
nilpotent groups of class  
$<c$, the inductive hypothesis implies that
$[G,G]$ and $G/[G,G]$ are finite groups. Thus 
$G$ is finite. % of order $|[G,G]|(G:[G,G])$.
\end{proof}

%\begin{lemma}
%	\label{lemma:kgenerators}
%	Sea $G$ un grupo y sea $G=G_0\supseteq G_1\supseteq\cdots\supseteq G_k=1$
%	una sucesión de subgrupos de $G$ tal que cada $G_{i+1}$ es normal en $G_i$
%	y cada $G_{i}/G_{i+1}$ es cíclico. Todo subgrupo de $G$ es finitamente
%	generado por $k$ elementos.
%\end{lemma}
%
%\begin{proof}
%	Procedemos por inducción en $k$. Supongamos primero que $k=1$. Entonces
%	$G\simeq G_0/G_1$ es cíclico y luego todo subgrupo de $G$ está generado por
%	un elemento. Supongamos ahora que el resultado es válido para $k\geq1$. Sea
%	$H$ un subgrupo de $G$, sea $N=G_{1}$ y sea $\pi\colon G\to G/N$ el
%	morfismo canónico. El grupo 
%	\[
%		\pi(H)\simeq H/H\cap N
%	\]
%	es cíclico pues un un subgrupo del grupo cíclico $G_k/G_{k-1}=G/N$. Como
%	existe $h\in H$ tal que $\pi(H)$ está generado por $\pi(h)$, se concluye que 
%	$H=\langle \pi(h),H\cap N\rangle$. Por hipótesis
%	inductiva, $H\cap N$ está generado por $k-1$ elementos y luego $H$ está
%	generado por $k$ elementos.
%\end{proof}
%
%\begin{theorem}
%	Sea $G$ un grupo nilpotente y finitamente generado. Entonces $T(G)$ es
%	finito.
%\end{theorem}
%
%%%% aca hay que hacer producto tensorial para construir una serie con factores cíclicos
%%%% ver libro de Khukhro
%%%% Nilpotent Groups and Their Automorphisms
%\begin{proof}
%	Sabemos por el teorema~\ref{theorem:} que existe una sucesión
%	$G=G_0\supseteq G_1\supseteq\cdots\supseteq G_k=G$ de subgrupos normales de
%	$G$ con factores cíclicos. 
%\end{proof}




%\subsection{Grupos finitos nilpotentes}
\subsection{Finite nilpotent groups}

Before studying finite nilpotent groups, we need a lemma. 

\begin{lemma}
\label{lem:normalizador}
Let $G$ be a finite group and $p$ a prime number dividing $|G|$. 
If 
$P\in\Syl_p(G)$, then 
\[
N_G(N_G(P))=N_G(P). 
\]
\end{lemma}

\begin{proof}
Let $H=N_G(P)$. Since $P$ is normal in $H$, $P$ is the only Sylow $p$-subgroup of $H$. 
To prove that $N_G(H)=H$, it is enough to see that $N_G(H)\subseteq
H$. Let $g\in N_G(H)$. Since  
\[
gPg^{-1}\subseteq gHg^{-1}=H,
\]
$gPg^{-1}\in\Syl_p(H)$ and $H$ has only one Sylow $p$-subgroup, 
$P=gPg^{-1}$.  Hence $g\in N_G(P)=H$. 
\end{proof}

\begin{theorem}
\label{thm:nilpotente:eq}
Let $G$ be a finite group. The following statements are equivalent:
\begin{enumerate}
	\item $G$ is nilpotent. 
	\item Every Sylow subgroup of $G$ is normal in $G$. 
	\item $G$ is a direct product of its Sylow subgroups. 
\end{enumerate}
\end{theorem}

\begin{proof}
We first prove that $(1)\implies(2)$. Let $P\in\Syl_p(G)$. We aim to prove that $P$ 
is normal in $G$, that is $N_G(P)=G$. By Lemma \ref{lem:normalizador}, 
$N_G(N_G(P))=N_G(P)$. Now the normalizer condition (Lemma \ref{lem:normalizadora}) implies that 
$N_G(P)=G$.

We now prove that $(2)\implies(3)$. Let $p_1,\cdots,p_k$ be the prime factors of 
$|G|$. For each $i\in\{1,\dots,k\}$, let  $P_i\in\Syl_{p_i}(G)$.
By assumption, each $P_j$ is normal in $G$.

We claim that $P_1\cdots P_j\simeq P_1\times\cdots\times P_j$ for all $j$.
The case $j=1$ is trivial. So assume the result holds for some 
$j\geq 1$. Since 
\[
N=P_1\cdots P_j\simeq P_1\times\cdots\times P_j
\]
is normal in $G$ and it has order coprime with $|P_{j+1}|$, 
\[
N\cap P_{j+1}=\{1\}.
\]
Hence 
\[
	NP_{j+1}\simeq N\times P_{j+1}\simeq P_1\times\cdots\times P_j\times P_{j+1}, 
\]
as $P_{j+1}$ is normal in $G$. 
Since now $P_1\cdots P_k\simeq P_1\times\cdots\times P_k$ is a subgroup of 
$G$ of order $|G|$, we conclude that $G=P_1\times\cdots\times P_k$.

Finally, we prove that $(3)\implies(1)$. We just need to note that 
every 
$p$-group is nilpotent (Proposition~\ref{pro:nilpotent_pgroups}) and that the direct product
of nilpotent groups is nilpotent. 
%(ejercicio~\ref{exercise:HxK_nilpotente}).
\end{proof}

\begin{exercise}
\label{xca:truco}
Let $G$ be a finite group. Prove that if $P\in\Syl_p(G)$ and $M$ is a subgroup of $G$ such that 
$N_G(P)\subseteq M$, then $M=N_G(M)$. 
\end{exercise}

% \begin{svgraybox}
% 	Sea $x\in N_G(M)$. Como $P\subseteq M$ y $M$ es normal en $N_G(M)$,
% 	$xPx^{-1}\subseteq M$.  Como $P$ y $xPx^{-1}$ son $p$-subgrupos de Sylow de
% 	$M$, existe $m\in M$ tal que 
% 	\[
% 	mPm^{-1}=xPx^{-1}.
% 	\]
% 	Luego $x\in M$ pues
% 	$m^{-1}x\in N_G(P)\subseteq M$. 
% \end{svgraybox}

\begin{exercise}
\label{xca:normalizadora}
Let $G$ be a finite group. Prove that the following statements are equivalent:
\begin{enumerate}
	\item $G$ is nilpotent.
	\item If $H\subsetneq G$ is a subgroup of $G$, then $H\subsetneq N_G(H)$.
	\item Every maximal subgroup of $G$ is normal in $G$.
\end{enumerate}
\end{exercise}

% \begin{svgraybox}
% 	Para demostrar que $(1)\implies(2)$ simplemente usamos el
% 	lema~\ref{lemma:normalizadora}. Para demostrar que $(2)\implies(3)$ hacemos
% 	lo siguiente: si $M$ es un subgrupo maximal, como $M\subsetneq N_G(M)$ por
% 	hipótesis, $N_G(M)=G$ por maximalidad. Finalmente demostremos que
% 	$(3)\implies(1)$.  Sea $P\in\Syl_p(G)$. Si $P$ no es normal en $G$,
% 	$N_G(P)\ne G$ y entonces existe un subgrupo maximal $M$ tal que
% 	$N_G(P)\subseteq M$. Como $M$ es normal en $G$, el
% 	ejercicio~\ref{exercise:truco} implica que $M=N_G(M)=G$, una contradicción.
% 	Luego $P$ es normal en $G$ y entonces $G$ es nilpotente por el
% 	teorema~\ref{theorem:nilpotente:eq}.
% \end{svgraybox}

% ejercicio: G finito. Es nilpotente si y solo si dos elementos de ordenes coprimos conmnutan
% 5.41 rotman

\begin{theorem}
Let $G$ be a finite nilpotent group. If $p$ is a prime number dividing 
$|G|$, there exist a minimal normal subgroup of order $p$ and 
there exists a maximal subgroup of index $p$. 
\end{theorem}

\begin{proof}
Assume that $|G|=p^{\alpha}m$ with $\gcd(p,m)=1$. 
Write $G=P\times H$, where $P\in\Syl_p(G)$.  Since $Z(P)$ is a non-trivial normal subgroup of
$P$, every subgroup of $Z(P)$ that is minimal normal in $G$ has order $p$ (and such subgroups exist because $G$ is finite). Since $P$ contains a subgroup of index $p$, 
it is maximal. Hence $P\times H$ contains 
a maximal subgroup of index $p$.
\end{proof}

\begin{exercise}
\label{xca:pgrupos}
Let $p$ be a prime number and $G$ be a non-trivial group 
of order $p^n$.
Prove the following statements:
\begin{enumerate}
	\item $G$ has a normal subgroup of order $p$.
	\item For every $j\in\{0,\dots,n\}$ there exists a normal subgroup 
            of $G$ of order $p^j$. 
\end{enumerate}
\end{exercise}

% \begin{svgraybox}
% 	\begin{enumerate}
% 		\item Sabemos que $Z(G)\ne1$. Sea $g\in Z(G)$ tal que $g\ne 1$.
% 			Supongamos que el orden de $g$ es $p^k$ para algún $k\geq1$.
% 			Entonces $g^{p^{k-1}}$ tiene orden $p$ y luego genera un subgrupo
% 			central de orden $p$. 
% 		\item Procederemos por inducción en $n$. Si $n=1$ el resultado es
% 			trivial.  Supongamos entonces que el resultado vale para un cierto
% 			$n\geq2$. Por el punto anterior, $G$ posee un subgrupo normal $N$
% 			de orden $p$. Luego $G/N$ tiene orden $p^{n-1}$. Sea $\pi\colon G\to G/N$ el morfismo canónico. 
% 			Por hipótesis
% 			inductiva, para cada $j\in\{0,\dots,n-1\}$. Por el teorema de la
% 			correspondecia, cada subgrupo normal $S_j$ de $G/N$ de orden $p^j$ se
% 			corresponde con un subgrupo $\pi^{-1}(S_j)$ de $G$ de orden $p^{j+1}$ pues, como
% 			$\pi$ es sobreyectiva, se tiene $\pi(\pi^{-1}(S_j))=S_j$, y luego
% 			\[
% 			p^j=|S_j|=|\pi(\pi^{-1}(S_j))|=\frac{|\pi^{-1}(S_j)|}{|\pi^{-1}(S_j)\cap N|}=\frac{|\pi^{-1}(S_j)|}{|N|}=\frac{|\pi^{-1}(S_j)|}{p}.
% 			\]
% 	\end{enumerate}
% \end{svgraybox}

\begin{exercise}
\label{xca:nilpotente_equivalencia}
Let $G$ be a finite group. Prove that the following statements are equivalent:
\begin{enumerate}
	\item $G$ is nilpotent.
	\item Any two elements of coprime order commute. 
 	\item Every non-trivial quotient of $G$ has a non-trivial center.
	\item If $d$ divides $|G|$, then there exists a normal subgroup of $G$ of order $d$. 
 \end{enumerate}
\end{exercise}

% \begin{svgraybox}
% 	Veamos que $(1)\implies(2)$. Sabemos que $G$ es producto directo de sus
% 	subgrupos de Sylow, digamos $G=\prod_{i=1}^k S_i$, donde los $S_i$ son los
% 	distintos subgrupos de Sylow de $G$.  Sean
% 	$x=(x_1,\dots,x_k),y=(y_1,\dots,y_k)\in G$. Como $|x|$ y $|y|$ son
% 	coprimos, para cada $i\in\{1,\dots,k\}$ se tiene $x_i=1$ o $y_i=1$. Luego
% 	\[
% 		[x,y]=([x_1,y_1],[x_2,y_2],\dots,[x_k,y_k])=1. 
% 	\]
% 	Demostremos ahora que $(2)\implies(1)$. Supongamos que
% 	$|G|=p_1^{\alpha_1}\cdots p_k^{\alpha_k}$, donde los $p_j$ son primos
% 	distintos y para cada $j$ sea $P_j\in\Syl_{p_j}(G)$. Como elementos de
% 	órdenes coprimos conmutan, la función $P_1\times\cdots\times P_k\to G$,
% 	$(x_1,\dots,x_k)\mapsto x_1\cdots x_k$, es un morfismo inyectivo de grupos.
% 	Como entonces $G\simeq P_1\times\cdots P_k$, y cada $P_j$ es nilpotente,
% 	$G$ es nilpotente. 

% 	Para demostrar que $(1)\implies(3)$ simplemente hay que observar que todo
% 	cociente de $G$ es nilpotente y luego utilizar el
% 	teorema~\ref{theorem:Z(nilpotent)}. Demostremos que $(3)\implies(1)$. Como
% 	todo cociente no trivial de $G$ tiene centro no trivial, en particular
% 	$Z_1=Z(G)$ es no trivial. Si $Z_1=G$ entonces $G$ es abeliano y no hay nada
% 	para demostrar. Si $Z_1\ne G$ entonces $G/Z_1\ne 1$; luego $Z(G/Z_1)\ne 1$.
% 	Si $\pi_1\colon G\to G/Z_1$ es el morfismo canónico,
% 	$Z_2=\pi_1^{-1}(Z(G/Z_1))$. Inductivamente, si tenemos construido el
% 	subgrupo $Z_i$, $Z_i\ne G$ y  $\pi_i\colon G\to G/Z_{i}$ es el morfismo
% 	canónico, se define el subgrupo $Z_{i+1}=\pi_i^{-1}(Z(G/Z_i))$. Por
% 	construcción, $Z_i\subseteq Z_{i+1}$ para todo $i$. Como $G$ es finito,
% 	existe $k$ tal que $Z_k=G$ y luego $G$ es nilpotente.

% 	Demostremos que $(1)\implies(4)$. Esta implicación es consecuencia
% 	inmediata del ejercicio~\ref{exercise:pgrupos}. 
% 	Como $G$ es nilpotente, $G$ producto
% 	directo de sus $p$-grupos de Sylow. Si $d=p_1^{\alpha_1}\cdots
% 	p_k^{\alpha_k}$ es un divisor del orden de $G$, basta tomar
% 	$H=H_1\times\cdots\times H_k$, 
% 	donde cada $H_j$ es un subgrupo normal del $p_j$-subgrupo de Sylow de $G$
% 	de orden $p_j^{\alpha_j}$. Para demostrar que $(4)\implies(1)$ simplemente
% 	se aplica la hipótesis a cada $p$-subgrupo de $G$ de orden maximal.
% \end{svgraybox}

\subsection{Baumslag--Wiegold theorem}

The following result can be proved with elementary tools 
and was discovered 
in 2014. 

\begin{theorem}[Baumslag--Wiegold]
\index{Baumslag--Wiegold theorem}
Let $G$ be a finite group such that $|xy|=|x||y|$ for all $x,y\in G$ of coprime orders. 
Then $G$ is nilpotent. 
\end{theorem}

\begin{proof}
Let $p_1,\dots,p_n$ be the prime factors of $|G|$. For 
each 
$i\in\{1,\dots,n\}$, let $P_i\in\Syl_{p_i}(G)$. We first prove that 
$G=P_1\cdots P_n$. To prove the non-trivial inclusion, we need to show that 
the map
\[
	\psi\colon P_1\times\cdots\times P_n\to G,\quad
	(x_1,\dots,x_n)\mapsto x_1\cdots x_n
\]
is surjective. We first show that $\psi$ is injective: If 
$\psi(x_1,\dots,x_n)=\psi(y_1,\dots,y_n)$, then 
\[
x_1\cdots x_n=y_1\cdots y_n. 
\]
If $y_n\ne x_n$, then $x_1\cdots x_{n-1}=(y_1\cdots
y_{n-1})y_nx_n^{-1}$. Since $x_1\cdots x_{n-1}$ has order coprime with 
$p_n$ and $y_1\cdots y_{n-1}y_nx_n^{-1}$ has order a multiple of 
$p_n$, we get a contradiction. Thus  $x_n=y_n$. The same argument shows that 
$\psi$ is injective. Since $|P_1\times\cdots\times
P_n|=|G|$, we conclude that $\psi$ is bijective. In particular, 
$\psi$ is surjective. 

We now prove that each $P_j$ is normal in $G$. Let $j\in\{1,\dots,n\}$ and 
$x_j\in P_j$. Let $g\in G$ and $y_j=gx_jg^{-1}$.  Since $y_j\in G$,
we can write $y_j=z_1\cdots z_n$ with $z_k\in P_k$ for all $k$.  Since
the order of $y_j$ is a power of $p_j$, the element $z_1\cdots
z_n$ has order a power of $p_j$. Thus $z_k=1$ for all $k\ne j$. Moreover, 
$y_j=z_j\in P_j$. Since every Sylow subgroup of $G$ is normal in $G$, 
we conclude that $G$ is nilpotent. 
\end{proof}

\subsection{It\^o's theorem}

\begin{definition}
\index{Grup!metabelian}
A group $G$ is said to be \textbf{metabelian} if $[G,G]$ is abelian. 
\end{definition}

\begin{exercise}
\label{xca:metabelian1}
Prove that a group $G$ is metabelian if and only if there exists a normal 
subgroup $K$ of $G$ such that $K$ and $G/K$ are abelian.
\end{exercise}

% \begin{remark}
% 	Los grupos metabelianos son solvables pues 
% 	si $G$ is metabeliano entonces 
% 	$G\supseteq [G,G]\supseteq 1$ is una serie solvable para $G$.
% \end{remark}

\begin{exercise}
\label{xca:metabelian2}
Let $G$ be a metabelian group. Prove the following statements: 
\begin{enumerate}
\item If $H$ is a subgroup of $G$, then $H$ is metabelian.
\item If $f\colon G\to H$ is a group homomorphism, then $f(H)$ is metabelian. 
\end{enumerate}
\end{exercise}

\begin{lemma}
In a group, the following formulas hold:
\begin{enumerate}
	\item $[a,bc]=[a,b]b[a,c]b^{-1}$. 
	\item $[ab,c]=a[b,c]a^{-1}[a,c]$.
\end{enumerate}
\end{lemma}

\begin{proof}
This is a straightforward calculation:
\begin{align*}
&[a,b]b[a,c]b^{-1}=aba^{-1}b^{-1}baca^{-1}c^{-1}b^{-1}=abca^{-1}c^{-1}b^{-1}=[a,bc],\\
&a[b,c]a^{-1}[a,c]=abcb^{-1}c^{-1}a^{-1}aca^{-1}c^{-1}=abcb^{-1}a^{-1}c^{-1}=[ab,c].\qedhere
\end{align*}
\end{proof}

\begin{example}
The group $\Sym_3$ is metabelian, as $\Alt_3\simeq C_3$ is a normal subgroup 
and the quotient $\Sym_3/\Alt_3\simeq C_2$ an abelian group. 
\end{example}

\begin{example}
The group $\Alt_4$ is metabelian, as the normal subgroup
\[
K=\{\id,(12)(34),(13)(24),(14)(23)\}
\]
is abelian and the quotient 
$\Alt_4/K\simeq C_3$ is abelian.
\end{example}

\begin{example}
The group $\SL_2(3)$ is not metabelian, as $[\SL_2(3),\SL_2(3)]\simeq Q_8$ 
is not abelian: 
\begin{lstlisting}
gap> IsAbelian(DerivedSubgroup(SL(2,3)));
false
gap> StructureDescription(DerivedSubgroup(SL(2,3)));
"Q8"
\end{lstlisting}
\end{example}

\begin{theorem}[It\^o]
\label{theorem:Ito}
Let $G=AB$ be a factorization of $G$ with $A$ and $B$ abelian 
subgroups of $G$. Then $G$ is metabelian.
\end{theorem}

\begin{proof}
Since $G=AB$ is a group, $AB=BA$. We claim that $[A,B]$ is a normal subgroup 
of $G$. Let $a,\alpha\in A$ and $b,\beta\in B$. Let  $a_1,a_2\in A$ and 
	$b_1,b_2\in B$ be such that $\alpha b\alpha^{-1}=b_1a_1$, $\beta
	a\beta^{-1}=a_2b_2$. Since 
	\begin{align*}
		&\alpha[a,b]\alpha^{-1}=a(\alpha b\alpha^{-1})a^{-1}(\alpha b^{-1}\alpha^{-1})=ab_1a_1a^{-1}a_1^{-1}b_1^{-1}=[a,b_1]\in [A,B]\\
		&\beta[a,b]\beta^{-1}=(\beta a\beta^{-1})\beta b\beta^{-1}(\beta a^{-1}\beta^{-1})b^{-1}=a_2b_2bb_2^{-1}a_2^{-1}b^{-1}=[a_2,b]\in [A,B],
	\end{align*}
	it follows that $[A,B]$ is normal in $G$. 

	We now claim that $[A,B]$ is abelian. Since 
	\begin{align*}
		&\beta\alpha[a,b]\alpha^{-1}\beta^{-1} = \beta[a,b_1]\beta^{-1}=(\beta a\beta^{-1})b_1(\beta a^{-1}\beta^{-1})b_1^{-1}=[a_2,b_1],\\
		&\alpha\beta[a,b]\beta^{-1}\alpha^{-1} = \alpha[a_2,b]\alpha^{-1}=a_2(\alpha b\alpha^{-1})a_2^{-1}(\alpha b\alpha^{-1})=[a_2,b_1],
	\end{align*}
	a direct calculation shows that 
	\[
		[\alpha^{-1},\beta^{-1}][a,b][\alpha^{-1},\beta^{-1}]^{-1}=[a,b].
	\]
	Two arbitrary generators of $[A,B]$ commute, so the group $[A,B]$ is abelian. 
	
	To finish the proof, note that $[G,G]=[A,B]$. In fact, 
	\[
	[a_1b_1,a_2b_2]=a_1[a_2,b_1]^{-1}a_1^{-1}a_2[a_1,b_2]a_2^{-1}\subseteq [A,B],
	\]
	as $[A,B]$ is normal in $G$. 
\end{proof}

In 1988 Sysak proved the following generalization 
of It\^o's theorem.  

\begin{theorem}[Sysak]
\index{Sysak's theorem}
    Let $A$ and $B$ be abelian subgroups of $G$. If $H$ is a subgroup of 
    $G$ contained in 
    $AB$, then $H$ is metabelian. 
\end{theorem}

For the proof, see \cite{MR988177}.

\subsection{Nilpotent groups of class two}

The following exercises go over groups 
of nilpotency class two. 

%\subsection{Grupos nilpotentes de clase dos}

\begin{exercise}
\label{xca:commutador}
Let $G$ be a group. Prove that 
if $x,y\in G$ are such that $[x,y]\in C_G(x)\cap C_G(y)$, then 
\[
[x,y]^n=[x^n,y]=[x,y^n]
\]
for all $n\in\Z$.
\end{exercise}

% \begin{proof}
% 	Procederemos por inducción en $n\geq0$. El caso $n=0$ es trivial. Supongamos entonces
% 	que el resultado vale para algún $n\geq0$. Entonces, como $[x,y]\in C_G(x)$, 
% 	\begin{align*}
% 		[x,y]^{n+1}&=[x,y]^n[x,y]
% 		=[x^n,y][x,y]=[x^n,y]xyx^{-1}y^{-1}=x[x^n,y]yx^{-1}y^{-1}=[x^{n+1},y].
% 	\end{align*}
% 	Para demostrar el lema en el caso $n<0$ basta observar que, como $[x,y]^{-1}=[x^{-1},y]$, 
% 	$[x,y]^{-n}=[x^{-1},y]^n=[x^{-n},y]$.
% \end{proof}

\begin{exercise}[Hall]
\label{xca:Hall}
Let $G$ be a class-two nilpotent group and 
$x,y\in G$. Prove that 
\[
(xy)^n=[y,x]^{n(n-1)/2}x^ny^n
\]
for all $n\geq1$.
\end{exercise}

% \begin{proof}
% 	Procederemos por inducción en $n$. Como el caso $n=1$ es trivial,
% 	supongamos que el resultado es válido para algún $n\geq1$. Entonces,
% 	gracias al lema anterior, 
% 	\begin{align*}
% 		(xy)^{n+1} &= (xy)^n(xy)=[y,x]^{n(n-1)/2}x^ny^{n-1}(yx)y\\
% 		&=[y,x]^{n(n-1)/2}x^{n}[y^n,x]xy^{n+1}=[y,x]^{n(n-1)/2}[y,x]^nx^{n+1}y^{n+1}.\qedhere 
% 	\end{align*}
% \end{proof}

\begin{exercise}
\label{xca:class2}
Let $p$ be an odd prime number and 
$P$ $p$-group of nilpotency class $\leq2$. 
Prove that if $[y,x]^p=1$ for all $x,y\in P$, then
$P\to [P,P]$,
$x\mapsto x^p$, is a group homomorphism. 
\end{exercise}

% \begin{proof}
% 	Por lema de Hall,
% 	$(xy)^p=[y,x]^{p(p-1)/2}x^py^p=x^py^p$. 	
% \end{proof}

\begin{exercise}
\label{xca:class2_torsion}
Let $p$ be an odd prime number and 
$P$ a $p$-group of nilpotency class $\leq2$. 
Prove that $\{x\in P:x^p=1\}$ is a subgroup of $P$.
\end{exercise}

% \begin{proof}
% 	Como $P$ tiene clase de nilpotencia dos, los conmutadores son centrales.
% 	Para cada $x\in G$, la función $g\mapsto [g,x]$ es un morfismo de grupos
% 	pues
% 	\[
% 		[gh,x]=ghxh^{-1}g^{-1}x^{-1}=g[h,x]xg^{-1}x^{-1}=[g,x][h,x].
% 	\]
% 	En particular, si $x,y\in P$ con $x^p=y^p=1$, entonces
% 	\[
% 		[x,y]^p=[x^p,y]=[1,y]=1.
% 	\]
% 	Luego, al usar el lema de Hall, se concluye que
% 	$(xy)^p=[y,x]^{p(p-1)/2}x^py^p=1$.
% \end{proof}


\subsection{Frattini subgroup}

\begin{definition}
\index{Frattini subgroup}
Let $G$ be a group. If $G$ has maximal subgroups, 
the \textbf{Frattini subgroup} is the intersection 
$\Phi(G)$ of all the maximal subgroups of $G$. 
Otherwise, 
$\Phi(G)=G$. 
\end{definition}

\begin{exercise}
\label{xca:Phi(G)char}
Prove that $\Phi(G)$ is a characteristic subgroup of $G$. 
\end{exercise}

\begin{example}
Sea $G=\Sym_3$. The maximal subgroups of $G$ are 
\[
M_1=\langle (123)\rangle,
\quad
M_2=\langle (12)\rangle,
\quad
M_3=\langle (23)\rangle,
\quad
M_4=\langle (13)\rangle.
\]
Hence $\Phi(G)=\{1\}$. 
\end{example}

\begin{example}
Let $G=\langle g\rangle\simeq C_{12}$. The subgroups of $G$ are 
\[
\{1\},\quad
\langle g^6\rangle\simeq C_2,\quad
\langle g^4\rangle\simeq C_3,\quad
\langle g^3\rangle\simeq C_4,\quad
\langle g^2\rangle\simeq C_6,\quad
G.
\]
Let us draw a picture:
\[\begin{tikzcd}
	& {C_{12}} \\
	{C_4} && {C_6} \\
	& {C_2} && {C_3} \\
	&& {\{1\}}
	\arrow[no head, from=4-3, to=3-2]
	\arrow[no head, from=4-3, to=3-4]
	\arrow[no head, from=3-4, to=2-3]
	\arrow[no head, from=2-3, to=1-2]
	\arrow[no head, from=3-2, to=2-1]
	\arrow[no head, from=2-1, to=1-2]
	\arrow[no head, from=3-2, to=2-3]
\end{tikzcd}\]
The maximal subgroups of $G$ are 
$\langle g^3\rangle\simeq C_4$ and $\langle
g^2\rangle\simeq C_6$. Hence $\Phi(G)=\langle g^3\rangle\cap \langle
g^2\rangle=\langle g^6\rangle\simeq C_2$. 
Let us see how to do this calculation with the computer:
\begin{lstlisting}
gap> G = CyclicGroup(12);;
gap> StructureDescription(FrattiniSubgroup(G));
"C2"
\end{lstlisting} 
\end{example}

\begin{lemma}[Dedekind]
\label{lem:Dedekind}
\index{Dedekind's!lemma}
Let $H$, $K$ and $L$ be subgroups of $G$ 
such that $H\subseteq L\subseteq G$. Then 
$HK\cap L=H(K\cap L)$.
\end{lemma}

\begin{proof}
One inclusion is trivial. Let us prove then that 
$HK\cap L\subseteq H(K\cap L)$. If 
$x=hk\in HK\cap L$ with $x\in L$, $h\in H$ and $k\in K$, then 
$k=h^{-1}x\in L\cap K$, as $H\subseteq L$. Thus $x=hk\in H(L\cap
	K)$.
\end{proof}

\begin{lemma}
\label{lem:G=HPhi(G)}
Let $G$ be a finite group and $H$ be a subgroup of $G$ such that 
$G=H\Phi(G)$. Then $H=G$.
\end{lemma}

\begin{proof}
If $H\ne G$, let $M$ be a maximal subgroup of $G$ such that 
$H\subseteq M$. Since $\Phi(G)\subseteq M$, $G=H\Phi(G)\subseteq M$, a 
contradiction. 
\end{proof}

\begin{proposition}
\label{pro:phi(N)phi(G)}
Let $N$ be a normal subgroup of a finite group $G$. Then 
$\Phi(N)\subseteq\Phi(G)$.
\end{proposition}

\begin{proof}
Since $\Phi(N)$ is characteristic in $N$ and $N$ 
is normal in $G$, $\Phi(N)$ is normal in $G$. 
If there exists a maximal subgroup $M$ such that 
$\Phi(N)\not\subseteq M$, then $\Phi(N)M=G$. (This happens
because, otherwise, $M=\Phi(N)M\supseteq\Phi(N)$.) By Dedekind's lemma (with  $H=\Phi(N)$, $K=M$ and $L=N$), 
\[
N=G\cap N=(\Phi(N)M)\cap N=\Phi(N)(M\cap N).
\]
By Lemma \ref{lem:G=HPhi(G)} (with $G=N$ and $H=M\cap N$), 
$\Phi(N)\subseteq N\subseteq M$, a contradiction. 
Hence every maximal subgroup of $G$ contains $\Phi(N)$ and therefore 
$\Phi(G)\supseteq\Phi(N)$. 
\end{proof}

The following proposition states that the 
elements of the Frattini subgroup are the \textbf{non-generators} of 
the group. 

\begin{proposition}
	\label{pro:nongenerators}
	Let $G$ be a finite group. Then 
 	\[
	\Phi(G)=\{x\in G:\text{if $G=\langle x,Y\rangle$ for some $Y\subseteq G$, then $G=\langle Y\rangle$}\}.
	\]
\end{proposition}

\begin{proof}
We first prove $\supseteq$. Let $x\in G$. If $M$ is a maximal subgroup of $G$ such that $x\not\in M$, then, since $G=\langle
	x,M\rangle$, we obtain that $G=\langle M\rangle=M$, a contradiction. Thus $x\in M$ for all maximal subgroup $M$ of $G$. Hence 
 $x\in \Phi(G)$. 

We now prove $\subseteq$. Let $x\in\Phi(G)$ be such that $G=\langle
	x,Y\rangle$ for some subset $Y$ of $G$. If $G\ne \langle Y\rangle$,
	there exists a maximal subgroup $M$ such that $\langle Y\rangle\subseteq M$. Since
	$x\in M$, $G=\langle x,Y\rangle\subseteq M$, a contradiction. 
\end{proof}

\begin{example}
For a prime number $p$, let $G$ be an elementary $p$-group, that is 
$G\simeq C_p^m$ for some $m\geq1$. Assume that 
	$G=\langle x_1\rangle\times\cdots\times\langle x_m\rangle$ with $\langle x_j\rangle\simeq C_p$.  
	We claim that $\Phi(G)$ is trivial. 
	For $j\in\{1,\dots,m\}$, let $n_j\in\{1,\dots,p-1\}$. Since 
	\[
	\{x_1,\dots,x_{j-1},x_j^{n_j},x_{j+1},\dots,x_m\}
	\]
	generates $G$ and $\{x_1,\dots,x_{j-1},x_{j+1},\dots,x_m\}$ does not, 
	$x_j^{n_j}\not\in\Phi(G)$ by Proposition \ref{pro:nongenerators}. 
	Hence $\Phi(G)=\{1\}$.
\end{example}

\begin{theorem}[Frattini]
\label{thm:Frattini}
\index{Frattini's!theorem}
Let $G$ be a finite group. Then $\Phi(G)$ is nilpotent.
\end{theorem}

\begin{proof}
Let $P\in\Syl_p(\Phi(G))$ for some prime number $p$. Since $\Phi(G)$ is normal in 
$G$, Lemma~\ref{lem:Frattini_argument} (Frattini's argument) implies that 
$G=\Phi(G)N_G(P)$. By Lemma~\ref{lem:G=HPhi(G)},
$G=N_G(P)$. Since every Sylow subgroup of $\Phi(G)$ is normal in $G$,
$\Phi(G)$ is nilpotent. 
\end{proof}

\begin{exercise}
\label{xca:G/M}
Let $G$ be a group and $M$ be a normal subgroup of $G$. Prove that if  
$M$ is maximal, then 
$G/M$ is cyclic of prime order. 
\end{exercise}

% \begin{svgraybox}
% 	Por el teorema de la correspondencia, $G/M$ no tiene subgrupos no trivales.
% 	Luego $G/M\simeq C_p$ para algún primo $p$.
% \end{svgraybox}

\begin{theorem}[Gasch\"utz]
	\label{thm:Gaschutz}
	\index{Gasch\"utz'!theorem}
	If $G$ is a finite group, then 
	\[
	[G,G]\cap Z(G)\subseteq\Phi(G).
	\]
\end{theorem}

\begin{proof}
Let $D=[G,G]\cap Z(G)$. Assume that $D$ is not contained in $\Phi(G)$.
Since $\Phi(G)$ is contained in every maximal subgroup of $G$, 
there is a maximal subgroup $M$ of $G$ not containing $D$. Then
$G=MD$. Since $D\subseteq Z(G)$, $M$ is normal in $G$, as 
	$g=md\in G=MD$ implies 
	\[
		gMg^{-1}=(md)Md^{-1}m^{-1}=mMm^{-1}=M.
	\]
	Since $G/M$ is cyclic of prime order, 
	$G/M$ is, in particular, abelian and hence $[G,G]\subseteq M$. Therefore 
	$D\subseteq [G,G]\subseteq M$, a contradiction.
\end{proof}

\begin{lemma}
\label{lem:N_G(H)=H}
Let $G$ be a finite group and $P\in\Syl_p(G)$. If $H$ is a subgroup of $G$ such that
$N_G(P)\subseteq H$, then $N_G(H)=H$.
\end{lemma}

\begin{proof}
Let $x\in N_G(H)$. Since $P\in\Syl_p(H)$ and $Q=xPx^{-1}\in\Syl_p(H)$, the second Sylow's theorem 
implies that there exists 
$h\in H$ such that $hQh^{-1}=(hx)P(hx)^{-1}=P$. Then $hx\in
N_G(P)\subseteq H$ and hence $x\in H$. 
\end{proof}

\begin{theorem}[Wielandt]
\label{thm:Wielandt}
\index{Wielandt's!theorem}
A finite group $G$ is nilpotent if and only if 
$[G,G]\subseteq\Phi(G)$.
\end{theorem}

\begin{proof}
Assume that $[G,G]\subseteq\Phi(G)$. Let $P\in\Syl_p(G)$. If $N_G(P)\ne
G$, then $N_G(P)\subseteq M$ for some maximal subgroup $M$ of $G$. If 
$g\in G$ and $m\in M$, then, since 
\[
	gmg^{-1}m^{-1}=[g,m]\in [G,G]\subseteq\Phi(G)\subseteq M,
\]
$M$ is normal in $G$. Furthermore $N_G(P)\subseteq M$. 
By Lemma~\ref{lem:N_G(H)=H},
\[
G=N_G(M)=M,
\]
a contradiction.
Thus $N_G(P)=G$ and every Sylow subgroup of $G$ si normal in $G$. Therefore 
$G$ is nilpotent. 

Conversely, assume that $G$ is nilpotent. Let $M$ be a maximal subgroup of $G$.
Since $M$ is normal in $G$ and maximal, $G/M$ has no proper non-trivial subgroups. 
Then $G/M\simeq C_p$ for some prime number $p$. In particular, $G/M$ is abelian
and $[G,G]\subseteq M$. Since $[G,G]$ is contained in every maximal subgroup of $G$, 
$[G,G]\subseteq\Phi(G)$.
\end{proof}

\begin{theorem}
\label{the:G/phi(G)}
A finite group $G$ is nilpotent if and only if 
$G/\Phi(G)$ is nilpotent. 
\end{theorem}

%%% TODO: la demostración no está bien explicada!
\begin{proof}
If $G$ is nilpotent, then $G/\Phi(G)$ is nilpotent. Conversely, assume that 
$G/\Phi(G)$ is nilpotent. Let $P\in\Syl_p(G)$. Since 
$\Phi(G)P/\Phi(G)\in\Syl_p(G/\Phi(G))$ and $G/\Phi(G)$ is nilpotent,
$\Phi(G)P/\Phi(G)$ is a normal subgroup of $G/\Phi(G)$. By the correspondence theorem, 
$\Phi(G)P$ is a normal subgroup of $G$.
Since $P\in\Syl_p(\Phi(G)P)$, Frattini's argument 
(Lemma~\ref{lem:Frattini_argument}) implies that 
\[
G=\Phi(G)PN_G(P)=\Phi(G)N_G(P), 
\]
as $P\subseteq N_G(P)$. Thus $G=N_G(P)$ by Lemma~\ref{lem:G=HPhi(G)}). Hence 
$P$ is normal in $G$ and therefore $G$ is nilpotent. 
\end{proof}

\begin{theorem}[Hall]
\index{Hall's!theorem}
\label{thm:Hall_nilpotente}
Let $G$ be a finite group with a normal subgroup $N$. If both $N$ and 
$G/[N,N]$ are nilpotent, then $G$ is nilpotent.
\end{theorem}

\begin{proof}
Since $N$ is nilpotent, $[N,N]\subseteq\Phi(N)$ by 
Wielandt's theorem~\ref{thm:Wielandt}. 
By Proposition~\ref{pro:phi(N)phi(G)},
$[N,N]\subseteq\Phi(N)\subseteq\Phi(G)$. 
By the universal property, there exists a surjective group homomorphism 
$G/[N,N]\to G/\Phi(G)$ such that the diagram 
    \[
    \begin{tikzcd}
	G & {G/\Phi(G)} \\
	{G/[N,N]}
	\arrow[from=1-1, to=1-2]
	\arrow[from=1-1, to=2-1]
	\arrow[dashed, from=2-1, to=1-2]
    \end{tikzcd}
    \]
%     \[
% 	\xymatrix{
% 	G
% 	\ar[d]
% 	\ar[r]
% 	& G/\Phi(G)
% 	\\
% 	G/[N,N]\ar@{-->}[ur]
% 	}
%    \xymatrix{ & P\ar[d]^f\ar@{-->}[ld]_h\\ M\ar[r]^g & N\ar[r] & 0 }
%    \]
is commutative. Since $G/[N,N]$ is nilpotent, $G/\Phi(G)$ is nilpotent by 
Exercise~\ref{xca:nilpotente}. Thus $G$ is nilpotent by the previous theorem. 
\end{proof}

\begin{definition}
A \textbf{minimal generating set} of a group $G$ is a set 
$X$ of generators of $G$ such that no proper subset of $X$ generates $G$. 
\end{definition}

Note that a minimal generating set does not necessarily have minimal size. 
	
\begin{example}
Let $G=\langle g\rangle\simeq C_6$.  If $a=g^2$ and 
$b=g^3$, then $\{a,b\}$ is a minimal generating set of $G$ that does not have
minimal size, as $G=\langle ab\rangle$.
\end{example}
	
For a prime number $p$, we write $\F_p$ to denote the field of $p$ elements. 

\begin{lemma}
\label{lem:Burnside:minimal}
Let $p$ be a prime number and 
$G$ be a finite $p$-group. Then $G/\Phi(G)$ is a vector space over $\F_p$.
\end{lemma}

\begin{proof}
Let $K$ be a maximal subgroup of $G$. Since $G$ is nilpotent 
(see Proposition~\ref{pro:nilpotent_pgroups}), 
$K$ is normal in $G$ (Exercise~\ref{xca:normalizadora}). 
Thus $G/K\simeq C_p$ because it is a simple $p$-group. 
	
It is enough to prove that $G/\Phi(G)$ is an elementary abelian $p$-group. It is a 
$p$-group because $G$ is a $p$-group. Let $K_1,\dots,K_m$ be the maximal subgroups 
of $G$. If $x\in G$, then $x^p\in K_j$ for all $j\in\{1,\dots,m\}$. Hence 
$x^p\in\Phi(G)=\cap_{j=1}^m K_j$. Moreover,  $G/\Phi(G)$ is abelian, as 
$[G,G]\subseteq \Phi(G)$ because $G$ is nilpotent (Wielandt's theorem~\ref{thm:Wielandt}). 
\end{proof}

\begin{theorem}[Burnside]
\label{thm:Burnside:basis}
Let $p$ be a prime number and $G$ a finite $p$-group. If $X$ is a minimal 
generating set of $G$, then $|X|=\dim G/\Phi(G)$. 
\end{theorem}

\begin{proof}
By Lemma \ref{lem:Burnside:minimal}, $G/\Phi(G)$ is a vector space over $\F_p$. 
Let $\pi\colon G\to G/\Phi(G)$ be the canonical map and 
$\{x_1,\dots,x_n\}$ be a minimal generating set of $G$.
We claim that $\{\pi(x_1),\dots,\pi(x_n)\}$ is a linearly independent subset of $G/\Phi(G)$. 
Assume this is not the case. Without loss of generality, let us assume that 
$\pi(x_1)\in\langle \pi(x_2),\dots,\pi(x_n)\rangle$. There exists $y\in
\langle x_2,\dots,x_n\rangle$ such that $x_1y^{-1}\in\Phi(G)$. Since $G$ is generated by 
$\{x_1y^{-1},x_2,\dots,x_n\}$ and $x_1y^{-1}\in\Phi(G)$, Proposition~\ref{pro:nongenerators}
implies that $G$ is generated by 
$\{x_2,\dots,x_n\}$, a contradiction to the minimality of $\{x_1,\dots,x_n\}$. 
Therefore $n=\dim G/\Phi(G)$.
\end{proof}

% TODO: agregar una aplicación (teorema de Hall). Ver Passman permutation groups, 11.7, pag 47

\begin{exercise}
Let $p$ be a prime number and $G$ a finite $p$-group. Prove that if $x\not\in\Phi(G)$, then 
$x$ belongs to some minimal set of generators of $G$. 
\end{exercise}
