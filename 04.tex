\section{07/03/2024}

\subsection{The Fitting subgroup}

\begin{definition}
\index{$p$-radical}
Let $G$ be a finite group and $p$ be a prime number. 
The \textbf{$p$-radical} of $G$ is the subgroup 
\[
O_p(G)=\bigcap_{P\in\Syl_p(G)}P.
\]
\end{definition}

\begin{lemma}
\label{lem:core:Op(G)}
Let $G$ be a finite group and $p$ be a prime number. The following statements hold: 
\begin{enumerate}
    \item $O_p(G)$ is normal in $G$.
    \item If $N$ is a normal subgroup of $G$ contained in some 
    $P\in\Syl_p(G)$, then $N\subseteq O_p(G)$.
\end{enumerate}
\end{lemma}

\begin{proof}
Let $P\in\Syl_p(G)$. Let $G$ act on $G/P$ by left multiplication. There is a group homomorphism 
$\rho\colon G\to\Sym_{G/P}$ with kernel 
\begin{align*}
		\ker\rho&=\{x\in G:\rho_x=\id\}\\
		&=\{x\in G:xgP=gP\;\forall g\in G\}\\
		&=\{x\in G:x\in gPg^{-1}\;\forall g\in G\}\\
    &=\bigcap_{g\in G}gPg^{-1}\\
    &=O_p(G).
\end{align*}
Then $O_p(G)$ is normal in $G$.

Let $N$ be a normal subgroup of $G$ such that $N\subseteq P$. Since 
$N=gNg^{-1}\subseteq gPg^{-1}$ for all $g\in G$, we conclude that 
$N\subseteq O_p(G)$.
\end{proof}

\begin{definition}
\index{Fitting!subgroup}
Let $G$ be a finite group and $p_1,\dots,p_k$ be the prime divisors of 
$|G|$. The \textbf{Fitting subgroup} of $G$ is the subgroup 
\[
F(G)=O_{p_1}(G)\cdots O_{p_k}(G)
\]
\end{definition}

\begin{exercise}
\label{xca:Fitting_char}
Prove that $F(G)$ is characteristic in $G$.
\end{exercise}

% \begin{svgraybox}
% 	Sea $f\in\Aut(G)$ y sea $p$ un primo. Como $f$ 
% 	permuta los $p$-subgrupos de Sylow de $G$, $f(O_p(G))=O_p(G)$. Luego
% 	$f(F(G))=F(G)$.
% \end{svgraybox}

\begin{example}
Let $G=\Sym_3$. Then $O_2(G)=\{1\}$ and $O_3(G)=\langle
(123)\rangle$. Hence $F(G)=\langle (123)\rangle$.
\end{example}

\begin{theorem}[Fitting]
\label{thm:Fitting}
\index{Fitting's theorem}
Let $G$ be a finite group. Then $F(G)$ is a nilpotent and normal in $G$.
Moreover, $F(G)$ contains every nilpotent normal subgroup of $G$.
\end{theorem}

\begin{proof}
By definition, $|F(G)|=\prod_p|O_p(G)|$.
Since $O_p(G)\in\Syl_p(F(G))$, we conclude that $F(G)$ is nilpotent, as it contains 
a normal Sylow $p$-subgroup for every prime $p$. 
Hence $F(G)$ is nilpotent by Theorem~\ref{thm:nilpotente:eq}.

Let $N$ be a nilpotent normal subgroup of $G$ and $P\in\Syl_p(N)$. Since 
$N$ is nilpotent, $P$ is normal in $N$ and hence $P$ is the only 
Sylow $p$-subgroup of $N$. Thus $P$ is characteristic in $N$ and 
$P$ is normal in $G$. Since $N$ is nilpotent, $N$ is a direct product of its Sylow subgroups. 
Therefore $N\subseteq O_p(G)$ by Lemma~\ref{lem:core:Op(G)}.
\end{proof}

\begin{corollary}
\label{cor:Z(G)subsetF(G)}
If $G$ is a finite group, then $Z(G)\subseteq F(G)$.
\end{corollary}

\begin{proof}
Since $Z(G)$ is nilpotent (in fact, it is abelian) and 
normal in $G$, by Fitting's theorem~\ref{thm:Fitting} we conclude that 
$Z(G)\subseteq F(G)$ 
\end{proof}

\begin{corollary}[Fitting]
\label{cor:Fitting}
Let $K$ and $L$ be nilpotent normal subgroups of a finite group $G$. 
Then $KL$ is nilpotent. 
\end{corollary}

\begin{proof}
By Fitting's theorem \ref{thm:Fitting}, $K\subseteq F(G)$ and 
$L\subseteq F(G)$. Then $KL\subseteq F(G)$ and $KL$ is nilpotent, as 
$F(G)$ is nilpotent. 
\end{proof}

\begin{corollary}
\label{cor:McapF(G)}
Let $N$ be a normal subgroup of a finite group $G$. Then 
$N\cap F(G)=F(N)$.
\end{corollary}

\begin{proof}
Since $F(N)$ is characteristic in $N$, $F(N)$ is normal in $G$. Then 
$F(N)\subseteq N\cap F(G)$ because $F(N)$ is nilpotent.  
Conversely, since $F(G)$ is normal in $G$, the subgroup $F(G)\cap N$ is normal in $N$. Since $F(G)\cap N$
is nilpotent, $F(G)\cap N\subseteq F(N)$. 
\end{proof}

We now discuss an application to finite solvable groups.

\begin{theorem}
Let $G$ be a non-trivial solvable group. Every normal non-trivial subgroup $N$ 
contains a normal abelian non-trivial subgroup. Moreover, this subgroup is contained in $F(N)$. 
% un grupo finito no trivial y resoluble. Todo subgrupo normal $N$ no
% trivial contiene un subgrupo normal abeliano no trivial y este subgrupo está en realidad 
% contenido en $F(N)$. 
\end{theorem}

\begin{proof}
Note that $N\cap G^{(0)}=N\ne\{1\}$. Since $G$ is solvable, there exists 
$m\geq1$ such that $N\cap G^{(m)}=\{1\}$. Let $n$ be the largest positive integer such that 
$N\cap G^{(n)}\ne\{1\}$. Since $[N,N]\subseteq N$ and $[G^{(n)},G^{(n)}]=G^{(n+1)}$, 
\[
[N\cap G^{(n)},N\cap G^{(n)}]\subseteq N\cap G^{(n+1)}=\{1\}.
\]
Then $N\cap G^{(n)}$ is an abelian subgroup of $G$. Moreover, it is normal and 
nilpotent. Hence 
\[
N\cap G^{(n)}\subseteq N\cap F(G)=F(N).\qedhere
\]
\end{proof}

\begin{theorem}
\label{thm:F(G)centraliza}
If $N$ is a minimal normal subgroup of a finite group $G$, then 
$F(G)\subseteq C_G(N)$.
\end{theorem}

\begin{proof}
By Fitting's theorem \ref{thm:Fitting}, $F(G)$ is a normal nilpotent group. 
The subgroup $N\cap F(G)$ is normal in $G$. Moreover, 
$[F(G),N]\subseteq N\cap F(G)$. If $N\cap F(G)=\{1\}$, then
$[F(G),N]=\{1\}$. Otherwise, $N=N\cap F(G)\subseteq F(G)$ by the minimality of $N$. 
By Hirsch's theorem, $N\cap Z(F(G))\ne \{1\}$ 
%~\ref{theorem:Z(nilpotent)}. 
Since $Z(F(G))$ is characteristic in $F(G)$ and 
$F(G)$ is normal in $G$, $Z(F(G))$ is normal in $G$. Since $\{1\}\ne N\cap Z(F(G))$ is 
normal in $G$, the minimality of $N$ implies that 
$N=N\cap Z(F(G))\subseteq
Z(F(G))$. Hence $[F(G),N]=\{1\}$. 
\end{proof}

\begin{corollary}
Let $G$ be a finite solvable group. The following statements hold: 
\begin{enumerate}
\item If $N$ is a minimal normal subgroup, then $N\subseteq Z(F(G))$. 
\item If $H$ is a non-trivial normal subgroup, then $H\cap F(G)\ne\{1\}$.
\end{enumerate}
\end{corollary}

\begin{proof}
Let us prove the first claim. Since $N$ is a $p$-group by Lemma~\ref{lemma:minimal_normal}, 
$N$ is nilpotent and hence $N\subseteq F(G)$. Moreover, $F(G)\subseteq C_G(N)$ by the previous theorem. 
Therefore $N\subseteq Z(F(G))$. 

Let us prove now the second claim. The subgroup $H$ contains a minimal normal subgroup $N$ and 
$N\subseteq F(G)$. Then $H\cap F(G)\ne\{1\}$. 
\end{proof}

\begin{theorem}
Let $G$ be a finite group. The following statements hold:
\begin{enumerate}
\item $\Phi(G)\subseteq F(G)$ and $Z(G)\subseteq F(G)$.
\item $F(G)/\Phi(G)\simeq F(G/\Phi(G))$.
\end{enumerate}
\end{theorem}

\begin{proof}
Let us prove the first claim. Since $\Phi(G)$ is normal in $G$, nilpotent by 
Frattini's Theorem~\ref{thm:Frattini} and $F(G)$ contains every normal nilpotent subgroup of $G$, 
$\Phi(G)\subseteq F(G)$. Moreover, $Z(G)$ is normal in $G$ and nilpotent. Hence 
$Z(G)\subseteq F(G)$.

Let us prove the second claim. Let $\pi\colon G\to G/\Phi(G)$ be the canonical map. Since 
$F(G)$ is nilpotent, $\pi(F(G))$ is nilpotent. Hence 
\[
\pi(F(G))\subseteq F(G/\Phi(G))
\]
by Fitting's Theorem~\ref{thm:Fitting}. Let 
$H=\pi^{-1}(F(G/\Phi(G)))$. By the correspondence theorem, $H$ is a normal 
subgroup of $G$ containing $\Phi(G)$. If $P\in\Syl_p(H)$, then 
$\pi(P)\in\Syl_p(\pi(H))$. In fact, $\pi(P)\simeq P/P\cap \Phi(G)$ is a $p$-group and 
$(\pi(H):\pi(P))$ is coprime with $p$ because 
	\[
	(\pi(H):\pi(P))
	=\frac{|\pi(H)|}{|\pi(P)|}
	=\frac{|H/\Phi(G)|}{|P/P\cap \Phi(G)|}
	=\frac{(H:P)}{(\Phi(G):P\cap\Phi(G))}
	\]
divides $(H:P)$, a number coprime with $p$. Since $\pi(H)$ is nilpotent, 
$\pi(P)$ is characteristic in $\pi(H)$. Then $\pi(P)$ is normal
in $\pi(G)=G/\Phi(G)$ and $P\Phi(G)=\pi^{-1}(\pi(P))$ is normal 
in $G$. Since $P\in\Syl_p(P\Phi(G))$, Frattini's argument (Lemma~\ref{lem:Frattini_argument}) implies that 
$G=\Phi(G)N_G(P)$. Therefore $P$
is normal in $G$ by Lemma~\ref{lem:G=HPhi(G)}. Since $P$ is nilpotent and normal in $G$, 
$P\subseteq F(G)$ by Fitting's theorem~\ref{thm:Fitting}. Hence 
$H\subseteq F(G)$ and 
$F(G/\Phi(G))=\pi(H)\subseteq \pi(F(G))$.
\end{proof}

%\begin{exercise}
%	Sea $G$ un grupo finito. Demuestre que
%	$F(G)/Z(G)\simeq F(G/Z(G))$.
%\end{exercise}
%
%\begin{svgraybox}
%	Sea $\pi\colon G\to G/Z(G)$ el morfismo canónico. 
%	Como $Z(G)$ es abeliano, $\pi(Z(G))$ es nilpotente y luego $\pi(Z(G))\subseteq F(G/Z(G))$. 
%\end{svgraybox}<++>