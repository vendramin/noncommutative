\section*{Some topics for final projects}

\fancyhf{}
\fancyfoot[R]{\thepage}
\fancyhead[L]{\course}
\fancyhead[R]{Final projects}
\setlength{\headheight}{14pt}

We collect here some topics for final presentations. Some topics
can also be used as bachelor's or master's theses.

\subsection*{Formanek's theorem}

Kaplansky zero divisors conjecture (Conjecture \ref{conjecture:zero}) 
holds for super-solvable groups. 
This is Formanek's theorem, see Theorem \ref{thm:Formanek:zerodivisors}. 

\subsection*{Carter subgroups}

The existence of Carter subgroups appears in Theorem \ref{thm:Carter}. For the proof of Carter's result, we refer to \cite{MR0123603}.

\subsection*{It\^o's theorem}

This nice theorem is surprisingly easy to prove. See Theorem \ref{thm:Ito}. There is a generalization
found by Sysak, see Theorem \ref{thm:Sysak}. 

\subsection*{The Mann subgroup}

How the set of sizes of conjugacy classes of a finite group influence 
the structure of the group? A partial answer goes back to Mann. See Theorem \ref{thm:Mann}. 

\subsection*{The Chermak--Delgado subgroup}

In \cite{MR0994774}, 
Chermak and Delgado presented a measuring argument for finite groups. They obtained some results about special families of finite groups, and powerful applications to finite simple groups and finite p-groups. See Theorem \ref{thm:ChermakDelgado}. 

\subsection*{Hall's subgroups}

For finite solvable groups, Hall developed a theory similar to that of Sylow. The proofs
presented in Theorems \ref{theorem:HallE} and \ref{theorem:HallC} (see page \pageref{theorem:HallC}) use
the Schur--Zassenhaus theorems. 

%\subsection*{Sylow systems}

\subsection*{Hall's marriage theorem}

Given a set of $n$ employees, fill out a list of the jobs each of them
would be able to preform. We can give each person the ``perfect'' job
if and only if for every $k\in\{1,\dots,n\}$ 
the union of any $k$ of the lists contains at least $k$ jobs. 

For a proof of the Hall marriage theorem, see \cite{MR33330}. 
The theorem is equivalent to several other
powerful theorems in combinatorics. Hall's theorem 
provides a combinatorial way of solving 
Exercise \ref{xca:Hall:cosets}. 

%\subsection*{Burnside's normal complement theorem}

\subsection*{The Wielandt automorphism tower theorem}

One striking application of subnormality is a beautiful result 
of Wielandt about the automorphism tower of a finite group 
with trivial center. See 
Theorem~\ref{thm:Wielandt:automorphism}.

\subsection*{Gardam's theorem}

Let $K$ be a field and $G$ be a torsion-free group.
What do the units of $K[G]$ look like? 
Gardam solved the Kaplansky units problem, answering 
negatively Question~\ref{question:units}. 
See Theorem~\ref{thm:Gardam_char2} for his solution
for fields of characteristic two.  

\subsection*{Passman's theorem}

There are several interesting open problems in the theory of group algebras. One 
of these problems is about zero-divisors in group algebras and another one 
is about group algebras being reduced as rings. Passman's theorem states
that these problems are equivalent. See Theorem \ref{thm:Passman}. 

\subsection*{The Alperin--Kuo theorem}

One way to prove the Alperin--Kuo theorem (see Theorem~\ref{thm:AlperinKuo}) 
is to combine 
ring-theoretical tools with the transfer homomorphism. The original 
proof uses group cohomology; see \cite{MR214674}. 
