\section{21/03/2024}


\subsection{Super-solvable groups}

\begin{definition}
\index{Group!super-solvable}
A group $G$ is said to be \textbf{super-solvable} if there exists a sequence 
\[
G=G_0\supseteq G_1\supseteq\cdots\supseteq G_n=\{1\}
\]
of normal subgroups of $G$ such that every 
quotient $G_{i-1}/G_i$ is cyclic. 
\end{definition}

In the previous definition, we do not require the group to be finite. Hence the quotients 
could be finite cyclic groups or isomorphic to $\Z$. 

\begin{example}
The dihedral group $\D_{n}$ of order $2n$ is super-solvable, as 
\[	
\D_{n}\supseteq \langle
r\rangle\supseteq \{1\}
\]
is a sequence of normal subgroups with cyclic factors. 
\end{example}

Every solvable group is super-solvable. See Exercise~\ref{xca:solvable}.

\begin{example}
The alternating group $\Alt_4$ solvable but not super-solvable. The only 
proper non-trivial normal subgroup of $\Alt_4$ is 
	\[
	\{\id,(12)(34),(13)(24),(14)(23)\}\simeq C_2\times C_2.
	\]
Thus $\Alt_4$ does not have a sequence of normal subgroups 
with cyclic factors. 
\end{example}

\begin{exercise}
\label{xca:Aff_supersolvable}
Prove that $\Aff(\Z)$ is super-solvable. 
\end{exercise}
% aff(Z) es súper-resoluble

\begin{example}
The group $\SL_2(3)$ is solvable but not super-solvable. Here is a computer verification: 
\begin{lstlisting}
gap> IsSolvable(SL(2,3));
true
gap> IsSupersolvable(SL(2,3));
false
\end{lstlisting}
\end{example}

\begin{exercise}
\label{xca:super}
Prove the following statements: 
\begin{enumerate}
\item Every subgroup of a super-solvable group is super-solvable. 
\item Quotients of super-solvable groups are super-solvable. 
\end{enumerate}
\end{exercise}

% \begin{svgraybox}
% 	Sea $G$ un grupo súper-resoluble y sea 			
% 	\[ 
% 	G=G_0\supseteq G_1\supseteq \cdots\supseteq G_n=1 
% 	\] 
% 	una sucesión de subgrupos normales
% 	donde cada cociente $G_{i-1}/G_{i}$ es cíclico. 
% 	\begin{enumerate}
% 		\item Sea $H$ un subgrupo de $G$. Como $G$ es
% 			súper-resoluble, Sea 
% 			\[
% 			H=H\cap G_0\supseteq H\cap G_1\supseteq\cdots\supseteq H\cap G_n=1
% 			\]
% 			una sucesión de subgrupos de $H$. Cada $H\cap G_i$ es normal en $H$
% 			pues $G_i$ es normal en $G$. Fijemos $i\in\{1,\dots,n\}$ y sea
% 			$\pi_{i-1}\colon G_{i-1}\to G_{i-1}/G_{i}$ el morfismo canónico. La
% 			restricción de $\pi_{i-1}$ al subgrupo $H\cap G_{i-1}$ es un morfismo con
% 			núcleo $G_{i}\cap H$.  Al usar el teorema de isomorfismos vemos que 
% 			\[
% 			\frac{H\cap G_{i-1}}{H\cap G_{i}}\simeq \pi_{i-1}(H\cap G_i)\subseteq G_{i-1}/G_i
% 			\]
% 			es un grupo cíclico por ser subgrupo de un grupo cíclico. 
% 		\item Sea $K$ un subgrupo normal de $G$ y sea $\pi\colon G\to G/K$ el
% 			morfismo canónico. Para cada $i$ sea $Q_i=\pi(G_i)$. Cada $Q_i$ es
% 			normal en $Q_n=\pi(G_n)=G/K$ pues $G_i$ es normal en $G$. Como
% 			$G_{i-1}K=G_{i-1}(G_iK)$ para todo $i$, 
% 			el grupo
% 			\begin{align*}
% 			Q_{i-1}/Q_i
% 			&\simeq\frac{G_{i-1}/G_{i-1}\cap K}{G_i/G_i\cap K}
% 			\simeq \frac{G_{i-1}K/K}{G_{i}K/K}\\
% 			&\simeq\frac{ G_{i-1}K}{G_iK}
% 			\simeq\frac{ G_{i-1}(G_iK)}{G_iK}
% 			\simeq\frac{ G_{i-1}}{G_iK\cap G_{i-1}}
% 			\simeq\frac{ G_{i-1}/G_i}{G_iK\cap G_{i-1}/G_i}
% 			\end{align*}
% 			es cíclico por ser un cociente de un grupo cíclico.
% 	\end{enumerate}
% \end{svgraybox}

\begin{exercise}
\label{xca:directosuper}
Prove that the direct product of super-solvable groups is super-solvable. 
\end{exercise}

% \begin{svgraybox}
% 	Supongamos que $G$ admite una sucesión $G=G_0\supseteq G_1\supseteq
% 	\cdots\supseteq G_n=1$ de de subgrupos normales tales que cada cociente
% 	$G_{i-1}/G_i$ es cíclico, y que $H$ admite una sucesión $H=H_0\supseteq
% 	H_1\supseteq \cdots\supseteq H_m=1$ de subgrupos normales donde cada
% 	$H_{i-1}/H_i$ es cíclico. Consideramos la sucesión 
% 	\[
% 		1=G_0\times H_0\supseteq G_1\times H_0\supseteq\cdots\supseteq G_n\times H_0\supseteq G_n\times H_1\supseteq \cdots\supseteq G_n\times H_m=G\times H
% 	\]
% 	tiene factores cíclicos pues 
% 	cada $G_{i-1}\times H_0/G_i\times H_0\simeq G_{i-1}/G_i$ es cíclico y cada 
% 	$G_n\times H_{j-1}/G_n\times H_j$ también pues
% 	\[
% 	G_n\times H_{j-1}/G_n\times H_j
% 	\simeq \frac{GH_{j-1}/G}{GH_j/G}
% 	\simeq \frac{H_{j-1}/H_{j-1}\cap G}{H_j/H_j\cap G}\simeq H_{j-1}/H_j.
% 	\]
% \end{svgraybox}

\begin{exercise}
\label{xca:super_quotient}
Let $H$ and $K$ be normal subgroups of a group $G$ such that $G/K$ and $G/H$
are super-solvable. Prove that $G/H\cap K$ is super-solvable. 
\end{exercise}

% \begin{svgraybox}
% 	El producto directo $G/H\times G/K$ es súper-resoluble. Sea $\partial\colon
% 	G\to G/H\times G/K$, $g\mapsto (gH,gK)$.  Como $\ker\partial=H\cap K$, se
% 	tiene que $G/H\cap K\simeq\partial(G)$, que es súper-resoluble por ser un
% 	subgrupo de un grupo súper-resoluble.
% \end{svgraybox}

\begin{exercise}
\label{xca:Nciclico}
Let $N$ be a cyclic normal subgroup of $G$. If $G/N$ is super-solvable, then 
$G$ is super-solvable. 
\end{exercise}

% todo: arreglar 

% \begin{proof}
% 	Sea $\pi\colon G\to G/N$ el morfismo canónico y sea $Q=G/N$. Como $Q$ es
% 	súper-resoluble, tenemos una sucesión
% 	\[
% 		Q=Q_0\supseteq Q_1\supseteq \cdots\supseteq Q_n=\{1\}
% 	\]
% 	de subgrupos normales de $Q$ tales que cada cociente $Q_{i-1}/Q_i$ es
% 	cíclico. Cada elemento de la sucesión
% 	\[
% 	G=\pi^{-1}(Q)\supseteq\pi^{-1}(Q_1)\supseteq\cdots\supseteq \pi^{-1}(Q_n)=N\supseteq \{1\}
% 	\]
% 	es normal en $G$ (por la correspondencia) y dejamos como 
% 	ejercicio demostrar que cada cociente es cíclico. 
% % 	cada cociente es cíclico $N$ es cíclico. 
% % 	Queda como ejercicio demostrar 
% % 	y cada 
% % 	\[
% % 	\frac{\pi^{-1}(Q_j)}{\pi^{-1}(Q_{j+1})}
% % 		=\frac{Q_jN}{Q_{j+1}N}
% % 		\simeq\frac{Q_jN/N}{Q_{j+1}N/N}
% % 		\simeq\frac{Q_j(Q_{j+1}N)}{Q_{j+1}N}
% % 		\simeq\frac{Q_j/Q_{j+1}}{Q_{j+1}N\cap Q_j}
% % 	\]
% % 	es cíclico por ser cociente de un grupo cíclico.
% \end{proof}

\begin{theorem}
\label{thm:ZorCp}
Let $G$ be a super-solvable non-trivial group. Then $G$ admits a sequence 
\[
G=G_0\supseteq G_1\supseteq\cdots\supseteq G_n=\{1\}
\]
of normal subgroups 
such that every quotient $G_{i-1}/G_i$ is cyclic of prime order or isomorphic to 
$\Z$.
\end{theorem}

\begin{proof}
Let $G=G_0\supseteq G_1\supseteq\cdots\supseteq G_n=\{1\}$ be a sequence of normal subgroups
of $G$ such that every quotient $G_{i-1}/G_i$ is cyclic. Let 
$i\in\{1,\dots,n\}$ be such that $G_{i-1}/G_i\simeq C_n$ for some non-prime  
$n$ and let $\pi\colon G_{i-1}\to G_{i-1}/G_i$ be the canonical map. 
Let $p$ be a prime divisor of $n$ and $H$ be a subgroup of $G$ such that 
$\pi(H)$ is a subgroup of $G_{i-1}/G_i$ of order $p$. By the correspondence theorem, 
$G_{i}\subseteq H\subseteq G_{i-1}$. 

We claim that $H$ is normal in $G$. Let $g\in G$. Since $\pi(gHg^{-1})$ is a subgroup of order $p$ of 
the cyclic group $G_{i-1}/G_i$, $\pi(gHg^{-1})=\pi(H)$. Then 
$gHg^{-1}=G_{i}H\subseteq H$ and hence $gHg^{-1}=H$. 
% 	\[
% 	\frac{gHg^{-1}}{G_i}=\frac{G_{i}H}{G_{i}}\simeq \frac{H}{G_i\cap H}=\frac{H}{G_i}
% 	\]
% 	y entonces $gHg^{-1}\subseteq H$.  

Note that $H/G_i$ is cyclic of prime order, as 
\[
H/G_i=H/H\cap G_i\simeq \pi(H)\simeq C_p. 
\]
Moreover, $G_{i-1}/H$ is cyclic, as 
\[
G_{i-1}/H\simeq\frac{G_{i-1}/G_i}{H/G_i}
\]
is the quotient of a cyclic group. 
	
We have shown that by adding $H$ to our sequence of normal subgroups, 
we obtain a sequence with cyclic factors where 
$H/G_{i}$ is cylic of prime order. Repeating this procedure, we obtain the desired result. 
\end{proof}

Let us discuss an immediate application. 

\begin{corollary}
A finite super-solvable group admits a sequence 
of normal subgroups where each quotient is cyclic of prime order. 
\end{corollary}

% \begin{proof}
% 	Es consecuencia inmediata del teorema~\ref{theorem:ZorCp}.
% \end{proof}

We now discuss other properties of super-solvable groups. 

\begin{theorem}
\label{thm:super_structure}
Let $G$ be a super-solvable group. The following statement hold:  
\begin{enumerate}
\item If $N$ is minimal normal in $G$, then $N\simeq C_p$ for some prime number $p$.
\item If $M$ is maximal in $G$, then $(G:M)=p$ for some prime number $p$.
\item The commutator subgroup $[G,G]$ is nilpotent. 
\item If $G$ is non-abelian, there exists a normal subgroup $N\ne G$ such that
	$Z(G)\subsetneq N$.
\end{enumerate}
\end{theorem}

\begin{proof}
Let us prove the first claim. Since $G$ is super-solvable, there exists a sequence 
\[
G=G_0\supseteq G_1\supseteq
G_2\supseteq\cdots\supseteq G_n=\{1\}
\]
of normal subgroups with cyclic factors. Since 
each $G_i\cap N$ is a normal subgroup of $G$ contained in $N$, 
the minimality implies that 
each $G_i\cap N$ is either trivial or equal to $N$. Moreover, $N\cap G_0=N$ and $N\cap
G_n=\{1\}$. Let $j$ be the smallest positive integer such that $N\cap G_j=\{1\}$. 
Since $N\subseteq G_{j-1}$ (because $N\cap G_{j-1}=N$), the composition 
	\[
	N\hookrightarrow G_{j-1}\to G_{j-1}/G_j
	\]
is an injective group homomorphism, as its kernel is equal $N\cap G_{j}=\{1\}$. 
Thus $N$ is cyclic, as it is isomorphic to a subgroup of the cyclic group $G_{i-1}/G_i$. 
If $G_{i-1}/G_i\simeq\Z$, then $N\simeq\Z$, a contradiction to the fact that $N$ is minimal normal. (For example, 
$2\Z$ is characteristic subgroup of $\Z$ and hence it is normal in $G$. Thus $N$ is cyclic and finite. Hence $N\simeq C_p$.)

We now prove the second claim. Let $M$ be a maximal subgroup of $G$. If $M$ is normal in $G$, 
then $G/M$ does not contain non-trivial proper subgroups. Then 
$G/M\simeq C_p$ for some prime number $p$. Assume that $M$ is not normal in $G$. 
Let $H=\cap_{g\in G}gMg^{-1}$ and $\pi\colon G\to G/H$ be the canonical map.  
Since $\pi(M)$ is maximal in 
	$\pi(G)=G/H$ and 
	\[
		(G:M)=(G/H:M/H)=(G/H:M/H\cap M)=(\pi(G):\pi(M)),
	\]
we may assume that $M$ does not contain non-trivial normal subgroups of $G$ (if needed, 
we just replace $G$ by $G/H$). Since $G$ is super-solvable, there exists a sequence 
$G=G_0\supseteq G_1\supseteq\cdots\supseteq G_n=\{1\}$ of normal subgroups of $G$ 
with factors either cyclic of prime order or isomorphic to $\Z$. Let 
$N=G_{n-1}$. Since $N$ is cyclic, every subgroup of $N$ is characteristic 
in $N$ and hence normal in $G$. In particular, $M\cap N$ is normal in 
$G$ and therefore $M\cap N=\{1\}$. Since $M\subseteq
NM\subseteq G$, the maximality of $M$ implies that either $M=NM$ or $G=NM$.
Since $N\subseteq NM=M$ yields a contradiction, we conclude that $G=NM$.

If $N\simeq C_p$ for some prime $p$, then $(G:M)=p$ and the proof is complete. 
Assume that $N\simeq\Z$. Let $H$ be a proper subgroup of $N$. Since 
$H$ is characteristic in $N$, $H$ is normal in $G$. Since 
$M\subseteq HM\subseteq NM=G$, the maximality of $M$ implies that either $HM=M$ or 
$HM=G$. Since $HM=M$ implies $H\subseteq M\cap N=\{1\}$,
we may assume that $HM=G$. If $n\in N\setminus H$, then $n=hm$ for some 
$h\in H$ and $m\in M$. Then $h=n$, as $h^{-1}n\in N\cap M=\{1\}$, a contradiction. 

We now prove the third claim. Since $G$ is super-solvable, 
there exists a sequence
	\[
	G=G_0\supseteq G_1\supseteq\cdots\supseteq G_n=\{1\}
	\]
of normal subgroups of $G$ such that each 
$G_i/G_{i+1}$ is cyclic. For  
	$i\in\{0,\dots,n\}$, let $H_i=[G,G]\cap G_i$. Since $[G,G]$ the each 
 $G_i$ are normal in $G$, one obtains a sequence 
	\[
	[G,G]=H_0\supseteq H_1\supseteq\cdots\supseteq H_n=\{1\}
	\]
of normal subgroups of $G$. Since $H_i$ and $H_{i+1}$ are normal in $G$, 
the group $G$ acts by conjugation on $H_i/H_{i+1}$. Thus there exists a group
homomorphism 
	$\gamma\colon G\to\Aut(H_i/H_{i+1})$. Since $H_i/H_{i+1}$ is cyclic, 
	$\Aut(H_i/H_{i+1})$ is abelian. Thus $[G,G]\subseteq\ker \gamma$. Therefore 
	$[G,G]$ acts trivially by conjugation on $H_{i}/H_{i+1}$. Hence 
 	\[
	H_i/H_{i+1}\subseteq Z([G,G]/H_{i+1}).
	\]

 Finally, we prove the fourth claim. Since $G$ is non-abelian,
	$Z(G)\ne G$. Let $\pi\colon G\to G/Z(G)$ be the canonical map. The group 
	$G/Z(G)$ is super-solvable and the sequence 
	\[
	G/Z(G)=\pi(G)\supseteq \pi(G_1)\supseteq\cdots\supseteq \pi(1)=\{1\}
	\]
is a sequence of normal subgroups of $G/Z(G)$ with cyclic quotients. 
In particular, $1\ne \pi(G_1)$ is a proper normal subgroup of $G/Z(G)$. By the correspondence theorem, $\pi^{-1}(\pi(G_1))\ne G$ is a normal subgroup of 
$G$ properly containing $Z(G)$. 
\end{proof}

There are solvable groups with a non-nilpotent derived subgroup. 

\begin{example}
The group $\Sym_4$ is solvable and $[\Sym_4,\Sym_4]=\Alt_4$ is not nilpotent.
\end{example}

\begin{proposition}
\label{pro:psuper}
Let $p$ be a prime number. Every finite $p$-group is super-solvable.
\end{proposition}

\begin{proof}
Let $G$ be a minimal counterexample. We may assume that $|G|=p^n$ for some 
$n>1$ (otherwise, if $n=1$, then $G$ is trivially super-solvable). 
The group $G$ is nilpotent and contains a normal subgroup $N$ of order $p$. 
Moreover, since $|G/N|=p^{n-1}$, the group $G/N$ is super-solvable. 
Since $N$ is cyclic and $G/N$ is super-solvable, 
$G$ is super-solvable by Exercise~\ref{xca:Nciclico}.
\end{proof}

% Como todo grupo finito nilpotente es producto directo de (finitos) subgrupos de
% Sylow, cada $p$-grupo es súper-resoluble y el producto directo de súper-resolubles es súper-resoluble, 
% se obtiene el siguiente resultado:

\begin{exercise}
\label{xca:nilpotent=>supersolvable}
Prove that finite nilpotent groups are super-solvable.
\end{exercise}

% \begin{proof}
% 	Todo grupo finito nilpotente es producto directo (finito) de subgrupos de
% 	Sylow. Como cada $p$-grupo es súper-resoluble por la
% 	proposición~\ref{proposition:psuper}, el resultado se obtiene
% 	inmediatamente del ejercicio~\ref{exercise:directosuper}.
% \end{proof}

\begin{theorem}
Super-solvable groups have maximal subgroups. 	
\end{theorem}

\begin{proof} 
We proceed by induction on the length of the super-solvable series. The claim holds for groups with a super-solvable series of length one, as in this case we are dealing with cyclic groups. So let 
$G$ be a group admitting a sequence
	\[
		G=G_0\supseteq\cdots\supseteq G_k=\{1\}
	\]
and suppose the theorem holds for super-solvable groups
with super-solvable series of length $<k$. Each  
$G_{k-1}$ is normal in $G$. Let $\pi\colon G\to
	G/G_{k-1}$ be the canonical map. 
The sequence 
 	\[
		G/G_{k-1}=\pi(G)\supseteq \pi(G_1)\supseteq\cdots\supseteq\pi(G_{k-1})=\{1\}
	\]
has length 
$<k$ and proves the super-solvability of $\pi(G)$. By the inductive hypothesis, 
$G/G_{k-1}$ admits maximal subgroups. By the correspondence theorem, 
$G$ admits maximal subgroups. 
\end{proof}

Solvable or nilpotent groups do not always admit maximal subgroups. Can you give an example?

\begin{definition}
	\index{Group!satisfying the maximal condition on subgroups}
A group $G$ satisfies the \textbf{maximal condition on subgroups} if
for every non-empty subset $\mathcal{S}$ of subgroups contains a maximal 
subgroup (i.e. a subgroup not contained in any other subgroup of $\mathcal{S}$). 
	%toda sucesión creciente
	%$S_1\subseteq S_2\subseteq S_3\subseteq\cdots$
	%de subgrupos es finita. 
	%%si todo subconjunto $\mathcal{S}$ 
	%%no vacío de subgrupos tiene un elemento maximal, es decir: existe
	%$M\in\mathcal{S}$ tal que $S\subseteq M$ para todo $S\in\mathcal{S}$.
\end{definition}

%\begin{lemma}
%	Un grupo $G$ satisface la la condición maximal para subgrupos si y sólo si
%	todo subconjunto $\mathcal{S}$ no vacío de subgrupos tiene un subgrupo
%	maximal (es decir, no contenido en ningún otro subgrupo de $\mathcal{S}$). 
%\end{lemma}

\begin{exercise}
\label{xca:MAX=fg}
A group satisfies the maximal condition on subgroups if and only if
every subgroup of $G$ is finitely generated. 
\end{exercise}

% \begin{proof}
% 	Supongamos que $G$ satisface la condición maximal para subgrupos y sea $H$
% 	un subgrupo de $G$.  Sea $\mathcal{S}$ el conjunto de subgrupos de $H$
% 	finitamente generados. Como $\mathcal{S}$ es no vacío (pues
% 	$1\in\mathcal{S}$), existe un elemento maximal $M\in\mathcal{S}$.  Sea
% 	$x\in H$. Como $\langle M,x\rangle\in\mathcal{S}$, $M=\langle M,x\rangle$ y
% 	luego $x\in M$. Como entonces $H=M$, $H$ es finitamente generado.
% 	%Supongamos que $G$ no es finitamente generado y satisface la condición maximal para subgrupos. Sea $1\ne g\in G$
% 	%y sea $S_1=\langle g_1\rangle$. Como $S_1\ne G$, existe $g_2\in G\setminus S_1$, y entonces 
% 	%$S_1\subseteq S_2=\langle x_1,x_2\rangle$. 

% 	Supongamos ahora que todo subgrupo de $G$ es finitamente generado. Si
% 	$\mathcal{S}$ es un subconjunto no vacío de subgrupos de $G$ sin elemento
% 	maximal, podemos construir una sucesión de subgrupos $S_1\subseteq
% 	S_2\subseteq\cdots$ que no se estabiliza (acá necesitamos utilizar el
% 	axioma de elección). Como la unión 
% 	\[
% 		S=\bigcup_{j\geq1}S_j 
% 	\]
% 	es un subgrupo de $G$, es finitamente generado y luego $S\subseteq S_k$
% 	para algún $k$ suficientemente grande, una contradicción.
% \end{proof}

% Una consecuencia inmediata. 

\begin{exercise}
Let $H$ be a subgroup of a group $G$. If $G$ satisfies 
the maximal condition on subgroups, then so does $H$. 
\end{exercise}

\begin{exercise}
\label{xca:max:G/N}
Let $G$ be a group and $N$ be a normal subgroup of $G$. If $G/N$ and $N$
satisfy the maximal condition on subgroups, then so does $G$.
\end{exercise}

% \begin{proof} 
% 	Sea $\pi\colon G\to G/N$ el morfismo canónico.  Sea $\mathcal{S}$ un
% 	subconjunto no vacío de subgrupos de $G$. El conjunto $\{S\cap
% 	N:S\in\mathcal{S}\}$ tiene un elemento maximal $A$ y el conjunto
% 	$\{\pi(S):S\in\mathcal{S},S\cap N=A\}$ tiene un elemento maximal $B$. Sea
% 	$S\in\mathcal{S}$ tal que $\pi(S)=B$ y $S\cap N=A$. Si $S$ no es maximal en
% 	$\mathcal{S}$, existe $T\in\mathcal{S}$ tal que $S\subseteq T$, $N\cap T=A$
% 	y $\pi(T)=B$. Sea $x\in T\setminus S$. Como $\pi(xN)=\pi(x)\in\pi(T)=B$,
% 	existe $y\in S$ tal que $xN=yN$. Luego $y^{-1}x\in N\cap T=A=N\cap S$, una
% 	contradicción pues $x\not\in S$. 
% \end{proof}

% TODO: agregar teorema de Huppert (ver por ejemplo Robinson, p. 268)
% corolario: G super si y sólo G/\Phi(G) super
% teorema de Iwasawa, Hall 342-345, 19.3
% teorema de Zappa-Ore, Duke 5 (1939), 431-460, Duke 6 (1940), 511-512

%\begin{definition}
%	\index{Grupo!que satisface la condición minimal para subgrupos}
%	Se dice que un grupo $G$ satisface la \textbf{condición minimal para
%	subgrupos} si todo subconjunto no vacío de subgrupos tiene un elemento
%	minimal.
%\end{definition}
%
%\begin{example}
%	El grupo $\Z$ no satisface la condición minimal para subgrupos pues
%	el conjunto $\{2^n\Z:n\in\N\}$ no posee elemento minimal. 
%\end{example}
%
%\begin{proposition}
%	Sea $G$ un grupo que satisface la condición minimal sobre subgrupos.
%	Entonces todo elemento de $G$ tiene orden finito.
%\end{proposition}
%
%\begin{proof}
%	Si existe $x\in G$ de orden infinito, la sucesión $\mathcal{S}$ de subgrupos 
%	\[
%	\langle x\rangle\supsetneq\langle x^2\rangle\supsetneq\langle
%	x^4\rangle\supsetneq\cdots\supsetneq\langle x^{2^k}\rangle\supsetneq\cdots
%	\]
%	tiene infinitos elementos y luego no posee un elemento minimal. 
%\end{proof}
%
%\begin{exercise}
%	\label{exercise:min:N}
%	Sea $G$ un grupo y sea $H$ un subgrupo de $G$.  Si $G$ satisface la
%	condición minimal para subgrupos entonces $H$ también. 
%\end{exercise}
%
%\begin{svgraybox}
%	Si $\mathcal{S}$ es un subconjunto no vacío de subgrupos de $H$, entonces
%	$\mathcal{S}$ posee un elemento minimal por ser un subconjunto no vacío de
%	subgrupos de $G$.
%\end{svgraybox}
%
%\begin{proposition}
%	\label{proposition:min:G/N}
%	Sea $G$ un grupo y sea $N$ un subgrupo normal de $G$.  Si $G/N$ y $N$
%	satisfacen la condición minimal para subgrupos entonces $G$ también. 
%\end{proposition}
%
%\begin{proof}
%	
%\end{proof}

\begin{proposition}
\label{pro:superfg}
Super-solvable groups satisfy the maximal condition on subgroups. In particular, 
every super-solvable group is finitely generated. 
\end{proposition}

\begin{proof}
We proceed by induction on the length of the super-solvable sequence. If the length
is one, the result holds as the group is cyclic. 
So assume the result holds for super-solvable groups with 
super-solvable series of length $\leq n-1$.  Let $G$
be a non-trivial super-solvable group and 
	\[
	G=G_0\supsetneq
	G_1\supsetneq\cdots\supsetneq G_n=\{1\}
	\]
a sequence of normal subgroups of $G$ with cyclic factors. Since 
$G_{1}$ is super-solvable (Exercise~\ref{xca:super}),
	$G_{1}$ satisfies the maximal condition on subgroups by the inductive
 hypothesis. By Exercise~\ref{xca:max:G/N}, $G$ satisfies the maximal
 condition on subgroups, as 
 $G/G_{1}$ is cyclic. 
\end{proof}

%\begin{proposition}\
%	\begin{enumerate}
%		\item Si un grupo súper-resoluble admite una serie de composición,
%			entonces es finito. 
%		\item Si un grupo súper-resoluble satisface la condición de minimal en
%			subgrupos entonces es finito.
%	\end{enumerate}
%\end{proposition}
%
%\begin{proof}
%	%Para probar la segunda afirmación obsevemos que todo cociente de $G$ es súper-resoluble 
%	%y que por el teorema~\ref{theorem:ZorCp} todo factor de la serie debe ser finito pues
%	%$\Z$ no satisface la condición minimal para subgrupos.
%\end{proof}

\begin{example}
The abelian group $\Q$ is nilpotent but not super-solvable, 
as it is not finitely generated. 
\end{example}

%	El grupo $\Sym_3$ es súper-resoluble pero no es nilpotente. 

If $G$ is a group and $x_1,\dots,x_{n+1}\in G$, 
let 
\[
[x_1,x_2\dots,x_{n+1}]=\left[ x_1,[x_2,\dots,x_{n+1}]\right],\quad
n\geq1.
\]

We will prove in Theorem~\ref{thm:super=fg} that nilpotent groups are super-solvable 
if and only if they are finitely generated. For this, we need two lemmas. 

\begin{lemma}
	\label{lem:G_n}
	Let $G$ be a finite generated group, say $G=\langle X\rangle$ for some finite set $X$.  
	For $n\geq2$, let 
 	\[
		G_n=\langle g[x_1,\dots,x_n]g^{-1}:x_1,\dots,x_n\in X,\,g\in G\rangle.
	\]
	Then $G_n=\gamma_n(G)$ for all $n\geq2$. 
\end{lemma}

\begin{proof}
	Note that each $G_n$ is normal in $G$. We proceed by induction on $n$. The case 
    $n=2$ is trivial. So let us assume that 
    $\gamma_{n-1}(G)=G_{n-1}$ for some $n\geq2$. Let $x_1,\dots,x_n\in X$. Since 
	$[x_1,\dots,x_n]\in\gamma_{n}(G)$, $G_{n-1}\subseteq\gamma_n(G)$. Let 
	$N=G_n$ and $\pi\colon G\to G/N$ be the canonical map. The group $G/N$ is finitely generated. Since 
	\[
	[\pi(x_1),[\pi(x_2),\dots,\pi(x_{n})]]=\pi([x_1,\dots,x_n])=1,
	\]
	we obtain that $\pi([x_2,\dots,x_{n}])\in Z(G/N)$. Hence 
	$\pi(g[x_2,\dots,x_n]g^{-1})=1$ for all $g\in G$. By the inductive hypothesis, 
 	\[
	\pi(\gamma_{n-1}(G))=\pi(G_{n-1})\subseteq Z(G/N).
	\]
	Since  
	\[
	\pi(\gamma_{n}(G))=\pi([G,\gamma_{n-1}(G)])=[\pi(G),\pi(\gamma_{n-1}(G))]=\{1\},
	\]
	we conclude that $\gamma_n(G)\subseteq N=G_n$.
\end{proof}

\begin{lemma}
	\label{lem:gamma_n/gamma_n+1}
	Let $G$ be a finitely generated group. Then 
 	$\gamma_n(G)/\gamma_{n+1}(G)$ is finitely generated. 
\end{lemma}

\begin{proof}
	Assume that $G=\langle X\rangle$ for some finite set $X$. Write 
	\[
	g[x_1,\dots,x_n]g^{-1}=[g,[x_1,\dots,x_n]][x_1,\dots,x_n]. 
	\]
	By Lemma~\ref{lem:G_n}, 
    $[g,[x_1,\dots,x_n]]\in \gamma_{n+1}(G)=G_{n+1}$. Then 
	\[
	g[x_1,\dots,x_n]g^{-1}\equiv [x_1,\dots,x_n]\bmod \gamma_{n+1}(G). 
	\]
	Hence $\gamma_{n}(G)/\gamma_{n+1}(G)$ is generated by the finite set 
	\[
	\{[x_1,\dots,x_n]\gamma_{n+1}(G):x_1,\dots,x_n\in X\}. \qedhere 
	\]
\end{proof}

\begin{theorem}
    \label{thm:super=fg}
    Let $G$ be a nilpotent group. Then $G$ is super-solvable if and only if 
    $G$ is finitely generated. 
\end{theorem}

\begin{proof}
    If $G$ is super-solvable, it is then finitely generated by 
    Proposition~\ref{pro:superfg}.  
    
    Now assume that the nilpotent group $G$ is finitely generated. By Lemma~\ref{lem:gamma_n/gamma_n+1}, 
    each quotient $\gamma_{n}(G)/\gamma_{n+1}(G)$ is finitely generated, say by 
    the elements $y_1,\dots,y_m$. Let $\pi\colon G\to G/\gamma_{n+1}(G)$ the canonical map. 
    For $j\in\{1,\dots,m\}$, let 
    \[
    K_j=\langle \gamma_{n+1}(G),y_1,\dots,y_j\rangle.
    \]
    Since $[G,K_j]\subseteq [G,\gamma_n(G)]=\gamma_{n+1}(G)$, 
    we obtain that $\pi(K_j)$ is central in $\pi(G)$. Thus $\pi(K_j)$ is normal
    in $\pi(G)$. Hence $K_j$ is normal in $G$. Each quotient $K_j/K_{j-1}$
    is cyclic and generated by $y_jK_{j-1}$. Therefore, in between $\gamma_n(G)$ and 
    $\gamma_{n+1}(G)$, we have constructed a sequence of normal subgroups of $G$ 
    with cyclic factors. Since $G$ is nilpotent, there exists an integer $c$ such that 
    $\gamma_{c+1}(G)=\{1\}$. Hence $G$ is super-solvable. 
\end{proof}

\begin{corollary}
	\label{cor:nilpotente=>max}
    Every finitely generated nilpotent group satisfies the maximal condition on subgroups. 
\end{corollary}

\begin{proof}
    This is an immediate consequence of Proposition~\ref{pro:superfg} and 
    Theorem~\ref{thm:super=fg}.  
\end{proof}

\begin{theorem}
    Let $G$ be a nilpotent finitely generated group. Then $T(G)$ is finite. 
\end{theorem}

\begin{proof}
    Since $G$ is nilpotent, $G$ satisfies the maximal condition on subgroups 
    (Corollary~\ref{cor:nilpotente=>max}). Thus 
	every subgroup of $G$ is finitely generated. Since 
    $T(G)$ is a subgroup (Theorem~\ref{thm:T(nilpotent)}), it is a torsion finitely generated group. 
	Hence $T(G)$ is finite by Theorem~\ref{thm:T(G)finito}.
\end{proof}

\subsection{*Huppert's super-solvable theorem}

\begin{theorem}[Huppert]
\index{Hupper's super-solvable theorem}
\label{thm:Huppert}
Let $G$ be a finite group such that all its maximal subgroups 
are of prime index. Then $G$ is super-solvable. 
\end{theorem}

\begin{proof}
    See \cite[Theorem 10.5.8]{MR414669}.
\end{proof}

\subsection{*Formanek's zero divisors theorem}

We start recalling a conjecture formulated by Kaplansky as \cite[Problem 6]{MR0096696} in 1957. 

\begin{conjecture}[Kaplansky]
\index{Kaplansky's zero divisors conjecture}
\label{conjecture:zero}
    Let $K$ be a field and $G$ be a torsion-free group. 
    Then $K[G]$ has no zero divisors. 
\end{conjecture}

In 1973 Formanek proved the following result:

\begin{theorem}[Formanek]
    \index{Formanek's zero divisors theorem}
    \label{thm:Formanek:zerodivisors}
        Let $G$ be a torsion-free super-solvable 
        group and $K$ be a field. Then $K[G]$ has no zero divisors. 
\end{theorem}

\begin{proof}
    See \cite[Thoerem 13.3.9]{MR798076}.
\end{proof}

Conjecture \ref{conjecture:zero} is also known to hold, for example, when $G$ admits a bi-order (proved by Malcev and independently by Neumann), when $G$ is polycyclic-by-finite (proved by Brown and Farkas--Snider), or when $G$ has the unique product property (proved by Cohen).  In full generality, the conjecture is still open. See~\cite[Chapter 13]{MR798076} for more information. 


\subsection{The Schur--Zassenhaus theorem}

% Para leer este capítulo es conveniente haber entendido el capítulo \ref{derivaciones}, ya
% que demostraremos el teorema de Schur--Zassenhaus gracias al uso de algunos trucos que
% involucran derivaciones. También es necesario utilizar el subgrupo de Frattini, capítulo \ref{Frattini}. 
% Daremos una aplicación 
% del teorema de Schur--Zassenhaus a grupos súper-resolubles, estudiados en el capítulo \ref{super_resoluble}. 

Recall that a group $Q$ \textbf{acts by automorphisms} on a group $K$ if 
there exists a map $Q\times K\to K$, $(q,k)\mapsto q\cdot k$, 
such that 
\begin{enumerate}
    \item $1\cdot a=a$ for all $a\in K$, 
    \item $x\cdot (y\cdot a)=(xy)\cdot a$ for all $x,y\in Q$ and $a\in K$, 
    \item $x\cdot 1=1$ for all $x\in Q$, and 
    \item $x\cdot (ab)=(x\cdot a)(x\cdot b)$ for all $x\in Q$ and $a,b\in K$, 
\end{enumerate}
For example, if $K$ is a normal subgroup of $G$, 
then $G$ acts by automorphisms on $K$ by conjugation. 

\begin{definition}
\index{1-cocycle}
Let $Q$ and $K$ be groups, where $Q$ acts by automorphisms on $K$. 
A map 
$\varphi\colon Q\to K$ is said to be a \textbf{1-cocycle} if 
\[
	\varphi(xy)=\varphi(x)(x\cdot\varphi(y))
\]
for all $x,y\in Q$.  
\end{definition}

Let $Q$ and $K$ be groups, where $Q$ acts by automorphisms on $K$. 
The set of 1-cocycles $Q\to K$ will be denoted by 
\[
Z^1(Q,K)=\{\delta\colon Q\to K:\text{$\delta$ is a 1-cocycle}\}.
\]

\begin{example}
Let $Q$ be a group acting by automorphisms on $K$. 
The semidirect product $K\rtimes Q$ 
is a group $G$ that contains a normal subgroup isomorphic to $K$ 
and a subgroup isomorphic to such that 
$G=KQ$ and $K\cap Q=\{1\}$. Under the obvious identifications, 
$Q$ acts on $K$ by conjugation. For each $k\in K$, the map 
$Q\to K$, $x\mapsto [k,x]=kxk^{-1}x^{-1}$, is a 1-cocycle. 
\end{example}

\begin{exercise}
\label{xca:1cocycle}
Let $\varphi\colon Q\to K$ be a 1-cocycle. Prove the following statements:
\begin{enumerate}
	\item $\varphi(1)=1$.
	\item $\varphi(y^{-1})=(y^{-1}\cdot\phi(y))^{-1}=y^{-1}\cdot\phi(y)^{-1}$.
	\item The set $\ker\varphi=\{x\in Q:\varphi(x)=1\}$ is a subgroup of $Q$. 
\end{enumerate}
\end{exercise}

\begin{lemma}
\label{lem:1cocycle}
Let $G$ be a group with a normal subgroup $N$. 
If $\varphi\colon G\to N$ is a 1-cocycle (where $G$ acts on $N$ by conjugation)
with kernel 
\[
K=\ker\varphi=\{g\in G:\varphi(g)=1\}, 
\]
then 
$\varphi(x)=\varphi(y)$ if and only if $xK=yK$. In particular,
$(G:K)=|\varphi(G)|$. 
\end{lemma}

\begin{proof}
If $\varphi(x)=\varphi(y)$, then, since  
\[
\varphi(x^{-1}y)
=\varphi(x^{-1})(x^{-1}\cdot\varphi(y))
=\varphi(x^{-1})(x^{-1}\cdot\varphi(x))
=\varphi(x^{-1}x)=\varphi(1)
=1,
\]
we obtain that $xK=yK$. Conversely, if $x^{-1}y\in K$, then, since 
\[
1=\varphi(x^{-1}y)=\varphi(x^{-1})(x^{-1}\cdot \varphi(y)),
\]
we obtain that $\varphi(y)=x\cdot\varphi(x^{-1})^{-1}$. We conclude that 
$\varphi(x)=\varphi(y)$.

The second claim now is clear, as $\varphi$ is constant in each coclass of $K$ 
and takes $(G:K)$ different values. 
\end{proof}

\begin{lemma}
	\label{lem:d}
	Let $G$ be a finite group, $N$ be an abelian normal subgroup of $G$ and $S$, $T$ and $U$
    be transversals of $N$ in $G$. Let 
	\[
	d(S,T)=\prod st^{-1}\in N,
	\]
	where the product runs over all elements $s\in S$ and $t\in T$ such that 
	$sN=tN$. The following statements hold: 
	\begin{enumerate}
		\item $d(S,T)d(T,U)=d(S,U)$.
		\item $d(gS,gT)=gd(S,T)g^{-1}$ for all $g\in G$.
		\item $d(nS,S)=n^{(G:N)}$ for all $n\in N$.
	\end{enumerate}
\end{lemma}

\begin{proof}
	If $s\in S$, $t\in T$ and $u\in U$ are such that $sN=tN=uN$, then, since $N$ is 
	abelian and $(st^{-1})(tu^{-1})=su^{-1}$, we obtain that 
	\[
		d(S,T)d(T,U)=\prod (st^{-1})(tu^{-1})=\prod su^{-1}=d(S,U).
	\]

	Since $sN=tN$ if and only if $gsN=gtN$ for all $g\in G$, 
	\[
	g\left(\prod st^{-1}\right)g^{-1}=\prod gst^{-1}g^{-1}=\prod (gs)(gt)^{-1}=d(gS,gT).
	\]

	Finally, since $N$ is normal in $G$, $nsN=sN$ for all $n\in N$. Thus 
	\[
		d(nS,S)=\prod (ns)s^{-1}=n^{(G:N)}.\qedhere
	\]
\end{proof}

Recall that a subgroup $K$ of $G$ admits a \textbf{complement} $Q$ 
if $G$ factorizes as 
$G=KQ$ with $K\cap Q=\{1\}$. 
A typical example is the semidirect product $G=K\rtimes Q$, where $K$ is a normal subgroup of 
$G$ and $Q$ is a subgroup of $G$ such that $K\cap Q=\{1\}$. 

\begin{exercise}
\label{xca:complementos}
Let $Q$ act by automorphisms on $K$. Prove that there is a bijection 
between the set of complements of $K$ in $K\rtimes Q$ and the set 
$Z^1(Q,K)$.
\end{exercise}

% \begin{sol}{xca:complementos}
% 	El grupo $Q$ actúa en $K$ por conjugación, entonces $\delta\in\Der(Q,K)$ si
% 	y sólo si $\delta(xy)=\delta(x)x\delta(y)x^{-1}$, $x,y\in Q$. En este caso,
% 	las fórmulas del ejercicio anterior quedan así:
% 	$\delta(1)=1$, $\delta(x^{-1})=x^{-1}\delta(x)^{-1}x$.
	
% 	Sea $\mathcal{C}$ el conjunto de complementos de $K$ en $K\rtimes Q$.  Sea
% 	$C\in\mathcal{C}$. Si $x\in Q$, sabemos que 
% 	existen únicos $k\in K$ y $c\in C$ tales que $x=k^{-1}c$. Queda bien
% 	definida entonces la función $\delta_C\colon Q\to K$, $x\mapsto k$. Vale
% 	que $\delta(x)x=c\in C$. 
	
% 	Veamos que $\delta_C\in\Der(Q,K)$. Si $x,x_1\in Q$, escribimos $x=k^{-1}c$
% 	y $x_1=k_1^{-1}c_1$, donde $k,k_1\in K$ y $c,c_1\in C$. Como $K$ es normal
% 	en $K\rtimes Q$, podemos escribir a $xx_1$ como $xx_1=k_2c_2$, donde
% 	$k_2=k^{-1}(ck_1^{-1}c^{-1})\in K$, $c_2=cc_1\in C$. Luego 
% 	\[
% 		\delta(xx_1)xx_1=cc_1=\delta(x)x\delta(x_1)x_1
% 	\]
% 	implica que $\delta(xx_1)=\delta(x)x\delta(x_1)x^{-1}$. 
% 	Tenemos así una función $F\colon\mathcal{C}\to\Der(Q,K)$, $F(C)=\delta_C$.

% 	Vamos a construir ahora $G\colon\Der(Q,K)\to\mathcal{C}$. 
% 	Para
% 	cada $\delta\in\Der(Q,K)$ vamos a definir un complemento $\Delta$ de $K$ en $K\rtimes Q$: 
% 	\[
% 	\Delta=\{\delta(x)x:x\in Q\}.
% 	\]

% 	Veamos que $\Delta$ es un subgrupo de $K\rtimes Q$. Como $\delta(1)=1$,
% 	$1\in X$. Si $x,y\in Q$ entonces
% 	$\delta(x)x\delta(y)y=\delta(x)x\delta(y)x^{-1}xy=\delta(xy)xy\in \Delta$.
% 	Por último si $x\in Q$ entonces
% 	$(\delta(x)x)^{-1}=x^{-1}\delta(x)^{-1}xx^{-1}=\delta(x^{-1})x^{-1}$.
	
	
% 	Veamos que $\Delta\cap K=\{1\}$. Si $x\in Q$ es tal que $\delta(x)x\in K$
% 	entonces, como $\delta(x)\in K$, $x\in K\cap Q=\{1\}$. Si $g\in G$ entonces
% 	existen únicos $k\in K$, $x\in Q$ tales que $g=kx$. Escribimos
% 	$g=k\delta(x)^{-1}\delta(x)x$. Como $k\delta(x)^{-1}\in K$ y $\delta(x)x\in
% 	\Delta$, se concluye que $G=K\Delta$. Queda bien definida entonces la
% 	función $G\colon\Der(Q,K)\to\mathcal{C}$, $G(\delta)=\Delta$.

% 	Veamos ahora que $G\circ F=\id_{\mathcal{C}}$. 
% 	Sea $C\in\mathcal{C}$. Entonces 
% 	\[
% 	G(F(C))=G(\delta_C)=\{\delta_C(x)x:x\in
% 	Q\}=C,
% 	\]
% 	por construcción. (Vimos que $\delta_C(x)x\in C$. Recíprocamente,  si $c\in
% 	C$, escribimos $c=kx$ para únicos $k\in K$, $x\in Q$ y luego $x=k^{-1}c$
% 	que implica $c=\delta_c(x)x$.)

% 	Por último veamos que $F\circ G=\id_{\Der(Q,K)}$. Sea $\delta\in\Der(Q,K)$.
% 	Entonces 
% 	\[
% 	F(G(\delta))=F(\Delta)=\delta_{\Delta}.
% 	\]
% 	Queremos demostrar que $\delta_\Delta=\delta$.  Sea $x\in Q$. Existe
% 	$\delta(y)y\in\Delta$ para algún $y\in Q$ tal que $x=k^{-1}\delta(y)y$.
% 	Luego $\delta_{\Delta}(x)x=\delta(y)y$ y luego $\delta(x)=\delta(y)$ por la
% 	unicidad de la escritura.
% \end{sol}

We are now ready to prove the first version of the
Schur--Zassenhaus theorem. 

\begin{theorem}[Schur--Zassenhaus]
	\index{Schur--Zassenhaus!theorem}
	\label{thm:SchurZassenhaus:abeliano}
	Let $G$ be a finite group and $N$ be an abelian normal subgroup of $G$. If 
 	$|N|$ and $(G:N)$ are coprime, then $N$ admits a complement in $G$. Moreover, 
    all complements of $N$ are conjugate. 
\end{theorem}

\begin{proof}
	Let $T$ be a transversal of $N$ in $G$ and $\theta\colon G\to N$,
	$\theta(g)=d(gT,T)$. Since $N$ is abelian, Lemma~\ref{lem:d} implies that 
	$\theta$ is a 1-cocycle, where $G$ acts on $N$ by conjugation: 
	\begin{align*}
		\theta(xy)&=d(xyT,T)
		=d(xyT,xT)d(xT,T)\\
		&=(xd(yT,T)x^{-1})d(xT,T)=(x\cdot\theta(y))\theta(x).
	\end{align*}

	\begin{claim}
		$\theta|_N\colon N\to N$ is surjective. 
	\end{claim}

	If $n\in N$, Lemma~\ref{lem:d} implies that 
	$\theta(n)=d(nT,T)=n^{(G:N)}$. Since $|N|$ and $(G:N)$ are coprime, 
	there exist $r,s\in\Z$ such that $r|N|+s(G:N)=1$. Thus 
	\[
		n=n^{r|N|+s(G:N)}=(n^s)^{(G:N)}=\theta(n^s).
	\]

	Let $H=\ker\theta$. We prove that $H$ is a complement of $N$. 
	By Exercise~\ref{xca:1cocycle}, $H$ is a subgroup of $G$. By Lemma~\ref{lem:1cocycle}, 
	\[
		|N|=|\theta(G)|=(G:H)=\frac{|G|}{|H|}. 
	\]
	
	Since $N\cap H$ is a subgroup of $N$ and a subgroup of $H$, $N\cap H=\{1\}$, as the numbers 
	$|N|$ and $(G:N)=|H|$ are coprime. Since $|NH|=|N||H|=|G|$, we conclude that 
	$G=NH$. Hence $H$ is a complement of~$N$. 

	We now prove that two complements of $N$ are conjugate. 
	Let $K$ be a complement of $N$ in $G$. Since $NK=G$ and $N\cap K=\{1\}$, $K$ is a transversal of $N$. 
 Let $m=d(T,K)\in N$. Since the restriction map $\theta|_N$ is surjective, 
	there exists $n\in N$ such that $\theta(n)=m$. By Lemma~\ref{lem:d}, 
	\[
	kmk^{-1}=kd(T,K)k^{-1}=d(kT,kK)=d(kT,K)=d(kT,T)d(T,K)=\theta(k)m
	\]
    for all $k\in K$. 
	Since $N$ is abelian,
	$\theta(n^{-1})=m^{-1}$. Thus 
	\begin{align*}
		\theta(nkn^{-1})&=\theta(n)n\theta(kn^{-1})n^{-1}
		=m\theta(kn^{-1})\\
		&=m\theta(k)k\theta(n^{-1})k^{-1}
		=m\theta(k)km^{-1}k^{-1}=1.
	\end{align*}
	Therefore $nKn^{-1}\subseteq H=\ker\theta$. Since 
	$|K|=(G:N)=|H|$, we conclude that $nKn^{-1}=H$.
\end{proof}


\begin{theorem}[Schur--Zassenhaus]
	\index{Schur--Zassenhaus!theorem}
	\label{thm:SchurZassenhaus}
	Let $G$ be a finite group and $N$ be a normal subgroup of $G$. If $|N|$ and 
	$(G:N)$ are coprime, then $N$ admits a complement in $G$. 
\end{theorem}

\begin{proof}
	We proceed by induction on $|G|$. If there is a proper subgroup $K$ of 
	$G$ such that $NK=G$, then, since $(K:K\cap N)=(G:N)$ and $|N|$ are coprime,
	$(K:K\cap N)=(G:N)$ is coprime with $|K\cap N|$. Since $K\cap N$ is normal in $K$,
	the inductive hypothesis implies that $K\cap N$ admits a complement in $K$. Thus there exists 
    a subgroup $H$ of $K$ such that $|H|=(K:K\cap N)=(G:N)$. 

    Assume that there is no proper subgroup $K$ of $G$ such that 
	$NK=G$. We may assume that $N\ne\{1\}$ (otherwise, $G$ would be a complement of $N$ in $G$). Since $N$ is contained in 
    every maximal subgroup of $G$ (because, if there is a maximal subgroup $M\subsetneq G$ such that 
	$N\not\subseteq M$, then $NM=G$), it follows that $N\subseteq\Phi(G)$. By Frattini's theorem~\ref{thm:Frattini}, 
    $\Phi(G)$ is nilpotent. Thus $N$ is nilpotent and then $Z(N)\ne\{1\}$. Let $\pi\colon G\to
	G/Z(N)$ be the canonical map. Since $N$ is normal in $G$ and $Z(N)$ is characteristic in $N$, 
    $Z(N)$ is normal in $G$.  Moreover, 
	\[
	(\pi(G):\pi(N))=\frac{|\pi(G)|}{|\pi(N)|}=\frac{|G/Z(N)|}{|N/N\cap Z(N)|}=(G:N)
	\]
	is coprime with $|N|$. Then $(\pi(G):\pi(N))$ is coprime with $|\pi(N)|$. By the inductive hypothesis, 
	$\pi(N)$ admits a complement in $G/Z(N)$, say $\pi(K)$
	for some subgroup $K$ of $G$. Hence $G=NK$, as 
	$\pi(G)=\pi(N)\pi(K)=\pi(NK)$. 
	Since $K=G$ (because there is no $K$ such that $G=NK$), 
	$\pi(N)$ is abelian, as 
	\[
		\pi(Z(N)=\pi(N)\cap\pi(K)=\pi(N)\cap\pi(G)=\pi(N).
	\]
	Thus $N\subseteq Z(N)$ is abelian. By Theorem~\ref{thm:SchurZassenhaus:abeliano}, the subgroup $N$ 
    admits a complement. 
\end{proof}

\begin{theorem}[Schur--Zassenhaus conjugation theorem]
	\label{thm:SchurZassenhaus:conjugation}
    Let $G$ be a finite group and $N$ be a normal subgroup of $G$ such that 
    $|N|$ and 
	$(G:N)$ are coprime. If either $N$ or $G/N$ is solvable, then 
    all complements of $N$ in $G$ are conjugate. 
\end{theorem}

%\begin{proof}
%	Sea $G$ un contraejemplo minimal, es decir: existen complementos $K_1$ y
%	$K_2$ a $N$ en $G$ que no son conjugados.
%
%	\begin{claim}
%		$N$ es minimal en $G$.
%	\end{claim}
%
%	Si $M\subseteq N$ es minimal normal en $G$, $M\ne1$ pues $N\ne1$. Sea
%	$\pi\colon G\to G/M$ el morfismo canónico. El grupo $\pi(G)$ contiene un
%	subgrupo normal $\pi(N)$ de índice coprimo con $|\pi(N)|$. Además
%	$\pi(K_1)$ y $\pi(K_2)$ complementan a $\pi(N)$. Como $|G|$ es minimal,
%	$\pi(K_1)$ y $\pi(K_2)$ son conjugados en $\pi(G)$, es decir: existe $x\in G$ tal que 
%	$\pi(K_1)=\pi(xK_2x^{-1})$.
%
%\end{proof}

\begin{proof}
	Let $G$ be a minimal counterexample to the theorem, that is there are complements $K_1$ and 
	$K_2$ of $N$ in $G$ such that $K_1$ and $K_2$ are not conjugate. 

	\begin{claim}
		Every subgroup $U$ of $G$ satisfies the assumptions of the theorem with respect to the normal subgroup 
        $U\cap N$.
%		Sea $U$ un subgrupo de $G$. Entonces $U$ satisface las hipótesis del
%		teorema con respecto al subgrupo normal $U\cap N$. Si $U$ contiene un
%		complemento $H$ para $N$ en $G$, entonces $H$ complementa a $U\cap N$
%		en $U$.
	\end{claim}
	
	Since $N$ is normal in $G$, $U\cap N$ is normal in $U$. Moreover, $|U\cap N|$ and 
	$(U:U\cap N)$ are coprime, as $|U\cap N|$ divides $|N|$ and $(U:U\cap
	N)=(UN:N)$ divides $(G:N)$.  If $G/N$ is solvable, then $U/U\cap N$
	is solvable, as $U/U\cap N$ is isomorphic to a subgroup of $G/N$. If $N$ is 
	solvable, then so is $U\cap N$.
%
%	Como $|H|$ divide a $|U|$ y $|H|$ es coprimo con $|U\cap N|$, se tiene que
%	$|H|$ divide a $(U:U\cap N)$. Como además $(U:U\cap N)$ divide a
%	$(G:N)=|H|$, se concluye que $|H|=|U:U\cap N|$. Luego $H$ complementa a
%	$U\cap N$ en $U$.

	\begin{claim}
		If there is a normal subgroup $L$ of $G$ such that $\pi(N)$ is normal in 
		$\pi(G)$, where $\pi\colon G\to G/L$ is the canonical map, then 
    	$\pi(G)$ satisfies the theorem's assumptions with respect to $\pi(N)$.
		In this case, if $H$  is a complement of $N$ in $G$, then $\pi(H)$ 
		is a complement of $\pi(N)$ in $\pi(G)$.
	\end{claim}

	If $N$ is solvable, then so is $\pi(N)$. If $G/N$ is solvable, then so is 
	$\pi(G)/\pi(N)\simeq G/NL$. Moreover, 
	$(\pi(G):\pi(N))=\frac{|G/L|}{|N/N\cap L|}$ divides $(G:N)$. 
	
	If $H$ is a complement of $N$ in $G$, $|\pi(H)|$ and $|\pi(N)|$ are 
	coprime. Then $\pi(H)$ is a complement of $\pi(N)$, as 
	$\pi(G)=\pi(N)\pi(H)=\pi(NH)$ and 
	$\pi(N)\cap\pi(H)=\{1\}$. 

	\begin{claim}
		$N$ is minimal normal in $G$.
	\end{claim}

	Let $M\ne\{1\}$ be a normal subgroup of $G$ such that $M\subseteq N$. Let $\pi\colon G\to G/M$ be the canonical map. 
	Then $\pi(G)$ satisfies the theorem's assumptions with respect to the normal subgroup 
	$\pi(N)$. By the minimality of $|G|$, there exists 
	$x\in G$ such that $\pi(xK_1x^{-1})=\pi(K_2)$. The subgroup 
	$U=MK_2$ satisfies the theorem's assumptions with respect to the normal subgroup 
	$U\cap N$. Since $xK_1x^{-1}\cup K_2\subseteq U$,
	we conclude that both $xK_1x^{-1}$ and $K_2$ complement $U\cap N$ in $U$.
	Hence $MK_2=G$, as $xK_1x^{-1}$ and $K_2$ are not conjugate and $G$ is a minimal counterexample. 
	minimal. Therefore $M=N$, as 
	\[
		\frac{|K_2|}{|M\cap K_2|}=(MK_2:M)=(G:M)=\frac{|NK_2|}{|M|}=(N:M)|K_2|.
	\]

	\begin{claim}
		$N$ is not solvable and $G/N$ is solvable. 
	\end{claim}
	
	Otherwise, by Lemma~\ref{lem:minimal_normal}, $N$ is abelian (because it is minimal normal). This contradicts
    Theorem~\ref{thm:SchurZassenhaus:abeliano}, as it states that 
	$K_1$ and $K_2$ are conjugate. 
 
	\medskip
	Let $p\colon G\to G/N$ be the canonical map and $S$ be a subgroup such that $p(S)$
	is minimal normal in $p(G)=G/N$.  By Lemma~\ref{lem:minimal_normal},
	$p(S)$ is a $p$-group for some prime number $p$. Since $G=NK_1=NK_2$ and $N\subseteq
	S$, Dedekind's lemma~\ref{lem:Dedekind} implies that 
	\[
	S=N(S\cap K_1)=N(S\cap K_2).
	\]
	Hence both $S\cap K_1$ and $S\cap K_2$
	complement $N$ in $S$. Since $p(S)=p(S\cap K_1)=p(S\cap K_2)$  is a $p$-group, 
 	$p$ divides $|S|$. The theorem's assumptions hold for $S$ with respect to the normal subgroup $N$, 
    so $|N|$ and $(S:N)$ are coprime. If $p\mid |N|$, then 
	$p\nmid (S:N)=|S\cap K_1|=|S\cap K_2|$, a contradiction. Thus $p\nmid |N|$ and 
	hence $p\nmid |S|$. This implies that both $S\cap K_1$ and $S\cap K_2$ are Sylow 
	$p$-subgroups of $S$, as 
	\[
		|S\cap K_1|=(S:N)=|S\cap K_2|.
	\]
	By Sylow's theorem, there exists $s\in
	S$ such that 
    \[
	S\cap sK_1s^{-1}=S\cap K_2.
	\]
	In particular, $S\ne G$ by the minimality of $G$.
	Let 
	\[
		L=S\cap K_2=S\cap sK_1s^{-1}\ne\{1\}.
	\]
	Since $S$ is normal in $G$, $sK_1s^{-1}\cup K_2\subseteq N_G(L)$ (because $L$
	is both normal in $sK_1s^{-1}$ and in $K_2$). The subgroups $sK_1s^{-1}\subseteq
	N_G(L)$ and $K_2\subseteq N_G(L)$ complement $N\cap N_G(L)$ in $N_G(L)$. Hence 
	$N_G(L)=G$ by the minimality of $G$ (if $N_G(L)\ne G$, then both 
	$sK_1s^{-1}$ and $K_2$ are conjugate in $G$ because they are conjugate in  $N_G(L)$). Therefore 
	$L$ is normal in $G$. 
	
	Let $\pi_L\colon G\to G/L$ be the canonical map. Since both 
	$\pi_L(K_1)$ and $\pi_L(K_2)$ complement $\pi_L(N)$ in $\pi_L(G)$, the minimality of 
	$|G|$ implies that there exists $g\in G$ such that $\pi_L(gK_1g^{-1})=\pi_L(K_2)$, that is 
	there exists $g\in G$ such that $(gK_1g^{-1})L=K_2L$.  Hence $gK_1g^{-1}\cup
	K_2\subseteq \langle K_2,L\rangle=K_2$, because $L\subseteq K_2$. In conclusion, 
	$gK_1g^{-1}=K_2$, a contradiction to the minimality of $|G|$. 
%	Sea $L$ un subgrupo maximal normal de $G$ tal que $N\subseteq L$. Por
%	definición $L\ne G$. Como $L\cap K_1$ y $L\cap K_2$ complementan a $N$ en
%	$L$, la minimalidad de $G$ implica que existe $x\in G$ tal que 
%	\[ 
%	L\cap K_2=x(L\cap K_1)x^{-1}=L\cap xK_1x^{-1}.
%	\]
%	Sea $D=L\cap K_2$. Como $L$ es normal en $G$, $D$ es normal en $K_2$ y en
%	$xK_1x^{-1}$. Como $K_2$ y $xK_1x^{-1}$ son complementos para 
%	$N$ en $G$ y además
%	$xK_1x^{-1}\cup K_2\subseteq N_G(D)$, la minimalidad de $G$ implica que $N_G(D)=G$.
%	Si $N$ es resoluble, $N\ne1$ (pues de lo contrario $G=H=K$ y no hay nada
%	para demostrar). Sea $L\subseteq N$ un subgrupo minimal normal de $G$. Por
%	el lema~\ref{lemma:minimal_normal}, $L$ es abeliano\dots
%
%	Si $G/N$ es resoluble,\dots 
\end{proof}


By the Feit--Thompson theorem, in the previous theorem 
we do not need to assume that either $N$ or $G/N$ is solvable. Since every group of odd order is solvable 
and $|N|$ and $(G:N)$ are coprime, one of these groups should have odd order. 

\begin{theorem}
	\label{thm:solvable_maximal}
	Let $G$ be a finite solvable group and $p$ a prime number dividing $|G|$. There exists a maximal 
    subgroup $M$ of $G$ of index a power of $p$. 
\end{theorem}

\begin{proof}
	We proceed by induction on $|G|$. If $G$ is a $p$-group, the result clearly holds. So we may assume that $|G|$ is divisible by at least two different prime numbers. 
    Let $p$ be a prime dividing $|G|$, $N$ be a minimal normal subgroup of $G$ and 
    $\pi\colon G\to G/N$ be the canonical map. Since $G$ is solvable, by Lemma~\ref{lem:minimal_normal}, 
    $N$ is a $q$-group for some prime $q$. Since $G/N$ is solvable, if $p$ divides 
	$(G:N)$, then, by the inductive hypothesis, $G/N$ has a maximal subgroup 
 	$M_1$ of index a power of $p$. By the correspondence theorem, 
  $M=\pi^{-1}(M_1)$ is a maximal subgroup of $G$ of index a power of $p$. 
  $p$. If $p$ does not divide $(G:N)$, then $p$ divides $|N|$. Thus 
	$N\in\Syl_p(G)$. Since $N$ is normal in $G$ and $|N|$ and $|G/N|$ are coprime, by 
	Schur--Zassenhaus theorem~\ref{thm:SchurZassenhaus}, 
	there exists a complement $K$ of $N$ in $G$, that is $G=NK$ and $N\cap K=\{1\}$. Let 
	$M$ be a maximal subgroup containing $K$. Then $(G:M)$ is a power of $p$. 
\end{proof}

We now discuss an application to finite super-solvable groups. 

\begin{definition}
	\index{Group!lagrangian}
	A finite group $G$ is said to be \textbf{lagrangian} if for each $d$ dividing $|G|$ 
	there exists a subgroup of $G$ of order $d$.
\end{definition}

The group $\Alt_4$ is not lagrangian, as it has no subgroups of order six. 

\begin{theorem}
	Every finite super-solvable group is lagrangian. 
\end{theorem}

\begin{proof}
	Let $p$ be a prime number dividing $|G|$. Since subgroups of super-solvable groups are super-solvable, it is enough to 
    show that there exists a subgroup of index $p$. 
	Since $G$ is solvable, there exists a maxima subgroup $M$ of index 
	$p^{\alpha}$ by Theorem~\ref{thm:solvable_maximal}. Since maximal subgroups of super-solvable groups have prime index 
    by Theorem~\ref{thm:super_structure}, we conclude that $\alpha=1$.
\end{proof}

See \cite{MR294497} for an elementary proof. 

\subsection{*Hall's theory for solvable groups}

As an application of the Schur--Zassenhaus theorem, 
we present Hall's theory of solvable groups. 
For an elementary presentation, see \cite{MR600654}. 

\begin{definition}
\index{$\pi$-number}
\index{$\pi$-group}
\index{$\pi$-subgroup}
Let $G$ be a finite group and $\pi$ be a set of prime numbers. We say that 
$G$ is a \textbf{$\pi$-group} if every prime dividing $|G|$ belongs to $\pi$. 
Similarly, a $\pi$-subgroup of $G$ is a subgroup of $G$ that is also a $\pi$-group.  
\end{definition}

For a set $\pi$ of prime numbers, 
we define a $\pi$-number as an integer whose prime divisors 
belong to $\pi$. The set of prime numbers not belonging to $\pi$ will be denoted 
as $\pi'$. Thus a $\pi'$-number is an integers not divisible by 
the prime numbers of $\pi$. 

\begin{definition}
	\index{Hall!subgroup}
	Let $G$ be a group and $\pi$ be a set of prime numbers. A subgroup $H$ of $G$ 
    is a \textbf{Hall $\pi$-subgroup} if $H$ is a $\pi$-subgroup of $G$ and 
    $(G:H)$ is a $\pi'$-number.
\end{definition}

We now prove that a finite solvable group of order $nm$ with $\gcd(n,m)=1$ 
always admits a subgroup of order $m$. 

\begin{theorem}[Hall's existence theorem]
	\index{Hall's existence!theorem}
 	\label{theorem:HallE}
	Let $\pi$ be a set of prime numbers and $G$ be a finite solvable group. 
    Then $G$ has a Hall $\pi$-subgroup. 
\end{theorem}

\begin{proof}
	Assume that $|G|=nm>1$ and $\gcd(n,m)=1$. We want to show that $G$ admits a
	subgroup of order $m$. We proceed by induction on $|G|$. Let $K$ be a minimal
	normal subgroup of $G$ and $\pi\colon G\to G/N$ be the canonical map. Since $G$
	is solvable, $K$ is a $p$-group (Lemma~\ref{lem:minimal_normal}).
	
	There are two cases to consider. Assume first that $p$ divides $m$. Since
	$|G/K|<|G|$, the inductive hypothesis and the correspondence theorem imply that
	there exists a subgroup $J$ of $G$ containing $K$ such that $\pi(J)$ is a
	subgroup of 
    $\pi(G)=G/K$ of order $m/|K|$. Then $J$ has order $m$, as 
    \[
	m/|K|=|\pi(J)|=\frac{|J|}{|K\cap J|}=(J:K).
	\]

	Assume now that $p$ does not divide $m$. By the inductive hypothesis and the
	correspondence theorem, there exists a subgroup $H$ of $G$ containing $K$ such
	that $\pi(H)$ is a subgroup of $G/K$ of order $m$. Since $|H|=m|K|$, $K$ is
	normal in $H$ and $|K|$ is coprime with $|H:K|$, the Schur--Zassenhaus theorem
	(Theorem~\ref{thm:SchurZassenhaus}) implies that there exists a complement $J$
	of $K$ in $H$. Hence $J$ is a subgroup of $G$ such that $|J|=m$.
\end{proof}

\begin{example}
	The group $\Alt_5$ contains a Hall $\{2,3\}$-subgroups isomorphic to 
 	$\Alt_4$.
\end{example}

\begin{example}
	The simple group $\PSL_3(2)$ of order $168$ does not contain Hall $\{2,7\}$-subgroups.
\end{example}

\begin{theorem}[Hall's conjugation theorem]
	\index{Hall's conjugation theorem}
	\label{theorem:HallC}
	Let $G$ be a finite solvable group and $\pi$ be a set of prime numbers. 
    Then all two Hall $\pi$-subgroups of $G$ are conjugate. 
\end{theorem}

\begin{proof}
	We may assume that $G\ne\{1\}$. We proceed by induction on $|G|$.  Let $H$
	and $K$ be Hall $\pi$-subgroups of $G$. Let $M$ be a minimal normal subgroup of 
    $G$ and $\pi\colon G\to G/M$ be the canonical map. Since $G$ is solvable, 
	$M$ is a $p$-group for some prime number $p$ (Lemma~\ref{lem:minimal_normal}). 
    Since $\pi(H)$ and $\pi(K)$ are both Hall 
	$\pi$-subgroups of $G/M$, by the inductive hypothesis, 
    the subgroups $\pi(H)$ and $\pi(K)$ are 
	conjugate in $G/M$. Thus there exists $g\in G$ such that $gHMg^{-1}=KM$. 

	There are two cases to consider. Assume first that $p\in\pi$. Since $|HM|$ and 
	$|KM|$ are $\pi$-numbers and $|H|=|K|$ is the largest $\pi$-number dividing $|G|$, 
    we conclude that $H=HM$ and $K=KM$. In particular, $gHg^{-1}=K$. 

	Assume now that $p\not\in\pi$. Then $K$ admits a complement $M$ in 
 	$KM$, as $K\cap M=\{1\}$. We claim that $gHg^{-1}$ complements $M$ in $KM$. Since 
	$M$ is normal in $G$, 
 	\[
	(gHg^{-1})M=gHMg^{-1}=KM,
	\]
	and $gHg^{-1}\cap M=\{1\}$, as $p\not\in\pi$. These complements are conjugate 
    by the Schur--Zassenhaus theorem~\ref{thm:SchurZassenhaus:conjugation}.
\end{proof}

\begin{corollary}
	Let $G$ be a finite group, $N$ a normal subgroup of $G$ and $n=|N|$. 
 	Assume that either $N$ of $G/N$ is solvable. 
    If $|G:N|=m$ is coprime with $n$ and 
    $m_1$ divide a $m$, then every subgroup of $G$ of order $m_1$ 
    is contained in some subgroup of order $m$. 
\end{corollary}

\begin{proof}
	Let $H$ be a complement of $N$ in $G$. Then $|H|=m$. Let $H_1$
	be a subgroup of $G$ such that $|H_1|=m_1$. 
	Since $\gcd(n,m)=1$, $m_1=|H_1|=|H\cap NH_1|$, as 
	\[
	\frac{|H||N||H_1|}{|H\cap NH_1|}=
	\frac{|H||NH_1|}{|H\cap NH_1|}=|H(NH_1)|=|G|=|NH|=|N||H|.
	\]
	Since both $H_1$ and $H\cap NH_1$ are complements of $N$ in $NH_1$, and both 
    groups have orders coprime with $n$, there exists 
	$g\in G$ such that $H_1=g(H\cap NH_1)g^{-1}$. Thus  
	$H_1\subseteq gHg^{-1}$ and hence $|gHg^{-1}|=m$. 
\end{proof}
