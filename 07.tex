\section{Lecture: Week 7}

\subsection{The Schur--Zassenhaus theorem}

% Para leer este capítulo es conveniente haber entendido el capítulo \ref{derivaciones}, ya
% que demostraremos el teorema de Schur--Zassenhaus gracias al uso de algunos trucos que
% involucran derivaciones. También es necesario utilizar el subgrupo de Frattini, capítulo \ref{Frattini}. 
% Daremos una aplicación 
% del teorema de Schur--Zassenhaus a grupos súper-resolubles, estudiados en el capítulo \ref{super_resoluble}. 

Recall that a group $Q$ \emph{acts by automorphisms} on a group $K$ if 
there exists a map $Q\times K\to K$, $(q,k)\mapsto q\cdot k$, 
such that 
\begin{enumerate}
    \item $1\cdot a=a$ for all $a\in K$, 
    \item $x\cdot (y\cdot a)=(xy)\cdot a$ for all $x,y\in Q$ and $a\in K$, 
    \item $x\cdot 1=1$ for all $x\in Q$, and 
    \item $x\cdot (ab)=(x\cdot a)(x\cdot b)$ for all $x\in Q$ and $a,b\in K$, 
\end{enumerate}
For example, if $K$ is a normal subgroup of $G$, 
then $G$ acts by automorphisms on $K$ by conjugation. 

\begin{definition}
\index{1-cocycle}
Let $Q$ and $K$ be groups, where $Q$ acts by automorphisms on $K$. 
A map 
$\varphi\colon Q\to K$ is said to be a \emph{1-cocycle} if 
\[
	\varphi(xy)=\varphi(x)(x\cdot\varphi(y))
\]
for all $x,y\in Q$.  
\end{definition}

Let $Q$ and $K$ be groups, where $Q$ acts by automorphisms on $K$. 
The set of 1-cocycles $Q\to K$ will be denoted by 
\[
Z^1(Q,K)=\{\delta\colon Q\to K:\text{$\delta$ is a 1-cocycle}\}.
\]

\begin{example}
Let $Q$ be a group acting by automorphisms on $K$. 
The semidirect product $K\rtimes Q$ 
is a group $G$ that contains a normal subgroup isomorphic to $K$ 
and a subgroup isomorphic to such that 
$G=KQ$ and $K\cap Q=\{1\}$. Under the obvious identifications, 
$Q$ acts on $K$ by conjugation. For each $k\in K$, the map 
$Q\to K$, $x\mapsto [k,x]=kxk^{-1}x^{-1}$, is a 1-cocycle. 
\end{example}

\begin{exercise}
\label{xca:1cocycle}
Let $\varphi\colon Q\to K$ be a 1-cocycle. Prove the following statements:
\begin{enumerate}
	\item $\varphi(1)=1$.
	\item $\varphi(y^{-1})=(y^{-1}\cdot\varphi(y))^{-1}=y^{-1}\cdot\varphi(y)^{-1}$.
	\item The set $\ker\varphi=\{x\in Q:\varphi(x)=1\}$ is a subgroup of $Q$. 
\end{enumerate}
\end{exercise}

\begin{lemma}
\label{lem:1cocycle}
Let $G$ be a group with a normal subgroup $N$. 
If $\varphi\colon G\to N$ is a 1-cocycle (where $G$ acts on $N$ by conjugation)
with kernel 
\[
K=\ker\varphi=\{g\in G:\varphi(g)=1\}, 
\]
then 
$\varphi(x)=\varphi(y)$ if and only if $xK=yK$. In particular,
$(G:K)=|\varphi(G)|$. 
\end{lemma}

\begin{proof}
If $\varphi(x)=\varphi(y)$, then, since  
\[
\varphi(x^{-1}y)
=\varphi(x^{-1})(x^{-1}\cdot\varphi(y))
=\varphi(x^{-1})(x^{-1}\cdot\varphi(x))
=\varphi(x^{-1}x)=\varphi(1)
=1,
\]
we obtain that $xK=yK$. Conversely, if $x^{-1}y\in K$, then, since 
\[
1=\varphi(x^{-1}y)=\varphi(x^{-1})(x^{-1}\cdot \varphi(y)),
\]
we obtain that $\varphi(y)=x\cdot\varphi(x^{-1})^{-1}$. We conclude that 
$\varphi(x)=\varphi(y)$.

The second claim now is clear, as $\varphi$ is constant in each coclass of $K$ 
and takes $(G:K)$ different values. 
\end{proof}

\begin{lemma}
	\label{lem:d}
	Let $G$ be a finite group, $N$ be an abelian normal subgroup of $G$ and $S$, $T$ and $U$
    be transversals of $N$ in $G$. Let 
	\[
	d(S,T)=\prod st^{-1}\in N,
	\]
	where the product runs over all elements $s\in S$ and $t\in T$ such that 
	$sN=tN$. The following statements hold: 
	\begin{enumerate}
		\item $d(S,T)d(T,U)=d(S,U)$.
		\item $d(gS,gT)=gd(S,T)g^{-1}$ for all $g\in G$.
		\item $d(nS,S)=n^{(G:N)}$ for all $n\in N$.
	\end{enumerate}
\end{lemma}

\begin{proof}
	If $s\in S$, $t\in T$ and $u\in U$ are such that $sN=tN=uN$, then, since $N$ is 
	abelian and $(st^{-1})(tu^{-1})=su^{-1}$, we obtain that 
	\[
		d(S,T)d(T,U)=\prod (st^{-1})(tu^{-1})=\prod su^{-1}=d(S,U).
	\]

	Since $sN=tN$ if and only if $gsN=gtN$ for all $g\in G$, 
	\[
	g\left(\prod st^{-1}\right)g^{-1}=\prod gst^{-1}g^{-1}=\prod (gs)(gt)^{-1}=d(gS,gT).
	\]

	Finally, since $N$ is normal in $G$, $nsN=sN$ for all $n\in N$. Thus 
	\[
		d(nS,S)=\prod (ns)s^{-1}=n^{(G:N)}.\qedhere
	\]
\end{proof}

Recall that a subgroup $K$ of $G$ admits a \emph{complement} $Q$ 
if $G$ factorizes as 
$G=KQ$ with $K\cap Q=\{1\}$. 
A typical example is the semidirect product $G=K\rtimes Q$, where $K$ is a normal subgroup of 
$G$ and $Q$ is a subgroup of $G$ such that $K\cap Q=\{1\}$. 

\begin{exercise}
\label{xca:complementos}
Let $Q$ act by automorphisms on $K$. Prove that there is a bijection 
between the set of complements of $K$ in $K\rtimes Q$ and the set 
$Z^1(Q,K)$.
\end{exercise}

% \begin{sol}{xca:complementos}
% 	El grupo $Q$ actúa en $K$ por conjugación, entonces $\delta\in\Der(Q,K)$ si
% 	y sólo si $\delta(xy)=\delta(x)x\delta(y)x^{-1}$, $x,y\in Q$. En este caso,
% 	las fórmulas del ejercicio anterior quedan así:
% 	$\delta(1)=1$, $\delta(x^{-1})=x^{-1}\delta(x)^{-1}x$.
	
% 	Sea $\mathcal{C}$ el conjunto de complementos de $K$ en $K\rtimes Q$.  Sea
% 	$C\in\mathcal{C}$. Si $x\in Q$, sabemos que 
% 	existen únicos $k\in K$ y $c\in C$ tales que $x=k^{-1}c$. Queda bien
% 	definida entonces la función $\delta_C\colon Q\to K$, $x\mapsto k$. Vale
% 	que $\delta(x)x=c\in C$. 
	
% 	Veamos que $\delta_C\in\Der(Q,K)$. Si $x,x_1\in Q$, escribimos $x=k^{-1}c$
% 	y $x_1=k_1^{-1}c_1$, donde $k,k_1\in K$ y $c,c_1\in C$. Como $K$ es normal
% 	en $K\rtimes Q$, podemos escribir a $xx_1$ como $xx_1=k_2c_2$, donde
% 	$k_2=k^{-1}(ck_1^{-1}c^{-1})\in K$, $c_2=cc_1\in C$. Luego 
% 	\[
% 		\delta(xx_1)xx_1=cc_1=\delta(x)x\delta(x_1)x_1
% 	\]
% 	implica que $\delta(xx_1)=\delta(x)x\delta(x_1)x^{-1}$. 
% 	Tenemos así una función $F\colon\mathcal{C}\to\Der(Q,K)$, $F(C)=\delta_C$.

% 	Vamos a construir ahora $G\colon\Der(Q,K)\to\mathcal{C}$. 
% 	Para
% 	cada $\delta\in\Der(Q,K)$ vamos a definir un complemento $\Delta$ de $K$ en $K\rtimes Q$: 
% 	\[
% 	\Delta=\{\delta(x)x:x\in Q\}.
% 	\]

% 	Veamos que $\Delta$ es un subgrupo de $K\rtimes Q$. Como $\delta(1)=1$,
% 	$1\in X$. Si $x,y\in Q$ entonces
% 	$\delta(x)x\delta(y)y=\delta(x)x\delta(y)x^{-1}xy=\delta(xy)xy\in \Delta$.
% 	Por último si $x\in Q$ entonces
% 	$(\delta(x)x)^{-1}=x^{-1}\delta(x)^{-1}xx^{-1}=\delta(x^{-1})x^{-1}$.
	
	
% 	Veamos que $\Delta\cap K=\{1\}$. Si $x\in Q$ es tal que $\delta(x)x\in K$
% 	entonces, como $\delta(x)\in K$, $x\in K\cap Q=\{1\}$. Si $g\in G$ entonces
% 	existen únicos $k\in K$, $x\in Q$ tales que $g=kx$. Escribimos
% 	$g=k\delta(x)^{-1}\delta(x)x$. Como $k\delta(x)^{-1}\in K$ y $\delta(x)x\in
% 	\Delta$, se concluye que $G=K\Delta$. Queda bien definida entonces la
% 	función $G\colon\Der(Q,K)\to\mathcal{C}$, $G(\delta)=\Delta$.

% 	Veamos ahora que $G\circ F=\id_{\mathcal{C}}$. 
% 	Sea $C\in\mathcal{C}$. Entonces 
% 	\[
% 	G(F(C))=G(\delta_C)=\{\delta_C(x)x:x\in
% 	Q\}=C,
% 	\]
% 	por construcción. (Vimos que $\delta_C(x)x\in C$. Recíprocamente,  si $c\in
% 	C$, escribimos $c=kx$ para únicos $k\in K$, $x\in Q$ y luego $x=k^{-1}c$
% 	que implica $c=\delta_c(x)x$.)

% 	Por último veamos que $F\circ G=\id_{\Der(Q,K)}$. Sea $\delta\in\Der(Q,K)$.
% 	Entonces 
% 	\[
% 	F(G(\delta))=F(\Delta)=\delta_{\Delta}.
% 	\]
% 	Queremos demostrar que $\delta_\Delta=\delta$.  Sea $x\in Q$. Existe
% 	$\delta(y)y\in\Delta$ para algún $y\in Q$ tal que $x=k^{-1}\delta(y)y$.
% 	Luego $\delta_{\Delta}(x)x=\delta(y)y$ y luego $\delta(x)=\delta(y)$ por la
% 	unicidad de la escritura.
% \end{sol}

We are now ready to prove the first version of the
Schur--Zassenhaus theorem. 

\begin{theorem}[Schur--Zassenhaus]
	\index{Schur--Zassenhaus!theorem}
	\label{thm:SchurZassenhaus:abeliano}
	Let $G$ be a finite group and $N$ be an abelian normal subgroup of $G$. If 
 	$|N|$ and $(G:N)$ are coprime, then $N$ admits a complement in $G$. Moreover, 
    all complements of $N$ are conjugate. 
\end{theorem}

\begin{proof}
	Let $T$ be a transversal of $N$ in $G$ and $\theta\colon G\to N$,
	$\theta(g)=d(gT,T)$. Since $N$ is abelian, Lemma~\ref{lem:d} implies that 
	$\theta$ is a 1-cocycle, where $G$ acts on $N$ by conjugation: 
	\begin{align*}
		\theta(xy)&=d(xyT,T)
		=d(xyT,xT)d(xT,T)\\
		&=(xd(yT,T)x^{-1})d(xT,T)=(x\cdot\theta(y))\theta(x).
	\end{align*}

	\begin{claim}
		$\theta|_N\colon N\to N$ is surjective. 
	\end{claim}

	If $n\in N$, Lemma~\ref{lem:d} implies that 
	$\theta(n)=d(nT,T)=n^{(G:N)}$. Since $|N|$ and $(G:N)$ are coprime, 
	there exist $r,s\in\Z$ such that $r|N|+s(G:N)=1$. Thus 
	\[
		n=n^{r|N|+s(G:N)}=(n^s)^{(G:N)}=\theta(n^s).
	\]

	Let $H=\ker\theta$. We prove that $H$ is a complement of $N$. 
	By Exercise~\ref{xca:1cocycle}, $H$ is a subgroup of $G$. By Lemma~\ref{lem:1cocycle}, 
	\[
		|N|=|\theta(G)|=(G:H)=\frac{|G|}{|H|}. 
	\]
	
	Since $N\cap H$ is a subgroup of $N$ and a subgroup of $H$, $N\cap H=\{1\}$, as the numbers 
	$|N|$ and $(G:N)=|H|$ are coprime. Since $|NH|=|N||H|=|G|$, we conclude that 
	$G=NH$. Hence $H$ is a complement of~$N$. 

	We now prove that two complements of $N$ are conjugate. 
	Let $K$ be a complement of $N$ in $G$. Since $NK=G$ and $N\cap K=\{1\}$, $K$ is a transversal of $N$. 
 Let $m=d(T,K)\in N$. Since the restriction map $\theta|_N$ is surjective, 
	there exists $n\in N$ such that $\theta(n)=m$. By Lemma~\ref{lem:d}, 
	\[
	kmk^{-1}=kd(T,K)k^{-1}=d(kT,kK)=d(kT,K)=d(kT,T)d(T,K)=\theta(k)m
	\]
    for all $k\in K$. 
	Since $N$ is abelian,
	$\theta(n^{-1})=m^{-1}$. Thus 
	\begin{align*}
		\theta(nkn^{-1})&=\theta(n)n\theta(kn^{-1})n^{-1}
		=m\theta(kn^{-1})\\
		&=m\theta(k)k\theta(n^{-1})k^{-1}
		=m\theta(k)km^{-1}k^{-1}=1.
	\end{align*}
	Therefore $nKn^{-1}\subseteq H=\ker\theta$. Since 
	$|K|=(G:N)=|H|$, we conclude that $nKn^{-1}=H$.
\end{proof}


\begin{theorem}[Schur--Zassenhaus]
	\index{Schur--Zassenhaus!theorem}
	\label{thm:SchurZassenhaus}
	Let $G$ be a finite group and $N$ be a normal subgroup of $G$. If $|N|$ and 
	$(G:N)$ are coprime, then $N$ admits a complement in $G$. 
\end{theorem}

\begin{proof}
	We proceed by induction on $|G|$. If there is a proper subgroup $K$ of 
	$G$ such that $NK=G$, then, since $(K:K\cap N)=(G:N)$ and $|N|$ are coprime,
	$(K:K\cap N)=(G:N)$ is coprime with $|K\cap N|$. Since $K\cap N$ is normal in $K$,
	the inductive hypothesis implies that $K\cap N$ admits a complement in $K$. Thus there exists 
    a subgroup $H$ of $K$ such that $|H|=(K:K\cap N)=(G:N)$. 

    Assume that there is no proper subgroup $K$ of $G$ such that 
	$NK=G$. We may assume that $N\ne\{1\}$ (otherwise, $G$ would be a complement of $N$ in $G$). Since $N$ is contained in 
    every maximal subgroup of $G$ (because, if there is a maximal subgroup $M\subsetneq G$ such that 
	$N\not\subseteq M$, then $NM=G$), it follows that $N\subseteq\Phi(G)$. By Frattini's theorem~\ref{thm:Frattini}, 
    $\Phi(G)$ is nilpotent. Thus $N$ is nilpotent and then $Z(N)\ne\{1\}$. Let $\pi\colon G\to
	G/Z(N)$ be the canonical map. Since $N$ is normal in $G$ and $Z(N)$ is characteristic in $N$, 
    $Z(N)$ is normal in $G$.  Moreover, 
	\[
	(\pi(G):\pi(N))=\frac{|\pi(G)|}{|\pi(N)|}=\frac{|G/Z(N)|}{|N/N\cap Z(N)|}=(G:N)
	\]
	is coprime with $|N|$. Then $(\pi(G):\pi(N))$ is coprime with $|\pi(N)|$. By the inductive hypothesis, 
	$\pi(N)$ admits a complement in $G/Z(N)$, say $\pi(K)$
	for some subgroup $K$ of $G$. Hence $G=NK$, as 
	$\pi(G)=\pi(N)\pi(K)=\pi(NK)$. 
	Since $K=G$ (because there is no $K$ such that $G=NK$), 
	$\pi(N)$ is abelian, as 
	\[
		\pi(Z(N))=\pi(N)\cap\pi(K)=\pi(N)\cap\pi(G)=\pi(N).
	\]
	Thus $N\subseteq Z(N)$ is abelian. By Theorem~\ref{thm:SchurZassenhaus:abeliano}, the subgroup $N$ 
    admits a complement. 
\end{proof}

\begin{theorem}[Schur--Zassenhaus conjugation theorem]
	\label{thm:SchurZassenhaus:conjugation}
    Let $G$ be a finite group and $N$ be a normal subgroup of $G$ such that 
    $|N|$ and 
	$(G:N)$ are coprime. If either $N$ or $G/N$ is solvable, then 
    all complements of $N$ in $G$ are conjugate. 
\end{theorem}

%\begin{proof}
%	Sea $G$ un contraejemplo minimal, es decir: existen complementos $K_1$ y
%	$K_2$ a $N$ en $G$ que no son conjugados.
%
%	\begin{claim}
%		$N$ es minimal en $G$.
%	\end{claim}
%
%	Si $M\subseteq N$ es minimal normal en $G$, $M\ne1$ pues $N\ne1$. Sea
%	$\pi\colon G\to G/M$ el morfismo canónico. El grupo $\pi(G)$ contiene un
%	subgrupo normal $\pi(N)$ de índice coprimo con $|\pi(N)|$. Además
%	$\pi(K_1)$ y $\pi(K_2)$ complementan a $\pi(N)$. Como $|G|$ es minimal,
%	$\pi(K_1)$ y $\pi(K_2)$ son conjugados en $\pi(G)$, es decir: existe $x\in G$ tal que 
%	$\pi(K_1)=\pi(xK_2x^{-1})$.
%
%\end{proof}

\begin{proof}
	Let $G$ be a minimal counterexample to the theorem, that is there are complements $K_1$ and 
	$K_2$ of $N$ in $G$ such that $K_1$ and $K_2$ are not conjugate. 

	\begin{claim}
		Every subgroup $U$ of $G$ satisfies the assumptions of the theorem with respect to the normal subgroup 
        $U\cap N$.
%		Sea $U$ un subgrupo de $G$. Entonces $U$ satisface las hipótesis del
%		teorema con respecto al subgrupo normal $U\cap N$. Si $U$ contiene un
%		complemento $H$ para $N$ en $G$, entonces $H$ complementa a $U\cap N$
%		en $U$.
	\end{claim}
	
	Since $N$ is normal in $G$, $U\cap N$ is normal in $U$. Moreover, $|U\cap N|$ and 
	$(U:U\cap N)$ are coprime, as $|U\cap N|$ divides $|N|$ and $(U:U\cap
	N)=(UN:N)$ divides $(G:N)$.  If $G/N$ is solvable, then $U/U\cap N$
	is solvable, as $U/U\cap N$ is isomorphic to a subgroup of $G/N$. If $N$ is 
	solvable, then so is $U\cap N$.
%
%	Como $|H|$ divide a $|U|$ y $|H|$ es coprimo con $|U\cap N|$, se tiene que
%	$|H|$ divide a $(U:U\cap N)$. Como además $(U:U\cap N)$ divide a
%	$(G:N)=|H|$, se concluye que $|H|=|U:U\cap N|$. Luego $H$ complementa a
%	$U\cap N$ en $U$.

	\begin{claim}
		If there is a normal subgroup $L$ of $G$ such that $\pi(N)$ is normal in 
		$\pi(G)$, where $\pi\colon G\to G/L$ is the canonical map, then 
    	$\pi(G)$ satisfies the theorem's assumptions with respect to $\pi(N)$.
		In this case, if $H$  is a complement of $N$ in $G$, then $\pi(H)$ 
		is a complement of $\pi(N)$ in $\pi(G)$.
	\end{claim}

	If $N$ is solvable, then so is $\pi(N)$. If $G/N$ is solvable, then so is 
	$\pi(G)/\pi(N)\simeq G/NL$. Moreover, 
	$(\pi(G):\pi(N))=\frac{|G/L|}{|N/N\cap L|}$ divides $(G:N)$. 
	
	If $H$ is a complement of $N$ in $G$, $|\pi(H)|$ and $|\pi(N)|$ are 
	coprime. Then $\pi(H)$ is a complement of $\pi(N)$, as 
	$\pi(G)=\pi(N)\pi(H)=\pi(NH)$ and 
	$\pi(N)\cap\pi(H)=\{1\}$. 

	\begin{claim}
		$N$ is minimal normal in $G$.
	\end{claim}

	Let $M\ne\{1\}$ be a normal subgroup of $G$ such that $M\subseteq N$. Let $\pi\colon G\to G/M$ be the canonical map. 
	Then $\pi(G)$ satisfies the theorem's assumptions with respect to the normal subgroup 
	$\pi(N)$. By the minimality of $|G|$, there exists 
	$x\in G$ such that $\pi(xK_1x^{-1})=\pi(K_2)$. The subgroup 
	$U=MK_2$ satisfies the theorem's assumptions with respect to the normal subgroup 
	$U\cap N$ of $U$. Since $xK_1x^{-1}\cup K_2\subseteq U$,
	we conclude that both $xK_1x^{-1}$ and $K_2$ complement $U\cap N$ in $U$.
	Hence $MK_2=G$, as $xK_1x^{-1}$ and $K_2$ are not conjugate and $G$, as $G$
        is a minimal counterexample. 
	Therefore $M=N$, as 
	\[
		\frac{|K_2|}{|M\cap K_2|}=(MK_2:M)=(G:M)=\frac{|NK_2|}{|M|}=(N:M)|K_2|.
	\]

	\begin{claim}
		$N$ is not solvable and $G/N$ is solvable. 
	\end{claim}
	
	Otherwise, by Lemma~\ref{lem:minimal_normal}, $N$ is abelian (because it is minimal normal). This contradicts
    Theorem~\ref{thm:SchurZassenhaus:abeliano}, as it states that 
	$K_1$ and $K_2$ are conjugate. 
 
	\medskip
	Let $p\colon G\to G/N$ be the canonical map and $S$ be a subgroup such that $p(S)$
	is minimal normal in $p(G)=G/N$.  By Lemma~\ref{lem:minimal_normal},
	$p(S)$ is a $p$-group for some prime number $p$. Since $G=NK_1=NK_2$ and $N\subseteq
	S$, Dedekind's lemma~\ref{lem:Dedekind} implies that 
	\[
	S=N(S\cap K_1)=N(S\cap K_2).
	\]
	Hence both $S\cap K_1$ and $S\cap K_2$
	complement $N$ in $S$. Since $p(S)=p(S\cap K_1)=p(S\cap K_2)$  is a $p$-group, 
 	$p$ divides $|S|$. The theorem's assumptions hold for $S$ with respect to the normal subgroup $N$, 
    so $|N|$ and $(S:N)$ are coprime. If $p\mid |N|$, then 
	$p\nmid (S:N)=|S\cap K_1|=|S\cap K_2|$, a contradiction. Thus $p\nmid |N|$ and 
	hence $p\nmid |S|$. This implies that both $S\cap K_1$ and $S\cap K_2$ are Sylow 
	$p$-subgroups of $S$, as 
	\[
		|S\cap K_1|=(S:N)=|S\cap K_2|.
	\]
	By Sylow's theorem, there exists $s\in
	S$ such that 
    \[
	S\cap sK_1s^{-1}=S\cap K_2.
	\]
	In particular, $S\ne G$ by the minimality of $G$.
	Let 
	\[
		L=S\cap K_2=S\cap sK_1s^{-1}\ne\{1\}.
	\]
	Since $S$ is normal in $G$, $sK_1s^{-1}\cup K_2\subseteq N_G(L)$ (because $L$
	is both normal in $sK_1s^{-1}$ and in $K_2$). The subgroups $sK_1s^{-1}\subseteq
	N_G(L)$ and $K_2\subseteq N_G(L)$ complement $N\cap N_G(L)$ in $N_G(L)$. Hence 
	$N_G(L)=G$ by the minimality of $G$ (if $N_G(L)\ne G$, then both 
	$sK_1s^{-1}$ and $K_2$ are conjugate in $G$ because they are conjugate in  $N_G(L)$). Therefore 
	$L$ is normal in $G$. 
	
	Let $\pi_L\colon G\to G/L$ be the canonical map. Since both 
	$\pi_L(K_1)$ and $\pi_L(K_2)$ complement $\pi_L(N)$ in $\pi_L(G)$, the minimality of 
	$|G|$ implies that there exists $g\in G$ such that $\pi_L(gK_1g^{-1})=\pi_L(K_2)$, that is 
	there exists $g\in G$ such that $(gK_1g^{-1})L=K_2L$.  Hence $gK_1g^{-1}\cup
	K_2\subseteq \langle K_2,L\rangle=K_2$, because $L\subseteq K_2$. In conclusion, 
	$gK_1g^{-1}=K_2$, a contradiction to the minimality of $|G|$. 
%	Sea $L$ un subgrupo maximal normal de $G$ tal que $N\subseteq L$. Por
%	definición $L\ne G$. Como $L\cap K_1$ y $L\cap K_2$ complementan a $N$ en
%	$L$, la minimalidad de $G$ implica que existe $x\in G$ tal que 
%	\[ 
%	L\cap K_2=x(L\cap K_1)x^{-1}=L\cap xK_1x^{-1}.
%	\]
%	Sea $D=L\cap K_2$. Como $L$ es normal en $G$, $D$ es normal en $K_2$ y en
%	$xK_1x^{-1}$. Como $K_2$ y $xK_1x^{-1}$ son complementos para 
%	$N$ en $G$ y además
%	$xK_1x^{-1}\cup K_2\subseteq N_G(D)$, la minimalidad de $G$ implica que $N_G(D)=G$.
%	Si $N$ es resoluble, $N\ne1$ (pues de lo contrario $G=H=K$ y no hay nada
%	para demostrar). Sea $L\subseteq N$ un subgrupo minimal normal de $G$. Por
%	el lema~\ref{lemma:minimal_normal}, $L$ es abeliano\dots
%
%	Si $G/N$ es resoluble,\dots 
\end{proof}


By the Feit--Thompson theorem, in the previous theorem, 
we do not need to assume that either $N$ or $G/N$ is solvable. Since every group of odd order is solvable 
and $|N|$ and $(G:N)$ are coprime, one of these groups should have odd order. 

