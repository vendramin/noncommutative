\section{18/04/2024}

\begin{theorem}
	\label{thm:solvable_maximal}
	Let $G$ be a finite solvable group and $p$ a prime number dividing $|G|$. There exists a maximal 
    subgroup $M$ of $G$ of index a power of $p$. 
\end{theorem}

\begin{proof}
	We proceed by induction on $|G|$. If $G$ is a $p$-group, the result clearly holds. So we may assume that $|G|$ is divisible by at least two different prime numbers. 
    Let $p$ be a prime dividing $|G|$, $N$ be a minimal normal subgroup of $G$ and 
    $\pi\colon G\to G/N$ be the canonical map. Since $G$ is solvable, by Lemma~\ref{lem:minimal_normal}, 
    $N$ is a $q$-group for some prime $q$. Since $G/N$ is solvable, if $p$ divides 
	$(G:N)$, then, by the inductive hypothesis, $G/N$ has a maximal subgroup 
 	$M_1$ of index a power of $p$. By the correspondence theorem, 
  $M=\pi^{-1}(M_1)$ is a maximal subgroup of $G$ of index a power of $p$. 
  $p$. If $p$ does not divide $(G:N)$, then $p$ divides $|N|$. Thus 
	$N\in\Syl_p(G)$. Since $N$ is normal in $G$ and $|N|$ and $|G/N|$ are coprime, by 
	Schur--Zassenhaus theorem~\ref{thm:SchurZassenhaus}, 
	there exists a complement $K$ of $N$ in $G$, that is $G=NK$ and $N\cap K=\{1\}$. Let 
	$M$ be a maximal subgroup containing $K$. Then $(G:M)$ is a power of $p$. 
\end{proof}

We now discuss an application to finite super-solvable groups. 

\begin{definition}
	\index{Group!lagrangian}
	A finite group $G$ is said to be \textbf{lagrangian} if for each $d$ dividing $|G|$ 
	there exists a subgroup of $G$ of order $d$.
\end{definition}

The group $\Alt_4$ is not lagrangian, as it has no subgroups of order six. 

\begin{theorem}
	Every finite super-solvable group is lagrangian. 
\end{theorem}

\begin{proof}
	Let $p$ be a prime number dividing $|G|$. Since subgroups of super-solvable groups are super-solvable, it is enough to 
    show that there exists a subgroup of index $p$. 
	Since $G$ is solvable, there exists a maxima subgroup $M$ of index 
	$p^{\alpha}$ by Theorem~\ref{thm:solvable_maximal}. Since maximal subgroups of super-solvable groups have prime index 
    by Theorem~\ref{thm:super_structure}, we conclude that $\alpha=1$.
\end{proof}

See \cite{MR294497} for an elementary proof. 

\subsection{*Hall's theory for solvable groups}

As an application of the Schur--Zassenhaus theorem, 
we present Hall's theory of solvable groups. 
For an elementary presentation, see \cite{MR600654}. 

\begin{definition}
\index{$\pi$-number}
\index{$\pi$-group}
\index{$\pi$-subgroup}
Let $G$ be a finite group and $\pi$ be a set of prime numbers. We say that 
$G$ is a \textbf{$\pi$-group} if every prime dividing $|G|$ belongs to $\pi$. 
Similarly, a $\pi$-subgroup of $G$ is a subgroup of $G$ that is also a $\pi$-group.  
\end{definition}

For a set $\pi$ of prime numbers, 
we define a $\pi$-number as an integer whose prime divisors 
belong to $\pi$. The set of prime numbers not belonging to $\pi$ will be denoted 
as $\pi'$. Thus a $\pi'$-number is an integers not divisible by 
the prime numbers of $\pi$. 

\begin{definition}
	\index{Hall!subgroup}
	Let $G$ be a group and $\pi$ be a set of prime numbers. A subgroup $H$ of $G$ 
    is a \textbf{Hall $\pi$-subgroup} if $H$ is a $\pi$-subgroup of $G$ and 
    $(G:H)$ is a $\pi'$-number.
\end{definition}

We now prove that a finite solvable group of order $nm$ with $\gcd(n,m)=1$ 
always admits a subgroup of order $m$. 

\begin{theorem}[Hall's!existence theorem]
	\index{Hall!existence theorem}
 	\label{theorem:HallE}
	Let $\pi$ be a set of prime numbers and $G$ be a finite solvable group. 
    Then $G$ has a Hall $\pi$-subgroup. 
\end{theorem}

\begin{proof}
	Assume that $|G|=nm>1$ and $\gcd(n,m)=1$. We want to show that $G$ admits a
	subgroup of order $m$. We proceed by induction on $|G|$. Let $K$ be a minimal
	normal subgroup of $G$ and $\pi\colon G\to G/N$ be the canonical map. Since $G$
	is solvable, $K$ is a $p$-group (Lemma~\ref{lem:minimal_normal}).
	
	There are two cases to consider. Assume first that $p$ divides $m$. Since
	$|G/K|<|G|$, the inductive hypothesis and the correspondence theorem imply that
	there exists a subgroup $J$ of $G$ containing $K$ such that $\pi(J)$ is a
	subgroup of 
    $\pi(G)=G/K$ of order $m/|K|$. Then $J$ has order $m$, as 
    \[
	m/|K|=|\pi(J)|=\frac{|J|}{|K\cap J|}=(J:K).
	\]

	Assume now that $p$ does not divide $m$. By the inductive hypothesis and the
	correspondence theorem, there exists a subgroup $H$ of $G$ containing $K$ such
	that $\pi(H)$ is a subgroup of $G/K$ of order $m$. Since $|H|=m|K|$, $K$ is
	normal in $H$ and $|K|$ is coprime with $|H:K|$, the Schur--Zassenhaus theorem
	(Theorem~\ref{thm:SchurZassenhaus}) implies that there exists a complement $J$
	of $K$ in $H$. Hence $J$ is a subgroup of $G$ such that $|J|=m$.
\end{proof}

\begin{example}
	The group $\Alt_5$ contains a Hall $\{2,3\}$-subgroups isomorphic to 
 	$\Alt_4$.
\end{example}

\begin{example}
	The simple group $\PSL_3(2)$ of order $168$ does not contain Hall $\{2,7\}$-subgroups.
\end{example}

\begin{theorem}[Hall's conjugation theorem]
	\index{Hall!conjugation theorem}
	\label{theorem:HallC}
	Let $G$ be a finite solvable group and $\pi$ be a set of prime numbers. 
    Then all two Hall $\pi$-subgroups of $G$ are conjugate. 
\end{theorem}

\begin{proof}
	We may assume that $G\ne\{1\}$. We proceed by induction on $|G|$.  Let $H$
	and $K$ be Hall $\pi$-subgroups of $G$. Let $M$ be a minimal normal subgroup of 
    $G$ and $\pi\colon G\to G/M$ be the canonical map. Since $G$ is solvable, 
	$M$ is a $p$-group for some prime number $p$ (Lemma~\ref{lem:minimal_normal}). 
    Since $\pi(H)$ and $\pi(K)$ are both Hall 
	$\pi$-subgroups of $G/M$, by the inductive hypothesis, 
    the subgroups $\pi(H)$ and $\pi(K)$ are 
	conjugate in $G/M$. Thus there exists $g\in G$ such that $gHMg^{-1}=KM$. 

	There are two cases to consider. Assume first that $p\in\pi$. Since $|HM|$ and 
	$|KM|$ are $\pi$-numbers and $|H|=|K|$ is the largest $\pi$-number dividing $|G|$, 
    we conclude that $H=HM$ and $K=KM$. In particular, $gHg^{-1}=K$. 

	Assume now that $p\not\in\pi$. Then $K$ admits a complement $M$ in 
 	$KM$, as $K\cap M=\{1\}$. We claim that $gHg^{-1}$ complements $M$ in $KM$. Since 
	$M$ is normal in $G$, 
 	\[
	(gHg^{-1})M=gHMg^{-1}=KM,
	\]
	and $gHg^{-1}\cap M=\{1\}$, as $p\not\in\pi$. These complements are conjugate 
    by the Schur--Zassenhaus theorem~\ref{thm:SchurZassenhaus:conjugation}.
\end{proof}

\begin{corollary}
	Let $G$ be a finite group, $N$ a normal subgroup of $G$ and $n=|N|$. 
 	Assume that either $N$ of $G/N$ is solvable. 
    If $|G:N|=m$ is coprime with $n$ and 
    $m_1$ divide a $m$, then every subgroup of $G$ of order $m_1$ 
    is contained in some subgroup of order $m$. 
\end{corollary}

\begin{proof}
	Let $H$ be a complement of $N$ in $G$. Then $|H|=m$. Let $H_1$
	be a subgroup of $G$ such that $|H_1|=m_1$. 
	Since $\gcd(n,m)=1$, $m_1=|H_1|=|H\cap NH_1|$, as 
	\[
	\frac{|H||N||H_1|}{|H\cap NH_1|}=
	\frac{|H||NH_1|}{|H\cap NH_1|}=|H(NH_1)|=|G|=|NH|=|N||H|.
	\]
	Since both $H_1$ and $H\cap NH_1$ are complements of $N$ in $NH_1$, and both 
    groups have orders coprime with $n$, there exists 
	$g\in G$ such that $H_1=g(H\cap NH_1)g^{-1}$. Thus  
	$H_1\subseteq gHg^{-1}$ and hence $|gHg^{-1}|=m$. 
\end{proof}


\subsection{Subnormality}

\begin{definition}
	\index{Subgroup!subnormal}
	Let $G$ be a group. A subgroup $H$ of $G$ is said to be \text{subnormal} in $G$ if there is a sequence 
    of subgroups 
	\[
		H=H_0\subseteq H_1\subseteq\cdots\subseteq H_k=G		
	\]
	with $H_i$ normal in $H_{i+1}$ for all $i\in\{0,\dots,k-1\}$. 
\end{definition}

\begin{example}
	Let $G=\Sym_4$. Then $K=\{\id,(12)(34),(13)(24),(14)(23)\}$ is normal in $G$. 
	The subgroup $L=\{\id,(12)(34)\}$ is subnormal in $G$ (and not normal). 
	es subnormal. 
\end{example}

\begin{exercise}
\label{xca:correspondence_subnormality}
    Prove that the correspondence theorem preserves subnormality. 
\end{exercise}

\begin{theorem}
	\label{thm:subnormal}
	Let $G$ be a finite group. Then $G$ is nilpotent if and only if every subgroup of $G$ is subnormal in $G$. 
\end{theorem}

\begin{proof}
	Assume first that every subgroup of $G$ is subnormal in $G$. Let $H$ be a subnormal subgroup of $G$, where 
	\[
		H=H_0\subseteq H_1\subseteq\cdots\subseteq H_k=G
	\]
	with $H_i$ normal in $H_{i+1}$. Without loss of generality, we may assume that 
	$H\subsetneq H_1$. Since $H\subsetneq H_1\subseteq N_G(H)$, 
	$G$ is nilpotent by Exercise~\ref{xca:normalizadora}.

	Assume now that $G$ is nilpotent. Let $H$ be a subgroup of $G$.
	We proceed by induction on $(G:H)$. If $(G:H)=1$, then $H=G$ and the theorem holds. If 
	$H\ne G$, since $H\subsetneq N_G(H)$ by Lemma~\ref{lem:normalizadora}, 
	\[
		(G:N_G(H))<(G:H).
	\]
	By the inductive hypothesis, $N_G(H)$ is subnormal in $G$. Since $H$ is 
	normal in $N_G(H)$, we conclude that $H$ is subnormal in $G$.
\end{proof}

\index{Subgroup!central}

\begin{corollary}
	Let $G$ be a group and $K$ be a central subgroup of $G$ (that is, $K\subseteq Z(G)$).
	Then $G$ is nilpotent if and only if $G/K$ is nilpotent. 
\end{corollary}

\begin{proof}
	If $G$ is nilpotent, then so is $G/K$. Conversely, let 
	$\pi\colon G\to G/K$ be the canonical map and $U$ be a subgroup of $G$. Since 
	$G/K$ is nilpotent, Theorem~\ref{thm:subnormal} implies that 
	$\pi(U)$ is a subnormal subgroup of $G/K$. By the correspondence theorem, 
 	$UK$ is a subnormal subgroup of $G$. Since $K$ is central, $U$ is normal in 
	$UZ$. Hence $U$ is subnormal in $G$ and therefore $G$ is nilpotent by Theorem~\ref{thm:subnormal}.
\end{proof}

\begin{theorem}
	\label{thm:F(G)subnormalidad}
	Let $G$ be a finite group and $H$ be a subgroup of $G$. Then $H$ is nilpotent and subnormal in $G$ if and only if 
    $H\subseteq F(G)$.
\end{theorem}

\begin{proof}
	Assume first that $H\subseteq F(G)$. Since $F(G)$ is nilpotent by Theorem~\ref{thm:Fitting}, 
    so is $H$. Moreover, since $H$ is subnormal in $F(G)$ (Theorem~\ref{thm:subnormal}) 
    and $F(G)$ is normal in $G$, $H$ is
	subnormal in $G$.

	Assume now that $H$ is nilpotent and subnormal in $G$. We proceed by induction on $|G|$. 
    If $H=G$, then the result holds. Assume then that $H\ne G$. Since $H$ is subnormal in $G$, there is a sequence 
	\[
		H=H_0\subseteq H_1\subseteq\cdots\subseteq H_k=G
	\]
    of subgroups of $G$ with $H_i$ normal in $H_{i+1}$ for all $i$. 
	Let $M=H_{k-1}$. Since $M\ne G$ and $M$ is normal in $G$, 
	$H\subseteq F(M)$ by the inductive hypothesis. Thus $H\subseteq F(M)=M\cap
	F(G)\subseteq F(G)$ by Corollary~\ref{cor:McapF(G)}.
\end{proof}

Before proving another important theorem of Wielandt, we need a lemma. 

\begin{lemma}
	\label{lem:McapN=1}
	Let $M$ and $N$ be normal subgroups of $G$ such that $M\cap N=\{1\}$.
	Then $M\subseteq C_G(N)$.
\end{lemma}

\begin{proof}
	Let $m\in M$ and $n\in N$. Then $[n,m]=(nmn^{-1})m\in M$, since $M$ is normal in $G$ and 
	Moreover, $[n,m]=n(mn^{-1}m^{-1})\in N$, since $N$ is normal in 
	$G$. Thus $[n,m]\in M\cap N=\{1\}$.
\end{proof}

\begin{exercise}
\label{xca:characteristically_simple}
\index{Group!characteristically simple}
    A group $G$ is said to be \textbf{characteristically simple} if $G$ is non-trivial 
    and has no proper characteristic subgroups. 
    Prove that any minimal normal subgroup of $G$ is characteristically simple. 
\end{exercise}

\begin{definition}
    \index{Socle}
    Let $G$ be a group. If $G$ admits minimal normal subgroups, the \textbf{socle} of $G$ 
    is defined as the subgroup $\Soc(G)$ of $G$ generated by all minimal normal subgroups of $G$. If $G$ admits no minimal normal subgroups, 
    then $\Soc(G)=\{1\}$. 
\end{definition}

For example, $\Soc(\Z)=\{0\}$ and $\Soc(\SL_2(3))\simeq C_2$. 

\begin{exercise}
\label{xca:Soc_direct_product}
% Robinson 3.3.11 with $\Omega=Inn(G)$
    Prove that the socle of a group is a direct product of minimal normal subgroups. 
\end{exercise}

\begin{exercise}
\label{xca:caracteristically_simple}
% Robinson 3.3.15, page 87
Prove the following statements:
    \begin{enumerate}
        \item A direct product of isomorphic simple groups is characteristically simple.
        \item A characteristically simple group with at least one minimal normal subgroup is a direct product of isomorphic simple groups. 
    \end{enumerate}
\end{exercise}

\begin{theorem}[Wielandt]
	\label{thm:MsubsetNG(S)}
    \index{Wielandt's!theorem}
	Let $G$ be a finite group. If $S$ is a subnormal group of $G$ and 
	$M$ is a minimal normal subgroup of $G$, then $M\subseteq N_G(S)$.
\end{theorem}

\begin{proof}
	We proceed by induction on $|G|$. If $S=G$ the result holds. So assume that 
	$S\ne G$.  Since $S$ is subnormal in $G$, there exists a sequence 
	\[
		S=S_0\subseteq S_1\subseteq\cdots\subseteq S_{k-1}\subseteq S_k=G
	\]
    of subgroups of $G$ such that $S_i$ is normal in $S_{i+1}$ for all $i$. 
	Let $N=S_{k-1}$. 

	If $M\cap N\ne\{1\}$, then $M\subseteq N$ (because since $M$ and $N$ are both normal in $G$, 
	$M\cap N=M$ by the minimality of $M$). We claim that 
	$M\subseteq\Soc(N)$.  Since $M\ne\{1\}$ and $M$ is normal in $N$,
	$M\cap\Soc(N)\ne\{1\}$. Moreover, since $\Soc(N)$ is characteristic in $N$ and $N$ is 
	normal in $G$, it follows that $\Soc(N)$ is normal in $G$. Hence $M\cap\Soc(N)$ is a normal subgroup of $G$. 
	Since $\{1\}\ne M\cap\Soc(N)\subseteq M$, we conclude that 
	$M\cap\Soc(N)=M$ by the minimality of $M$. By the inductive hypothesis, 
	every minimal normal subgroup of $N$ normalizes $S$. Thus 
	$\Soc(N)\subseteq N_N(S)\subseteq N_G(S)$ and therefore 
	\[
	M\subseteq\Soc(N)\subseteq N_G(S).
	\]
	If $M\cap N=1$, Lemma~\ref{lem:McapN=1} implies that 
	\[
	M\subseteq C_G(N)\subseteq C_G(S)\subseteq N_G(S). \qedhere
	\]
\end{proof}

\begin{corollary}
    Let $G$ be a finite group and $S$ be a subnormal subgroup of $G$. Then 
    \[
    \Soc(G)\subseteq N_G(S).
    \]
\end{corollary}

\begin{proof}
	By Theorem~\ref{thm:MsubsetNG(S)}, every minimal normal subgroup of $G$ is contained in $N_G(S)$. Then 
	$\Soc(G)=\langle M:M\text{ minimal normal subgroup of $G$}\rangle\subseteq N_G(S)$.
\end{proof}

\begin{theorem}[Wielandt]
	\label{thm:STsubnormal}
    \index{Wielandt's lattice theorem}
	Let $G$ be a finite group and $S$ and $T$ be subnormal subgroups of $G$. Then $S\cap T$ and 
	$\langle S,T\rangle$ are subnormal in $G$.
\end{theorem}

\begin{proof}
	We first prove that $S\cap T$ is subnormal in $G$. Since subnormality is a transitive relation, it is enough to see that 
	$S\cap T$ is subnormal in $T$.
	Since $S$ is subnormal in $G$, there exists a sequence 
	\[
		S=S_0\subseteq S_1\subseteq \cdots\subseteq S_k=G
	\]
    of subgroups of $G$ such that $S_i$ is normal in $S_{i+1}$ for all $i$. 
	Each $S_{j-1}\cap T$ is normal in $S_j\cap T$. Then $S\cap T$ is 
	subnormal in $T$.
	
	
	We now prove that $\langle S,T\rangle$ is subnormal in $G$ 
	We proceed by induction on $|G|$. Assume that $G\ne\{1\}$. Let $M$ be a minimal normal subgroup of $G$ and 
    $\pi\colon G\to G/M$ be the canonical map. Since 
	both $\pi(S)$ and $\pi(T)$ subnormal in $G/M$ and $|G/M|<|G|$,
	the inductive hypothesis implies that 
	\[
	\pi(\langle S,T\rangle M)=\pi(\langle S,T\rangle)=\langle \pi(S),\pi(T)\rangle
	\]
	is subnormal in $G/M$. By the correspondence theorem, $\langle S,T\rangle M$ is 
	subnormal in $G$. Theorem~\ref{thm:MsubsetNG(S)}
	implies that $M\subseteq N_G(S)$ and $M\subseteq N_G(T)$. Hence $M\subseteq
	N_G(\langle S,T\rangle)$. Since $\langle S,T\rangle$ is normal in
	$\langle S,T\rangle M$ and $\langle S,T\rangle M$ is subnormal in $G$, we conclude that 
	$\langle S,T\rangle$ is subnormal in $G$.
\end{proof}


%\section{Torres de automorfismos}
%
%\begin{exercise}
%	\label{exercise:Inn(G)}
%	Sea $G$ un grupo. Demuestre las siguientes afirmaciones:
%	\begin{enumerate}
%		\item La conjugación $\gamma\colon G\to\Inn(G)$, $g\mapsto \gamma_g$,
%			es morfismo con núcleo $Z(G)$.
%		\item $\Inn(G)$ es normal en $\Aut(G)$.
%		\item Si $Z(G)=1$ entonces $C_{\Aut(G)}(\Inn(G))=1$.
%		\item Si $Z(G)=1$ entonces $Z(\Aut(G))=1$.
%	\end{enumerate}
%\end{exercise}
%
%\begin{svgraybox}
%	\begin{enumerate}
%		\item Es morfismo pues
%			\[
%			\gamma_{gh}(x)=(gh)x(gh)^{-1}=\gamma_g\gamma_h(x)
%			\]
%			y el núcleo es
%			\[
%			\ker\gamma=\{g\in G:\gamma_g=\id\}=\{g\in G:gxg^{-1}=x\text{ para todo $x\in G$}\}=Z(G).
%			\]
%		\item Es trivial pues $f\gamma_gf^{-1}=\gamma_{f(g)}$.
%		\item Sea $f\in C_{\Aut(G)}(\Inn(G))$. Como
%			$f\gamma_gf^{-1}=\gamma_{f(g)}$, 
%	Sea $\sigma\in\Aut(\Aut(G))$. Sabemos por el
%	ejercicio~\ref{exercise:Inn(G)} que $\Inn(G)$ es normal en $\Aut(G)$, y
%	entonces $\sigma(\Inn(G))$ es normal en $\Aut(G)$. El grupo
%	\[
%	\Inn(G)\simeq G/Z(G)\simeq G
%	\]
%	es simple y $\sigma(\Inn(G))\cap \Inn(G)$ es normal en $\Inn(G)$; entonces
%	hay dos posibilidades: $\Inn(G)\cap\sigma(\Inn(G))=1$ o bien
%	$\Inn(G)\cap\sigma(\Inn(G))=\sigma(\Inn(G))$. Basta demostrar que
%	$\Inn(G)\cap\sigma(\Inn(G))\ne1$. 
%
%	Si $\Inn(G)\cap\sigma(\Inn(G))=1$ entonces, al usar la normalidad de
%	$\sigma(\Inn(G)$ y de $\Inn(G)$ en $\Aut(G)$, tendríamos
%	$[\Inn(G),\sigma(\Inn(G))]\in \Inn(G)\cap \sigma(\Inn(G))=1$.  Luego, por
%	el el ejercicio ~\ref{exercise:Inn(G)}, 
%	\[
%	G\simeq \Inn(G)\simeq\sigma(\Inn(G))\subseteq
%	C_{\Aut(G)}(\Inn(G))=1,
%	\]
%	una contradicción.
%\end{proof}
%
%\begin{theorem}
%	\label{theorem:simple=>completo}
%	Sea $G$ un grupo simple no abeliano. Entonces $\Aut(G)$ es completo.
%\end{theorem}
%
%\begin{proof}
%	Como $Z(G)=1$, el ejercicio~\ref{exercise:Inn(G)} implica que
%	$Z(\Aut(G))=1$. Por el lema~\ref{lemma:Inn(G)char}, $\Inn(G)$ es
%	característico en $\Aut(G)$. Queremos demostrar que 
%	$\Aut(G)\subseteq \Inn(G)$.
%	
%	Sean $\sigma\in\Aut(\Aut(G))$, $g\in G$ y $\gamma_g\in\Inn(G)$. Como
%	$\sigma(\Inn(G))\subseteq\Inn(G)$, existe $\alpha(g)\in G$ tal que
%	$\sigma(\gamma_g)=\gamma_{\alpha(g)}$. Queda definida entonces una función
%	$\alpha\colon G\to G$ tal que $\sigma(\gamma_g)=\gamma_{\alpha(g)}$ para
%	todo $g\in G$.
%
%	\begin{claim}
%		$\alpha\in\Aut(G)$. 
%	\end{claim}
%
%	Veamos que $\alpha$ es inyectiva: si $\alpha(g)=\alpha(h)$ entonces
%	\[
%	\sigma(\gamma_g)=\gamma_{\alpha(g)}=\gamma_{\alpha(h)}=\sigma(\gamma_h)
%	\]
%	y luego $\gamma_g=\gamma_h$; esto implica que $h=g$ pues
%	$\ker(\gamma)=Z(G)=1$.  Luego $\alpha$ es biyectiva. Además $\alpha$ es
%	morfismo pues 
%	\[
%	\gamma_{\alpha(gh)}=\sigma(\gamma_{gh})=\sigma(\gamma_h\gamma_h)=\sigma(\gamma_h)\sigma(\gamma_h)=\gamma_{\alpha(g)}\gamma_{\alpha(h)}.
%	\]
%
%	\begin{claim}
%		$\sigma=\gamma_\alpha$.
%	\end{claim}
%
%	Sea $\tau=\sigma\gamma_{\alpha}^{-1}$ y sea $h\in G$. Entonces
%	\[
%		\tau(\gamma_h)=\sigma\gamma_{\alpha}^{-1}\gamma_h
%		=\sigma(\alpha^{-1}\gamma_h\alpha)
%		=\sigma\gamma_{\alpha^{-1}(h)}
%		=\gamma_{\alpha\alpha^{-1}(h)}=\gamma_h.
%	\]
%	Si $\beta\in\Aut(G)$ y $g\in G$ entonces 
%	\[
%		\beta\gamma_g\beta^{-1}
%		=\gamma_{\beta(g)}
%		=\tau(\gamma_{\beta(g)})
%		=\tau(\beta)\tau(\gamma_g)\tau(\beta)^{-1}
%		=\tau(\beta)\gamma_g\tau(\beta)^{-1}.
%	\]
%	Como entonces $\tau(\beta)\beta^{-1}\in C_{\Aut(G)}(\Inn(G))=1$ para todo
%	$\beta$, $\tau=\id$ y luego $\sigma=\gamma_{\alpha}$.
%\end{proof}
%
%Si $G$ es un grupo con centro trivial entonces $G\hookrightarrow\Aut(G)$ pues 
%\[
%G\simeq
%G/Z(G)\simeq\Inn(G)\subseteq\Aut(G).
%\]
%Como también $\Aut(G)$ tiene centro trivial, al iterar este procedimiento
%obtenemos una sucesión
%\begin{equation}
%	\label{equation:Aut(G)}
%G\hookrightarrow\Aut(G)\hookrightarrow\Aut(\Aut(G))\hookrightarrow\cdots
%\end{equation}
%Como aplicación del concepto de subnormalidad veremos un teorema de Wielandt
%que afirma que la sucesión~\eqref{equation:Aut(G)} se estabiliza. 
%
%\begin{lemma}
%	\label{lemma:CG(S)=1}
%	Sea $G$ un grupo y sea $S=S_1\triangleleft
%	S_2\triangleleft\cdots\triangleleft S_r=G$.  Si $C_{S_{i+1}}(S_i)=1$ para
%	todo $i\in\{1,\dots,r-1\}$ entonces $C_G(S)=1$. 
%\end{lemma}
%
%\begin{proof}
%	Procederemos por inducción en $r$. El caso $r=2$ es trivial pues
%	$C_{G}(S)=C_{S_2}(S_1)=1$. Supongamos entonces que $r>2$. Al usar la
%	hipótesis inductiva al grupo $S_{r-1}$ obtenemos 
%	$C_G(S)\cap S_{r-1}=C_{S_{r-1}}(S)=1$.
%	Como $S_1$ es normal en $S_2$, $C_{G}(S)$ también es normal en $S_2$ pues
%	si $x\in C_G(S)$, $s_1\in S_1$, $s_2\in S_2$ entonces $s_2^{-1}s_1s_2\in
%	S_1$ y luego 
%	\[
%		[s_2xs_2^{-1},s_1]=s_2x(s_2^{-1}s_1s_2)x^{-1}s_2^{-1}s_1^{-1}=1.
%	\]
%	La normalidad de $S_{r-1}$ en $G$ implica que 
%	$[C_G(S),S_2]\subseteq C_G(S)\cap S_{r-1}=1$.
%	Luego $C_G(S)\subseteq C_G(S_2)$
%	Al usar la hipótesis inductiva en la sucesión 
%	$S_2\triangleleft\cdots\triangleleft S_r=G$, se concluye que $C_G(S)=1$.
%\end{proof}
%
%\begin{lemma}
%	\label{lemma:CG(N)=Z(N)}
%	Sea $N$ un subgrupo normal de un grupo finito $G$ tal que $C_G(N)\subseteq
%	N$. Entonces $|G|$ divide a $|Z(N)||\Aut(N)|$. En particular, $|G|$ divide
%	al factorial de $|N|$.
%\end{lemma}
%
%\begin{proof}
%	Al hacer actuar a $G$ en $N$ por conjugación obtenemos un morfismo
%	$\rho\colon G\to\Aut(N)$ con núcleo
%	\[
%	\ker\rho=\{g\in G:gng^{-1}=n\text{ para todo $n\in N$}\}=C_G(N).
%	\]
%	Como $C_G(N)\subseteq N$, $\ker\rho=C_G(N)=Z(N)$ y luego $G/Z(N)$ es
%	isomorfo a un subgrupo de $\Aut(N)$. 
%
%	Por el teorema de Lagrange, $|Z(N)|$ divide a $|N|$. Como 
%	$\Aut(N)$ actúa fielmente en el conjunto $N\setminus\{1\}$, se tiene 
%	un morfismo inyectivo $\Aut(N)\to\Sym_{|N|-1}$. Luego $|G|$ divide a $|N|!=|N|(|N|-1)!$.
%	% recordemos que actuar fielmente quiere decir que $f\cdot n=n$ para todo $n$ implica que $f=\id$. 
%\end{proof}
%
%Recordemos que si $G$ es un grupo, existe un único subgrupo normal minimal
%$G^{\infty}$ con la siguiente propiedad: el cociente $G/G^{\infty}$ es
%nilpotente.
%
%\begin{lemma}
%	\label{lemma:Ginf=Sinf}
%	Sea $G$ un grupo finito tal que $G=SF$ para algún subgrupo $S$ subnormal en
%	$G$ y algún subgrupo nilpotente $F$ normal en $G$. Entonces
%	$G^{\infty}=S^{\infty}$.
%\end{lemma}
%
%\begin{proof}
%	Sin perder generalidad podemos suponer que $S\ne G$.
%\end{proof}

\subsection{Wielandt's zipper theorem}

\begin{theorem}[Wielandt]
	\index{Wielandt's!zipper theorem}
	\label{thm:zipper}
	Let $G$ be a finite group and $S$ be a subgroup of $G$ subnormal in every 
    proper subgroup of $G$ containing $S$. If $S$ is not subnormal in $G$, 
    then there exists a unique maximal subgroup of $G$ containing $S$. 
\end{theorem}

\begin{proof}
	We proceed by induction on $(G:S)$. If $S$ is not subnormal in $G$, then 
	$S\ne G$ and the case where $(G:S)=1$ holds. 

	Since $S$ is not subnormal in $G$, $N_G(S)\ne G$. Then $S\subseteq
	N_G(S)\subseteq M$ for some maximal subgroup $M$ of $G$. Assume that 
	$S\subseteq K$ for some maximal subgroup $K$ of $G$. We claim that 
	que $K=M$. Since $S\subseteq K\ne G$, $S$ is subnormal in $K$. If $S$ is 
	normal in $K$, then $K\subseteq N_G(S)\subseteq M$. Hence $K=M$ by the maximality of $K$. 
	If $S$ is not normal in $K$, there exist a sequence 
	$S_0,\dots,S_m$ of subgroups of $K$ such that 
	\[
		S=S_0\subseteq S_1\subseteq\cdots\subseteq S_m=K,
	\]
    where $S_i$ is normal in $S_{i+1}$ for all $i$ and 
	$S$ is not normal in $S_2$. Let $x\in S_2$ be such that $xSx^{-1}\ne S$ and 
	$T=\langle S,xSx^{-1}\rangle\subseteq K$. 

	Since $xSx^{-1}\subseteq xS_1x^{-1}=S_1\subseteq N_G(S)$, we obtain that 
	$T\subseteq N_G(S)\subseteq M$. Moreover, $S$ is normal in $T$. Thus $T\ne G$. 

	We claim that $T$ satisfies the theorem's assumptions. If $T$ is subnormal in $G$, then, since 
	$S$ is normal in $T$, $S$ is subnormal in $G$. If $H$ is a proper subgroup of $G$ 
    containing $T$, then, since 
	$S\subseteq H$, $S$ is subnormal in $H$. Moreover, $xSx^{-1}$ is subnormal in $H$. Hence 
	$T$ is subnormal in $H$ by Theorem~\ref{thm:STsubnormal}.

	Since $S\subsetneq T$, $(G:T)<(G:S)$. By the inductive hypothesis, $T$ is contained in a unique maximal subgroup of $G$. Therefore
	$K=M$, since $T\subseteq M$ and 
	$T\subseteq K$.
\end{proof}

Before giving an application, we need a lemma. 

\begin{lemma}
	\label{lem:H=G}
	Let $G$ be a group and $H$ be a subgroup of $G$. If $(xHx^{-1})H=G$ for some 
	$x\in G$, then $H=G$.
\end{lemma}

\begin{proof}
	Write $x=uv$ for some $u\in xHx^{-1}$ and $v\in H$. Since $u\in xHx^{-1}$ and
	$u^{-1}x=v\in H$, we obtain that $H=vHv^{-1}=u^{-1}(xHx^{-1})u=xHx^{-1}$. Thus
	$G=H$. 
\end{proof}

Recall that two subgroups $S$ and $T$ of a group $G$ are said to be
\textbf{permutable} if $ST=TS$. 

\begin{theorem}
	Let $G$ be a finite group and $S$ be a subgroup of $G$ permutable with any of
	its conjugates. Then $S$ is subnormal in $G$. 
\end{theorem}

\begin{proof}
	We proceed by induction on $|G|$. Assume that $S$ is subnormal in 
	every subgroup $H$ such that $S\subseteq H\subsetneq G$.  If $S$ is not subnormal in $G$, 
	then, by Theorem~\ref{thm:zipper}, there exists a unique maximal subgroup $M$ of $G$ 
	such that $S\subseteq M$. Let $x\in G$ and 
	$T=xSx^{-1}$. By Lemma~\ref{lem:H=G}, $ST\ne G$ (because $S\ne G$). Thus 
	$ST$ is contained in some maximal subgroup of $G$. Since 
	$S\subseteq ST$ and $S$ is contained in a unique maximal subgroup of $G$, we conclude that 
	$T\subseteq ST\subseteq M$.  Since $S^G=\langle xSx^{-1}:x\in
	G\rangle\subseteq M\ne G$, the inductive hypothesis implies that $S$ is subnormal in
	$S^G$. Hence $S$ is subnormal in $G$ since $S^G$ is normal in $G$, a contradiction. 
\end{proof}

\subsection{Baer's theorem}

\begin{theorem}[Baer]
	\index{Baer's theorem}
	\label{thm:Baer}
	Let $G$ be a finite group and $H$ be a subgroup of $G$. Then $H\subseteq
	F(G)$ if and only if $\langle H,xHx^{-1}\rangle$ is nilpotent for all 
	$x\in G$.
\end{theorem}

\begin{proof}
	If $H\subseteq F(G)$, then $xHx^{-1}\subseteq F(G)$ for all $x\in G$, since
	$F(G)$ is normal in $G$. Thus $\langle H,xHx^{-1}\rangle$ is nilpotent, as it
	is a subgroup of $F(G)$.

	Conversely, assume that $\langle H,xHx^{-1}\rangle$ is nilpotent for all  $x\in
	G$. Since $H\subseteq \langle H,xHx^{-1}\rangle$, $H$ is nilpotent. By
	Theorem~\ref{thm:F(G)subnormalidad}, it is enough to see that $H$ is subnormal
	in $G$. We proceed by induction on $|G|$. Suppose that $H$ is not subnormal in
	$G$. If $H$ is properly contained in some subgroup $K$, then, since $\langle
	H,kHk^{-1}\rangle$ is nilpotent for all $k\in K$, $H$ is subnormal in $K$ by
	the inductive hypothesis. By Theorem~\ref{thm:zipper}, there exists a unique
	maximal subgroup $M$ of $G$ containing $H$. There are two cases to consider.
    
    Assume first that $G=\langle H,xHx^{-1}\rangle$ for some $x\in G$. Since $G$
    is nilpotent, $H$ subnormal in $G$ by Theorem~\ref{thm:subnormal}, a
    contradiction. 

    Assume now that $\langle H,xHx^{-1}\rangle\ne G$ for all $x\in G$. For each 
	$x\in G$, there exists a maximal subgroup containing $\langle
	H,xHx^{-1}\rangle$. Since $H\subseteq \langle H,xHx^{-1}\rangle$ and $H$
	is contained in a unique maximal subgroup, we conclude that $\langle
	H,xHx^{-1}\rangle\subseteq M$ for all $x\in G$. In particular, the normal closure 
	$H^G$ of $H$ is properly contained in $G$. By the inductive hypothesis, 
	$H$ is subnormal in $H^G$ and $H^G$ is normal in $G$, we conclude that 
	$H$ is subnormal in $G$, a contradiction. 
\end{proof}

\subsection{Zenkov's theorem}

\begin{theorem}[Zenkov]
    \index{Zenkov's!theorem}
    \label{thm:Zenkov}
    Let $G$ be a finite group and $A$ and $B$ be abelian subgroups of $G$. Let
    $M\in\{A\cap gBg^{-1}:g\in G\}$ such that no $A\cap gBg^{-1}$ is properly
    contained in $M$. Then $M\subseteq F(G)$.
\end{theorem}

\begin{proof}
	Without loss of generality, we may assume that $M=A\cap B$. Using induction on 
	$|G|$, we prove that $M\subseteq F(G)$.

	Assume first that $G=\langle A,gBg^{-1}\rangle$ for some $g\in G$. Since $A$
	and $B$ are both abelian, 
 \[
 A\cap gBg^{-1}\subseteq Z(G)
 \]
 and hence 
	\[
		A\cap gBg^{-1}=g^{-1}(A\cap gBg^{-1})g\subseteq A\cap B=M.
	\]
	By the minimality of $M$, 
    \[
    M=A\cap gBg^{-1}\subseteq Z(G)\subseteq F(G)
    \]
	by Corollary~\ref{cor:Z(G)subsetF(G)}.

	Assume now that $G\ne \langle A,gBg^{-1}\rangle$ for all $g\in G$.
	Let $g\in G$, $H=\langle A,gBg^{-1}\rangle\ne G$ and $C=B\cap H$.
	Since $A\subseteq H$, we obtain that 
 	$M=A\cap B=A\cap C$ and 
	$A\cap hCh^{-1}=A\cap hBh^{-1}$
	for all $h\in H$. This implies that no 
	$A\cap hCh^{-1}$ is properly contained in $A\cap C$. 
    By the inductive hypothesis on $H$, 
 	\[
		M=A\cap B=A\cap C\subseteq F(H).
	\]

    We now prove that every Sylow $p$-subgroup $P$ of $M$ is contained in $F(G)$. 
	Since $M$ is generated by its Sylow subgroups, 
    $M\subseteq F(G)$.
	If $P\in\Syl_p(M)$, then $P\subseteq M\subseteq F(H)$. Since $O_p(H)$ is 
    the only Sylow $p$-subgroup of $F(H)$, $P\subseteq O_p(H)$. Since 
	$P\subseteq M\subseteq B$, 
	\[
	gPg^{-1}\subseteq gBg^{-1}\subseteq H
	\]
	for all $g\in G$. Thus $O_p(H)(gPg^{-1})$ is a $p$-subgroup of $H$ containing 
	$\langle P,gPg^{-1}\rangle$. Hence $\langle P,gPg^{-1}\rangle$
	is nilpotent for all $g\in G$, since it is a $p$-group. By Baer's theorem~\ref{thm:Baer}, 
    $P\subseteq F(G)$ for all Sylow $p$-subgroup $P$ of $M$. 
\end{proof}

\begin{corollary}
	\label{cor:Zenkov}
	Let $G$ be a non-trivial finite group and $A$ be an abelian subgroup of $G$ such that 
 	$|A|\geq(G:A)$. Then $A\cap F(G)\ne\{1\}$.
\end{corollary}

\begin{proof}
	Let $g\in G$. We may assume that $G\ne A$. Then $(gAg^{-1})A\ne G$ by Lemma~\ref{lem:H=G}. Since 
	$|gAg^{-1}||A|=|A|^2\geq |A|(G:A)=|G|$, 
	\[
		|G|>|gAg^{-1}A|
		=\frac{|A||gAg^{-1}|}{|A\cap gAg^{-1}|}
		\geq \frac{|G|}{|A\cap gAg^{-1}|}.
	\]
	Hence $A\cap gAg^{-1}\ne 1$ for all $g\in G$. In particular, no
	$A\cap gAg^{-1}$ is properly contained in $A$. By 
	Zenkov's theorem~\ref{thm:Zenkov}, $A\subseteq F(G)$.
\end{proof}

\begin{corollary}
	Let $G=NA$ be a finite group, where $N$ is a normal subgroup of $G$, $A$ is an abelian subgroup of $G$ and 
	$C_A(N)=\{1\}$. If $F(N)=\{1\}$, then $|A|<|N|$. 
\end{corollary}

\begin{proof}
	Since $N$ is normal in $G$, 
	\[
    N\cap F(G)=F(N)=\{1\}
    \]
    by Corollary~\ref{cor:McapF(G)}. Thus $[N,F(G)]=\{1\}$, since 
	both $N$ and $F(G)$ are normal in $G$. Since 
	\[
	|A|\geq |N|\geq \frac{|N|}{|N\cap A|}=(NA:A)=(G:A),
	\]
	$A\cap F(G)\ne\{1\}$ by Corollary~\ref{cor:Zenkov}. If $1\ne a\in
	A\cap F(G)$, then $a\in C_A(N)=1$, a contradiction. 
\end{proof}

\subsection{Brodkey's theorem}

\begin{theorem}[Brodkey]
	\index{Brodkey's!theorem}
	\label{thm:Brodkey}
	Let $G$ be a finite group such that there exists an abelian $P\in\Syl_p(G)$. Then
    there exist $S,T\in\Syl_p(G)$ such that $S\cap T=O_p(G)$.
\end{theorem}

\begin{proof}
	By applying Zenkov's theorem (Theorem~\ref{thm:Zenkov}) with $A=B=P$, 
	\[
    P\cap gPg^{-1}\subseteq F(G)
    \]
    for some $g\in G$. Since $O_p(G)$ is the only Sylow $p$-subgroup of 
	$F(G)$, $P\cap gPg^{-1}\subseteq O_p(G)$.
	Hence $P\cap gPg^{-1}=P_p(G)$, since $O_p(G)$ is contained in every Sylow $p$-subgroup 
	of $G$. 
\end{proof}

\begin{corollary}
	\label{corollary:GP2}
	Let $G$ be a finite group. If there exists an abelian $P\in \Syl_p(G)$, 
	\[
	(G:O_p(G))\leq (G:P)^2. 
	\]
\end{corollary}

\begin{proof}
	By Brodkey's theorem, there exist $S,T\in\Syl_p(G)$
	such that $S\cap T=O_p(G)$. Then 
	\[
		|G|\geq |ST|=\frac{|S||T|}{|S\cap T|}=\frac{|P|^2}{|O_p(G)|},
	\]
	which implies the claim. 
\end{proof}

\begin{corollary}
	Let $G$ be a finite group. If there exists an abelian $P\in\Syl_p(G)$ such that 
	$|P|<\sqrt{|G|}$, then $O_p(G)\ne\{1\}$.
\end{corollary}

\begin{proof}
	Since $(G:P)^2<|G|$, the previous corollary implies that 
	$O_p(G)\ne\{1\}$.
\end{proof}

% \begin{exercise}
% 	\label{xca:G/Z(G)}
% 	Sea $G$ un grupo y sea Sea $K\subseteq Z(G)$. Demuestre que si $G/K$ es
% 	cíclico entonces $G$ es abeliano.
% \end{exercise}

% \begin{sol}{xca:G/Z(G)}
% 	Sean $g,h\in G$ y sea $\pi\colon G\to G/K$ el morfismo canónico. Como $G/K$
% 	es cíclico, existe $x\in G$ tal que $G/K=\langle xK\rangle$. Sean $k,l$ tales que 
% 	$\pi(g)=x^k$, $\pi(h)=x^l$. Entonces existen $z_1,z_2\in K$ tales que 
% 	$g=x^kz_1$, $h=x^lz_2$. Luego $[g,h]=[x^k,x^l]=1$. 
% \end{svgraybox}


\subsection{Lucchini's theorem}

\begin{theorem}[Lucchini]
	\index{Lucchini's!theorem}
	\label{thm:Lucchini}
	Let $G$ be a finite group and $A$ be a proper cyclic subgroup of $G$. If 
	$K=\Core_G(A)$, then $(A:K)<(G:A)$.
\end{theorem}

\begin{proof}
	We proceed by induction on $|G|$. Let $\pi\colon G\to G/K$ be the canonical map. Note that $\Core_{G/K}\pi(A)$ is trivial. 

	Assume first that $K\ne\{1\}$. Since $\pi(A)$ is a proper cyclic subgroup of 
	$G/K$ and $K\subseteq A$, the inductive hypothesis implies that 
	\[
		(A:K)=|\pi(A)|=(\pi(A):\pi(K))<(\pi(G):\pi(A))=\frac{(G:K)}{(A:K)}=(G:A).
	\]

	Assume now that $K=\{1\}$. We want to prove that $|A|<(G:A)$. Suppose that 
	$|A|\geq (G:A)$. Since $A\ne G$, $A\cap F(G)\ne\{1\}$ by Corollary~\ref{cor:Zenkov}. In particular, $F(G)\ne\{1\}$. Let $E$ be a minimal normal subgroup of 
	such that $E\subseteq F(G)$. By Theorem~\ref{thm:Hirsch}, $E\cap Z(F(G))\ne\{1\}$.  Since 
	$E\cap Z(F(G))$ is normal in $G$ and $E$ is minimal, $E\cap Z(F(G))=E$, that is 
	$E\subseteq Z(F(G))$. In particular, $E$ is abelian. By the minimality of 
	$E$, there is a prime number $p$ such that $x^p=1$ for all $x\in E$. 

	\begin{claim}
		$A\cap F(G)$ is a normal subgroup of $EA$.
	\end{claim}

	Since $E$ is normal in $G$, $EA$ is a subgroup of $G$. Since $A\cap
	F(G)\subseteq A$, $A\cap F(G)$ is a subgroup of $EA$.  Since $F(G)$ is 
	normal in $G$, $a(A\cap F(G))a^{-1}=A\cap F(G)$ for all $a\in A$. Moreover, 
    $E\subseteq Z(F(G))$ and $A\cap F(G)\subseteq F(G)$ imply that 
	$x(A\cap F(G))x^{-1}=A\cap F(G)$ for all $x\in E$. 

	\begin{claim}
		$EA\ne G$.
	\end{claim}

	If $G=EA$, then, since $A\cap F(G)$ is a normal subgroup of $G$
	contained in $A$, we conclude that $\{1\}\ne A\cap F(G)\subseteq K=1$, a 
	contradiction. 
%como $F(G)$ es normal en $G$, 
%	\[
%	A\cap F(G)=g(A\cap F(G))g^{-1}=gAg^{-1}\cap F(G)\subseteq gAg^{-1}
%	\]
	%para todo $g\in G$. 
%    Luego $1\ne A\cap F(G)\subseteq K$, una contradicción pues $K=1$.

	\medskip
	Let $p\colon G\to G/E$ the canonical map. By the correspondence theorem,
	there exists a normal subgroup $M$ of $G$ such that $E\subseteq M$ and 
	$p(M)=\Core_{G/E}(p(A))$. Since $EA\ne G$, $p(A)$ is a proper cyclic subgroup 
	of $p(G)$. Since $p(A)\simeq A/A\cap E\simeq EA/E$ and $p(M)\simeq
	M/E$, the inductive hypothesis implies that 
	$(EA:M)<(G:EA)$, as 
	\[
	\frac{|EA/E|}{|M/E|}
	=(p(A):p(M))
	<(p(G):p(A))
	=\frac{|G/E|}{|EA/E|}.
	\]

	\begin{claim}
		$MA=EA$. 
	\end{claim}

	Since $E\subseteq M$, $EA\subseteq MA$. Conversely, if $m\in M$, 
	then, since $p(m)\in\Core_{G/E}(p(A))$, we obtain that 
	$p(m)\in p(A)$. Thus $m\in EA$. 

	\medskip
	Let $B=A\cap M$. Since $(AE:M)<(G:EA)$, 
	\[
	(A:B)=|A/A\cap M|=|AM/M|=(EA:M).
	\]
	By the inductive hypothesis, 
	\begin{equation}
		\label{eq:(M:B)leq|B|}
	\begin{aligned}
		(M:B)&=(M:A\cap M)=(MA:A)\\
		&=(EA:A)
		=\frac{(G:A)}{(G:EA)}
		<\frac{(G:A)}{(AE:M)}
		=\frac{(G:A)}{(A:B)}\leq |B|, 
	\end{aligned}
	\end{equation}
	as $|A|\geq (G:A)$. 

	\begin{claim}
		$M=EB$.
	\end{claim}

	Since $E\cup B\subseteq M$, $EB\subseteq M$. Conversely, if 
	$m\in M$, then $m=ea$ for some $e\in E$ and $a\in A$. Since $e^{-1}m=a\in
	A\cap M=B$ (because $E\subseteq M$), $m\in EB$.

	\begin{claim}
		$M$ is not abelian. 
	\end{claim}

	Suppose that $M$ is abelian. The map $f\colon M\to M$, $m\mapsto
	m^p$, is a group homomorphism such that $E \subseteq\ker f$. Since $M=EB$,
	$f(M)\subseteq f(B)\subseteq B\subseteq A$. Since $M$ is normal in $G$,
	$f(M)$ is normal in $G$. Thus $f(M)=\{1\}$, as $K=\Core_G(A)=\{1\}$ is the largest normal subgroup of $G$ contained in $A$. In particular, since $B$ is normal in $M=EB$, $M/B$ is a $p$-group. Since $B\subseteq M$,  $f(B)=\{1\}$. Moreover, since 
	$B\subseteq A$ is cyclic, $|B|\leq p$. By using~\eqref{eq:(M:B)leq|B|}, 
	$(M:B)<|B|\leq p$. This implies that $M=B\subseteq A$ and $M=E=1$ (because 
	$M$ is normal in $G$ and $\Core_G(A)=K=\{1\}$ is the largest normal subgroup of $G$ containing $A$), a contradiction. 
	
	\begin{claim}
		$Z(M)$  is cyclic. 
	\end{claim}

	Since $M$ is not abelian and $M/E=EB/E\simeq B/E\cap B$ is cyclic,
	$E\not\subseteq Z(M)$, that is $E\cap
	Z(M)\subsetneq E$. Thus  
	\begin{equation}
		\label{equation:EcapZ(M)}
		E\cap Z(M)=\{1\}
	\end{equation}
	by the minimality of $E$. Hence  
	\[
	Z(M)=Z(M)/Z(M)\cap E\simeq p(Z(M))\subseteq p(M)=\Core_{G/E}p(A)\subseteq p(A)
	\]
	and therefore $Z(M)$ is cyclic, since $p(A)$ is cylic. 

	\medskip
	Since $B\subseteq A$ is abelian and $(M:B)<|B|$, $B\cap F(M)\ne1$ by Corollary~\ref{cor:Zenkov}. Then $[E,F(M)]=1$ (because $E\subseteq
	Z(F(G))$ and $F(M)\subseteq F(G)$ by Corollary~\ref{cor:McapF(G)}).
	Hence $B\cap F(M)\subseteq Z(M)$, since $M=BE$, $[B\cap F(M),E]\subseteq
	[F(M),E]=1$ and $[B\cap F(M),B]=1$ as $B$ is abelian. Since 
	$Z(M)$ is cyclic, $B\cap F(M)$ is characteristic in $Z(M)$. Since 
	$Z(M)$ is normal in $G$, $\{1\}\ne B\cap F(M)$ is a normal subgroup of $G$
	contained in $A$, a contradiction. 
\end{proof}

\subsection{Horosevskii's theorem}

To conclude this section, we present a striking application of Lucchini's theorem.

\begin{corollary}[Horosevskii]
	\index{Horosevskii's!theorem}
	Let $G$ be a finite non-trivial group and $\sigma\in\Aut(G)$. Then 
	$|\sigma|<|G|$.
\end{corollary}

\begin{proof}
	Let $A=\langle\sigma\rangle$ act by automorphisms on $G$ and 
	 $\Gamma=G\rtimes A$. The group operation of $\Gamma$ is 
	\[
	(g,\sigma^k)(h,\sigma^l)=(g\sigma^k(h),\sigma^{k+l}).
	\]
	Identity $A$ with $\{1\}\times A$ and $G$ with $G\times\{1\}$. 
	Since $K\cap G\subseteq A\cap G=\{1\}$ and $A\cap C_{\Gamma}(G)=\{1\}$, 
	\[
		K\subseteq A\cap C_{\Gamma}(G)=\{1\}.
	\]
	If $k\in K$ and $g\in G$, then $gkg^{-1}k^{-1}\in G\cap K=\{1\}$ (because
	$K$ and $G$ are both normal in $\Gamma$). By Lucchini's theorem, 
    $(A:K)<(\Gamma:A)$, that is
	\[
		|\sigma|=|A|=(A:K)<(\Gamma:A)=|G|.\qedhere
	\]
\end{proof}

\subsection{*Wielandt's automorphism tower theorem}

We now present without proof a beautiful theorem of Wielandt. 
For $G$ a finite group with trivial center, let 
$A_1=G$ and $A_{k+1}=\Aut(A_k)$ for $k\geq1$. Note that 
identifying $G$ with $\Inn(G)$, one gets a sequence
\begin{equation}
\label{eq:automorphism_groups}
A_1\subseteq A_2\subseteq A_3\subseteq\cdots 
\end{equation}
where $A_i$ is normal in $A_{i+1}$. 

\begin{definition}
    \index{Group!complete}
    A group $G$ is said to be \textbf{complete} if $Z(G)=\{1\}$ and
    $\Aut(G)=\Inn(G)$. 
\end{definition}

\begin{example}
For example, the group $\Sym_3$ is complete:
\begin{lstlisting}
gap> G := SymmetricGroup(3);;
gap> IsTrivial(Center(G));
true
gap> AutomorphismGroup(G)=InnerAutomorphismGroup(G);
true    
\end{lstlisting}
In particular, the sequence \eqref{eq:automorphism_groups}
stabilizes. 
\end{example}

\begin{example}
    Let $G=\Sym_3\times\Sym_3$. Then $|G|=36$ and $Z(G)=\{1\}$. Moreover, 
    $|\Aut(G)|=72$ and $|\Aut(\Aut(G))|=144$. Since the group
    $\Aut(\Aut(G))$ is complete, the sequence \eqref{eq:automorphism_groups} stabilizes. Let us do this 
    with the computer: 
\begin{lstlisting}
gap> G := DirectProduct(SymmetricGroup(3),SymmetricGroup(3));;
gap> A1 := G;;
gap> A2 := AutomorphismGroup(G);;
gap> A3 := AutomorphismGroup(A2);;
gap> Order(A2);
72
gap> Order(A3);
144
gap> IsTrivial(Center(A3));
true
gap> AutomorphismGroup(A3)=InnerAutomorphismGroup(A3);
true    
\end{lstlisting}
\end{example}

Let $G$ be a group and $g\in G$. Let
$\gamma_g\colon G\to G$, $x\mapsto gxg^{-1}$, denote the
conjugation map. Then 
\[
\Inn(G)=\{\gamma_g:g\in G\}
\]
is a normal subgroup of $\Aut(G)$. Moreover, $G/Z(G)\simeq\Inn(G)$. 

% Rotman 
\begin{exercise}
\label{xca:Aut}
    Let $G$ be a non-abelian simple group, $A=\Aut(G)$ and
    $I=\Inn(G)$. Prove the following statements:
    \begin{enumerate}
        \item $C_A(I)=\{\id\}$. 
        \item $f(I)=I$ for all $f\in \Aut(A)$.
        \item Every $f\in\Aut(A)$ is inner. 
        \item $\Aut(G)$ is complete. 
    \end{enumerate}    
\end{exercise}

% \begin{sol}{xca:Aut}
%     \begin{enumerate}
%     \item 
%     \end{enumerate}
% \end{sol}

The following result is known as the \textbf{Wielandt automorphism tower theorem}. 

\begin{theorem}[Wielandt]
    \index{Wielandt's automorphism tower theorem}
    \label{thm:Wielandt:automorphism}
    Let $G$ be a finite group with trivial center. Up to isomorphism, there
    are finitely many groups among the terms of the sequence \eqref{eq:automorphism_groups}.  
\end{theorem}

\begin{proof}
    See \cite[Theorem 9.10]{MR2426855}.
\end{proof}