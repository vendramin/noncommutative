\section{Lecture: Week 10}



\subsection{The transfer map}

If $H$ is a subgroup of $G$, recall that 
a \textbf{transversal} of $H$ in $G$ is a complete
set of coset representatives of $G/H$. 

\begin{lemma}
	\label{lem:d:transfer}
	Let $G$ be a group and $H$ be a subgroup of $G$ of finite index.  Let $R$
	and $S$ be transversals of $H$ in $G$ and let $\alpha\colon H\to H/[H,H]$
	be the canonical map. Then 
	\[
		d(R,S)=\prod \alpha(rs^{-1}),
	\]
	where the product is taken over all pairs 
	$(r,s)\in R\times S$ such that $Hr=Hs$,
	is well-defined and satisfies the following properties:
	\begin{enumerate}
		\item $d(R,S)^{-1}=d(S,R)$.
		\item $d(R,S)d(S,T)=d(R,T)$ for all transversal $T$ of $H$ in $G$.
		\item $d(Rg,Sg)=d(R,S)$ for all $g\in G$.
		\item $d(Rg,R)=d(Sg,S)$ for all $g\in G$.
	\end{enumerate}
\end{lemma}

\begin{proof}
	The product that defines $d(R,S)$ is well-defined since $H/[H,H]$ is 
	an abelian group. The first three claim are trivial. Let us prove
	4). By 2), 
	\[
		d(Rg,Sg)d(Sg,S)d(S,R)=d(Rg,S)d(S,R)=d(Rg,R).
	\]
	Since $H/[H,H]$ is abelian, 1) and 3) imply that 	
	\[
		d(Rg,Sg)d(Sg,S)d(S,R)=d(R,S)d(S,R)d(Sg,S)=d(Sg,S).\qedhere
	\]
\end{proof}

\begin{theorem}
	\label{thm:transfer}
	Let $G$ be a group and $H$ be a finite-index subgroup of $G$. The map 	
	\[
		\nu\colon G\to H/[H,H],\quad
		g\mapsto d(Rg,R),
	\]
	does not depend on the transversal $R$ of $H$ in $G$ and is a group
	homomorphism. 
\end{theorem}

\begin{proof}
	The previous lemma implies that the map does not depend on the transversal used. 
	Moreover, $\nu$ is a group homomorphism, as 
	\begin{align*}
		\nu(gh)&=d(R(gh),R)
		=d(R(gh),Rh)d(Rh,R)
		=d(Rg,R)d(Rh,R)=\nu(g)\nu(h).\qedhere
	\end{align*}
\end{proof}

The theorem justifies the following definition: 

\begin{definition}
\index{Transfer homomorphism}
	Let $G$ be a group and $H$ be a finite-index subgroup of $G$. The
	\textbf{transfer map} of $G$ in $H$ is the group homomorphism 
	\[
		\nu\colon G\to H/[H,H],
		\quad
		g\mapsto d(Rg,R),
	\]
	of Theorem~\ref{thm:transfer}, where $R$ is some transversal of $H$ in $G$.
\end{definition}

We need methods for computing the transfer map. If $H$ is a subgroup of 
$G$
and $(G:H)=n$, let $T=\{x_1,\dots,x_n\}$ be a transversal of $H$. For $g\in G$ let  
\[
	\nu(g)=\prod \alpha(xy^{-1}),
\]
where the product is taken over all pairs $(x,y)\in (Tg)\times T$ such that $Hx=Hy$
and $\alpha\colon H\to H/[H,H]$ is the canonical map. 
If we write 
$x=x_ig$ for some $i\in\{1,\dots,n\}$, then  
$Hx_ig=Hx_{\sigma(i)}$ for some permutation $\sigma\in\Sym_n$. Thus 
\[
	\nu(g)=\prod_{i=1}^n\alpha(x_igx_{\sigma(i)}^{-1}).
\]
The cycle structure of $\sigma$ turns out to be important. 
For example, if $\sigma=(12)(345)$ and $n=5$, then a direct calculation shows that 
\[
\prod_{i=1}^5\alpha\left(x_igx_{\sigma(i)}^{-1}\right)
=\alpha(x_1g^2x_1^{-1})\alpha(x_3g^3x_3^{-1}).
\]
This is precisely the content of the following lemma. 



% \begin{lemma}
% 	\label{lem:transfer}
% 	Let $G$ be a group and $H$ be a subgroup such that $(G:H)=n$. Let 
% 	$T$ be a transversal of $H$ in $G$. 
% 	For each $g\in G$ there exist $k$ and 
% 	positive integers 
% 	$n_1,\dots,n_k$ such that $n_1+\cdots+n_k=n$ and elements 
% 	$t_1,\dots,t_k\in T$ such that  
% 	\[
% 		\nu(g)=\prod_{i=1}^k \alpha(t_ig^{n_i}t_i^{-1}),
% 	\]
% 	where $\alpha\colon H\to H/[H,H]$ is the canonical map.
% \end{lemma}

\begin{lemma}
	\label{lem:transfer}
	Let $G$ be a group and $H$ be a subgroup of index $n$. Let 
	$T=\{t_1,\dots,t_n\}$ be a transversal of $H$ in $G$.  For each $g\in G$ there exist 
	$m\in\Z_{>0}$ and elements 
	$s_{1},\dots,s_{m}\in T$ and positive integers $n_1,\dots,n_m$
    such that 
	$s_i^{-1}g^{n_i}s_i\in H$,
	$n_1+\cdots+n_m=n$ and 
	\[
		\nu(g)=\prod_{i=1}^m \alpha(s_i^{-1}g^{n_i}s_i).
	\]
\end{lemma}

\begin{proof}
	For each $i$ there exist $h_1,\dots,h_n\in H$ and $\sigma\in\Sym_n$ such that 
	$gt_i=t_{\sigma(i)}h_i$. Write $\sigma$ as a product of disjoint cycles, say 
	\[
		\sigma=\alpha_1\cdots\alpha_m.
	\]

	Let $i\in\{1,\dots,n\}$ and write 
	$\alpha_i=(j_{1}\cdots j_{n_i})$. Since   
	\[
		g t_{j_k}=t_{\sigma(j_k)}h_{j_k}=\begin{cases}
			t_{j_1}h_{j_k} & \text{if $k=n_i$},\\
			t_{j_{k+1}}h_{j_k} & \text{otherwise},
		\end{cases}
	\]
	then 
	\begin{align*}
	t_{j_1}^{-1}g^{n_i}t_{j_1}
	&=t_{j_1}^{-1}g^{n_i-1}gt_{j_1}\\
	&=t_{j_1}^{-1}g^{n_i-1}t_{j_2}h_{j_1}\\
	&=t_{j_1}^{-1}g^{n_i-2}gt_{j_2}h_{j_1}\\
	&=t_{j_1}^{-1}g^{n_i-2}t_{j_3}h_{j_2}h_{j_1}\\
	&\phantom{=}\vdots\\
	&=t_{j_1}^{-1}gt_{j_{n_i}}h_{n_{i-1}}\cdots h_{j_2}h_{j_1}\\
	&=t_{j_1}^{-1}t_{j_1}h_{j_{n_i}}\cdots h_{j_2}h_{j_1}\in H. 	
	\end{align*}
	Thus $s_i=t_{j_1}\in T$. It only remains to note that $\nu(g)=h_1\cdots h_n$. 
\end{proof}

% \begin{proof}
% 	There exists $\sigma\in\Sym_n$ such that 
% 	\[
% 	\nu(g)=\prod_{i=1}^n \alpha( t_igt_{\sigma(i)}^{-1}). 
% 	\]
% 	Write $\sigma$ as a product of $k$ disjoint cycles
% 	$\sigma=\alpha_1\cdots\alpha_k$, where each $\alpha_j$ is a cycle of length 
% 	$n_j$. For every cycle of the form $(i_1\cdots i_{n_j})$
% 	we reorder the product in such a way that 
% 	\[
% 		\alpha(x_{i_1}gx_{i_2}^{-1})\alpha(x_{i_2}gx_{i_3}^{-1})\cdots \alpha(x_{i_{n_j}}gx_{i_1}^{-1})=\alpha(x_{i_1}g^{n_1}x_{i_1}^{-1}).
% 	\]
% 	There exist $t_1,\dots,t_k\in T$ such that 
% 	$\nu(g)=\prod_{j=1}^k \alpha(t_ig^{n_i}t_i^{-1}$). 
% \end{proof}

Gauss's lemma in number theory gives conditions for an integer to be a quadratic residue. The lemma appears in some proof of the quadratic reciprocity. Gauss's Lemma is basically a computation of the transfer homomorphism. 

\begin{exercise}[Gauss' lemma]
	\index{Gauss'!lemma}
	Let $p$ be a prime number. Let $G=\F_p^\times$ and $H=\{-1,1\}$. 
    \begin{enumerate} 
    \item Prove that the transfer homomorphism 
	\[
		\nu\colon G\to H,\quad
		\nu(x)=x^{\frac{p-1}{2}}=\legendre{x}{p}=\begin{cases}
			1 & \text{if $x$ is a square},\\
			-1 & \text{otherwise}.
		\end{cases}.
	\]
	\item For a transversal $T=\{1,2,\dots,\frac{p-1}{2}\}$ and elements $x\in G$ and $t\in T$, let 
 	\[
	\epsilon(x,t)=\begin{cases}
		1 & \text{if $xt\in T$},\\
		-1 & \text{if $xt\not\in T$}.
	\end{cases}
	\]
	Prove that  
	\[
	\legendre{x}{p}=\prod_{t\in T}\epsilon(x,t).
	\]
    \end{enumerate}
\end{exercise}

\subsection{Other applications of the transfer homomorphism}

\begin{lemma}
	\label{lem:sigma}
	Let $G$ be a group, $H$ be a finite-index subgroup and $n=(G:H)$. 
    Let $S=\{s_1,\dots,s_n\}$ and $T=\{t_1,\dots,t_n\}$ be transversals of $H$ in $G$. 
    For each $g\in G$, there exist unique $h_1,\dots,h_n\in H$ and 
	$\sigma\in\Sym_n$ such that 
	\[
		gt_i=s_{\sigma(i)}h_i,\quad
		i\in\{1,\dots,n\}.
	\]
\end{lemma}

\begin{proof}
	If $i\in\{1,\dots,n\}$, there exists  a unique $j\in\{1,\dots,n\}$ such that 
    $gt_i\in
	s_jH$. Thus $gt_i=s_jh_i$ for a unique $h_i\in H$. Thus we have constructed a 
	$\sigma\colon\{1,\dots,n\}\to\{1,\dots,n\}$, $\sigma(i)=j$.  We need to show that 
	$\sigma\in\Sym_n$. It is enough to prove that $\sigma$ is injective. If
	$\sigma(i)=\sigma(k)=j$, since $gt_i=s_jh_i$ and $gt_k=s_jh_k$, we obtain that 
	$t_i^{-1}t_k=h_i^{-1}h_k\in H$. Hence $i=k$, since $t_iH=t_kH$.
\end{proof}


\begin{theorem}
	\label{thm:P_nonabelian}
	Let $G$ be a finite group and $p$ be a prime number dividing $|[G,G]\cap
	Z(G)|$. If $P\in\Syl_p(G)$, then $P$ is non-abelian. 
\end{theorem}

\begin{proof}
	Assume that $P$ is abelian. Let $T=\{t_1,\dots,t_n\}$ be a transversal of $P$ in $G$. Since 
	$[G,G]\cap Z(G)$ is a normal subgroup of $G$, we may assume that 
	$P\cap [G,G]\cap Z(G)\ne\{1\}$. Let $z\in P\cap [G,G]\cap Z(G)\setminus\{1\}$. 
 
    Let $\nu\colon G\to P$ be the transfer homomorphism. We compute  
    $\nu(z)$ with Lemma~\ref{lem:sigma}. For $i\in\{1,\dots,n\}$, let 
    $x_1,\dots,x_n\in P$ and $\sigma\in\Sym_n$ be such that 
	$zt_i=t_{\sigma(i)}x_i$. Since $z\in Z(G)$, 
	$t_i=t_{\sigma(i)}x_iz^{-1}$. By the uniqueness of Lemma~\ref{lem:sigma}, 
	$\sigma=\id$ and $x_i=z$ for all $i$. Therefore  
	\[
	\nu(z)=z^{|T|}=z^{(G:P)}. 
	\]

	Since $P$ is abelian, $[G,G]\subseteq\ker\nu$. Thus $\nu(z)=1$, a contradiction, since 
    $1\ne z\in P$ and $z^{(G:P)}=1$ implies that $z$ has order not divisible by $p$. 
\end{proof}




Another application:

\begin{proposition}
	\label{pro:center}
	If $G$ is a group such that $Z(G)$ has finite index $n$, then
	$(gh)^n=g^nh^n$ for all $g,h\in G$.	
\end{proposition}

\begin{proof}
	Note that we may assume that $\alpha=\id$, as $Z(G)$ is
	abelian. Let $g\in G$. By Lemma~\ref{lem:transfer} there are positive integers 
    $n_1,\dots,n_k$ such that $n_1+\cdots+n_k=n$ and elements 
	$t_1,\dots,t_k$ of a transversal of $Z(G)$ in $G$ such that 
	\[
		\nu(g)=\prod_{i=1}^k t_ig^{n_1}t_i^{-1}.
	\]
	Since $g^{n_i}\in Z(G)$ for all $i\in\{1,\dots,k\}$ (as $t_ig^{n_i}t_i^{-1}\in Z(G)$), 
	it follows that 
	\[
	\nu(g)=g^{n_1+\cdots+n_k}=g^n.
	\]
	Now Theorem~\ref{thm:transfer} implies the claim.
\end{proof}

The same idea implies the following property:

\begin{exercise}
\label{xca:K_central}
	If $G$ is a group and $K$ is a central subgroup of finite index $n$, then
	$(gh)^n=g^nh^n$ for all $g,h\in G$.	
\end{exercise}

\begin{proposition}
	\label{prop:semidirecto}
	Let $G$ be a finite group and $H$ a central subgroup of index $n$, where 
	$n$ is coprime with $|H|$. Then
	$G\simeq N\rtimes H$.
\end{proposition}

\begin{proof}
	Since $H$ is abelian, $H=H/[H,H]$. Let  
	$\nu\colon G\to H$ be the transfer map and $h\in H$. 
	By Lemma~\ref{lem:transfer}, 
	\[
		\nu(h)
		=\prod_{i=1}^m s_i^{-1}h^{n_i}s_i,
	\]
	where each $s_i^{-1}h^{n_i}s_i\in H$. Since 
	$h^{n_i}\in H\subseteq Z(G)$ for all $i$, it follows that 
	$s_i^{-1}h^{n_i}s_i=h^{n_i}$ for all $i$. Thus 
	\[
		\nu(h)
		=\prod_{i=1}^m s_i^{-1}h^{n_i}s_i
		=\prod_{i=1}^mh^{n_i}
		=h^{\sum_{i=1}^m n_i}=h^n.
	\] 
	The composition $f\colon H\hookrightarrow G\xrightarrow{\nu} H$ is a group homomorphism. 
	We claim that it is an isomorphism. It is injective: If $h^n=1$, then 
	$|h|$ divides both $|H|$ and $n$. Since $n$ and $|H|$ are
	coprime, $h=1$. It is surjectice: Since $n$ and $|H|$ are coprime, there exists 
	$m\in\Z$ such that $nm\equiv 1\bmod |H|$. If $h\in H$, then $h^m\in
	H$ and $\nu(h^m)=h^{nm}=h$. 
	
	Let $N=\ker f$. We claim that $G=N\rtimes H$. 
	By definition, $N$ is normal in $G$ and $N\cap
	H=\{1\}$. To show that $G=NH$ note that 
	$|NH|=|N||H|$ and $G/N\simeq H$.
\end{proof}

\begin{exercise}
	Let $H$ be a central subgroup of a finite group $G$. If $|H|$
	and $|G/H|$ are coprime, then $G\simeq H\times G/H$.
\end{exercise}

%\begin{proof}
%	Es consecuencia inmediata del corolario~\ref{corollary:semidirecto} pues
%	$H$ es normal por ser un subgrupo central.
%\end{proof}

% TODO: Transitivity of the transfer

% serre, 7.12
We now present a nice 
application to infinite groups taken from Serre's book 
\cite[7.12]{MR3469786}. 

\begin{theorem}
	Let $G$ be a torsion-free group that contains a finite-index subgroup isomorphic to  
	$\Z$. Then $G\simeq\Z$.
\end{theorem}

\begin{proof}
	We may assume that $G$ contains a finite-index normal subgroup isomorphic to $\Z$. Indeed, 
	if $H$ is a finite-index subgroup of $G$ such that $H\simeq\Z$, then 
	$K=\cap_{x\in G}xHx^{-1}$ is a non-trivial normal subgroup of $G$ (because $K=\Core_G(H)$ and 
	$G$ has no torsion) and hence $K\simeq\Z$ (because  
	$K\subseteq H$) and $(G:K)=(G:H)(H:K)$ is finite.
	The action of $G$ on $K$ by conjugation induces a group homomorphism  
	$\epsilon\colon G\to\Aut(K)$. Since $\Aut(K)\simeq\Aut(\Z)=\{-1,1\}$, 
	there are two cases to consider.
	
	Assume first that $\epsilon=\id$. Since $K\subseteq Z(G)$, let
	$\nu\colon G\to K$ be the transfer homomorphism. By
	Proposition~\ref{pro:center} (more precisely, 
	by Exercise \ref{xca:K_central}), $\nu(g)=g^n$, where $n=(G:K)$. Since
	$G$ has no torsion, $\nu$ is injective. Thus
	$G\simeq\Z$ because it is isomorphic to a subgroup of $K$.

	Assume now that $\epsilon\ne\id$. Let $N=\ker\epsilon\ne G$. Since
	$K\simeq\Z$ is abelian, $K\subseteq N$. The result proved in the previous paragraph 
	applied to $\epsilon|_N=1$ implies that $N\simeq\Z$, as 
	$N$ contains a finite-index subgroup isomorphic to $\Z$. Let $g\in G\setminus N$. 
	Since $N$ is normal in $G$, $G$ acts by conjugation on $N$ and hence 
	there exists a group homomorphism $c_g\in\Aut(N)\simeq\{-1,1\}$. Since
	$K\subseteq N$ and $g$ acts non-trivially on $K$, 
	\[
	c_g(n)=gng^{-1}=n^{-1}
	\]
	for all $n\in N$.  Since 
	$g^2\in N$, 
	\[
		g^2=gg^2g^{-1}=g^{-2}.
	\]
	Therefore $g^4=1$, a contradiction since $g\ne1$ and $G$ has no torsion.
\end{proof}

\subsection{Dietzman's theorem}

\begin{theorem}[Dietzmann]
	\index{Dietzmann's theorem}
	\label{thm:Dietzmann} 
	Let $G$ be a group and $X\subseteq G$ be a finite subset of $G$ closed by
	conjugation. If there exists $n$ such that $x^n=1$ for all $x\in X$, then
	$\langle X\rangle$ is a finite subgroup of $G$.
\end{theorem}

\begin{proof}
	Let $S=\langle X\rangle$. Since $x^{-1}=x^{n-1}$, every element of $S$ can be 
	written as a finite product of elements of $X$. 
	Fix $x\in X$. We claim that if $x\in X$ appears $k\geq 1$ times 
	in the word $s$, then we can write $s$ as a product of $m$
	elements of $X$, where the first $k$ elements are equal to $x$. Suppose that 
	\[
	s=x_1x_2\cdots x_{t-1}xx_{t+1}\cdots x_m,
	\]
	where $x_j\ne x$ for all $j\in\{1,\dots,t-1\}$. Then 
	\[
		s=x(x^{-1}x_1x)(x^{-1}x_2x)\cdots (x^{-1}x_{t-1}x)x_{t+1}\cdots x_m
	\]
	is a product of $m$ elements of $X$ since $X$ is closed under conjugation and 
	the first element is $x$. The same argument implies that $s$
	can be written as 
	\[
		s=x^ky_{k+1}\cdots y_m,
	\]
	where each $y_j$ belongs to $X\setminus\{x\}$.

	Let $s\in S$ and write $s$ as a product of $m$ elements of 
	$X$, where $m$ is minimal. We need to show that 
	$m\leq (n-1)|X|$. 
	If $m>(n-1)|X|$, 
	then at least one $x\in X$ appears exactly $n$ 
	times in the representation of 
	$s$. Without loss of generality, we write 
	\[
		s=x^nx_{n+1}\cdots x_m=x_{n+1}\cdots x_m,
	\]
	a contradiction to the minimality of $m$. 
\end{proof}

\subsection{Schur's commutator theorem}

\begin{theorem}[Schur]
\index{Schur's!commutator theorem}
\label{thm:Schur}
	Let $G$ be a group. 
	If $Z(G)$ has finite index in $G$, then $[G,G]$ is finite.
\end{theorem}

\begin{proof}
	Let $n=(G:Z(G))$ and  
	$X$ be the set of commutators of $G$. We claim that $X$ is finite, in fact
	$|X|\leq n^2$.
	A routine calculation shows that the map 
	\[
		\varphi\colon X\to G/Z(G)\times G/Z(G),\quad [x,y]\mapsto (xZ(G),yZ(G)),
	\]
	is well-defined. It is, moreover, 
	injective: if $(xZ(G),yZ(G))=(uZ(G),vZ(G))$, then $u^{-1}x\in Z(G)$, 
	$v^{-1}y\in Z(G)$. Thus 
	\begin{align*}
		[u,v]&=uvu^{-1}v^{-1}=uv(u^{-1}x)x^{-1}v^{-1}=xvx^{-1}(v^{-1}y)y^{-1}=xyx^{-1}y^{-1}=[x,y].
	\end{align*}
	Moreover, $X$ is closed under conjugation, as 
	\[
		g[x,y]g^{-1}=[gxg^{-1},gyg^{-1}]
	\]
	for all $g,x,y\in G$. Since $G\to Z(G)$, $g\mapsto g^n$ is a group
	homomorphism, Proposition~\ref{pro:center} implies that $[x,y]^n=[x^n,y^n]=1$ for
	all $[x,y]\in X$.  The theorem follows from applying Dietzmann's theorem. 
\end{proof}

\begin{exercise}
    Let $G$ be the group with generators $a,b,c$ and 
    relations $ab=ca$, $ac=ba$ and $bc=ab$. Prove the following statements:
    \begin{enumerate}
        \item $G$ is infinite and non-abelian.
        \item $Z(G)$ has finite index in $G$ and every conjugacy class of $G$ is finite.
        \item $[G,G]$ is finite. 
        \item The subgroup $N=\langle a^2\rangle$ of $G$ 
        generated by $a^2$ is central 
        and $G/N$ is finite.
    \end{enumerate}
\end{exercise}

We conclude the section with some results similar to that of Schur. 

\begin{theorem}[Niroomand]
\index{Niroomand's theorem}
\label{thm:Niroomand}
	If the set of commutators of a group $G$ is finite, then 
	$[G,G]$ is finite.
\end{theorem}

\begin{proof}
 	Let $C=\{[x_1,y_1],\dots,[x_k,y_k]\}$ be the (finite) set of commutators of $G$ and  
	\[
    H=\langle x_1,x_2,\dots,x_k,y_1,y_2,\dots,y_k\rangle.
    \]
    Since $C$ is a set of commutators of $H$, 
	it follows that 
	$[G,G]=\langle C\rangle\subseteq [H,H]$. To simplify the notation we write 
	$H=\langle h_1,\dots,h_{2k}\rangle$. 	
 	Since $h\in Z(H)$ if and only if $h\in C_H(h_i)$ for all 
	$i\in\{1,\dots,2k\}$, we conclude that $Z(H)=C_H(h_1)\cap\cdots\cap C_H(h_{2k})$. Moreover, if 
	$h\in H$, then $hh_ih^{-1}=ch_i$ for some $c\in C$. Thus the conjugacy class of each 
	$h_i$ contains at most as many elements as $C$. This implies that 
	\[
		|H/Z(H)|=|H/\cap_{i=1}^{2k} C_H(h_i)|\leq\prod_{i=1}^{2k} (H:C_H(h_i))\leq |C|^{2k}.
	\]
	Since $H/Z(H)$ is finite, $[H,H]$ is finite. Hence  
	$[G,G]=\langle C\rangle\subseteq [H,H]$ 
	is a finite group. 
\end{proof}

\begin{theorem}[Hilton--Niroomand]
	\index{Hilton--Niroomand theorem}
	\label{thm:HiltonNiroomand}
	Let $G$ be a finitely generated group. If $[G,G]$ is finite and $G/Z(G)$ is generated by
	$n$ elements, then  
	\[
	|G/Z(G)|\leq |[G,G]|^n. 
	\]
\end{theorem}

\begin{proof}
	Assume that $G/Z(G)=\langle x_1Z(G),\dots,x_nZ(G)\rangle$. Let 
	\[
		f\colon G/Z(G)\to [G,G]\times\cdots\times [G,G],
		\quad
		y\mapsto ([x_1,y],\dots,[x_n,y]).
	\]
	Note that $f$ is well-defined: If $y\in G$ and $z\in Z(G)$, then $[x_i,y]=[x_i,yz]$ for all $i$. 
	Then $f(yz)=f(y)$.
	 
	The map $f$ is injective. Assume that $f(xZ(G))=f(yZ(G))$. Then 
	$[x_i,x]=[x_i,y]$ for all $i\in\{1,\dots,n\}$. For each $i$ we compute  
	\begin{align*}
		[x^{-1}y,x_i] &= x^{-1}[y,x_i]x[x^{-1},x_i]\\
		&=x^{-1}[y,x_i][x_i,x]x=x^{-1}[x_i,y]^{-1}[x_i,x]x=x^{-1}[x_i,y]^{-1}[x_i,y]x=1.
	\end{align*}
	This implies that $x^{-1}y\in Z(G)$. Indeed, since  
	every $g\in G$ can be written as $g=x_kz$ for some $k\in\{1,\dots,n\}$ and some $z\in Z(G)$, 
	it follows that 
    \[
    [x^{-1}y,g]=[x^{-1}y,x_kz]=[x^{-1}y,x_k]=1.
    \]
    Since $f$ is injective, 
	$|G/Z(G)|\leq |[G,G]|^n$. 
\end{proof}

\begin{exercise}
    Prove Theorem~\ref{thm:HiltonNiroomand} from Theorem~\ref{thm:Niroomand}. 
\end{exercise}


