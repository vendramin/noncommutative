\subsection{Project: How many finite groups are there?}

In \cite{MR2410121}, the function $\operatorname{gnu}(n)$ is defined, which returns the number of isomorphism classes of groups of order $n$. For example, \[
\operatorname{gnu}(1)=\operatorname{gnu}(2)=\operatorname{gnu}(3)=\operatorname{gnu}(5)=1\text{ and }\operatorname{gnu}(4)=\operatorname{gnu}(6)=2.
\]
The name comes from \emph{\emph{g}roups \emph{nu}mber}.

Burnside's theorem can be reformulated as follows:
\[
\operatorname{gnu}(n)=1 \Longleftrightarrow \text{$n$ is cyclic} \Longleftrightarrow \gcd(n,\phi(n))=1.
\]

In \cite{MR2410121}, Conway, Dietrich, and O'Brien characterized the $n\in\mathbb{N}$ such that $\operatorname{gnu}(n)=2$, $\operatorname{gnu}(n)=3$, $\operatorname{gnu}(n)=4$.

In the Encyclopedia of Integer Sequences, the sequence
\[
\operatorname{gnu}(1),\operatorname{gnu}(2),\operatorname{gnu}(3),\operatorname{gnu}(4)\dots
\]
is \lstinline{A000001}, see \url{http://oeis.org/A000001} for more information.

There exists a powerful database of small groups. It was 
written by Besche, Eick and O'Brien and is now a fundamental tool in group theory \cite{MR1935567}. In particular, this database allows us to easily compute some values of the function $\operatorname{gnu}$.

The function \lstinline{NrSmallGroups} returns the number of isomorphism classes of groups of a certain order. We then define the function $\operatorname{gnu}$ and compute some examples:

\begin{lstlisting}
gap> gnu := NrSmallGroups;;
gap> gnu(16);
14
gap> gnu(32);
51
gap> gnu(64);
267
gap> gnu(27);
5
gap> gnu(81);
15
gap> gnu(128);
2328
gap> gnu(512);
10494213
\end{lstlisting}

It is known that $\operatorname{gnu}(1024)=49487365422$, although this value cannot be obtained directly with the computer, as there are way too many 
groups of order 1024. The groups of order 1024 were classified by Besche, Eick, and O'Brien, and the announcement was made in \cite{MR1826989}.

More than 99\% of the groups of order $<2000$ are of order $1024$. In fact, as we saw, there are 49487365422 groups of order 1024, and the number of isomorphism classes of groups of order $n\ne1024$ with $n<2016$ is 423164131.

\begin{lstlisting}
gap> Sum([1..1023], gnu);
423164131
\end{lstlisting}

The classification of groups is a difficult problem. A particularly difficult case is that of groups of order $p^3$, where $p$ is a prime number. For example, there are exactly 2 groups of order $2^3$ and 5 groups of order $3^3$. For general $p$, the question is open. For example, there are 14 groups of order $2^4$, 51 groups of order $2^5$, and 267 groups of order $2^6$. The sequences are found in the Encyclopedia of Integer Sequences, sequences \lstinline{A000679}, \lstinline{A000880}, and \lstinline{A000881}.

% Finally, we mention some references on the subject. The classic book on the classification of finite simple groups is \cite{MR1915964}. For groups of small order, see \cite{MR1250465} and \cite{MR1357169}.

\begin{lstlisting}[language=gap]
gap> Sum([1..1023], gnu)+Sum(List([1025..2015], gnu));
423164131
\end{lstlisting}

These numbers give us approximately 99.15\%.

These observations naturally suggest the following conjecture, which seems to be part of mathematical folklore:

\begin{conjecture}
Almost every finite group is a $2$-group.
\end{conjecture}

The numerology we did allows us to avoid having to make precise what \textquotedblleft almost every finite group\textquotedblright\ means. Similar problems appear in Chapter 22 of the book \cite{MR2382539}.

\begin{problem}
Calculate $\operatorname{gnu}(2048)$.
\end{problem}

It is known that $\operatorname{gnu}(2048)>1774274116992170$, which is the number of subgroups of order 2048 of a certain class.

\begin{conjecture}
\label{conjecture:gnu}x
Let $n\geq1$. The sequence
\[
\operatorname{gnu}(n),\operatorname{gnu}^2(n),\operatorname{gnu}^3(n)\dots
\]
stabilizes at 1.
\end{conjecture}

In Conjecture \ref{conjecture:gnu}, 
$\operatorname{gnu}^{1}(n)=\operatorname{gnu}(n)$ and 
$\operatorname{gnu}^{k+1}(n)=\operatorname{gnu}(\operatorname{gnu}^k(n))$ for $k\geq1$. 

It is not difficult to verify that the conjecture is true for $n<2000$.

The following conjecture appeared independently in several places. Apparently, the first more or less explicit appearance was around 1930 and was due to Miller. Independently, MacHale formulated it forty years later.

\begin{conjecture}
The function $\Z_{\geq1}\to\Z_{\geq1}$, 
$n\mapsto\operatorname{gnu}(n)$, is surjective.
\end{conjecture}

For more information about this conjecture, we refer to \cite[\S21.6]{MR2382539}.

Although there are many conjectures about the behavior of the function that counts the number of isomorphism classes of finite groups, there are several results. The following one is completely elementary:

\begin{theorem}
If $n\geq1$, then $\operatorname{gnu}(n)\leq n^{n\log_2 n}$.
\end{theorem}

\begin{proof}
If $G$ is a group, let
\[
d(G)=\min\{k:\text{there exist }g_1,\dots,g_k\in G\text{ such that }G=\langle g_1,\dots,g_k\rangle\}.
\]
We are going to prove that if $|G|=n$, then $d(G)\leq\log_2 n$. Let
\[
\{1\}=G_0\subsetneq G_1\subsetneq\cdots\subsetneq G_r=G
\]
be a maximal sequence of subgroups. For each $i\in\{1,\dots,r\}$ let
$g_i\in G_i\setminus G_{i-1}$. It is easily shown
that
\[
G_i=\langle g_1,\dots,g_i\rangle
\]
for all $i$. 
Indeed, if there exists some $i$ such that $G_i\ne\langle g_1,\dots,g_i\rangle$, then
there exists $g\in G_i\setminus\langle g_1,\dots,g_i\rangle$ and then
\[
\langle g_1,\dots,g_i\rangle\subsetneq \langle G_i,g\rangle\subsetneq G_{i+1}
\]
which contradicts the maximality of the sequence of subgroups. In particular, $G$ is generated 
by $r$ elements.

By Lagrange's theorem, 
\[
n=|G|=\prod_{i=1}^r(G_i:G_{i-1})\geq 2^r
\]
and then $r\leq\lfloor \log_2 n\rfloor$. 
As $G$ is isomorphic to a 
subgroup of $\Sym_n$ by Cayley's theorem, then 
\begin{align*}
\operatorname{gnu}(n)
&\leq\text{number of subgroups of order $n$ of $\Sym_n$}\\
&\leq\text{number of subgroups of $\Sym_n$ generated by $\lfloor\log_2n\rfloor$ elements}\\
&\leq\text{number of subsets of $\Sym_n$ of $\lfloor\log_2n\rfloor$ elements}.
\end{align*}
Since the number of subsets of $\Sym_n$ of $\lfloor\log_2n\rfloor$ elements
is
\[
\binom{n!}{\lfloor\log_2n\rfloor}\leq(n!)^{\log_2n}
\]
since $\binom{n}{k}\leq n^k$, we conclude that $\operatorname{gnu}(n)\leq n^{n\log_2n}$.
\end{proof}
% The proof is simple and elementary and can be found in the second
% chapter of the book \cite{MR2382539}.

In the case of $p$-groups, it can be shown by elementary methods 
that
\[
\operatorname{gnu}(p^n)\leq p^{\frac16(n^3-n)},
\]
see \cite[Theorem 5.1]{MR2382539}. Much more sophisticated bounds are known:

\begin{theorem}[Higman--Sims]
\index{Higman--Sims Theorem}
If $p$ is a prime number and $n\geq1$, then
\[
p^{\frac{2}{27}n^3-O(n^2)}\leq\operatorname{gnu}(p^n)\leq p^{\frac{2}{27}n^3+O(n^{\frac{8}{3}})}.
\]
\end{theorem}

The proof of the above theorem results from combining the works of Higman \cite{MR113948}
and Sims \cite{MR169921}. A modern presentation can be found in the book
\cite{MR2382539}.

In \cite{MR123605}, Higman conjectured that $\operatorname{gnu}(p^n)$ is a polynomial function of $p$ and $p$ modulo $N$ for a certain
finite number of integers $N$. 

\begin{conjecture}[Higman]
\index{Higman's PORC conjecture}
Let $n\geq1$. Then there exist $N$ polynomials
\[
P_{0}(X),P_{1}(X),\dots,P_{N-1}(X)
\]
such that
if $p\equiv i\bmod N$, then $\operatorname{gnu}(p^n)=P_{i}(p)$.
\end{conjecture}

PORC stands for \emph{P}olynomial \emph{O}n \emph{R}esidue \emph{C}lasses.

It is known that the conjecture is true for $n\leq7$, although the problem remains open for $n\geq8$.
In \cite{MR2921623}, a work of over seventy pages,
du Sautoy and Vaughan--Lee constructed a family of groups
of order $p^{10}$ suggesting that the PORC conjecture might not be true. Nevertheless, Higman's PORC conjecture remains open.

\begin{theorem}[Pyber]
\index{Pyber's Theorem}
If $n\geq1$, then
\[
\operatorname{gnu}(n)\leq n^{\frac{2}{27}\mu(n)^2+O\left(\mu(n)^{\frac{5}{3}}\right)},
\]
where $\mu(n)$ is the largest exponent
appearing in the prime factorization of $n$.
\end{theorem}

The proof appears in \cite{MR1200081} and in the case of non-solvable groups uses the classification of simple groups. A detailed presentation can be found in the book
\cite{MR2382539} by Blackburn, Neumann, and Venkataraman.
