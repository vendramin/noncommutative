\subsection{Project: Mann subgroup}

Let $G$ be a finite group with conjugacy classes sizes 
\[
1=n_1<n_2<\cdots<n_k.
\]
For example, for $G=\Sym_3$ one has $k=3$ and $(n_1,n_2,n_3)=(1,2,3)$. How the number $k$ of class sizes affect the structure of the group?

\begin{problem}
    What is the connection between the conjugacy classes sizes and the nilpotency of a group? 
\end{problem}

For example, if $k=1$, then $G$ is abelian. 
It\^o proved that if $k=2$, then $G$ is nilpotent; see \cite{MR61597}. 
And Ishikawa proved that the nilpotency class of $G$ is at most three. 

\begin{definition}
    \index{Mann subgroup}
    Let $G$ be a finite group. The \emph{Mann subgroup} $M(G)$ is defined as the subgroup
    of $G$ generated by all elements lying in conjugacy classes of size $\leq n_2$.
\end{definition}

Ishikawa's theorem
follows from the following theorem. 

\begin{theorem}[Mann]
\label{thm:Mann}
\index{Mann's theorem}
Let $G$ be a nilpotent finite group. Then $M(G)$ has nilpotency class $\leq3$.
\end{theorem}

\begin{proof}
    See \cite[Theorem 4.14]{MR2426855}.
\end{proof}
