\subsection{Project: Nilpotent groups of class two}

The following exercises go over groups 
of nilpotency class two. 

\begin{exercise}
    \label{xca:class2}
    Let $G$ be a finite group. Prove that the following statements are equivalent:
    \begin{enumerate}
        \item $G$ is nilpotent of class $\leq2$.
        \item $[G,G]\subseteq Z(G)$. 
        \item Any triple commutator in $G$ is trivial. 
        \item The inner automorphism group of $G$ is abelian. 
    \end{enumerate}
\end{exercise}

\begin{exercise}
    Let $G$ be a nilpotent group of class two and $g\in G$. Prove 
    that the map $G\to G$, $x\mapsto [g,x]$, is a group homomorphism. 
\end{exercise}

\begin{exercise}
\label{xca:commutador}
Let $G$ be a group. Prove that 
if $x,y\in G$ are such that $[x,y]\in C_G(x)\cap C_G(y)$, then 
\[
[x,y]^n=[x^n,y]=[x,y^n]
\]
for all $n\geq0$.
\end{exercise}

% \begin{proof}
% 	Procederemos por inducción en $n\geq0$. El caso $n=0$ es trivial. Supongamos entonces
% 	que el resultado vale para algún $n\geq0$. Entonces, como $[x,y]\in C_G(x)$, 
% 	\begin{align*}
% 		[x,y]^{n+1}&=[x,y]^n[x,y]
% 		=[x^n,y][x,y]=[x^n,y]xyx^{-1}y^{-1}=x[x^n,y]yx^{-1}y^{-1}=[x^{n+1},y].
% 	\end{align*}
% 	Para demostrar el lema en el caso $n<0$ basta observar que, como $[x,y]^{-1}=[x^{-1},y]$, 
% 	$[x,y]^{-n}=[x^{-1},y]^n=[x^{-n},y]$.
% \end{proof}

\begin{exercise}[Hall]
\label{xca:Hall}
Let $G$ be a class-two nilpotent group and 
$x,y\in G$. Prove that 
\[
(xy)^n=[y,x]^{n(n-1)/2}x^ny^n
\]
for all $n\geq1$.
\end{exercise}

% \begin{proof}
% 	Procederemos por inducción en $n$. Como el caso $n=1$ es trivial,
% 	supongamos que el resultado es válido para algún $n\geq1$. Entonces,
% 	gracias al lema anterior, 
% 	\begin{align*}
% 		(xy)^{n+1} &= (xy)^n(xy)=[y,x]^{n(n-1)/2}x^ny^{n-1}(yx)y\\
% 		&=[y,x]^{n(n-1)/2}x^{n}[y^n,x]xy^{n+1}=[y,x]^{n(n-1)/2}[y,x]^nx^{n+1}y^{n+1}.\qedhere 
% 	\end{align*}
% \end{proof}

\begin{exercise}
\label{xca:class2_homomorphism}
Let $p$ be an odd prime number and 
$P$ a $p$-group of nilpotency class $\leq2$. 
Prove that if $[y,x]^p=1$ for all $x,y\in P$, then
$P\to P$,
$x\mapsto x^p$, is a group homomorphism. 
\end{exercise}

% \begin{proof}
% 	Por lema de Hall,
% 	$(xy)^p=[y,x]^{p(p-1)/2}x^py^p=x^py^p$. 	
% \end{proof}

\begin{exercise}
\label{xca:class2_torsion}
Let $p$ be an odd prime number and 
$P$ a $p$-group of nilpotency class $\leq2$. 
Prove that $\{x\in P:x^p=1\}$ is a subgroup of $P$.
\end{exercise}

% \begin{proof}
% 	Como $P$ tiene clase de nilpotencia dos, los conmutadores son centrales.
% 	Para cada $x\in G$, la función $g\mapsto [g,x]$ es un morfismo de grupos
% 	pues
% 	\[
% 		[gh,x]=ghxh^{-1}g^{-1}x^{-1}=g[h,x]xg^{-1}x^{-1}=[g,x][h,x].
% 	\]
% 	En particular, si $x,y\in P$ con $x^p=y^p=1$, entonces
% 	\[
% 		[x,y]^p=[x^p,y]=[1,y]=1.
% 	\]
% 	Luego, al usar el lema de Hall, se concluye que
% 	$(xy)^p=[y,x]^{p(p-1)/2}x^py^p=1$.
% \end{proof}
