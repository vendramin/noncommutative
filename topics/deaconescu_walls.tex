\section{Project: The Deaconescu--Walls theorem}

Let $A$ be a group acting on automorphisms on a finite group $G$. Then 
$C_{G}(A)=\{g\in G:a\cdot g=g\,\,\forall a\in A\}$ acts by left multiplication 
on the set of 
$A$-orbits by 
\[
  c(A\cdot g)
  =\{c(a\cdot g):a\in A\}
  =\{(a\cdot c)(a\cdot g):a\in A\}
  =\{a\cdot (cg):a\in A\}
  =A\cdot (cg)
\]
for all $g\in G$ and $c\in C_G(A)$.

%The following theorem first appeared in~\cite{MR2164558}. 
%The proof presented here goes back to Isaacs, see~\cite{MR2922681}. 

\begin{theorem}[Deaconescu--Walls]
	\index{Deaconescu--Walls theorem}
	\label{thm:DeaconescuWalls}
	Let $A$ be a group acting by automorphisms on a finite group $G$. Let
	$C=C_{G}(A)$ and $N=C\cap [A,G]$,
	where $[A,G]$ is the subgroup of $G$ generated by $[a,g]=(a\cdot g)g^{-1}$,
	$a\in A$, $g\in G$.  Then $(C:N)$ divides the number of $A$-orbits of 
	$G$. 
\end{theorem}

\begin{proof}
  The group $C$ acts by left multiplication on the set $\Omega$ of 
  $A$-orbits of $G$. Let $X=A\cdot g\in\Omega$ and $C_X$ be the stabilizer of 
  $C$ in $X$. If $c\in C_X$, then $cX=X$. In particular, if $c\in C_X$, then 
  $cg=a\cdot g$ for some $a\in A$, that is $c=(a\cdot
  g)g^{-1}=[a,g]\in [A,G]$. Thus $C_X\subseteq N$.

  To show that $(C:N)$ divides $|\Omega|$, it is enough to show that 
  $(C:N)$ divides the size of each $C$-orbit. If $X\in\Omega$, then $C\cdot
  X$ has size 
  \[
	(C:C_X)=(C:N)(N:C_X).
  \]
  Hence $(C:N)$ divides the size of the orbit $C\cdot X$.
\end{proof}

\begin{corollary}
	\label{cor:Z(G)subset[G,G]}
  Let $G$ be a non-trivial finite group with $k(G)$ conjugacy classes. 
  If the order of $Z(G)$ is coprime with $k(G)$, then  
  $Z(G)\subseteq[G,G]$.
\end{corollary}

\begin{proof}
	The group $A=G$ acts on $G$ by conjugation. Since $C_G(A)=Z(G)$ and 
	$[A,G]=[G,G]$, Theorem~\ref{thm:DeaconescuWalls} implies that the index 
	$(Z(G):Z(G)\cap [G,G])$ divides $k(G)$. Since $k(G)$ and $|Z(G)|$ are coprime, we conclude that $Z(G)=Z(G)\cap [G,G]\subseteq [G,G]$. 
\end{proof}

\begin{definition}
	\index{Central automorphism}
 	Let $G$ be a group and $f\in\Aut(G)$. We say that $f$ is \emph{central} if 
	$f(x)x^{-1}\in Z(G)$ for all $x\in G$.
\end{definition}

An automorphism $f$ of a group $G$ 
is central if and only if $f\in C_{\Aut(G)}(\Inn(G))$.

\begin{corollary}
	Let $G$ be a finite group with $k(G)$ conjugacy classes and $c(G)$
	central automorphisms. If $\gcd(|G|,k(G)c(G))=1$, then 
	$[G,G]=Z(G)$.
\end{corollary}

\begin{proof}
	By Corollary~\ref{cor:Z(G)subset[G,G]}, $Z(G)\subseteq [G,G]$. Conversely, let 
	$A=C_{\Aut(G)}(\Inn(G))$. Since $|G|$ and $k(G)c(G)$ are coprime 
	and $(C_G(A):C_G(A)\cap [A,G])$ divides $c(G)$ by 
	Theorem~\ref{thm:DeaconescuWalls}, we obtain that $C_G(A)=C_G(A)\cap [A,G]$. 
	Since 
	\[
		a\cdot [x,y]=[(a\cdot x)x^{-1}x,(a\cdot y)y^{-1}y]=[x,y]
	\]
	for all $a\in A$ and $x,y\in G$, 
    $[G,G]\subseteq C_G(A)$. Moreover, 
    $[A,G]\subseteq Z(G)$. Thus 
	\[
	[G,G]\subseteq C_G(A)=C_G(A)\cap [A,G]\subseteq [A,G]\subseteq Z(G).\qedhere 
	\]
\end{proof}

\begin{exercise}
    Let $p$ be a prime number and $G$ be a group with $p$ conjugacy classes. 
    Prove that either $Z(G)\subseteq[G,G]$ or $|G|=p$. 
\end{exercise}

% \begin{proof}
%   Hacemos actuar a $G$ en $G$ por conjugación.  Como cada elemento de $C=Z(G)$
%   es una clase de conjugación, $|C|\leq p$. Si $|C|=p$ entonces $G=C=Z(G)$
%   tiene orden $p$. Si no, $|C|$ es coprimo con $p$ y luego $C\subseteq
%   N=[G,G]$.
% \end{proof}

