\subsection{*The Alperin--Kuo theorem}

% We first start stating a general result known as the transitively of the transfer map. 

% \begin{theorem}
%     Let $G$ be a group and $H\subseteq K$ be subgroups of $G$ 
%     $(G:H)$ is finite. If $U\colon G\to K$, $W\colon K\to H$ and 
%     $V\colon G\to H$ are pretransfer maps, 
%     then $V(g)=W(U(g))\bmod [H,H]$ for all $g\in G$. 
% \end{theorem}

% \begin{proof}
%     See \cite[Theorem 10.8]{MR2426855}.
% \end{proof}



\begin{theorem}[Alperin--Kuo]
\index{Alperin--Kuo theorem}
\label{thm:AlperinKuo}
Let $G$ be a finite group and $A=[G,G]\cap Z(G)$. Then $g^{(G:A)}=1$ for all $g\in G$. 
\end{theorem}

\index{Transitivity of the transfer}
One way to prove Theorem~\ref{thm:AlperinKuo} uses non-trivial properties of 
the transfer map. More precisely, the proof of the Alperin--Kuo theorem 
combines the transitivity of the transfer (see \cite[Theorem 10.8]{MR2426855}) 
with the following theorem:

\begin{theorem}[Furtwr\"angler]
\index{Furtwr\"angler's theorem}
\label{thm:Furtwrangler}
    Let $G$ be a finite group. Then the transfer homomorphism 
    $G\to G^{(1)}/G^{(2)}$ is the trivial map. 
\end{theorem}

\begin{proof}
See \cite[Theorem 10.18]{MR2426855} for a ring-theoretical proof. 
\end{proof}