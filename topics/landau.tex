\section{Project: Landau's Theorem}

In 1903, Landau demonstrated that there exists only a finite number of groups with finite conjugacy classes. The proof is entirely elementary and is based on the following lemma:

\begin{lemma}[Landau]
  \label{lem:Landau}
  \index{Landau's!lemma}
  For each $k \in \mathbb{N}$, the equation 
  \[
	\frac{1}{n_1}+\cdots+\frac{1}{n_k}=1
  \]
  has only finitely many solutions.
\end{lemma}

\begin{proof}
  Suppose $0 < n_1 \leq n_2 \leq \cdots \leq n_k$. Then $n_1 \leq k$.
  We prove by induction that 
  \[
    n_j \leq \frac{k+1-j}{1-\left(\frac{1}{n_1}+\cdots+\frac{1}{n_{j-1}}\right)}
  \]
  for all $j \in \{2,\dots,k\}$. Since for each $j \in \{2,\dots,k\}$, $n_j \leq n_2$, then $1 \leq \frac{1}{n_1}+\frac{k-1}{n_2}$ and hence $n_2 \leq \frac{k-1}{1-\frac{1}{n_1}}$. If we assume that the result holds for $j \geq 2$, say $n_p \geq n_j$ for all $p \geq j$, then
  \[
	1 \leq \sum_{i=1}^{j-1}\frac{1}{n_i}+\frac{k-j+1}{n_j},
  \]
  which implies the inequality we wanted to prove.
\end{proof}

\begin{theorem}[Landau]
\index{Landau's!theorem}
  Let $k\geq1$. There exists only a finite number of finite groups that have exactly $k$ conjugacy classes.
\end{theorem}

\begin{proof}
  Let $G$ be a group with $k$ conjugacy classes, say $C_1,\dots,C_k$, and let $1=g_1,\dots,g_k$ be representatives of these classes. When we decompose $G$ as $G=C_1\cup\cdots\cup C_k$, we have 
  \[
    |G|=|C_1|+\cdots+|C_k|=(G:C_G(g_1))+\cdots(G:C_G(g_k)).
  \]
  For each $j \in \{1,\dots,k\}$, let $n_j=|C_G(g_j)|$. Then 
  \[
	1=\frac{1}{n_1}+\cdots+\frac{1}{n_k}.
  \]
  As we saw in Lemma~\ref{lem:Landau}, this equation has only finitely many solutions. In particular, $n_k=|G|$ is bounded by a function of $k$.
\end{proof}

Landau's method allows tackling certain classification results. Let us 
see some examples.

\begin{example}
  Let $G$ be a finite group that has two conjugacy classes. Since $G\setminus\{1\}$ is a conjugacy class, $|G|-1$ divides $|G|$, and thus $|G|=2$.
\end{example}

\begin{example}
  Let $G$ be a finite non-abelian group with three conjugacy classes. The solutions of the equation $1/n_1+1/n_2+1/n_3=1$ with $n_1\leq n_2\leq n_3$ are $(3,3,3)$, $(2,3,6)$, and $(2,4,4)$. The only possibility is $(2,3,6)$. Then $G\simeq\Sym_3$.
\end{example}

Now let us see a bound that can be easily obtained from Landau's method.

\begin{theorem}[Neumann]
\index{Neumann's!theorem}
If $G$ is a finite group of order $n$ with $k$ conjugacy classes, then
\[
k\geq\frac{\log\log n}{\log 4}.
\]
\end{theorem}

\begin{proof}
We proceed as we did in the proof of Landau's theorem. Let $C_1,\dots,C_k$ be the conjugacy classes, and let $1=g_1,\dots,g_k$ be representatives of these classes. When we decompose $G$ as a disjoint union of conjugacy classes, we have 
\[
n=|G|=|C_1|+\cdots+|C_k|=(G:C_G(g_1))+\cdots(G:C_G(g_k)).
\]
For each $j \in \{1,\dots,k\}$, let $n_j=|C_G(g_j)|$. Then 
\[
	1=\frac{1}{n_1}+\cdots+\frac{1}{n_k}.
\]

We claim that 
\[
\max_{1\leq i\leq k}n_i\leq k^{2^{k-1}}.
\]
Without loss of generality, we can assume that $n_1\leq n_2\leq\cdots\leq n_k$. Then $n_1\leq k$, because otherwise, 
\[
\sum_{i=1}^k\frac{1}{n_i}<\sum_{i=1}^k\frac{1}{k}=1,
\]
a contradiction. Let $r\in\{1,\dots,k-1\}$. We write
\[
\sum_{i=r+1}^k\frac{1}{n_i}=1-\sum_{i=1}^r\frac{1}{n_i}=\frac{x}{n_1\cdots n_r}
\]
for some positive integer $x$. Then
\[
\frac{k-r}{n_{r+1}}\geq\frac{1}{n_1\cdots n_r}
\]
and hence $n_{r+1}\leq (k-r)n_1\cdots n_r<kn_1\cdots n_r$.

To complete the proof, we need to prove that 
\begin{equation}
    \label{eq:Neumann}
    n_r\leq k^{2^{r-1}}
\end{equation}
for all $r$. We proceed by induction on $r$. The case $r=1$ is trivial. Suppose then that the result is valid
for all $j\leq r$. By the inductive hypothesis, 
\[
n_{r+1}\leq kn_1\cdots n_r\leq k\prod_{j=1}^k2^{2^{j-1}}=k^{2^k}.\qedhere
\]
\end{proof}

%Landau's method opens the door to tackling certain classification results. Let's %see some examples.

\begin{problem}[Brauer]
\index{Brauer problem}
Find good bounds for the order $n$ of a group with a fixed number $k$ of conjugacy classes. It is expected that the bounds will be considerably better than those obtained from Landau's method.
\end{problem}
