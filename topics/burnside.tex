\section{Project: Burnside's cyclic numbers theorem}

We now mention some problems and results related to the number of isomorphism classes of finite groups of a given order. This classification problem is obviously almost as old as group theory itself. When taking the first steps in group theory, we encounter some easy-to-prove results:
\begin{itemize}
    \item There exists a unique finite group of prime order, and it is cyclic.
    \item There are two groups of order four, both abelian.
    \item There are two groups of order six, one of which is non-abelian.
    \item Groups of order $p^2$ are abelian.
\end{itemize}

With the Sylow theorems, we can go a bit further. It is easy to prove, for example, that there exists a unique group of order 15, and it is cyclic. The same result can be shown for other orders, such as 455 and 615.

A natural question arises. For which values of $n$ does a unique group (which will obviously be isomorphic to $C_n$) of order $n$ exist? The answer was given by Burnside.

\begin{definition}
\index{Cyclic number}
A number $n\in\mathbb{N}$ is called \textbf{cyclic} if $C_n$ is the only group (up to isomorphism) of order $n$.
\end{definition}

Some examples of cyclic numbers are: $2$, $3$, $15$, and $615=3\cdot 5\cdot 41$.

\begin{theorem}[Burnside]
\index{Burnside's cyclic number theorem}
    Let $n\in\mathbb{N}$. Then $n$ is cyclic if and only if $n$ and $\phi(n)$ are coprime.
\end{theorem}

\begin{proof}
    Suppose $n$ is cyclic. Without loss of generality, we can assume that $n$ is square-free (because otherwise, if $n=p^am$ with $m\in\mathbb{N}$, $p$ prime such that $\gcd(p,m)=1$, and $a\geq2$, the group $C_m\times C_p^a$ has order $n$ and is not cyclic). We then write
    \[
        n=p_1\cdots p_k
    \]
    with the $p_j$ distinct primes and $\phi(n)=(p_1-1)\cdots(p_k-1)$. If $\gcd(n,\phi(n))\ne1$, there exist distinct primes $p$ and $q$ such that $p$ divides $q-1$. The group $G= C_m\times (C_p\rtimes C_q)$ has order $n=pqm$ and is not cyclic.

    Suppose $\gcd(n,\phi(n))=1$ and $n$ is not cyclic. Let $G$ be a group of minimum order $n$ that is not cyclic.
    Without loss of generality, we can assume that $n$ is square-free: if $n=p^\alpha m$ with $p$ prime, $m$ coprime with $p$ and $\alpha\geq2$, then, as $\phi(n)=p^{\alpha-1}(p-1)\phi(m)$, $p$ divides 
    $\gcd(n,\phi(n))$. Then
    \[
        n=p_1\cdots p_k,
    \]
    with the $p_j$ distinct primes.

    \begin{claim}
    Every subgroup of $G$ and every quotient of $G$ is cyclic.
    \end{claim}

    If $m$ divides $n$, then $\gcd(m,\phi(m))=1$ since $n$ and $\phi(n)=(p_1-1)\cdots(p_k-1)$ are coprime. Therefore, every subgroup and every proper quotient is cyclic by the minimality of $n$.

    \begin{claim}
    $Z(G)=\{1\}$.
    \end{claim}

    For each $i\in\{1,\dots,k\}$, let $x_i\in G$ be an element of order $p_i$. If $G$ were abelian, then $G$ would be cyclic: $x_1\cdots x_k$ would be an element of order $n$. Hence $Z(G)\ne G$. Now, if $1<|Z(G)|<n$, then $G/Z(G)$ would be cyclic (since every quotient of $G$ is), and then $G$ would be abelian.

    \begin{claim}
    If $M$ is a maximal subgroup of $G$ and $x\in M\setminus\{1\}$, then $M=C_G(x)$. In particular, if $M$ and $N$ are distinct maximal subgroups, then $M\cap N=\{1\}$.
    \end{claim}

    Since $Z(G)\ne\{1\}$, $C_G(x)\ne G$. And since $M$ is cyclic, $M\subseteq C_G(x)$. Therefore, by maximality, $M=C_G(x)$. If $M$ and $N$ are two maximal subgroups and $x\in M\cap N\setminus\{1\}$, then $M=N=C_G(x)$.

    \begin{claim}
    If $M$ is a maximal subgroup, then $M=N_G(M)$.
    \end{claim}

    Let $x\in N_G(M)\setminus\{1\}$ and let $\alpha\in\Aut(M)$ be given by $y\mapsto xyx^{-1}$. Since $M$ is cyclic, if $m=|M|$, then $|\Aut(M)|$ has order $\phi(m)$. On the other hand, since $|x|$ divides $n$, $|\alpha|$ divides $n$. Hence $|\alpha|$ divides $\gcd(n,\phi(m))=1$. This means that $x\in C_G(M)$, i.e., $N_G(M)\subseteq C_G(M)$. Since $M\subseteq N_G(M)\subseteq C_G(M)$, $M=N_G(M)=C_G(M)$.

    Let $M_1,\dots,M_l$ be the representatives of the conjugacy classes of maximal subgroups of $G$. For each $j\in\{1,\dots,l\}$, let $m_j=|M_j|$. Since $M_{j}=N_G(M_j)$ for each $j$, the orbit of $M_j$ has $n/m_j$ elements.

    Since for each $g\in G\setminus\{1\}$ there exists a unique maximal subgroup $M$ such that $g\in M$, we have
    \begin{equation}
    \label{eq:particion}
    n=1+\sum_{j=1}^l \frac{n}{m_j}(m_j-1).
    \end{equation}
    If $l=1$ then $n=m_1$, a contradiction. If $l>1$ then, since for each $j$ we have $m_j\geq2$, rewriting~\eqref{eq:particion}, we have
    \begin{align*}
    \frac{1}{n}+l-1=\sum_{j=1}^l\frac{1}{m_j}\leq\frac{l}{2}.
    \end{align*}
    From this inequality, we obtain $nl\leq 2n-2<2n$ and then $l<2$, a contradiction. 
\end{proof}

Similarly, abelian and nilpotent numbers can be defined. These numbers are classified, and an elementary proof can be found in \cite{MR1786236}. There is also the notion of a solvable number. Thanks to the Feit--Thompson theorem, every odd number is a solvable number. These numbers are also classified, although the proof is much more difficult as it relies on a very deep theorem of Thompson and the famous Feit--Thompson theorem.
