\chapter{}

\topic{Radical rings and solutions}

Let $S$ be a non-unitary ring. Consider $S_1=\Z\times S$ with the addition defined component-wise and  multiplication
\[
(k,a)(l,b)=(kl,kb+la+ab)
\]
for all $k,l\in\Z$ and $a,b\in S$. 
Then $S_1$ is a ring and $(1,0)$ is its unit element. 
Furthermore, $\{0\}\times S$ is an ideal of $S_1$. 
Note that $\{0\}\times S\simeq S$ as non-unitary rings. Also 
note that if $(k,x)\in S_1$ is invertible, 
then $k\in\{-1,1\}$. 

\begin{definition}
    A non-unitary ring $S$ is a (Jacobson) \textbf{radical ring} 
    if it is isomorphic to the Jacobson radical of a unitary ring.
\end{definition}

Let $R$ be a ring. The (Jacobson) \textbf{radical} $J(R)$ of $R$ is defined as the intersection
of all maximal left ideals of $R$. One proves that $J(R)$ is an ideal of $R$. Moreover, 
$x\in J(R)$ if and only if $1+rx$ is invertible for all $r\in R$.

\begin{proposition}
\label{pro:radical}
	Let $S$ be a non-unitary ring. The following statements are equivalent.
	\begin{enumerate}
		\item $S$ is a radical ring.
		\item For all $a\in S$ there exists a unique $b\in S$ such that $a+b+ab=a+b+ba=0$.
		\item $S\cong J(S_1)$. 
	\end{enumerate}
\end{proposition}  
	
\begin{proof}
    Let us first prove that $1)\implies2)$. Let $R$ be a unitary ring such that 
    $S\simeq J(R)$ and let $\psi\colon S\rightarrow R$ be an injective homomorphism 
    of non-unitary rings $\psi(S)=J(R)$. Let $a\in S$. Since  
    $1+\psi(a)\in R$ is invertible, there exists $c\in R$ such that 
    \[
    (1+\psi(a))(1+c)=(1+c)(1+\psi(a))=1.
    \]
    Thus 
    $c=-\psi(a)c-\psi(a)\in J(R)$. 
    Hence there exists $b\in S$ such that $\psi(b)=c$. Therefore 
    \[
    a+b+ab=a+b+ba=0.
    \]
    It is an exercise to prove that $b$ is unique. 
    
    We now prove that $2)\implies 3)$. We first note that if 
    $a\in S$, then there exists $b\in S$ such that $a+b+ab=a+b+ba=0$. 
    Thus every 
    $(1,a)\in S_1$ is invertible, as 
    \[
    (1,a)(1,b)=(1,0)=(1,b)(1,a).
    \]

    We claim that $J(S_1)=\{0\}\times S$. Let us prove that 
    $J(S_1)\supseteq \{0\}\times S$. If $(k,a)\in J(S_1)$, then, in particular, 
    \[
    (1+3k,3a)=(1,0)+(3,0)(k,a)
    \]
    is invertible, which implies that either $1+3k=1$ or $1+3k=-1$. Since
    $k\in\Z$, it follows that $k=0$ and hence $(k,a)=(0,a)\in\{0\}\times S$. 
    To prove that 
    $J(S_1)\supseteq \{0\}\times S$ note that
    if $(0,x)\in\{0\}\times S$, then
    \[
    (1,0)+(k,a)(0,x)=(1,kx+ax)
    \]
    is invertible, as $kx+ax\in S$. 
    
    The implication $3)\implies1)$ is trivial.
\end{proof}

A \textbf{nil ring} is a non-unitary ring $S$ such that every 
element of $S$ is nilpotent. Every nil ring is a radical ring.

\begin{example} 
    $X\C[[X]]$ is a radical ring and it is not a nil ring.
\end{example}

Let $S$ be a ring (unitary or non-unitary, it is not important here). 
Define on $S$ the binary operation 
\[
(a,b)\mapsto a\circ b=a+b+ab
\]
for all $a,b\in S$. Then $(S,\circ)$ is a monoid with neutral element $0$.
Note that $S$ is a radical ring if and only if $(S,\circ)$ is a group. 
If $a\in S$ is invertible in the monoid $(S,\circ)$, we will denote by $a'$ its inverse.

\begin{example}
	For $n>1$ let $A=\left\{\frac{nx}{ny+1}:x,y\in\Z\right\}\subseteq \Q$. 
	Note that $A$ is a (non-unitary) subring of $\Q$. In fact, $A$ is a radical ring. A straightforward computation shows that 
	\[
	\left(\frac{nx}{ny+1}\right)'=\frac{-nx}{n(x+y)+1}.
	\]
\end{example}

We now go back to study solutions to the YBE and discuss the intriguing interplay
between radical rings and involutive solutions. 

\begin{definition}
	\index{Solution!involutive}
	A solution $(X,r)$ is said to be \emph{involutive} if $r^2=\id$. 
\end{definition}

Note that if $(X,r)$ is an  involutive solution, then 
\[
(x,y)=r^2(x,y)=r(\sigma_x(y),\tau_y(x))=(\sigma_{\sigma_x(y)}\tau_y(x),\tau_{\tau_y(x)}\sigma_x(y)).
\]
Hence 
\begin{equation}
	\label{eq:involutive}
	\tau_y(x)=\sigma_{\sigma_x(y)}^{-1}(x),
	\quad
	\sigma_x(y)=\tau_{\tau_y(x)}^{-1}(y)
\end{equation}
for all $x,y\in X$. Thus for involutive solutions
it is enough to know $\{\sigma_x:x\in X\}$, as from this we obtain the
set $\{\tau_x:x\in X\}$.

\begin{example}
	Let $X$ be a non-empty set and $\sigma$ be a bijection on $X$. Then 
	$(X,r)$, where $r(x,y)=(\sigma(y),\sigma^{-1}(x))$, is an involutive solution. 
\end{example}

\index{Jacobson!radical ring}
\index{Radical ring}
We now present a very important family of involutive solutions. 


% The following fundamental family of solutions appears in~\cite{MR2278047}. 
% It turns out to be fundamental in the study of 
% set-theoretic solutions to the YBE. 

\begin{theorem}[Rump]
	\label{thm:Rump}
	\index{Rump's theorem}
	Let $R$ be a radical ring. Then $(R,r)$, where 
	\[
	r(x,y)=( -x+x\circ y,(-x+x\circ y)'\circ x\circ y)
	\]
	is an involutive solution.
\end{theorem}

The proposition can demonstrated using Theorem~\ref{thm:LYZ}. We will
prove a more general result later. 

\topic{Skew braces}


 

