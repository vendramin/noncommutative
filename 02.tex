\section{22/02/2024}

\subsection{Wielandt's theorem}

\begin{lemma}
	\label{lemma:4Wielandt}
	Let $G$ be a finite group and $H$ and $K$ be subgroups of $G$ 
        of coprime indices. Then $G=HK$ and $(H:H\cap K)=(G:K)$.
\end{lemma}

\begin{proof}
	Let $D=H\cap K$. Since 
	\[
	(G:D)=\frac{|G|}{|H\cap K|}=(G:H)(H:H\cap K),
	\]
	$(G:H)$ divides $(G:D)$. Similarly, $(G:K)$ divides 
	$(G:D)$. Since $(G:H)$ and $(G:K)$ are coprime, $(G:H)(G:K)$
	divides $(G:D)$. In particular, 
	\[
	\frac{|G|}{|H|}\frac{|G|}{|K|}=(G:H)(G:K)\leq (G:D)=\frac{|G|}{|H\cap K|} 
	\]
	and hence $|G|=|HK|$. Since 
 \[
 |G|=|HK|=|H||K|/|H\cap K|,
 \]
 we conclude that 
	$(G:K)=(H:H\cap K)$.
\end{proof}

\begin{definition}
\index{Normal closure}
Let $G$ be a group and $H$ be a subgroup of $G$. The 
\textbf{normal closure} $H^G$ of $H$ in $G$ is the subgroup 
$H^G=\langle xHx^{-1}:x\in G\rangle$.
\end{definition}

\begin{exercise}
	Let $G$ be a group and $H$ a subgroup of $G$. Prove that $H^G$ is normal in $G$ and that 
	$H^G$ is the smallest normal subgroup of $G$ containing $H$. 
\end{exercise}

% \begin{svgraybox}
% 	Es trivial demostrar que $H^G$ is normal in $G$.  Sea $N$ un subgrupo
% 	normal de $G$ tal que $H\subseteq N$. Since $xHx^{-1}\subseteq xNx^{-1}=N$
% 	para todo $x\in G$, $H\subseteq H^G\subseteq N$. 
% \end{svgraybox}

\begin{example}
Let $G=\Alt_4$ and $H=\{\id,(12)(34)\}$. The normal closure of $H$ in $G$
is 
 \[
 H^G=\{\id,(12)(34),(13)(24),(14)(23)\}\simeq C_2\times C_2.
 \]
\end{example}

\begin{theorem}[Wielandt]
	\label{theorem:Wielandt:solvable}
	Let $G$ be a finite group and $H$, $K$ and $L$ be 
        subgroups of $G$ with pairwise coprime indices. 
        If $H$, $K$ and $L$ are solvable, then $G$ is 
	solvable.
\end{theorem}

%%% TODO: revisar la demostración para escribirla mejor!

\begin{proof}
    Let $G$ be a minimal counterexample. Then $G\ne\{1\}$. Let $N$
	be a minimal normal subgroup of $G$ and $\pi\colon G\to G/N$ be the canonical map. Since $H$, $K$ and $L$ are solvable, 
	the subgroups $\pi(H)=\pi(HN)$, $\pi(K)=\pi(KN)$ and $\pi(L)=\pi(LN)$ of $\pi(G)=G/N$ are solvable. 
	By the correspondence theorem, $\pi(H)$, $\pi(K)$ and $\pi(L)$ 
        have pairwise coprime indices. 
% 	pues por ejemplo\footnote{El núcleo de la restricción
% 	$\ker(\pi|_H)=\ker \pi\cap N$ and entonces $\pi(H)\simeq H/N\cap H$.}
% 	\[
% 	(\pi(G):\pi(H))=(G/N:H/N\cap H)=(G:NH)
% 	\]
% 	divide a $(G:N)$. 
    By the minimality of $G$, $\pi(G)$ is solvable. If $H=\{1\}$, then 
    $|G|=(G:H)$ is coprime with $(G:K)$ and thus $G=K$ is solvable. If 
    $H\ne \{1\}$, let $M$ be a minimal normal subgroup of $H$. By Lemma~\ref{lemma:minimal_normal}, 
    $M$ is a $p$-group for some prime number $p$. 
    Without loss of generality, we may assume that $p$ does not divide $(G:K)$ (otherwise, if $p$ divides $(G:K)$, then 
    $p$ does not divide $(G:L)$ and we just need to change $K$ by $L$). 
    There exists $P\in\Syl_p(G)$ such that $P\subseteq K$. Sylow subgroups are conjugate, so there exists 
    $g\in G$ such that $M\subseteq
	gKg^{-1}$. Since $(G:gKg^{-1})=(G:K)$ is coprime with $(G:H)$, Lemma~\ref{lemma:4Wielandt} implies that
    $G=(gKg^{-1})H$. 
	
    We claim that all conjugates of $M$ are in $gKg^{-1}$. 
    If $x\in G$, write $x=uv$ some some $u\in 
	gKg^{-1}$ and $v\in H$. Since $M$ is normal in $H$, 
	\[
	xMx^{-1}=(uv)M(uv)^{-1}=uMu^{-1}\subseteq gKg^{-1}.
	\]
	In particular, $\{1\}\ne M^G\subseteq gKg^{-1}$ is solvable, as $gKg^{-1}$ is 
	solvable. By the minimality of $M$, the group $G/M^G$ is solvable. Hence 
        $G$ is solvable. 
\end{proof}

\subsection{Hall's theorem}

\begin{definition}
\index{$p$-complement}
Let $G$ be a finite group of order $p^{\alpha}m$, where $p$ is a prime number such that 
$\gcd(p,m)=1$. A subgroup 
$H$ of $G$ is said to be a \textbf{$p$-complement} if $|H|=m$. 
\end{definition}

\begin{example}
Let $G=\Sym_3$. Then $H=\langle (123)\rangle$ is a $2$-complement and 
$K=\langle (12)\rangle$ is a $3$-complement.
\end{example}

\begin{theorem}[Hall]
\label{theorem:Hall:solvable}
 \index{Hall's theorem}
Let $G$ be a finite group that admits a $p$-complement for every prime divisor $p$ of $|G|$. 
Then $G$ is solvable. 
\end{theorem}

\begin{proof}
	Let $|G|=p_1^{\alpha_1}\cdots
	p_k^{\alpha_k}$ with $p_1<\cdots<p_k$ prime numbers. We proceed by induction on $k$. 
	If $k=1$, then the claim holds, as $G$ is a $p$-group. If $k=2$, the result holds by
        Burnside's theorem. Assume now that 
	$k\geq3$. For $j\in\{1,2,3\}$, let $H_j$ be a $p_j$-complement in 
	$G$. Since $|H_j|=|G|/p_j^{\alpha_j}$, the subgroups $H_j$ have pairwise coprime indices.

	We claim that $H_1$ is solvable. Note that $|H_1|=p_2^{\alpha_2}\cdots
	p_k^{\alpha_k}$. Let $p$ be a prime number dividing $|H_1|$ and $Q$ be a 
	$p$-complement in $G$. 
	Since $(G:H_1)$ and $(G:Q)$ are
	coprime, Lemma~\ref{lemma:4Wielandt} implies that  
	\[
	(H_1:H_1\cap Q)=(G:Q). 
	\]
	Then $H_1\cap Q$ is a $p$-complement in $H_1$.  Therefore $H_1$ is
	solvable by the inductive hypothesis. Similarly, both $H_2$ and 
	$H_3$ are solvable.

	Since $H_1$, $H_2$ and $H_3$ are solvable of pairwise coprime indices, 
        the theorem follows from Wieland's theorem. 
\end{proof}

\subsection{Nilpotent groups}

For a group $G$ and $x,y,z\in G$, conjugation will be considered as a left action of $G$ on $G$ 
and we will use the following notation: $\prescript{x}{}y=xyx^{-1}$. The commutator between $x$ and $y$ 
will be written as 
\[
[x,y]=xyx^{-1}y^{-1}=(\prescript{x}{}y)y^{-1}.
\]

We will also use the following notation:  
$[x,y,z]=[x,[y,z]]$. For subgroups $X$, $Y$ and $Z$ of $G$, we write 
$[X,Y,Z]=\left[ X,[Y,Z] \right]$. Note that $[X,Y]=[Y,X]$. 

\begin{exercise}[The Hall--Witt identity]
	\label{exercise:HallWitt}
	\index{Hall--Witt identity}
	\index{Hall, P.}
	\index{Witt, E.}
	Let $G$ be a group and $x,y,z\in G$. Prove that 
	\begin{equation}
		\label{eq:HallWitt}
	\left(\prescript{y}{}[x,y^{-1},z]\right)\left(\prescript{z}{}[y,z^{-1},x]\right)\left(\prescript{x}{}[z,x^{-1},y]\right)=1.
	\end{equation}
\end{exercise}

\index{Jacobi, G.}
\index{Jacobi identity}
If $G$ is a group and $[G,G]$ is central in $G$,
then the Hall-Witt becomes Jacobi's identity.

%\begin{proof}
%	Como la demostración es simplemente un cálculo de rutina, la dejamos como ejercicio.
%	Calculamos 
%	\begin{align*}
%	&\prescript{y}{}[x,y^{-1},z]=yxy^{-1}zyz^{-1}x^{-1}zy^{-1}z^{-1},\\
%	&\prescript{z}{}[y,z^{-1},x]=zyz^{-1}xzx^{-1}y^{-1}xz^{-1}x^{-1},\\
%	&\prescript{x}{}[z,x^{-1},y]=xzx^{-1}yxy^{-1}z^{-1}yx^{-1}y^{-1}.
%	\end{align*}
%\end{proof}
\begin{lemma}[Hall's three subgroups lemma]
	\label{lemma:3subgrupos}
	\index{Lema!de los tres subgrupos}
	Let $X$, $Y$ and $Z$ be subgroups of $G$ 
 such that $[X,Y,Z]=[Y,Z,X]=\{1\}$.
	Then $[Z,X,Y]=\{1\}$.
\end{lemma}

\begin{proof}
Since $[x,y]\in C_G(z)$ implies $[X,Y]\subseteq C_G(Z)$, 
it is enough to prove that $[z,x^{-1},y]=1$ for all $x\in X$, $y\in Y$ and $z\in Z$. Since $[y^{-1},z]\in [Y,Z]$, $[x,y^{-1},z]\in [X,Y,Z]=\{1\}$. Thus $\prescript{y}{}[x,y^{-1},z]=1$. Similarly, $\prescript{z}{}[y,z^{-1},x]=1$. Using the Hall--Witt identity, we conclude that $[z,x^{-1},y]=1$.
\end{proof}

\begin{exercise}
\label{xca:3subgroups}
Let $N$ be a normal subgroup of $G$ and 
$X$, $Y$ and $Z$ be subgroups of $G$. If $[X,Y,Z]\subseteq N$ and $[Y,Z,X]\subseteq N$, then 
$[Z,X,Y]\subseteq N$.
\end{exercise}

%\begin{sol}{xca:3subgroups}
%	Sea $\pi\colon G\to G/N$ el morfismo canónico. Como $[X,Y,Z]\subseteq N$, 
%	\begin{align*}
%		\{1\}&=\pi([X,Y,Z])=\pi([ X,[Y,Z]])\\
%		&=[\pi(X),\pi([Y,Z])]=[ \pi(X),[\pi(Y),\pi(Z)]]=[\pi(X),\pi(Y),\pi(Z)]. 
%	\end{align*}
%	Similarmente $[\pi(Y),\pi(Z),\pi(X)]=\{1\}$. Entonces, gracias al lema de los
%	tres subgrupos, $[\pi(Z),\pi(X),\pi(Y)]=\{1\}$, es decir
%	$[Z,X,Y]\subseteq N$.
%\end{sol}

% \begin{svgraybox}
% 	Sean $h\in H$ y $g\in G$. $hKgK=gKhK$ si y sólo si $[h,g]\in K$.
% \end{svgraybox}

\begin{definition}
\index{Lower central series}
Let $G$ be a group. The \textbf{lower central series} 
is the sequence $\gamma_k(G)$ of subgroups defined inductively 
as 
\[
\gamma_1(G)=G,\quad
\gamma_{i+1}(G)=[G,\gamma_i(G)]\quad i\geq 1.
\]
\end{definition}

\begin{definition}
\index{Group!nilpotent}
\index{Nilpotency index}
A group $G$ is said to be \textbf{nilpotent} if there exists a positive integer $c$ such that 
$\gamma_{c+1}(G)=\{1\}$. The smallest $c$ with $\gamma_{c+1}(G)=\{1\}$ is 
the \textbf{nilpotency class} of $G$.
\end{definition}

\begin{exercise}
\label{xca:nilpotent=>solvable}
Prove that every nilpotent group is solvable. 
\end{exercise}

A group is nilpotent of nilpotency class one if and only if it is abelian. 

\begin{example}
The group $\Sym_3$ is solvable, as 
$\Sym_3\supseteq \Alt_3\supseteq\{1\}$ is a composition series 
with abelian factors. However, $\Sym_3$ is not nilpotent, as 
\[
\gamma_1(\Sym_3)=\Alt_3,\quad
\gamma_2(\Sym_3)=[\Alt_3,\Sym_3]=\Alt_3, 
\]
and therefore $\gamma_i(\Sym_3)\ne\{1\}$ for all $i\geq1$. 
\end{example}

\begin{example}
The group $G=\Alt_4$ is not nilpotent, as 
\[
\gamma_1(G)=G,\quad
\gamma_j(G)=\{\id,(12)(34),(13)(24),(14)(23)\}\simeq C_2\times C_2
\]
for all $j\geq2$. We can do this with the computer:
\begin{lstlisting}
gap> IsNilpotent(AlternatingGroup(4));
false
\end{lstlisting}
Let us do the calculation of the lower central series with the computer: 
\begin{lstlisting}
gap> List(LowerCentralSeries(AlternatingGroup(4)),\
StructureDescription);
[ "A4", "C2 x C2" ]
\end{lstlisting}
Here is an alternative:
\begin{lstlisting}
gap> G := AlternatingGroup(4);;
gap> gamma_1 := G;;
gap> gamma_2 := DerivedSubgroup(G);;
gap> gamma_3 := CommutatorSubgroup(gamma_2,G);;
gap> StructureDescription(gamma_1);
"A4"
gap> StructureDescription(gamma_2);
"C2 x C2"
gap> StructureDescription(gamma_3);
"C2 x C2"
\end{lstlisting}
\end{example}

\begin{exercise}
\label{xca:gamma}
Let $G$ be a group. Prove the following statements: 
\begin{enumerate}
\item Each $\gamma_i(G)$ is a characteristic subgroup of $G$. 
\item $\gamma_i(G)\supseteq\gamma_{i+1}(G)$ for all $i\geq1$.
\item If $f\colon G\to H$ is a surjective group homomorphism, then  
$f(\gamma_i(G))=\gamma_i(H)$ for all $i\geq1$.
\end{enumerate}
\end{exercise}

\begin{exercise}
\label{xca:HxK_nilpotente}
Prove that if $H$ and $K$ are nilpotent groups, then 
$H\times K$ is nilpotent. 
\end{exercise}

\begin{exercise}
\label{xca:nilpotente}
Let $G$ be a nilpotent group. Prove the following statements: 
\begin{enumerate}
\item Subgroups of $G$ are nilpotent. 
\item If $f\colon G\to H$ is a surjective homomorphism, then $H$ is nilpotent.  
\end{enumerate}
\end{exercise}

%\begin{sol}{xca:nilpotent}
%	La primera afirmación es cierta pues $\gamma_i(H)\subseteq\gamma_i(G)$ para
%	todo $i\geq1$. La segunda afirmación: si existe $c$ tal que $\gamma_{c+1}(G)=\{1\}$
%	entonces \[
%	\gamma_{c+1}(H)=f(\gamma_{c+1}(G))=f(\{1\})=\{1\}.\qedhere
%	\]
%\end{sol}

\begin{exercise}
   True or false? If $G$ is a nilpotent group and $N$ is normal 
   subgroup of $G$ such that $N$ and $G/N$ are nilpotent, then 
   $G$ is nilpotent. 
\end{exercise}

%\begin{sol}{xca:nilpotent_notexact}
%    The claim is false. Take $G=\Sym_3$
%    and $N=\langle(123)\rangle$. Then both $N\simeq C_3$ and $G/N\simeq C_2$ are nilpotent, but $G$ is not nilpotent. 
%\end{sol}

\begin{proposition}
\label{pro:nilpotent_pgroups}
Finite $p$-groups are nilpotent.
\end{proposition}

\begin{proof}
We proceed by induction on $|G|$. The case $G=\{1\}$ is trivial. 
Assume the result holds for $p$-groups of order $<|G|$. Since 
$G$ is a $p$-group, $Z(G)\ne\{1\}$. By the inductive hypothesis, 
$G/Z(G)$ is nilpotent. There exists $c$ such that 
$\gamma_{c+1}(G/Z(G))=\{1\}$. 
	
Let $\pi\colon G\to G/Z(G)$ be the canonical map.  
By Exercise~\ref{xca:gamma},
\[
\pi(\gamma_{c+1}(G))=\gamma_{c+1}(G/Z(G))=\{1\}. 
\]
Then $\gamma_{c+1}(G)\subseteq \ker\pi=Z(G)$. Hence
$G$ is nilpotent, as 
\[	
\gamma_{c+2}(G)=[G,\gamma_{c+1}(G)]=[G,Z(G)]=\{1\}.\qedhere
\]
\end{proof}

\begin{theorem}
\label{thm:gamma}
If $G$ is a group, then $[\gamma_i(G),\gamma_j(G)]\subseteq
\gamma_{i+j}(G)$ for all $i,j\geq1$.	
\end{theorem}

\begin{proof}
We proceed by induction on $i$. The case $i=1$ is trivial, as
$[G,\gamma_j(G)]=\gamma_{j+1}(G)$ by definition. Assume that
the result holds for some $i\geq1$ and all $j\geq 1$. 
	
First note that 
\begin{equation*}
[G,\gamma_i(G),\gamma_j(G)]\subseteq [G,\gamma_{i+j}(G)]=\gamma_{i+j+1}(G)
\end{equation*}
by the inductive hypothesis. Moreover, by the inductive hypothesis, 
\begin{equation*}
[\gamma_i(G),\gamma_j(G),G]=[\gamma_i(G),G,\gamma_j(G)]=[\gamma_{i}(G),\gamma_{j+1}(G)]\subseteq \gamma_{i+j+1}(G)
\end{equation*}
By using Exercise~\ref{xca:3subgroups} with $X=G$, $Y=\gamma_i(G)$ and $Z=\gamma_j(G)$,
we get that 
\[
[\gamma_j(G),G,\gamma_i(G)]\subseteq \gamma_{i+j+1}(G).
\]
Hence  
\[
[\gamma_{i+1}(G),\gamma_{j}(G)]=[\gamma_{j}(G),\gamma_{i+1}(G)]=[\gamma_j(G),G,\gamma_i(G)]\subseteq \gamma_{i+j+1}(G).\qedhere
\]
\end{proof}

\index{Weigth of a commutator}
We consider arbitrary commutators, not necessarily associated on the right.   
For example, both $[G,G,G]=[G,[G,G]]$ and $[[G,G],G]$ are commutators of \textbf{weight} three. 

\begin{corollary}
In a group $G$, every weight $n$ commutator is contained in 
$\gamma_n(G)$.
\end{corollary}

\begin{proof}
We proceed by induction on $n$. The case $n=1$ is trivial. Assume that $n\geq1$ and 
the result holds for all $k\leq n$. An arbitrary commutator of weight $n+1$ 
is of the form $[A,B]$, where $A$ is a commutator of weight $k$,
$B$ is a commutator of weight $l$ and $n+1=k+l$. Since $k<n$ and $l<n$, 
the inductive hypothesis implies that  $A\subseteq \gamma_k(G)$ and $B\subseteq
\gamma_l(G)$. Hence $[A,B]\subseteq [\gamma_k(G),\gamma_l(G)]\subseteq
\gamma_{k+l}(G)$ by the previous theorem.
\end{proof}

\begin{exercise}
\label{xca:inclusion}
    Let $G$ be a group. Prove that $G^{(k)}\subseteq G^{2^k}$ for all $k\geq1$. 
\end{exercise}

\begin{exercise}
\label{xca:derived_length}
    Let $G$ be a nilpotent group of class $m$. Prove that
    the derived length of $G$ is $\leq 1+\log_2m$.
\end{exercise}

The following lemma is important. It states that nilpotent groups satisfy 
the \textbf{normalizer condition}. 

\begin{lemma}[normalizer condition]
\label{lem:normalizadora}
\index{Normalizer condition}
Let $G$ be a nilpotent group. If $H$ is a proper subgroup of $G$, then 
$H\subsetneq N_G(H)$.
\end{lemma}

\begin{proof}
There exists $c$ such that $G=\gamma_1(G)\supseteq\cdots\supseteq\gamma_{c+1}(G)=\{1\}$. Since 
$\{1\}=\gamma_{c+1}(G)\subseteq H$ and $\gamma_1(G)\not\subseteq H$, 
let $k$ be the smallest positive integer such that  $\gamma_k(G)\subseteq H$. 
Since
\[
[H,\gamma_{k-1}(G)]\subseteq [G,\gamma_{k-1}(G)]=\gamma_k(G)\subseteq H,
\]
we obtain that  
	%$xhx^{-1}h^{-1}\in H$ para todo $x\in\gamma_{k-1}(G)$ y $h\in
	%H$. Esto implica que 
$xHx^{-1}\subseteq H$ for all $x\in\gamma_{k-1}(G)$,
that is $\gamma_{k-1}(G)\subseteq N_G(H)$. If $N_G(H)=H$, then
$\gamma_{k-1}(G)\subseteq H$, a contradiction to the minimality of $k$. 
\end{proof}

%\begin{example}
%	Un grupo $G$ es nilpotente de clase dos si y sólo $\gamma_2(G)

%\end{example}

For a group $G$, we define the sequence  $\zeta_0(G),\zeta_1(G),\dots$
recursively as follows: 
\[
	\zeta_0(G)=\{1\},\quad
	\zeta_{i+1}(G)=\{g\in G:[x,g]\in\zeta_{i}(G)\text{ for all $x\in G$}\},\quad i\geq 0.
\]
For example, $\zeta_1(G)=Z(G)$.

\begin{lemma}
\label{lem:central_ascendente}
Let $G$ be a group. For every $i\geq0$, the set $\zeta_i(G)$ 
is a normal subgroup of $G$. 
\end{lemma}

\begin{proof}
We proceed by induction on $i$. The case $i=0$ is trivial, as 
$\zeta_0(G)=\{1\}$. Assume the result holds for some $i$.
We claim that $\zeta_{i+1}(G)$ is normal subgroup of $G$. 
Let $g,h\in \zeta_{i+1}(G)$ and $x\in G$. By the inductive hypothesis, 
\begin{align*}
	&[x,g^{-1}]=(xg^{-1})[x^{-1},g](xg^{-1})^{-1}\zeta_i(G)(xg^{-1})^{-1}=\zeta_i(G),\\
	&[x,gh]=[x,h][hxh^{-1},g]\in \zeta_{i}(G).
\end{align*}
Since $1\in\zeta_{i+1}(G)$, we conclude that each $\zeta_i(G)$ is a subgroup of $G$. 
Moreover, $xgx^{-1}\in\zeta_{i+1}(G)$, as  
	\[
	[xgx^{-1},y]=x[g,x^{-1}yx]x^{-1}\in\zeta_{i}(G)
	\]
para todo $y\in G$.
\end{proof}


\begin{definition}
\index{Ascending central series}
Let $G$ be a group. The \textbf{ascending central series} of $G$ 
is the sequence 
\[
\{1\}=\zeta_0(G)\subseteq\zeta_1(G)\subseteq\zeta_2(G)\subseteq\cdots
\]
\end{definition}

\begin{definition}
\index{Group!perfect}
A group $G$ is said to be \textbf{perfect} if $[G,G]=G$.
\end{definition}

\begin{theorem}[Gr\"un]
\label{thm:Grun}
\index{Gr\"un's theorem}
If $G$ is a perfect group, then $Z(G/Z(G))=\{1\}$. 
\end{theorem}

\begin{proof}
    By definition, $[G,\zeta_2(G)]\subseteq Z(G)$ and 
    $[\zeta_2(G),G]\subseteq Z(G)$. Then 
    \[
    [G,G,\zeta_2(G)]=[G,\zeta_2(G),G]=\{1\}.
    \]
    By using the three subgroups lemma with $X=Y=G$ and $Z=\zeta_2(G)$, 
    \[
    [\zeta_2(G),G]=[\zeta_2(G),[G,G]]=[\zeta_2(G),G,G]=\{1\}.
    \]
    Thus $\zeta_2(G)\subseteq Z(G)$. 
    
    We aim to prove that $Z(G/Z(G))$ is trivial. Let $\pi\colon G\to G/Z(G)$ be the canonical map and 
    $g\in G$ be such that $\pi(g)$ is central. Since 
    \[
    1=[\pi(x),\pi(g)]=\pi([x,g])
    \]
    for all $x\in G$, $[x,g]\in Z(G)=\zeta_1(G)$ for all $x\in G$. Hence 
    $g\in\zeta_2(G)\subseteq Z(G)$. 
\end{proof}

For subgroups $H$ and $K$ of $G$, let 
\[
[H,K]=\langle [h,k]:h\in H,\,k\in K\rangle.
\]

\index{Normalizer}
Let $G$ be a group and $K$ be a subgroup of $G$. We say that $K$ \textbf{normalizes} 
$H$ if $K\subseteq N_G(H)$.
\index{Centralizer}
We say that $K$ \textbf{centralizes} 
$H$ if $K\subseteq C_G(H)$, that is if and only if $[H,K]=\{1\}$.

\begin{exercise}
Let $K$ and $H$ be subgroups of $G$ such that $K\subseteq H$ and $K$ is normal in $G$.
Prove that $[H,G]\subseteq K$ if and only if $H/K\subseteq Z(G/K)$. 
\end{exercise}

\begin{lemma}
\label{lem:gamma_zeta}
Let $G$ be a group. There exists an integer $c$ such that 
$\zeta_c(G)=G$ if and only if 
$\gamma_{c+1}(G)=\{1\}$. In this case, 
\[
\gamma_{i+1}(G)\subseteq\zeta_{c-i}(G)
\]
for all $i\in\{0,1,\dots,c\}$. 
\end{lemma}

\begin{proof}
Assume first that $\zeta_c(G)=G$. To prove that 
$\gamma_{i+1}(G)\subseteq\zeta_{c-i}(G)$ holds for all $i$, we proceed by induction. 
The case $i=0$ is trivial. So assume that the result holds for some $i\geq0$. If
$g\in\gamma_{i+2}(G)=[G,\gamma_{i+1}(G)]$, then     
\[
g=\prod_{k=1}^N [x_k,g_k],
\]
for some $g_1,\dots,g_N\in\gamma_{i+1}(G)$ and $x_1,\dots,x_N\in G$. By the inductive 
hypothesis, 
	\[
	g_k\in\gamma_i(G)\subseteq\zeta_{c-i}(G)
	\]
for all $k$. Hence $[x_k,g_k]\in\zeta_{c-i-1}(G)$ for all $k$. Therefore   
$g\in\zeta_{c-(i+1)}(G)$. 
	
We now assume that $\gamma_{c+1}(G)=\{1\}$. We aim to prove that 
$\gamma_{i+1}(G)\subseteq\zeta_{c-i}(G)$ holds for all $i$. We proceed by backwards induction on $i$. 
The case $i=c$ is trivial. So assume the result holds for some $i+1\leq c$. 
Let $g\in\gamma_{i}(G)$. By the inductive hypothesis, 
	\[
	[x,g]\in [G,\gamma_i(G)]=\gamma_{i+1}(G)\subseteq\zeta_{c-i}(G).
	\]
Thus $g\in\zeta_{c-i+1}(G)$ by definition. 
\end{proof}

\begin{example}
Let $G=\Sym_3$. Then $\zeta_j(G)=\{1\}$ for all $j\geq 0$: 
\begin{lstlisting}
gap> UpperCentralSeries(SymmetricGroup(3));
[ Group(()) ]
\end{lstlisting}
\end{example}

\begin{definition}
\index{Central series}
Let $G$ be a group. A \textbf{central series} for $G$ 
is a sequence 
	\[
		G=G_0\supseteq G_1\supseteq\cdots\supseteq G_n=\{1\}
	\]
of normal subgroups of $G$ such that 
for each $i\in\{1,\dots,n\}$, 
$\pi_i(G_{i-1})$ is a subgroup of $Z(G/G_i)$, where  $\pi_i\colon G\to
G/G_i$ is the canonical map.
\end{definition}

\begin{lemma}
\label{lem:central_series}
Let $G=G_0\supseteq G_1\supseteq\cdots\supseteq G_n=\{1\}$ be 
a central series of a group $G$. Then $\gamma_{i+1}(G)\subseteq G_i$ for all $i$.
\end{lemma}

\begin{proof}
We proceed by induction on $i$. The case $i=0$ is trivial. So assume the result holds for some
$i\geq0$. Let  
	$\pi_i\colon G\to
	G/G_i$ be the canonical map. 
	Then  
	\[
	\gamma_{i+1}(G)=[G,\gamma_i(G)]\subseteq [G,G_{i-1}].
	\]
	Since $\pi_i(G_{i-1})\subseteq Z(G/G_{i})$, 
	\[
        \pi_i([G,G_{i-1}])=[\pi_i(G),\pi_i(G_{i-1})]=\{1\}.
        \]
        Hence $\gamma_{i+1}(G)=[G,G_{i-1}]\subseteq G_i$. 
\end{proof}

% extender el lema para ver qué pasa con zeta_i

\begin{theorem}
A group is nilpotent if and only if it admits a central series. 
\end{theorem}

\begin{proof}
Let $G$ be a group. If $G$ is nilpotent, then the $\gamma_j(G)$ form a central series of 
$G$. Conversely, if $G=G_0\supseteq
G_1\supseteq\cdots\supseteq G_n=\{1\}$ is a central series of $G$, 
then, by the previous lemma,  
	\[
	\gamma_{n+1}(G)\subseteq G_n=\{1\}.
	\]
Hence $G$ is nilpotent. 
\end{proof}

\begin{exercise}
\label{xca:nilpotente_central}
Let $G$ be a group. Prove that if $K$ is a subgroup of $Z(G)$ such that 
$G/K$ is nilpotent, then $G$ is nilpotent. 
\end{exercise}

\subsection{Hirsch's theorem}

\begin{theorem}[Hirsch]
\label{thm:Hirsch}
\index{Hirsch's theorem}
Let $G$ be a nilpotent group. If $H$ is a non-trivial normal subgroup of $G$, 
then $H\cap Z(G)\ne\{1\}$. In particular, $Z(G)\ne\{1\}$. 
\end{theorem}

\begin{proof}
Since $\zeta_0(G)=\{1\}$ and there exists an integer $c$ such that $\zeta_c(G)=G$, 
there exists 
\[
m=\min\{k:H\cap\zeta_k(G)\ne\{1\}\}.
\]
Since $H$ is normal in $G$, 
\[
[G,H\cap\zeta_m(G)]\subseteq H\cap[G,\zeta_m(G)]\subseteq H\cap\zeta_{m-1}(G)=\{1\}.
\]
Therefore $\{1\}\ne H\cap\zeta_m(G)\subseteq H\cap Z(G)$. If $H=G$, then $Z(G)\ne\{1\}$. 
\end{proof}

\begin{exercise}
\label{xca:nilpotente_minimalnormal}
Let $G$ be a nilpotent group and $M$ be a minimal normal subgroup of $G$. Prove 
that $M\subseteq Z(G)$.
\end{exercise}

% \begin{svgraybox}
% 	Como $M\cap Z(G)$ es normal en $G$, la minimalidad de $M$ implica que hay
% 	dos posibilidades: $M\cap Z(G)$ es trivial o bien $M=M\cap Z(G)\subseteq Z(G)$.
% 	Por el teorema~\ref{theorem:Z(nilpotent)}, $M\cap Z(G)\ne 1$.
% \end{svgraybox}

\begin{definition}
\index{Subgroup|Maximal normal}
    Let $G$ be a group. A subgroup $M$ is said to be \textbf{maximal normal} in $G$
    if $M\ne G$ and $M$ is the only proper normal subgroup of $G$ containing $M$. 
\end{definition}

\begin{corollary}
Let $G$ be a non-abelian nilpotent group and $A$ be a maximal normal and abelian 
subgroup un subgrup of $G$. Then $A=C_G(A)$.
\end{corollary}

\begin{proof}
Since $A$ is abelian, $A\subseteq C_G(A)$. Assume that $A\ne C_G(A)$.
The centralizer $C_G(A)$ is normal in $G$, as, since $A$ is normal in $G$, 
\[
gC_G(A)g^{-1}=C_G(gAg^{-1})=C_G(A).
\]
for all $g\in G$. Let $\pi\colon G\to G/A$ be the canonical map. 
Then $\pi(C_G(A))$ is a non-trivial normal subgroup of $\pi(G)$. Since 
$G$ is nilpotent, $\pi(G)$ is nilpotent. By Hirsch's theorem, 
\[
\pi(C_G(A))\cap Z(\pi(G))\ne\{1\}.
\]
Let $x\in C_G(A)\setminus A$ be such that $\pi(x)$ is central in $\pi(G)$.  Then 
$\langle A,x\rangle$ is abelian, as $x\in C_G(A)$. Moreover,  $\langle
A,x\rangle$ is normal in $G$, as $A$ is normal in $G$ and 
$gxg^{-1}x^{-1}\in A$ for all  $g\in G$ (because $\pi(x)$ is central). Hence
	$gxg^{-1}\in \langle A,x\rangle$ and therefore $A\subsetneq \langle
	A,x\rangle\subsetneq G$, a contradiction.
\end{proof}

\begin{theorem}
Let $G$ be a nilpotent group. The following statements hold: 
\begin{enumerate}
\item Every minimal normal subgroup of $G$ has prime order and is central. 
\item Every maximal subgroup of $G$ is normal of prime index and contains $[G,G]$. 
\end{enumerate}
\end{theorem}

\begin{proof}\
\begin{enumerate}
    \item Let $N$ be a minimal normal subgroup of $G$. Since 
        $N\cap Z(G)\ne\{1\}$ by Hirsch's theorem, $N\cap Z(G)$ is a normal subgroup of 
        $G$ contained in $N$. Then $N=N\cap Z(G)\subseteq
	Z(G)$ by the minimality of $N$. In particular, $N$ is abelian. Since every subgroup of 
         $N$ is normal in $G$, $N$ is simple. Hence $N\simeq
	C_p$ for some prime number $p$.
 \item  If $M$ is a maximal subgroup, then $M$
is normal in $G$ by the normalizer condition (Lemma \ref{lem:normalizadora}). By the maximality
of $M$, the quotient $G/M$ contains no proper non-trivial subgroups. Thus 
$G/M\simeq C_p$ for some prime $p$. Since 
	$G/M$ is abeliano, $[G,G]\subseteq M$. \qedhere 
\end{enumerate}
\end{proof}

The previous theorem does not prove the existence of maximal subgroups. For example, 
$\Q$ is a nilpotent group (as it is abelian) 
that contains no maximal subgroups. 

\begin{proposition}
\label{pro:g^n}
Let $G$ be a nilpotent group and $H$ be a subgroup with $(G:H)=n$. If 
$g\in G$, then $g^n\in H$.
\end{proposition}

\begin{proof}
We proceed by induction on $n$. The case $n=1$ is trivial. The case  
$n=2$ follows from the normality of  $H$. So assume the result 
holds for all groups of index $<n$. Let $H$ be a subgroup of $G$ such that $(G:H)=n$. 
Let $H_0=H$ and $H_{i+1}=N_G(H_i)$ for all $i\geq0$. By definition, $H_{i}$ is normal in 
$H_{i+1}$. Since $G$ is nilpotent, $H_i\ne G$
implies that $H_i\subsetneq H_{i+1}$ by the normalizer condition. 
Since $(G:H)$ is finite, there exists $k$ such that  $H_k=G$. Since
$(H_j:H_{j-1})<n$ for all $j$, the inductive hypotehsis implies that 
	$x^{(H_j:H_{j-1})}\in H_{j-1}$ for all $x\in H_j$ and all $j$. Hence 
	\[
		g^{(G:H)}=g^{(H_k:H_{k-1})(H_{k-1}:H_{k-2})\cdots (H_1:H_0)}\in H.\qedhere 
	\]
% 	El resultado es obvio en el caso en que $H$ sea un subgrupo normal.  Sea
% 	$H_0=H$ y $H_{i+1}=N_G(H_i)$ para $i\geq0$. Por definición, $H_{i}$ es
% 	normal en $H_{i+1}$ y además, como $G$ es nilpotente, si $H_i\ne G$
% 	entonces $H_i\subsetneq H_{i+1}$ por la condición normalizadora. 
% 	Como $(G:H)$ es finito, existe $k$ tal que $H_k=G$. Veamos que 
% 	\[
% 		g^{(G:H)}=g^{(H_k:H_{k-1})(H_{k-1}:H_{k-2})\cdots (H_1:H_0)}\in H.
% 	\]
% 	Observemos que $g^{(H_k:H_{k-1})}\in H_{k-1}$ pues $H_{k-1}$ es normal en $H_k=G$, y que, como 
% 	$g^{(H_k:H_{k-1})}\in H_k$, entonces 
% 	\[
% 	g^{(H_k:H_{k-2})}=g^{(H_k:H_{k-1})(H_{k-1}:H_{k-2})}=\left(g^{(H_k:H_{k-1})}\right)^{(H_{k-1}:H_{k-2})}\in H_{k-2}
% 	\]
% 	pues $H_{k-2}$ es normal en $H_{k-1}$. Al repetir este argumento, $g^{(G:H)}\in H$. 
\end{proof}

\begin{exercise}
\label{xca:g^n}
    Does the previous proposition hold for non-nilpotent groups? 
\end{exercise}

% \begin{sol}{xca:g^n}
% No. Take for example 
% $G=\Sym_3$ and $H=\{\id,(12)\}$. Then $H$ is index three in $G$ and 
% for  
% $g=(13)$ one has $g^{3}=(13)\not\in H$.
% \end{sol}

The following lemma is useful for performing induction 
in the nilpotency index of nilpotent groups. 

\begin{lemma}
\label{lem:a[GG]}
Let $G$ be a nilpotent group of class $c\geq2$. If $x\in G$, then 
the subgroup 
$\langle x,[G,G]\rangle$ is nilpotente of class $<c$.
\end{lemma}

\begin{proof}
Let $H=\langle x,[G,G]\rangle$.  If $x\in [G,G]$, the there is nothing to prove. 
So assume that $x\not\in [G,G]$. Note that 
	\[
		H=\{x^nc:n\in\Z,c\in [G,G]\},
	\]
as $[G,G]$ is normal in $G$. We need to show that
$[H,H]\subseteq\gamma_3(G)$. Let $h=x^nc,k=x^md\in H$
be such that $c,d\in [G,G]$. 
Since 
	\[
	[h,x^m]=[x^n,[c,x^m]][c,x^m]\in\gamma_4(G)\gamma_3(G)\subseteq\gamma_3(G),
	\]
then  
	\begin{align*}
		[h,k]&=[h,x^m][x^m,[h,d]][h,d]\\
			&=[x^n,[c,x^m]][c,x^m][x^m,[h,d]][h,d]\in\gamma_3(G).\qedhere
	\end{align*}
%	Como
%	$[G,G]=\gamma_2(G)\subseteq\zeta_{c-1}(G)$, al usar la definición de $H$ se
%	obtiene que $[G,G]\subseteq \zeta_{c-1}(G)\cap H\subseteq\zeta_{c-1}(H)$.
%	Luego $H/\zeta_{c-1}(H)$ es cíclico generado por $a\zeta_{c-1}(H)$. Si
%	$\zeta_{c-1}(H)=H$, no hay nada para demostrar. En caso contrario,
%	$Z(H)\subseteq\zeta_{c-1}(H)$ implicaría que $H$ es abeliano y luego $H$ es
%	nilpotente de clase uno.
\end{proof}

\begin{example}
Let $G=\D_{8}=\langle r,s:r^{8}=s^2=1,srs=r^{-1}\rangle$ the dihedral group of order 16.
Then $G$ is nilpotent of class three and 
$[G,G]=\{1,r^2,r^4,r^6\}\simeq C_4$. The subgroup $\langle
s,[G,G]\rangle\simeq\D_4$ is nilpotent of class two. 
\begin{lstlisting}
gap> G := DihedralGroup(IsPermGroup,16);;
gap> gens := GeneratorsOfGroup(G);;
gap> r := gens[1];;
gap> s := gens[2];;
gap> D := DerivedSubgroup(G);;
gap> S := Subgroup(G, Concatenation(Elements(D), [s]));;
gap> StructureDescription(S);
"D8"
gap> NilpotencyClassOfGroup(G);
3
gap> NilpotencyClassOfGroup(S);
2
	\end{lstlisting}
\end{example}

Let us discuss a concrete application of Lemma \ref{lem:a[GG]}.

\begin{theorem}
\label{thm:T(nilpotent)}
If $G$ is a nilpotent group, then 
\[
T(G)=\{g\in G:g^n=1\text{ for some $n\geq1$}\}
\]
is a subgroup of $G$. 
\end{theorem}

\begin{proof}
We proceed by induction on the nilpotency class of $G$. 
Let $a,b\in T(G)$ and 
\[
	A=\langle a,[G,G]\rangle,\quad
	B=\langle b,[G,G]\rangle.
\]
Since  $A$ and $B$ are nilpotent of class $<c$ by the previous lemma, 
the inductive hypothesis 
implies that $T(A)$ is a subgroup of $A$ and $T(B)$ is a subgroup of $B$.
Since $T(A)$ is characteristic in $A$ and $A$ is normal in $G$, $T(A)$ is normal in $G$. 
Similarly, $T(B)$ is normal in $G$. 

We claim that every element of 
$T(A)T(B)$ has finite order. If 
$x\in T(A)T(B)$, say $x=a_1b_1$ with
$a_1$ of order $m$, then $x$ has finite order, as 
\begin{align*}
x^m=(a_1b_1)^m&=%(a_1b_1a_1^{-1})(a_1^2b_1a_1^{-2})\cdots (a_1^{m-1} b_1 a_1^{-m+1}b_1)\\
(a_1b_1a_1^{-1})(a_1^2b_1a_1^{-2})\cdots (a^{m-1} b_1 a^{-m+1})b_1\in T(B).
\end{align*}

To see clearly what is what we did, let us work out a concrete 
example, say $m=3$. In this case, we obtain the following formula:
\begin{align*}
(a_1b_1)^3&=(a_1b_1)(a_1b_1)(a_1b_1)\\
&=(a_1b_1a_1^{-1})(a_1^2b_1a_1^{-2})a_1^3b_1
=(a_1b_1a_1^{-1})(a_1^2b_1a_1^{-2})b_1,
\end{align*}
as $a_1^3=1$.

With this trick, we prove that $ab$ and $a^{-1}$ have finite order. Hence $T(G)$ 
is a subgroup of $G$. 
\end{proof}

Another application:

\begin{theorem}
\label{thm:a=b}
Let $G$ be a torsion-free nilpotent group and $a,b\in G$. If there exists 
$n\ne 0$ such that $a^n=b^n$, then $a=b$.
\end{theorem}

\begin{proof}
We proceed by induction on the nilpotency order $c$ of $G$. 
The result clearly holds for abelian groups. Assume that $G$ is nilpotent of class $c\geq2$. 
Since $\langle a,[G,G]\rangle$ is a nilpotent subgroup of $G$ of class $<c$
and $bab^{-1}=[b,a]a\in \langle
a,[G,G]\rangle$, the inductive hypotehsis implies that 
$ba=ab$, as 
\[
	a^n=(bab^{-1})^n=b^n.
\]
Thus $(ab^{-1})^n=a^nb^{-n}=1$. Since $G$ has no torsion, we conclude that 
$a=b$.
\end{proof}

\begin{corollary}
Let $G$ be a torsion-free nilpotent group. If $x,y\in G$ are such that 
$x^ny^m=y^mx^n$ for some $n,m\ne 0$, then $xy=yx$.
\end{corollary}

\begin{proof}
Let $a=x$ and $b=y^nxy^{-n}$. Since $a^m=b^m$, the previous theorem implies that 
$a=b$. Thus $xy^n=y^nx$. Apply the previous theorem again, this time with 
$a=y$ and $b=xyx^{-1}$. Then we conclude that 
$xy=yx$. 
\end{proof}

Before proving another theorem, we recall a basic 
lemma about finitely generated groups. 

\begin{lemma}
\label{lem:fg}
Let $G$ be a finitely generated group and 
$H$ a finite-index subgroup. 
Then $H$ is finitely generated. 
\end{lemma}

\begin{proof}
Assume that $G$ is generated by $\{g_1,\dots,g_m\}$. Without loss of generality, we may assume that 
for each $i$ there exists $k$ such that $g_i^{-1}=g_k$. 
	
Let $\{1=t_1,\dots,t_n\}$ be a transversal of $H$ in $G$, that is 
a complete set of representatives of $G/H$. For $i\in\{1,\dots,n\}$ 
and $j\in\{1,\dots,m\}$, write 
\[
t_ig_j=h(i,j)t_{k(i,j)}.
\]
We claim that $H$ is generated by the $h(i,j)$. Let $x\in H$. Then  
	\begin{align*}
	x &=g_{i_1}\cdots g_{i_s}\\
	&= (t_1g_{i_1})g_{i_2}\cdots g_{i_s}\\
	&= h(1,i_1)t_{k_1}g_{i_2}\cdots g_{i_s}\\
	&= h(1,i_1)h(k_1,i_2)t_{k_2}g_{i_3}\cdots g_{i_s}\\
	&= h(1,i_1)h(k_1,i_2)\cdots h(k_{s-1},i_s)t_{k_s},
	\end{align*}
where $k_1,\dots,k_{s-1}\in\{1,\dots,n\}$. Since $t_{k_s}\in H$ (because $x\in H$), 
$t_{k_s}=1\in H$. Hence  $x$ is generated by the $h(i,j)$.
\end{proof}

Now the theorem:

\begin{theorem}
\label{thm:T(G)finito}
Let $G$ be a finitely generated torsion group that is nilpotent. 
Then $G$ is finite 
\end{theorem}

\begin{proof}
We proceed by induction on the nilpotency class $c$ of $G$. The case
$c=1$ is true, as $G$ is abelian. So assume the result holds for 
groups of class $c\geq1$. Since $[G,G]$ and $G/[G,G]$ are 
finitely generated (Lemma \ref{lem:fg}) torsion 
nilpotent groups of class  
$<c$, the inductive hypothesis implies that
$[G,G]$ and $G/[G,G]$ are finite groups. Thus 
$G$ is finite. % of order $|[G,G]|(G:[G,G])$.
\end{proof}

%\begin{lemma}
%	\label{lemma:kgenerators}
%	Sea $G$ un grupo y sea $G=G_0\supseteq G_1\supseteq\cdots\supseteq G_k=1$
%	una sucesión de subgrupos de $G$ tal que cada $G_{i+1}$ es normal en $G_i$
%	y cada $G_{i}/G_{i+1}$ es cíclico. Todo subgrupo de $G$ es finitamente
%	generado por $k$ elementos.
%\end{lemma}
%
%\begin{proof}
%	Procedemos por inducción en $k$. Supongamos primero que $k=1$. Entonces
%	$G\simeq G_0/G_1$ es cíclico y luego todo subgrupo de $G$ está generado por
%	un elemento. Supongamos ahora que el resultado es válido para $k\geq1$. Sea
%	$H$ un subgrupo de $G$, sea $N=G_{1}$ y sea $\pi\colon G\to G/N$ el
%	morfismo canónico. El grupo 
%	\[
%		\pi(H)\simeq H/H\cap N
%	\]
%	es cíclico pues un un subgrupo del grupo cíclico $G_k/G_{k-1}=G/N$. Como
%	existe $h\in H$ tal que $\pi(H)$ está generado por $\pi(h)$, se concluye que 
%	$H=\langle \pi(h),H\cap N\rangle$. Por hipótesis
%	inductiva, $H\cap N$ está generado por $k-1$ elementos y luego $H$ está
%	generado por $k$ elementos.
%\end{proof}
%
%\begin{theorem}
%	Sea $G$ un grupo nilpotente y finitamente generado. Entonces $T(G)$ es
%	finito.
%\end{theorem}
%
%%%% aca hay que hacer producto tensorial para construir una serie con factores cíclicos
%%%% ver libro de Khukhro
%%%% Nilpotent Groups and Their Automorphisms
%\begin{proof}
%	Sabemos por el teorema~\ref{theorem:} que existe una sucesión
%	$G=G_0\supseteq G_1\supseteq\cdots\supseteq G_k=G$ de subgrupos normales de
%	$G$ con factores cíclicos. 
%\end{proof}




%\subsection{Grupos finitos nilpotentes}
\subsection{Finite nilpotent groups}

Before studying finite nilpotent groups, we need a lemma. 

\begin{lemma}
\label{lem:normalizador}
Let $G$ be a finite group and $p$ a prime number dividing $|G|$. 
If 
$P\in\Syl_p(G)$, then 
\[
N_G(N_G(P))=N_G(P). 
\]
\end{lemma}

\begin{proof}
Let $H=N_G(P)$. Since $P$ is normal in $H$, $P$ is the only Sylow $p$-subgroup of $H$. 
To prove that $N_G(H)=H$, it is enough to see that $N_G(H)\subseteq
H$. Let $g\in N_G(H)$. Since  
\[
gPg^{-1}\subseteq gHg^{-1}=H,
\]
$gPg^{-1}\in\Syl_p(H)$ and $H$ has only one Sylow $p$-subgroup, 
$P=gPg^{-1}$.  Hence $g\in N_G(P)=H$. 
\end{proof}

\begin{theorem}
\label{thm:nilpotente:eq}
Let $G$ be a finite group. The following statements are equivalent:
\begin{enumerate}
	\item $G$ is nilpotent. 
	\item Every Sylow subgroup of $G$ is normal in $G$. 
	\item $G$ is a direct product of its Sylow subgroups. 
\end{enumerate}
\end{theorem}

\begin{proof}
We first prove that $(1)\implies(2)$. Let $P\in\Syl_p(G)$. We aim to prove that $P$ 
is normal in $G$, that is $N_G(P)=G$. By Lemma \ref{lem:normalizador}, 
$N_G(N_G(P))=N_G(P)$. Now the normalizer condition (Lemma \ref{lem:normalizadora}) implies that 
$N_G(P)=G$.

We now prove that $(2)\implies(3)$. Let $p_1,\cdots,p_k$ be the prime factors of 
$|G|$. For each $i\in\{1,\dots,k\}$, let  $P_i\in\Syl_{p_i}(G)$.
By assumption, each $P_j$ is normal in $G$.

We claim that $P_1\cdots P_j\simeq P_1\times\cdots\times P_j$ for all $j$.
The case $j=1$ is trivial. So assume the result holds for some 
$j\geq 1$. Since 
\[
N=P_1\cdots P_j\simeq P_1\times\cdots\times P_j
\]
is normal in $G$ and it has order coprime with $|P_{j+1}|$, 
\[
N\cap P_{j+1}=\{1\}.
\]
Hence 
\[
	NP_{j+1}\simeq N\times P_{j+1}\simeq P_1\times\cdots\times P_j\times P_{j+1}, 
\]
as $P_{j+1}$ is normal in $G$. 
Since now $P_1\cdots P_k\simeq P_1\times\cdots\times P_k$ is a subgroup of 
$G$ of order $|G|$, we conclude that $G=P_1\times\cdots\times P_k$.

Finally, we prove that $(3)\implies(1)$. We just need to note that 
every 
$p$-group is nilpotent (Proposition~\ref{pro:nilpotent_pgroups}) and that the direct product
of nilpotent groups is nilpotent. 
%(ejercicio~\ref{exercise:HxK_nilpotente}).
\end{proof}

\begin{exercise}
\label{xca:truco}
Let $G$ be a finite group. Prove that if $P\in\Syl_p(G)$ and $M$ is a subgroup of $G$ such that 
$N_G(P)\subseteq M$, then $M=N_G(M)$. 
\end{exercise}

% \begin{svgraybox}
% 	Sea $x\in N_G(M)$. Como $P\subseteq M$ y $M$ es normal en $N_G(M)$,
% 	$xPx^{-1}\subseteq M$.  Como $P$ y $xPx^{-1}$ son $p$-subgrupos de Sylow de
% 	$M$, existe $m\in M$ tal que 
% 	\[
% 	mPm^{-1}=xPx^{-1}.
% 	\]
% 	Luego $x\in M$ pues
% 	$m^{-1}x\in N_G(P)\subseteq M$. 
% \end{svgraybox}

\begin{exercise}
\label{xca:normalizadora}
Let $G$ be a finite group. Prove that the following statements are equivalent:
\begin{enumerate}
	\item $G$ is nilpotent.
	\item If $H\subsetneq G$ is a subgroup of $G$, then $H\subsetneq N_G(H)$.
	\item Every maximal subgroup of $G$ is normal in $G$.
\end{enumerate}
\end{exercise}

% \begin{svgraybox}
% 	Para demostrar que $(1)\implies(2)$ simplemente usamos el
% 	lema~\ref{lemma:normalizadora}. Para demostrar que $(2)\implies(3)$ hacemos
% 	lo siguiente: si $M$ es un subgrupo maximal, como $M\subsetneq N_G(M)$ por
% 	hipótesis, $N_G(M)=G$ por maximalidad. Finalmente demostremos que
% 	$(3)\implies(1)$.  Sea $P\in\Syl_p(G)$. Si $P$ no es normal en $G$,
% 	$N_G(P)\ne G$ y entonces existe un subgrupo maximal $M$ tal que
% 	$N_G(P)\subseteq M$. Como $M$ es normal en $G$, el
% 	ejercicio~\ref{exercise:truco} implica que $M=N_G(M)=G$, una contradicción.
% 	Luego $P$ es normal en $G$ y entonces $G$ es nilpotente por el
% 	teorema~\ref{theorem:nilpotente:eq}.
% \end{svgraybox}

% ejercicio: G finito. Es nilpotente si y solo si dos elementos de ordenes coprimos conmnutan
% 5.41 rotman

\begin{theorem}
Let $G$ be a finite nilpotent group. If $p$ is a prime number dividing 
$|G|$, there exist a minimal normal subgroup of order $p$ and 
there exists a maximal subgroup of index $p$. 
\end{theorem}

\begin{proof}
Assume that $|G|=p^{\alpha}m$ with $\gcd(p,m)=1$. 
Write $G=P\times H$, where $P\in\Syl_p(G)$.  Since $Z(P)$ is a non-trivial normal subgroup of
$P$, every subgroup of $Z(P)$ that is minimal normal in $G$ has order $p$ (and such subgroups exist because $G$ is finite). Since $P$ contains a subgroup of index $p$, 
it is maximal. Hence $P\times H$ contains 
a maximal subgroup of index $p$.
\end{proof}

\begin{exercise}
\label{xca:pgrupos}
Let $p$ be a prime number and $G$ be a non-trivial group 
of order $p^n$.
Prove the following statements:
\begin{enumerate}
	\item $G$ has a normal subgroup of order $p$.
	\item For every $j\in\{0,\dots,n\}$ there exists a normal subgroup 
            of $G$ of order $p^j$. 
\end{enumerate}
\end{exercise}

% \begin{svgraybox}
% 	\begin{enumerate}
% 		\item Sabemos que $Z(G)\ne1$. Sea $g\in Z(G)$ tal que $g\ne 1$.
% 			Supongamos que el orden de $g$ es $p^k$ para algún $k\geq1$.
% 			Entonces $g^{p^{k-1}}$ tiene orden $p$ y luego genera un subgrupo
% 			central de orden $p$. 
% 		\item Procederemos por inducción en $n$. Si $n=1$ el resultado es
% 			trivial.  Supongamos entonces que el resultado vale para un cierto
% 			$n\geq2$. Por el punto anterior, $G$ posee un subgrupo normal $N$
% 			de orden $p$. Luego $G/N$ tiene orden $p^{n-1}$. Sea $\pi\colon G\to G/N$ el morfismo canónico. 
% 			Por hipótesis
% 			inductiva, para cada $j\in\{0,\dots,n-1\}$. Por el teorema de la
% 			correspondecia, cada subgrupo normal $S_j$ de $G/N$ de orden $p^j$ se
% 			corresponde con un subgrupo $\pi^{-1}(S_j)$ de $G$ de orden $p^{j+1}$ pues, como
% 			$\pi$ es sobreyectiva, se tiene $\pi(\pi^{-1}(S_j))=S_j$, y luego
% 			\[
% 			p^j=|S_j|=|\pi(\pi^{-1}(S_j))|=\frac{|\pi^{-1}(S_j)|}{|\pi^{-1}(S_j)\cap N|}=\frac{|\pi^{-1}(S_j)|}{|N|}=\frac{|\pi^{-1}(S_j)|}{p}.
% 			\]
% 	\end{enumerate}
% \end{svgraybox}

\begin{exercise}
\label{xca:nilpotente_equivalencia}
Let $G$ be a finite group. Prove that the following statements are equivalent:
\begin{enumerate}
	\item $G$ is nilpotent.
	\item Any two elements of coprime order commute. 
 	\item Every non-trivial quotient of $G$ has a non-trivial center.
	\item If $d$ divides $|G|$, then there exists a normal subgroup of $G$ of order $d$. 
 \end{enumerate}
\end{exercise}

% \begin{svgraybox}
% 	Veamos que $(1)\implies(2)$. Sabemos que $G$ es producto directo de sus
% 	subgrupos de Sylow, digamos $G=\prod_{i=1}^k S_i$, donde los $S_i$ son los
% 	distintos subgrupos de Sylow de $G$.  Sean
% 	$x=(x_1,\dots,x_k),y=(y_1,\dots,y_k)\in G$. Como $|x|$ y $|y|$ son
% 	coprimos, para cada $i\in\{1,\dots,k\}$ se tiene $x_i=1$ o $y_i=1$. Luego
% 	\[
% 		[x,y]=([x_1,y_1],[x_2,y_2],\dots,[x_k,y_k])=1. 
% 	\]
% 	Demostremos ahora que $(2)\implies(1)$. Supongamos que
% 	$|G|=p_1^{\alpha_1}\cdots p_k^{\alpha_k}$, donde los $p_j$ son primos
% 	distintos y para cada $j$ sea $P_j\in\Syl_{p_j}(G)$. Como elementos de
% 	órdenes coprimos conmutan, la función $P_1\times\cdots\times P_k\to G$,
% 	$(x_1,\dots,x_k)\mapsto x_1\cdots x_k$, es un morfismo inyectivo de grupos.
% 	Como entonces $G\simeq P_1\times\cdots P_k$, y cada $P_j$ es nilpotente,
% 	$G$ es nilpotente. 

% 	Para demostrar que $(1)\implies(3)$ simplemente hay que observar que todo
% 	cociente de $G$ es nilpotente y luego utilizar el
% 	teorema~\ref{theorem:Z(nilpotent)}. Demostremos que $(3)\implies(1)$. Como
% 	todo cociente no trivial de $G$ tiene centro no trivial, en particular
% 	$Z_1=Z(G)$ es no trivial. Si $Z_1=G$ entonces $G$ es abeliano y no hay nada
% 	para demostrar. Si $Z_1\ne G$ entonces $G/Z_1\ne 1$; luego $Z(G/Z_1)\ne 1$.
% 	Si $\pi_1\colon G\to G/Z_1$ es el morfismo canónico,
% 	$Z_2=\pi_1^{-1}(Z(G/Z_1))$. Inductivamente, si tenemos construido el
% 	subgrupo $Z_i$, $Z_i\ne G$ y  $\pi_i\colon G\to G/Z_{i}$ es el morfismo
% 	canónico, se define el subgrupo $Z_{i+1}=\pi_i^{-1}(Z(G/Z_i))$. Por
% 	construcción, $Z_i\subseteq Z_{i+1}$ para todo $i$. Como $G$ es finito,
% 	existe $k$ tal que $Z_k=G$ y luego $G$ es nilpotente.

% 	Demostremos que $(1)\implies(4)$. Esta implicación es consecuencia
% 	inmediata del ejercicio~\ref{exercise:pgrupos}. 
% 	Como $G$ es nilpotente, $G$ producto
% 	directo de sus $p$-grupos de Sylow. Si $d=p_1^{\alpha_1}\cdots
% 	p_k^{\alpha_k}$ es un divisor del orden de $G$, basta tomar
% 	$H=H_1\times\cdots\times H_k$, 
% 	donde cada $H_j$ es un subgrupo normal del $p_j$-subgrupo de Sylow de $G$
% 	de orden $p_j^{\alpha_j}$. Para demostrar que $(4)\implies(1)$ simplemente
% 	se aplica la hipótesis a cada $p$-subgrupo de $G$ de orden maximal.
% \end{svgraybox}

\subsection{Baumslag--Wiegold theorem}

The following result can be proved with elementary tools 
and was discovered 
in 2014. 

\begin{theorem}[Baumslag--Wiegold]
\index{Baumslag--Wiegold theorem}
Let $G$ be a finite group such that $|xy|=|x||y|$ for all $x,y\in G$ of coprime orders. 
Then $G$ is nilpotent. 
\end{theorem}

\begin{proof}
Let $p_1,\dots,p_n$ be the prime factors of $|G|$. For 
each 
$i\in\{1,\dots,n\}$, let $P_i\in\Syl_{p_i}(G)$. We first prove that 
$G=P_1\cdots P_n$. To prove the non-trivial inclusion, we need to show that 
the map
\[
	\psi\colon P_1\times\cdots\times P_n\to G,\quad
	(x_1,\dots,x_n)\mapsto x_1\cdots x_n
\]
is surjective. We first show that $\psi$ is injective: If 
$\psi(x_1,\dots,x_n)=\psi(y_1,\dots,y_n)$, then 
\[
x_1\cdots x_n=y_1\cdots y_n. 
\]
If $y_n\ne x_n$, then $x_1\cdots x_{n-1}=(y_1\cdots
y_{n-1})y_nx_n^{-1}$. Since $x_1\cdots x_{n-1}$ has order coprime with 
$p_n$ and $y_1\cdots y_{n-1}y_nx_n^{-1}$ has order a multiple of 
$p_n$, we get a contradiction. Thus  $x_n=y_n$. The same argument shows that 
$\psi$ is injective. Since $|P_1\times\cdots\times
P_n|=|G|$, we conclude that $\psi$ is bijective. In particular, 
$\psi$ is surjective. 

We now prove that each $P_j$ is normal in $G$. Let $j\in\{1,\dots,n\}$ and 
$x_j\in P_j$. Let $g\in G$ and $y_j=gx_jg^{-1}$.  Since $y_j\in G$,
we can write $y_j=z_1\cdots z_n$ with $z_k\in P_k$ for all $k$.  Since
the order of $y_j$ is a power of $p_j$, the element $z_1\cdots
z_n$ has order a power of $p_j$. Thus $z_k=1$ for all $k\ne j$. Moreover, 
$y_j=z_j\in P_j$. Since every Sylow subgroup of $G$ is normal in $G$, 
we conclude that $G$ is nilpotent. 
\end{proof}

\subsection{It\^o's theorem}

\begin{definition}
\index{Grup!metabelian}
A group $G$ is said to be \textbf{metabelian} if $[G,G]$ is abelian. 
\end{definition}

\begin{exercise}
\label{xca:metabelian1}
Prove that a group $G$ is metabelian if and only if there exists a normal 
subgroup $K$ of $G$ such that $K$ and $G/K$ are abelian.
\end{exercise}

% \begin{remark}
% 	Los grupos metabelianos son solvables pues 
% 	si $G$ is metabeliano entonces 
% 	$G\supseteq [G,G]\supseteq 1$ is una serie solvable para $G$.
% \end{remark}

\begin{exercise}
\label{xca:metabelian2}
Let $G$ be a metabelian group. Prove the following statements: 
\begin{enumerate}
\item If $H$ is a subgroup of $G$, then $H$ is metabelian.
\item If $f\colon G\to H$ is a group homomorphism, then $f(H)$ is metabelian. 
\end{enumerate}
\end{exercise}

\begin{lemma}
In a group, the following formulas hold:
\begin{enumerate}
	\item $[a,bc]=[a,b]b[a,c]b^{-1}$. 
	\item $[ab,c]=a[b,c]a^{-1}[a,c]$.
\end{enumerate}
\end{lemma}

\begin{proof}
This is a straightforward calculation:
\begin{align*}
&[a,b]b[a,c]b^{-1}=aba^{-1}b^{-1}baca^{-1}c^{-1}b^{-1}=abca^{-1}c^{-1}b^{-1}=[a,bc],\\
&a[b,c]a^{-1}[a,c]=abcb^{-1}c^{-1}a^{-1}aca^{-1}c^{-1}=abcb^{-1}a^{-1}c^{-1}=[ab,c].\qedhere
\end{align*}
\end{proof}

\begin{example}
The group $\Sym_3$ is metabelian, as $\Alt_3\simeq C_3$ is a normal subgroup 
and the quotient $\Sym_3/\Alt_3\simeq C_2$ an abelian group. 
\end{example}

\begin{example}
The group $\Alt_4$ is metabelian, as the normal subgroup
\[
K=\{\id,(12)(34),(13)(24),(14)(23)\}
\]
is abelian and the quotient 
$\Alt_4/K\simeq C_3$ is abelian.
\end{example}

\begin{example}
The group $\SL_2(3)$ is not metabelian, as $[\SL_2(3),\SL_2(3)]\simeq Q_8$ 
is not abelian: 
\begin{lstlisting}
gap> IsAbelian(DerivedSubgroup(SL(2,3)));
false
gap> StructureDescription(DerivedSubgroup(SL(2,3)));
"Q8"
\end{lstlisting}
\end{example}

\begin{theorem}[It\^o]
\label{theorem:Ito}
Let $G=AB$ be a factorization of $G$ with $A$ and $B$ abelian 
subgroups of $G$. Then $G$ is metabelian.
\end{theorem}

\begin{proof}
Since $G=AB$ is a group, $AB=BA$. We claim that $[A,B]$ is a normal subgroup 
of $G$. Let $a,\alpha\in A$ and $b,\beta\in B$. Let  $a_1,a_2\in A$ and 
	$b_1,b_2\in B$ be such that $\alpha b\alpha^{-1}=b_1a_1$, $\beta
	a\beta^{-1}=a_2b_2$. Since 
	\begin{align*}
		&\alpha[a,b]\alpha^{-1}=a(\alpha b\alpha^{-1})a^{-1}(\alpha b^{-1}\alpha^{-1})=ab_1a_1a^{-1}a_1^{-1}b_1^{-1}=[a,b_1]\in [A,B]\\
		&\beta[a,b]\beta^{-1}=(\beta a\beta^{-1})\beta b\beta^{-1}(\beta a^{-1}\beta^{-1})b^{-1}=a_2b_2bb_2^{-1}a_2^{-1}b^{-1}=[a_2,b]\in [A,B],
	\end{align*}
	it follows that $[A,B]$ is normal in $G$. 

	We now claim that $[A,B]$ is abelian. Since 
	\begin{align*}
		&\beta\alpha[a,b]\alpha^{-1}\beta^{-1} = \beta[a,b_1]\beta^{-1}=(\beta a\beta^{-1})b_1(\beta a^{-1}\beta^{-1})b_1^{-1}=[a_2,b_1],\\
		&\alpha\beta[a,b]\beta^{-1}\alpha^{-1} = \alpha[a_2,b]\alpha^{-1}=a_2(\alpha b\alpha^{-1})a_2^{-1}(\alpha b\alpha^{-1})=[a_2,b_1],
	\end{align*}
	a direct calculation shows that 
	\[
		[\alpha^{-1},\beta^{-1}][a,b][\alpha^{-1},\beta^{-1}]^{-1}=[a,b].
	\]
	Two arbitrary generators of $[A,B]$ commute, so the group $[A,B]$ is abelian. 
	
	To finish the proof, note that $[G,G]=[A,B]$. In fact, 
	\[
	[a_1b_1,a_2b_2]=a_1[a_2,b_1]^{-1}a_1^{-1}a_2[a_1,b_2]a_2^{-1}\subseteq [A,B],
	\]
	as $[A,B]$ is normal in $G$. 
\end{proof}

In 1988 Sysak proved the following generalization 
of It\^o's theorem.  

\begin{theorem}[Sysak]
\index{Sysak's theorem}
    Let $A$ and $B$ be abelian subgroups of $G$. If $H$ is a subgroup of 
    $G$ contained in 
    $AB$, then $H$ is metabelian. 
\end{theorem}

For the proof, see \cite{MR988177}.

\subsection{Nilpotent groups of class two}

The following exercises go over groups 
of nilpotency class two. 

%\subsection{Grupos nilpotentes de clase dos}

\begin{exercise}
\label{xca:commutador}
Let $G$ be a group. Prove that 
if $x,y\in G$ are such that $[x,y]\in C_G(x)\cap C_G(y)$, then 
\[
[x,y]^n=[x^n,y]=[x,y^n]
\]
for all $n\in\Z$.
\end{exercise}

% \begin{proof}
% 	Procederemos por inducción en $n\geq0$. El caso $n=0$ es trivial. Supongamos entonces
% 	que el resultado vale para algún $n\geq0$. Entonces, como $[x,y]\in C_G(x)$, 
% 	\begin{align*}
% 		[x,y]^{n+1}&=[x,y]^n[x,y]
% 		=[x^n,y][x,y]=[x^n,y]xyx^{-1}y^{-1}=x[x^n,y]yx^{-1}y^{-1}=[x^{n+1},y].
% 	\end{align*}
% 	Para demostrar el lema en el caso $n<0$ basta observar que, como $[x,y]^{-1}=[x^{-1},y]$, 
% 	$[x,y]^{-n}=[x^{-1},y]^n=[x^{-n},y]$.
% \end{proof}

\begin{exercise}[Hall]
\label{xca:Hall}
Let $G$ be a class-two nilpotent group and 
$x,y\in G$. Prove that 
\[
(xy)^n=[y,x]^{n(n-1)/2}x^ny^n
\]
for all $n\geq1$.
\end{exercise}

% \begin{proof}
% 	Procederemos por inducción en $n$. Como el caso $n=1$ es trivial,
% 	supongamos que el resultado es válido para algún $n\geq1$. Entonces,
% 	gracias al lema anterior, 
% 	\begin{align*}
% 		(xy)^{n+1} &= (xy)^n(xy)=[y,x]^{n(n-1)/2}x^ny^{n-1}(yx)y\\
% 		&=[y,x]^{n(n-1)/2}x^{n}[y^n,x]xy^{n+1}=[y,x]^{n(n-1)/2}[y,x]^nx^{n+1}y^{n+1}.\qedhere 
% 	\end{align*}
% \end{proof}

\begin{exercise}
\label{xca:class2}
Let $p$ be an odd prime number and 
$P$ $p$-group of nilpotency class $\leq2$. 
Prove that if $[y,x]^p=1$ for all $x,y\in P$, then
$P\to [P,P]$,
$x\mapsto x^p$, is a group homomorphism. 
\end{exercise}

% \begin{proof}
% 	Por lema de Hall,
% 	$(xy)^p=[y,x]^{p(p-1)/2}x^py^p=x^py^p$. 	
% \end{proof}

\begin{exercise}
\label{xca:class2_torsion}
Let $p$ be an odd prime number and 
$P$ a $p$-group of nilpotency class $\leq2$. 
Prove that $\{x\in P:x^p=1\}$ is a subgroup of $P$.
\end{exercise}

% \begin{proof}
% 	Como $P$ tiene clase de nilpotencia dos, los conmutadores son centrales.
% 	Para cada $x\in G$, la función $g\mapsto [g,x]$ es un morfismo de grupos
% 	pues
% 	\[
% 		[gh,x]=ghxh^{-1}g^{-1}x^{-1}=g[h,x]xg^{-1}x^{-1}=[g,x][h,x].
% 	\]
% 	En particular, si $x,y\in P$ con $x^p=y^p=1$, entonces
% 	\[
% 		[x,y]^p=[x^p,y]=[1,y]=1.
% 	\]
% 	Luego, al usar el lema de Hall, se concluye que
% 	$(xy)^p=[y,x]^{p(p-1)/2}x^py^p=1$.
% \end{proof}
