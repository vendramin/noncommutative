\chapter{}

\topic{Braces and regular subgroups}

\index{Holomorph}
For an additive group $A$, the \textbf{holomorph} of $A$ is the semidirect product 
\[
	\Hol(A)=A\rtimes\Aut(A).
\]
This means that the operation is 
\[
	(a,f)(b,g)=(a+f(b),f\circ g),\quad
	a,b\in A,\quad
	f,g\in\Aut(A).
\]
%Write $\pi_1\colon\Hol(A)\to A$, $\pi_1(a,f)=a$ and $\pi_2\colon\Hol(A)\to\Aut(A)$, $\pi_2(a,f)=f$. 
Every subgroup $G$ of $\Hol(A)$ acts on $A$ by 
\[
	(x,f)\cdot a=\pi_1( (x,f)(a,\id) )=\pi_1(x+f(a),f)=x+f(a),\quad
	a,x\in A,\quad 
	f\in\Aut(A),
\]
where $\pi_1\colon \Hol(A)\to A$, $(a,f)\mapsto a$. 

\begin{exercise}
    The group $\Hol(A)$ acts transitively on $A$ 
	and the stabilizer $a\in A$ 
	is isomorphic to $\Aut(A)$. 
\end{exercise}

\index{Regular subgroup}
A subgroup $G$ of $\Hol(A)$ is said to be \emph{regular} 
if its acts regularly on $A$, this means that 
given $a,b\in A$ there exists a unique $(x,f)\in G$ such that
\[
	b=(x,f)\cdot a=x+f(a).
\]

\begin{exercise}
    \label{xca:bijective}
	Prove that 
	if $G$ is a regular subgroup of $\Hol(A)$, then $\pi_1\colon G\to A$ is
	bijective. 
\end{exercise}

Now we establish an important connection between braces and regular subgroups. 

\begin{theorem}
\label{thm:regular}
	If $A$ is a brace, then $\Delta=\{(a,\lambda_a):a\in A\}$ is a regular subgroup of 
	$\Hol(A,+)$. Conversely, if $A$ is an additive group and 
	$G$ is a regular subgroup of $\Hol(A)$, then $A$ is 
	a brace with 
	\[
		a\circ b=a+f(b),
	\]
	where $(\pi_1|_G)^{-1}(a)=(a,f)\in G$. 
\end{theorem}

\begin{proof}
    Assume first that $A$ is a brace. 
    Using~\eqref{eq:formulas} and that 
    $\lambda$ is a group homomorphism, it follows that
    $\Delta=\{(a,\lambda_a):a\in A\}$ is a subgroup of $\Hol(A,+)$, as 
    \begin{align*}
    &(a,\lambda_a)^{-1}=(\lambda^{-1}_a(-a),\lambda^{-1}_a)=(a',\lambda_{a'})\in\Delta,\\
    &(a,\lambda_a)(b,\lambda_b)=(a+\lambda_a(b),\lambda_a\circ\lambda_b)=(a\circ b,\lambda_{a\circ b})\in\Delta.    
    \end{align*}
    To see that $\Delta$ is a regular subgroup, note that $(c,\lambda_c)\cdot a=b$ 
    implies that $c=b\circ a'$, as $(A,\circ)$ is a group. 

	Assume now that $A$ is an additive group and that 
	$G$ is a regular subgroup of $\Hol(A)$. By Exercise~\ref{xca:bijective}, the restriction 
	$\pi_1|_G$ is bijective.  Use the bijection $\pi_1|_G$ to transport the
	operation of $G$ into $A$: 
	\[
        a\circ b=\pi_1|_G\left((\pi_1|_G)^{-1}(a)(\pi_1|_G)^{-1}(b)\right)=a+f(b),
	\]
	where $a,b\in A$ and $(\pi_1|_G)^{-1}(a)=(a,f)\in G$. 
	Then $(A,\circ)$ is a group isomorphic to $G$ and moreover $A$ is a brace, as 
	\begin{align*}
		a\circ (b+c)&=a+f(b+c)=a+f(b)+f(c)\\
		&=a+f(b)-a+a+f(c)=a\circ b-a+a\circ c
	\end{align*}
	holds for all $a,b,c\in A$. 
\end{proof}

%The following lemma is from~\cite{MR4113853}.

\begin{lemma}
\label{lem:BNY}
	Let $A$ be a group. If $H$ and $K$ are conjugate regular subgroups of
	$\Hol(A)$, then $H$ and $K$ are conjugate by an automorphism of $A$. 
\end{lemma}

\begin{proof}
    Assume that $H$ and $K$ are conjugate in $\Hol(A)$. Let $(b,g)\in\Hol(A)$ be such that $(b,g)^{-1}H(b,g)=K$. 
    Since $b\in A$, the regularity of $H$ implies that
    there exists $(a,f)\in H$ such that $a+f(b)=0$. Since $(a,f)\in H$, 
    \begin{align*}
    K=(b,g)^{-1}H(b,g)&=(b,g)^{-1}(a,f)^{-1}H(a,f)(b,g)\\
    &=(0,f\circ g)^{-1}H(0,f\circ g)=(f\circ g)^{-1}H(f\circ g).\qedhere
    \end{align*}
    %Hence $H$ and $K$ are conjugate by an automorphism of $A$.
\end{proof}

\begin{proposition}
\label{pro:regular}
	Let $A$ be an additive group.  There exists a bijective correspondence between
	isomorphism classes of 
	brace structures with additive group $A$ and conjugacy classes of regular subgroups of
	$\Hol(A)$. 

	\begin{proof}
        Assume that the additive group $A$ has two isomorphic brace structures given by
		$(a,b)\mapsto a\circ b$ and $(a,b)\mapsto a\times b$. Let 
		$f\colon A\to A$ be a bijective map such that $f(a+b)=f(a)+f(b)$ and $f(a\circ b)=f(a)\times f(b)$ for
		all $a,b\in A$. 
		We claim that the regular subgroups $\{(a,\lambda_a):a\in A$\} and
		$\{(a,\mu_a):a\in A\}$, where $\lambda_a(b)=-a+a\circ b$ and
		$\mu_a(b)=-a+a\times b$, are conjugate. 
		Since $f$ is an isomorphism of braces, 
		\[
		f\circ\lambda_a\circ f^{-1}=\mu_{f(a)}
		\]
		for all $a\in A$. This implies that $(0,f)(a,\lambda_a)(0,f)^{-1}=(f(a),\mu_{f(a)})$ 
		for all $a\in A$ and hence 
		and the first claim follows. 
		
		Conversely, let $H$ and $K$ be conjugate regular subgroups of $\Hol(A)$. Since $H$ and $K$ are conjugate in $\Hol(A)$, by 
		Lemma~\ref{lem:BNY} there exists $\varphi\in\Aut(A)$ such that $\varphi
		H\varphi^{-1}=K$. The brace structure on $A$ corresponding to the subgroup $H$ is given by 
		$a\circ b=a+f(b)$,
		where
		$(a,f)=(\pi_1|_H)^{-1}(a)\in H$, see Lemma~\ref{lem:bijective}. 
		Since 
        \[
        \varphi(f,a)\varphi^{-1}=(\varphi(a),\varphi\circ f\circ \varphi^{-1})\in K,
        \]
        it follows that 
        $(\pi_1|_K)^{-1}(\varphi(a))=(\varphi(a),\varphi\circ f\circ \varphi^{-1})$. Since $\varphi\in\Aut(A)$, 
        \begin{align*}
            \varphi(a)\times\varphi(b)&=\varphi(a)+(\varphi\circ f\circ\varphi^{-1})(\varphi(b))\\
            &=\varphi(a)+\varphi(f(b))=\varphi(a+f(b))=\varphi(a\circ b)
        \end{align*}
        and hence the braces corresponding to $H$ and $K$ are
		isomorphic.
	\end{proof}
\end{proposition}

With Theorem~\ref{thm:regular}, Lemma~\ref{lem:BNY} and
Proposition~\ref{pro:regular} one constructs small braces. In particular, the
method can be used to compute the number $s(n)$ of non-isomorphic braces of
size $n$. With small modifications, one computes the number $a(n)$ of
non-isomorphic braces of abelian type of size $n$. 
Some values for $s(n)$ and $a(n)$ appear in
Table~\ref{tab:braces}. 

\begin{table}[ht]
\centering
	\caption{The number of non-isomorphic braces.}
	\begin{tabular}{|c|cccccccccccc|}
		\hline
		$n$ & 1 & 2 & 3 & 4 & 5  & 6 & 7 & 8 & 9 & 10 & 11 & 12\tabularnewline
		$a(n)$ & 1  & 1  & 1  & 4  & 1  & 2  & 1  & 27  & 4  & 2  & 1  & 10\tabularnewline
		$s(n)$ & 1 & 1 & 1 & 4 & 1 & 6 & 1 & 47 & 4 & 6 & 1 & 38 \tabularnewline
		\hline
		$n$ & 13 & 14 & 15 & 16 & 17 & 18 & 19 & 20 & 21 & 22 & 23 & 24\tabularnewline
		$a(n)$ & 1  & 2  & 1  & 357  & 1  & 8  & 1  & 11  & 2  & 2  & 1  & 96\tabularnewline
		$s(n)$ & 1 & 6 & 1 & 1605 & 1 & 49 & 1 & 43 & 8 & 6 & 1 & 855 \tabularnewline
		\hline
		$n$ & 25 & 26 & 27 & 28 & 29 & 30 & 31 & 32 & 33 & 34 & 35 & 36\tabularnewline
		$a(n)$ & 4  & 2  & 37  & 9  & 1  & 4 & 1  & 25281  & 1  & 2  & 1  & 46\tabularnewline
		$s(n)$ & 4 & 6 & 101 & 29 & 1 & 36 & 1 & 1223061 &  1 & 6 & 1 & 400\tabularnewline
		\hline
		$n$ & 37 & 38 & 39 & 40 & 41 & 42 &  43 & 44 & 45 & 46 & 47 & 48\tabularnewline
		$a(n)$  & 1  & 2  & 2  & 106  & 1  & 6  & 1  & 9  & 4 & 2  & 1  & 1708\tabularnewline
		$s(n)$ & 1 & 6 & 8 & 944 & 1 & 78 & 1 & 29 & 4 & 6 & 1 & 66209 \tabularnewline
		\hline
		$n$    & 49 & 50 & 51 & 52 & 53 & 54 & 55 & 56 & 57 & 58 & 59 & 60\tabularnewline
		$a(n)$ &  4 & 8 & 1 & 11 & 1 & 80 & 2 & 91 & 2 & 2 & 1 & 28\tabularnewline
		$s(n)$ & 4 & 51 & 1 & 43 & 1 & 1028 &  12 & 815 & 8 & 6 & 1 & 418\tabularnewline

		\hline
		$n$ & 61 & 62 & 63 & 64 & 65 & 66 & 67 & 68 & 69 & 70 & 71 & 72\tabularnewline
		$a(n)$ & 1 & 2 & 11 & ? & 1 & 4 & 1 & 11 & 1 & 4 & 1 & 489\tabularnewline
		$s(n)$ & 1 & 6 & 11 & ? &  1 & 36 & 1 & 43 & 1 & 36 & 1 & 17790 \tabularnewline

		\hline
		$n$ & 73 & 74 & 75 & 76 & 77 & 78 & 79 & 80 & 81 & 82 & 83 & 84\tabularnewline
		$a(n)$ & 1 & 2 & 5 & 9 & 1 & 6 & 1 & 1985 & 804 & 2 & 1 & 34\tabularnewline
		$s(n)$ & 1 & 6 & 14 & 29 & 1 & 78 & 1 & 74120 & 8436 &  6 & 1 & 606\tabularnewline

		\hline
		$n$ & 85 & 86 & 87 & 88 & 89 & 90 & 91 & 92 & 93 & 94 & 95 & 96\tabularnewline
		$a(n)$ & 1 & 2 & 1 & 90 & 1 & 16 & 1 & 9 & 2 & 2 & 1 & 195971\tabularnewline
		$s(n)$ & 1 & 6 & 1 & 800 & 1 & 294 & 1 & 29 & 8 & 6 & 1 & ? \tabularnewline

		\hline
		$n$ & 97 & 98 & 99 & 100 & 101 & 102 & 103 & 104 & 105 & 106 & 107 & 108\tabularnewline
		$a(n)$ & 1 & 8 & 4 & 51 & 1 & 4 & 1 & 106 & 2 & 2 & 1 & 494 \tabularnewline
		$s(n)$ & 1 & 53 &  4 & 711 & 1 & 36 & 1 & 944 &  8 &  6 &  1 & 11223 \tabularnewline

		\hline
		$n$ & 109 & 110 & 111 & 112 & 113 & 114 & 115 & 116 & 117 & 118 & 119 & 120\tabularnewline
		$a(n)$ & 1 & 6 & 2 & 1671 & 1 & 6 & 1 & 11 & 11 & 2 & 1 & 395\tabularnewline
		$s(n)$ & 1 & 94 &  8 & 65485 &  1 & 78 &  1 & 43 &  47 &  6 &  1 & 22711 \tabularnewline
		\hline

		$n$ & 121 & 122 & 123 & 124 & 125 & 126 & 127 & 128 & 129 & 130 & 131 & 132\tabularnewline
		$a(n)$ & 4 & 2 & 1 &  9  & 49 & 36 & 1 & ? & 2 &  4 & 1 & 24\tabularnewline
		$s(n)$ & 4 & 6 & 1 & 29 & 213 & 990 & 1 & ? & 8 & 36 & 1 & 324 \tabularnewline
		\hline

		$n$  & 133 & 134 & 135 & 136 & 137 & 138 & 139 & 140 & 141 & 142 & 143 & 144\tabularnewline
		$a(n)$ & 1 & 2 & 37 & 108 & 1 & 4 & 1 & 27 & 1 & 2 & 1 & 10215\tabularnewline
		$s(n)$ & 1 & 6 & 101 & 986 & 1 & 36 & 1 & 395 & 1 & 6 & 1 & 3013486 \tabularnewline
		\hline
		
		$n$ & 145 & 146 & 147 & 148 & 149 & 150 & 151 & 152 & 153 & 154 & 155 & 156\tabularnewline
		$a(n)$ & 1 & 2 & 9 & 11 & 1 & 19 & 1 & 90 & 4 & 4 & 2 & 40\tabularnewline
		$s(n)$ & 1 & 6 & 123 & 43 & 1 & 401 & 1 & 800 & 4 & 36 & 12 & 782 \tabularnewline
		\hline

		$n$ & 157 & 158 & 159 & 160 & 161 & 162 & 163 & 164 & 165 & 166 & 167 & 168\tabularnewline
		$a(n)$ & 1 & 2 & 1 & 209513 & 1 & 1374 & 1 & 11 & 2 & 2 & 1 & 443\tabularnewline
		$s(n)$ & 1 & 6 & 1 & ? & 1 & 45472 & 1 & 43 & 12 & 6 & 1 & 28505 \tabularnewline
		\hline
	\end{tabular}
	\label{tab:braces}
\end{table}

\topic{A universal construction}



