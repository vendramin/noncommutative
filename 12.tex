\section{24/05/2024}

\subsection{*The Deaconescu--Walls theorem}

Let $A$ be a group acting on automorphisms on a finite group $G$. Then 
$C_{G}(A)=\{g\in G:a\cdot g=g\,\,\forall a\in A\}$ acts by left multiplication 
on the set of 
$A$-orbits by 
\[
  c(A\cdot g)
  =\{c(a\cdot g):a\in A\}
  =\{(a\cdot c)(a\cdot g):a\in A\}
  =\{a\cdot (cg):a\in A\}
  =A\cdot (cg)
\]
for all $g\in G$ and $c\in C_G(A)$.

%The following theorem first appeared in~\cite{MR2164558}. 
%The proof presented here goes back to Isaacs, see~\cite{MR2922681}. 

\begin{theorem}[Deaconescu--Walls]
	\index{Deaconescu--Walls theorem}
	\label{thm:DeaconescuWalls}
	Let $A$ be a group acting by automorphisms on a finite group $G$. Let
	$C=C_{G}(A)$ and $N=C\cap [A,G]$,
	where $[A,G]$ is the subgroup of $G$ generated by $[a,g]=(a\cdot g)g^{-1}$,
	$a\in A$, $g\in G$.  Then $(C:N)$ divides the number of $A$-orbits of 
	$G$. 
\end{theorem}

\begin{proof}
  The group $C$ acts by left multiplication on the set $\Omega$ of 
  $A$-orbits of $G$. Let $X=A\cdot g\in\Omega$ and $C_X$ be the stabilizer of 
  $C$ in $X$. If $c\in C_X$, then $cX=X$. In particular, if $c\in C_X$, then 
  $cg=a\cdot g$ for some $a\in A$, that is $c=(a\cdot
  g)g^{-1}=[a,g]\in [A,G]$. Thus $C_X\subseteq N$.

  To show that $(C:N)$ divides $|\Omega|$, it is enough to show that 
  $(C:N)$ divides the size of each $C$-orbit. If $X\in\Omega$, then $C\cdot
  X$ has size 
  \[
	(C:C_X)=(C:N)(N:C_X).
  \]
  Hence $(C:N)$ divides the size of the orbit $C\cdot X$.
\end{proof}

\begin{corollary}
	\label{cor:Z(G)subset[G,G]}
  Let $G$ be a non-trivial finite group with $k(G)$ conjugacy classes. 
  If the order of $Z(G)$ is coprime with $k(G)$, then  
  $Z(G)\subseteq[G,G]$.
\end{corollary}

\begin{proof}
	The group $A=G$ acts on $G$ by conjugation. Since $C_G(A)=Z(G)$ and 
	$[A,G]=[G,G]$, Theorem~\ref{thm:DeaconescuWalls} implies that the index 
	$(Z(G):Z(G)\cap [G,G])$ divides $k(G)$. Since $k(G)$ and $|Z(G)|$ are coprime, we conclude that $Z(G)=Z(G)\cap [G,G]\subseteq [G,G]$. 
\end{proof}

\begin{definition}
	\index{Central automorphism}
 	Let $G$ be a group and $f\in\Aut(G)$. We say that $f$ is \textbf{central} if 
	$f(x)x^{-1}\in Z(G)$ for all $x\in G$.
\end{definition}

An automorphism $f$ of a group $G$ 
is central if and only if $f\in C_{\Aut(G)}(\Inn(G))$.

\begin{corollary}
	Let $G$ be a finite group with $k(G)$ conjugacy classes and $c(G)$
	central automorphisms. If $\gcd(|G|,k(G)c(G))=1$, then 
	$[G,G]=Z(G)$.
\end{corollary}

\begin{proof}
	By Corollary~\ref{cor:Z(G)subset[G,G]}, $Z(G)\subseteq [G,G]$. Conversely, let 
	$A=C_{\Aut(G)}(\Inn(G))$. Since $|G|$ and $k(G)c(G)$ are coprime 
	and $(C_G(A):C_G(A)\cap [A,G])$ divides $c(G)$ by 
	Theorem~\ref{thm:DeaconescuWalls}, we obtain that $C_G(A)=C_G(A)\cap [A,G]$. 
	Since 
	\[
		a\cdot [x,y]=[(a\cdot x)x^{-1}x,(a\cdot y)y^{-1}y]=[x,y]
	\]
	for all $a\in A$ and $x,y\in G$, 
    $[G,G]\subseteq C_G(A)$. Moreover, 
    $[A,G]\subseteq Z(G)$. Thus 
	\[
	[G,G]\subseteq C_G(A)=C_G(A)\cap [A,G]\subseteq [A,G]\subseteq Z(G).\qedhere 
	\]
\end{proof}

\begin{exercise}
    Let $p$ be a prime number and $G$ be a group with $p$ conjugacy classes. 
    Prove that either $Z(G)\subseteq[G,G]$ or $|G|=p$. 
\end{exercise}

% \begin{proof}
%   Hacemos actuar a $G$ en $G$ por conjugación.  Como cada elemento de $C=Z(G)$
%   es una clase de conjugación, $|C|\leq p$. Si $|C|=p$ entonces $G=C=Z(G)$
%   tiene orden $p$. Si no, $|C|$ es coprimo con $p$ y luego $C\subseteq
%   N=[G,G]$.
% \end{proof}


\subsection{*The Chermak--Delgado subgroup}

\begin{definition}
\index{Chermak--Delgado!measure}
Let $G$ be a finite group and $H$ a subgroup of $G$. 
The \textbf{Chermak--Delgado measure} of $H$ 
is the number 
$m_G(H)=|H||C_G(H)|$.
\end{definition}

\begin{example}
If $G$ is abelian and $H$ is a subgroup of $G$, then 
$m_G(H)=|H||G|$.
\end{example}

\begin{example}
Let $G=\Sym_3$. The subgroups of $G$ are 
	\[
		H_0=1,\quad
		H_1=\langle (23)\rangle,\quad
		H_2=\langle (12)\rangle,\quad
		H_3=\langle (13)\rangle,\quad
		H_4=\langle (123)\rangle,\quad
		H_5=\Sym_3.
	\]
	A direct calculation shows that 
	\[
		m_G(H_j)=\begin{cases}
			6 & \text{if $j\in\{0,5\}$},\\
			4 & \text{if $j\in\{1,2,3\}$},\\
			9 & \text{if $j=4$}.
		\end{cases}
	\]
\end{example}

\begin{lemma}
\label{lem:CD1}
Let $G$ be a finite group and $H$ be a subgroup of $G$. Then 
\[
m_G(H)\leq m_G(C_G(H)).
\]
If the equality holds, then $H=C_G(C_G(H))$.
\end{lemma}

\begin{proof}
Let $C=C_G(H)$. 
Since $H\subseteq C_G(C)$, 
\[
m_G(C)=|C||C_G(C)|\geq |C||H|=m_G(H). 
\]
If $m_G(H)=m_G(C_G(H))$, then $|H|=|C_G(C_G(H))|$ and 
hence $H=C_G(C_G(H))$, as $H\subseteq C_G(C_G(H))$. 
\end{proof}

\begin{lemma}
	Let $G$ be a finite group and 
	$H$ and $,K$ be subgroups of $G$. Let $D=H\cap K$ and
    $J=\langle H,K\rangle$. Then 
	\[
		m_G(H)m_G(K)\leq m_G(D)m_G(J).
	\]
	If the equality holds, then $J=HK$ and $C_G(D)=C_G(H)C_G(K)$.
	\label{lem:CD2}
\end{lemma}

\begin{proof}
	Let $C_H=C_G(H)$, $C_K=C_G(K)$, $C_D=C_G(D)$, and $C_J=C_G(J)$. Then
	$C_J=C_H\cap C_K$ and $C_H\cup C_K\subseteq C_D$. Since 
	\[
		|J|\geq |HK|=\frac{|H||K|}{|D|},
		\quad
		|C_D|\geq |C_HC_K|=\frac{|C_H||C_K|}{|C_J|},
	\]
	we obtain that 
	\[
		m_G(D)
		=|D||C_D|\geq \frac{|H||K|}{|J|}\frac{|C_H||C_K|}{|C_J|}
		=\frac{m_G(H)m_G(K)}{m_G(J)}.
	\]
	The second claim is clear. 
\end{proof}

\begin{definition}
\index{Lattice of subgroups}
Let $G$ be a finite group and $\mathcal{L}$ be a collection of subgroups of $G$. We say that $\mathcal{L}$ is a \textbf{lattice} if for every $H,K\in\mathcal{L}$ one has that
$H\cap K\in\mathcal{L}$ and $\langle H,K\rangle\in\mathcal{L}$. 
\end{definition}

Since $G$ is finite, it makes sense to consider the set $\mathcal{L}(G)$ of 
subgroups of $G$ $G$ where the Chermak--Delgado gets its largest value,
say $M_G$. 

\begin{exercise}
	\label{xca:M_S}
	Let $G$ be a finite group and $H$ be a subgroup of $G$. Prove that 
	$M_H\leq M_G$.
\end{exercise}

% \begin{svgraybox}
% 	Sabemos que existe algún subgrupo $K$ de $H$ tal que $M_H=m_H(K)$. Como
% 	$C_H(K)\subseteq C_G(K)$, 
% 	\[
% 		M_H=m_H(K)=|H||C_H(K)|\leq |H||C_G(K)|\leq m_G(H)\leq M_G.
% 	\]
% \end{svgraybox}

\begin{example}
	\label{exa:D8_CD}
    Let $G=\D_8=\langle r,s:r^4=s^2=1,srs=r^{-1}\rangle$ be the dihedral group
    of eight elements. In the subgroups 
    \[
		G,
		\quad
		Z(G)=\{1,r^2\},\quad
		A=\{1,r,r^2,r^3\},\quad
		B=\{1, s,sr^2,r^2\},\quad
		C=\{1,sr,sr^3,r^2\},
	\]
	the Chermak--Delgado measure is $16$ and this is the largest possible value. Thus and $M_G=16$ and $\mathcal{L}(G)=\{G,Z(G),A,B,C\}$. 
	\begin{lstlisting}
gap> ChermakDelgado := function(group, subgroup)
> return Size(subgroup)\
> *Size(Centralizer(group, subgroup));
> end;
function( group, subgroup ) ... end
gap> gr := DihedralGroup(IsPermGroup, 8);;
gap> r := gr.1;;
gap> s := gr.2;;
gap> ChermakDelgado(gr, Subgroup(gr, [r]));
16
gap> ChermakDelgado(gr, Subgroup(gr, [s*r,s*r^3]));
16
gap> ChermakDelgado(gr, Subgroup(gr, [s,s*r^2]));
16
gap> ChermakDelgado(gr, Subgroup(gr, [r^2]));
16
gap> List(AllSubgroups(gr), x->ChermakDelgado(gr, x));
[ 8, 16, 8, 8, 8, 8, 16, 16, 16, 16 ]
	\end{lstlisting}
\end{example}

\begin{theorem}
	Let $G$ be a finite group. The following statements hold: 
	\begin{enumerate}
		\item $\mathcal{L}(G)$ is a lattice. 
		\item If $H,K\in\mathcal{L}(G)$, then $\langle H,K\rangle=HK$.
		\item If $H\in\mathcal{L}(G)$, then $C_G(H)\in\mathcal{L}(G)$ and $C_G(C_G(H))=H$.
	\end{enumerate}
	\label{thm:lattice}
\end{theorem}

\begin{proof}
	If $H,K\in\mathcal{L}(G)$, then $m_G(H)=m_G(K)=M_G$. Let $D=H\cap K$ and $J=\langle
	H,K\rangle$. By Lemma~\ref{lem:CD2}, 
	\[
		M_G^2=m_G(H)m_G(K)\leq m_G(D)m_G(J).
	\]
	Since $m_G(D)\leq M_G$ and $m_G(J)\leq M_G$ (because $M_G$ is the largest possible value), we conclude that $m_G(D)=m_G(J)=M_G$. Hence $\mathcal{L}(G)$ is a lattice. 

	Since $m_G(H)m_G(K)=m_G(D)m_G(J)=M_G^2$, Lemma~\ref{lem:CD2} implies that 
	$J=HK$. 

	By Lemma~\ref{lem:CD1}, 
	\[
	M_G=m_G(H)\leq m_G(C_G(H)).
	\]
	Since $M_G$ is the largest possible value, $m_G(C_G(H))=M_G$. Thus 
    $C_G(H)\in\mathcal{L}(G)$.  By Lemma~\ref{lem:CD1}, $C_G(C_G(H))=H$.
\end{proof}

\index{Chermak--Delgado!subgroup}
Theorem~\ref{thm:lattice} implies the existence 
of the \textbf{Chermak--Delgado subgroup}.

\begin{corollary}
	\label{cor:ChermakDelgado}
	Let $G$ be a finite group. There exists a unique subgroup $M$ minimal 
    suc that $m_G(M)$ is the largest possible value among all the subgroups 
    of $G$. Moreover, $M$ is characteristic, abelian and $Z(G)\subseteq M$. 
\end{corollary}

% f(C_G(H))=C_G(f(H))$ para todo $H$ y todo $f\in\Aut(G)$.

\begin{proof}
	By Theorem~\ref{thm:lattice}, $\mathcal{L}(G)$ is a lattice. Let 
	\[
		M=\bigcap_{H\in\mathcal{L}(G)}H\in\mathcal{L}(G).
	\]
	By Theorem~\ref{thm:lattice},  
	\[
    C_G(M)\in\mathcal{L}(G)
    \text{ and }M=C_G(C_G(M))\supseteq Z(G).
    \]Since $C_G(M)\in\mathcal{L}(G)$, $M\subseteq C_G(M)$. Hence $M$ is abelian. Moreover, $M$ is characteristic in $G$ because $f(M)\in\mathcal{L}(G)$
	for all $f\in\Aut(G)$.
\end{proof}

\begin{example}
	Let $G=\D_8$ be the dihedral group of eight elements. The Chermak--Delgado subgroup of $G$ is $Z(G)\simeq C_2$. See Example~\ref{exa:D8_CD}.
\end{example}

\begin{theorem}[Chermak--Delgado]
	\index{Chermak--Delgado theorem}
 	\label{thm:ChermakDelgado}
	Let $G$ be a finite group. Then $G$ has an abelian characteristic subgroup $M$ such that $(G:M)\leq (G:A)^2$ for every abelian subgroup 
	$A$ of $G$. 
\end{theorem}

\begin{proof}
	Let $M$ be the Chermak--Delgado subgroup of Corollary~\ref{cor:ChermakDelgado} 
    and $A$ be an abelian subgroup of 
	$G$. Then $A\subseteq C_G(A)$. Hence 
	\[
		m_G(M)\geq m_G(A)=|A||C_G(A)|\geq|A|^2
	\]
	and 
	\[
	(G:A)^2
	=\frac{|G|^2}{|A|^2}\geq\frac{|G|^2}{m_G(M)}
	=\frac{|G|}{|M|}\frac{|G|}{|C_G(M)|}
	=\frac{|G|}{|M|}
	=(G:M).\qedhere 
	\]
\end{proof}

\begin{corollary}
	Let $G$ be a non-abelian finite group and $H$ be a subgroup of $G$ such that 
	\[
	|H||C_G(H)|>|G|.
	\]
	Then $G$ is not simple. 
\end{corollary}

\begin{proof}
	Let $M$ be the Chermak--Delgado subgroup of $G$. 
	Then 
	\begin{equation}
		\label{equation:mG}
	m_G(M)\geq m_G(H)>|G|.
	\end{equation}
	This implies that $M\ne\{1\}$, since $m_G(M)=m_G(1)=|G|$. If $G$ is simple, then $G=M$ is abelian. 
\end{proof}

\begin{corollary}
	Let $G$ be a non-abelian finite group and $P\in\Syl_p(G)$. If $P$ is abelian and $|P|^2>|G|$, then $G$ is not simple. 
\end{corollary}

\begin{proof}
	Let $M$ be the Chermak--Delgado subgroup of $G$. Since $P$ is abelian, 
    \[
    (G:M)\leq (G:P)^2<|G|
    \]
    by Theorem~\ref{thm:ChermakDelgado}. Hence 
    $M\ne\{1\}$. If $G$ is simple, then $G=M$ is abelian. 
\end{proof}

We now discuss an application of the Wielandt zipper theorem 
to the Chermak--Delgado lattice. 

\begin{lemma}
	\label{lem:L(G)L(S)}
	Let $G$ be a finite group, $H\in\mathcal{L}(G)$ and  $S$ be a subgroup of $G$ such that 
	$HC_G(H)\subseteq S$. Then $H\in\mathcal{L}(S)$.
\end{lemma}

\begin{proof}
	Since $C_G(H)\subseteq S$, $C_G(H)=C_S(H)$. By Exercise~\ref{xca:M_S},
    $M_S\leq M_G$. Thus $M_G=M_S$, since 
	\[
		M_G=m_G(H)=|H||C_G(H)|=|H||C_S(H)|=m_S(H)\leq M_S.\qedhere 
	\]
\end{proof}

\begin{theorem}
	\label{thm:L(G)subnormal}
	Let $G$ be a finite group. Every $H\in\mathcal{L}(G)$ is subnormal in $G$.
\end{theorem}

\begin{proof}
	We proceed by induction on $|G|$. If $|G|=1$, the result is trivial. So assume the group $G$ is non-trivial. Let $H\in\mathcal{L}(G)$ and $K=HC_G(H)$. Since $H$ is normal in $K$, by the inductive hypotehsis, 
    it is enough to show that 
	$K$ is subnormal in $G$. If $K=G$, the claim holds. So assume that 
	$K\ne G$. 

	Assume that $K$ is not subnormal in $G$. By the inductive hypothesis and 
    Wielandt's zipper theorem (Theorem~\ref{thm:zipper}), there exists a unique 
    maximal subgroup $M$ containing $K$. By Theorem~\ref{thm:lattice},
	$C_G(H)\in\mathcal{L}(G)$ and $K=HC_G(H)\in\mathcal{L}(G)$. By Lemma~\ref{lem:L(G)L(S)},
	$H\in\mathcal{L}(M)$. Hence $K\in\mathcal{L}(M)$. By the inductive hypothesis, $K$ is subnormal in $M$. We claim that $M$ is normal in $G$. Let $x\in G$. Since 
	$m_G(xKx^{-1})=m_G(K)$, the subgroup $xKx^{-1}\in\mathcal{L}(G)$. Hence 
	$K(xKx^{-1})\in\mathcal{L}(G)$. 
	
	If $K(xKx^{-1})=G$, then, since there exist $k_1,k_2\in K$ such that 
	$k_1(xk_2x^{-1})=x^{-1}$, we obtain that $x\in K$, since $x^{-1}=k_2k_1\in K$. This implies that $xKx^{-1}\subseteq K$. Therefore $K=G$, a contradiction.

	Since $K(xKx^{-1})\ne G$, there exists a maximal subgroup $N$ such that 
	$K(xKx^{-1})\subseteq N$. Since $K\subseteq N$, $N=M$ because $M$ is the unique
	maximal subgroup containing $K$. Since $xKx^{-1}\subseteq M$, $K\subseteq
	x^{-1}Mx$. Hence $x^{-1}Mx=M$, because $x^{-1}Mx$ is a maximal subgroup containing $K$ and $M$ is the only maximal subgroup containing $K$. 
\end{proof}

\begin{corollary}
	Let$G$ be a non-abelian finite Then $\mathcal{L}(G)=\{1,G\}$. 
\end{corollary}

\begin{proof}
	Let $K\in\mathcal{L}(G)$. Then $K$ is subnormal in $G$ by Theorem~\ref{thm:L(G)subnormal}. Hence $K\in\{1,G\}$. Now the claim follows from $m_G(1)=m_G(G)$. 
\end{proof}

\begin{exercise}
	Let $n\geq5$. Prove that $\mathcal{L}(\Sym_n)=\{1,\Sym_n\}$. 
\end{exercise}

% \begin{proof}
% 	Let $G=\Sym_n$ y sea $K\in\mathcal{L}(G)$. Por el
% 	teorema~\ref{thm:L(G)subnormal}, $K$ es subnormal en $G$. Si $K\ne G$
% 	entonces se tiene una sucesión estrictamente creciente de subgrupos 
% 	\[
% 	K=K_1\triangleleft
% 	K_2\triangleleft\cdots\triangleleft K_{n-1}\triangleleft K_n=G.
% 	\]
% 	Como $K_{n-1}$ es normal en $G$, $K_{n-1}\in\{1,\Alt_n\}$ y luego $K=1$. 
% 	El corolario queda demostrado al observar que $m_G(1)=m_G(G)$. 
% \end{proof}



\subsection{Miller's double cosets theorem}

\index{Double coset}
Let $G$ be a group and $H$ and $K$ be subgroups of $G$. 
The group $L=H\times K$ acts on $G$ by
\[
(h,k)\cdot g=hgk^{-1},\quad h\in H,k\in K,g\in G.
\]
The orbits of this action are the set of the form 
\[
HgK=\{hgk:h\in H,\,k\in K\}.
\]
A set of the form $HgK$ for some $g\in G$ is called a \textbf{double coset} modulo $(H,K)$ 
with representative $g$. In particular, 
any two double cosets are either disjoint or equal, and $G$ decomposes
as a disjoint union 
\[
G=\bigcup_{i\in I}Hg_iK,
\]
for some set $I$. Let 
\[
L_g=\{(h,k)\in H\times K:hgk^{-1}=g\}=\{(h,g^{-1}hg)\in H\times K\}.
\]
Then
$|L_g|=|H\cap gKg^{-1}|$, 
because there is a bijection $L_g\to H\cap gKg^{-1}$.  
By the fundamental counting principle, 
\[
|HgK|=(L:L_g)=\frac{|H\times K|}{|H\cap gKg^{-1}|}=\frac{|H||K|}{|H\cap gKg^{-1}|}.
\]

We need a lemma. 

\begin{lemma}
\label{lem:Miller}
    Let $G$ be a finite group, $x\in G$, and $H$ and $K$ be subgroups of $G$. Then
    \[
    \#\{zK:zK\subseteq HxK\}=(H:xKx^{-1}\cap H).
    \]
\end{lemma}

\begin{proof}
    Let $L=xKx^{-1}\cap H$ and 
    \[
    \varphi\colon H/L\to\{yK:yK\subseteq H\times K\},\quad 
    hL\mapsto hxK.
    \]

    The map $\varphi$ is well-defined. If $hL=h_1L$, then $h^{-1}h_1\in L$. Thus 
    $h^{-1}h_1=xkx^{-1}$ for some $k\in K$. This means that
    \[
    (h_1x)^{-1}(hx)=x^{-1}h_1^{-1}hx=k\in K,
    \]
    that is $\varphi(hL)=hxK=h_1xK=\varphi(h_1L)$. 

    The map $\varphi$ is surjective: If $zK$ is such that $zK\subseteq HxK$, then 
    $z=hxk$ for some $k\in K$. In particular, 
    $zK=hxK$. Now $\varphi(hL)=hxK=zK$.

    The map $\varphi$ is injective: If $hxK=h_1xK$, then 
    $x^{-1}h_1^{-1}hx\in K$. Moreover, 
    $h_1^{-1}h\in xKx^{-1}\cap H=L$. Thus $h_1L=hL$. 
\end{proof}

\begin{exercise}
\label{xca:Miller}
    Let $G$ be a finite group, $H$ and $K$ be subgroups of $G$, and $x\in G$. Prove 
    that 
    \[
    \#\{Hy:Hy\subseteq HxK\}=(K:xHx^{-1}\cap K).
    \]
\end{exercise}

\begin{theorem}[Miller]
\index{Miller' theorem}
    Let $G$ be a finite group and $H$ and $K$ be subgroups of $G$ 
    of the same index. Then there exists a common complete set
    of representatives for the right cosets of $H$ in $G$ and the 
    left cosets of $K$ in $G$. 
\end{theorem}

\begin{proof}
    Let $Hy$ be a right coset and $zK$ be a left coset. Note that 
    $Hy$ and $zK$ have a common representative
    if and only if $Hy\cap zK\ne\emptyset$, as 
    \[
    Hy=Hx\text{ and }zK=xK
    \Longleftrightarrow 
    xy^{-1}\in H\text{ and }z^{-1}x\in K
    \Longleftrightarrow x\in Hy\cap zK.
    \]

    The group $G$ decomposes as a  
    disjoint union of finitely many double cosets. Each doble coset
    $HxK$ is a disjoint union of finitely many right cosets of $H$ 
    and a disjoint union of finitely many left cosets of $K$. Thus 
    \[
    HxK=\bigcup_{i=1}^kHy_i=\bigcup_{j=1}^lz_jK, 
    \]
    where the unions are disjoint. 
    Since $|H|=|K|$, by applying cardinality, it follows that $k=l$. To prove the theorem
    it is enough to show that each $Hy_i$ intersects every $z_jK$. 
    
    Note that for each $i\in\{1,\dots,k\}$ there exists $j\in\{1,\dots,k\}$ such that
    $Hy_i\cap z_jK\ne\emptyset$. 
    Without loss of generality, we may assume (reordering if needed) that 
    $Hy_1\cap z_jK\ne\emptyset$ for all $j\in\{1,\dots,m\}$, where $1\leq m\leq k$. Then
    \[
    Hy_1\subseteq\bigcup_{j=1}^mz_jK. 
    \]
    Then
    \[
    Hy_1K\subseteq\bigcup_{j=1}^mz_jK\subseteq \bigcup_{j=1}^kz_jK=HxK.
    \]
    Since $Hy_1K$ and $HxK$ are double cosets with non-empty intersection, 
    they are equal. Thus 
    \[
    |HxK|=|Hy_1K|\leq \sum_{j=1}^m|z_jK|=m|K|.
    \]
    By Lemma~\ref{lem:Miller}, 
    \[
    k=\#\{z_jK:z_jK\subseteq HxK\}=(H:xKx^{-1}\cap H). 
    \]
    Therefore
    \[
    k|K|=\frac{|H||K|}{|H\cap xKx^{-1}|}=|HxK|\leq m|K|
    \]
    and hence $k=m$. 
\end{proof}

\begin{exercise}[Hall]
\label{xca:Hall:cosets}
    Let $G$ be a finite group and $H$ be a subgroup of $G$ with $(G:H)=n$. 
    Then there exists $x_1,\dots,x_n\in G$ such that 
    $\{Hx_1,Hx_2,\dots,Hx_n\}=\{x_1H,x_2H,\dots,x_nH\}$. 
\end{exercise}

